\documentclass{note}
\usepackage{mathptm,mydef-ams,amsfonts}
\usepackage{alltt}
\usepackage[T1]{fontenc}
\usepackage[all,knot]{xy}
%\renewcommand{\ttdefault}{txtt}
%\usepackage{MinionPro}

\usepackage{hyperref}
\hypersetup{
    colorlinks, 
    citecolor=black, 
    filecolor=black,
    linkcolor=blue, 
    urlcolor=black
}


\begin{document}

\small

\vspace*{0.5cm}

\begin{center}
\textcolor{blue2}{\large\bf Meadow Use Cases}
\\
$$\xy
%\vtop{\vbox{
\xygraph{!{0;/r0.7pc/:} !{\vover}[u]
  !{\hcap[-2]} [d] !{\vover-} [ruu] !{\hcap[2]}}
%}\smallskip}
\endxy$$

\vspace*{0.8cm}

{\small\today}

\vspace*{0.8cm}

\end{center}

\tableofcontents

\section{Topologies}
\subsection{Pair}
\subsection{Chain}
\subsection{Ring}
\subsection{Clique}
\subsection{Tree}
\subsection{Systolic array}

\section{Classic Problems}

\subsection{Producer-consumer}
\subsubsection{One producer, one consumer}



\subsection{Readers-writers}
\subsection{Dining philosopher}


\section{Hadoop style flow}
\subsection{Coroutine}
\subsection{Ping-ping}
\subsection{Fork-join: Divide-and-conquer}
\[\xymatrix@-0.5pc{
  & *+[F]\txt<1cm>{fork} \ar[d]\ar[dl]\ar[dr]\\
 \txt<1cm>{job0} \ar[dr]& \txt<1cm>{job1}\ar[d] & \txt<1cm>{job2} \ar[dl] \\
  &*+[F]\txt<1cm>{join}&
}\]
An easy way to handle this is to see two points, \bb{fork} and \bb{join} as
the identical point. 

\[\xymatrix@-0.5pc{
  & *+[F]\txt<1cm>{fork/join} \ar@/_2ex/[dl] \ar@/_2ex/[d]\ar@/_2ex/[dr]\\
 \txt<1cm>{job0} \ar@/_2ex/[ur]& \txt<1cm>{job1}\ar@/_2ex/[u] & \txt<1cm>{job2} \ar@/_2ex/[ul] 
}\]
That is, \bb{fork-join} distributes the same token to three nodes and continue
following the flow graph when all three tokens come back.

\begin{alltt}
  \textcolor{red2}{process ForkJoin(dev0, dev1, dev2) \{
    function start() \{
      fork \{
        dev0.job0();
        dev1.job1();
        dev2.job2();
      \} join (?);
    \}
  \}}
\end{alltt}

\subsection{Fork-join-any}
\subsection{Fork-join-none}
\subsection{Map-reduce}
\subsection{Arbitrary graph}

\section{Workflow patterns}
\subsection{Workflow data patterns} 
\subsection{Workflow control patterns} 
\subsection{Workflow resource patterns} 
\subsection{Workflow exception handling patterns} 


\section{Service interaction patterns}

\section{Device selection patterns}
\subsection{Roles: Any device in the group}
Sometimes, we don't need some very specific device for a given role.
Any device, which can satisfy the given role would suffice. 

\paragraph{Situation} In Seoul, we want to get information where the bus \#51
that I'm waiting for is. Any bus with the number \#51 is good that is close to
me. 

\section{Communication patterns}
\subsection{Broadcasting}
\subsection{Multicasting}

\section{Data access/transfer patterns}
\subsection{Pipes}
\subsection{Blackboards}
\subsection{Shared variable}
Synchronization is required.
\subsection{Streaming}

\section{Applications}
\subsection{Messenger}
\subsection{Chatting}
\bibliographystyle{plain}
\bibliography{bib/mac,bib/softsys}
\nocite{HW03,RHAM06,RHEA04a,RHEA04b,BDH05}
\end{document}

%%  LocalWords:  MeadowView EPL Hadoop proc FSM API insertEvent TCP IP ev MV
%%  LocalWords:  Runtime workflow lookup Namespace stateful Meadow Jeong VM
%%  LocalWords:  Cheoljoo runtime HDLs Verilog VHDL HDL APIs softsys compsys
%%  LocalWords:  worklets worklet callee Namespaces FUNC aggregators pamsbook
%%  LocalWords:  Meadowview servicee halfEnabledA halfEnabledB
