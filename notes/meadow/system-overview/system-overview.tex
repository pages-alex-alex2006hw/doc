\documentclass{note}
\usepackage{mathptm,mydef}
%\usepackage{MinionPro}
\usepackage{color}
%\usepackage[scaled]{beramono}
%\usepackage[scaled]{ulgothic}
%\usepackage[ocr-a]{ocr}
%\usepackage{ocr}
%\usepackage{courier}
\usepackage{alltt}
\usepackage[all,knot]{xy}
%\renewcommand*\ttdefault{lmodern}
\usepackage[T1]{fontenc}

\usepackage{hyperref}
\hypersetup{
    colorlinks,
    citecolor=black, 
    filecolor=black, 
    linkcolor=blue, 
    urlcolor=black
}


%\setlength\oddsidemargin{-1.5cm}
%\setlength\evensidemargin{-1.5cm}
%\setlength\textwidth{19.3cm}
%\addtolength\topmargin{-1cm}
%\addtolength\textheight{2cm}
%\addtolength\columnsep{0.2cm}
%\newtheorem{theorem}{Theorem}
%\theoremstyle{definition}
%\newtheorem{definition}[theorem]{Definition}
%\def\EE{\mbox{\eufm{}E}}1

\begin{document}
\small

%\begin{center}
%{\Large\bf Meadow Event Processing Language: User Guide} \par \bigskip
%Cheoljoo Jeong (cjeongkr@gmail.com)
%\end{center}

\title{\large\bf{}\textcolor{blue2}{Meadow:  A Software System for Cooperating
    Devices}}
%\author{\normalsize Cheoljoo Jeong (cjeongkr@gmail.com)}
\author{
$$\xy
%\vtop{\vbox{
\xygraph{!{0;/r0.7pc/:} !{\vover}[u]
  !{\hcap[-2]} [d] !{\vover-} [ruu] !{\hcap[2]}}
%}\smallskip}
\endxy$$
}
\date{\normalsize\today}
\maketitle


\tableofcontents

\section{Introduction}
Programming multiple, autonomous devices to accomplish a task
is not easy. Consider programming $n$ devices to 
perform a task which requires interaction of these devices.
Assuming that they depend on {\em message passing\/} for interaction,
which is the norm in distributed computing, 
guaranteeing that the system works correctly under any order of
interactions is not trivial. Though huge amount of research efforts
have been devoted to the study of behavior of concurrent systems, 
most efforts have been focused on
{\em specification\/} and {\em verification\/} of systems, rather than 
``programming'' such systems.%~\cite{Hoare85,MP91,Milner89,Murata89}. 

As more programmable devices are connected, coordinating them
to perform tasks is becoming common activities. Many new 
technologies, such as Internet of Things, robot programming, industrial
automation, and smart cars, will only accelerate this trend.

In particular, \textcolor{blue2}{\textbf{the Meadow system}} is a software
system for programming cooperating devices, where users can program
interaction of devices from a single viewpoint, rather than from multiple,
different viewpoints (i.e. one per each participating device).
Devices are viewed as ``programmable objects'' in the Meadow system, 
which provide \textcolor{blue2}{services} to other devices.
Using the \textcolor{blue2}{\textbf{Meadowview scripting language}}, users 
can define a process over a set of devices, by composing existing device
services to accomplish a task.  
Later, the defined process can be installed in actual devices and then,
executed. 
 
\subsection{Benefits}
\paragraph{Device choreography}
When a task involves multiple devices, adding behaviors to devices
  which collectively perform a single task is not easy. The Meadow system
  allows to define a process over devices, which will then be synthesized into
  code fragments which will be deployed to devices. Execution of these devices
  will perform the task.

\paragraph{Workflows over devices}
Also, only using pre-deployed functionalities in devices, one can define a
workflow over a set of devices. This workflow will be instantiated using 

\paragraph{Deployment of reactive code to devices}


\paragraph{Dynamically typed language}
Provides a dynamically-typed language which expedites the development of
   the system and avoids all hassle of statically-typed languages at the 
   expense of performance.\footnote{Latency in distributed programs are
     already quite high so we may not need to worry too much about performance
    at this stage. Also, all the messes of type compatibility between
    different languages, which we might eventually support through language
    bindings could be avoided to some degree.}.

\paragraph{Dynamic device interface}
A device interface is defined as the {\em set of services\/}
provided by the device.
It represents the behavior of device, i.e. what it can provide to the rest of
the (Meadow) world. Users can dynamically change the services of devices
unlike many of the distributed computing frameworks such as CORBA. 

This
simplifies the deployment of new services in many devices and removes
the burden of interface repositories or service discovery. 
A device requests a service to another device, and when the device does not
have the requested service, then we \textcolor{blue2}{\bf{}let it fail} --
it's the developer's or system maintainer's responsibility to ensure that the
target device actually provides the service at the moment. 

\paragraph{Scalable event processing}
Complex event processing aims intelligent handling events by filtering and
processing countless events into a small set of value-added events.
However, processing events at a central server is difficult since the number
of events to be processed will increase very quickly. A better way to handle
is to let ``event generator'' to generate filtered, processed, and refined
events rather than generating raw, unrefined events. Meadow runtime allows to
add ``filtering/processing'' logic at the site of event geneation.

\paragraph{Runtime type checking}
Even if there are no static type checking, each value carries a type and
whenever an ``operation'' is executed, type checking on the operands is
performed. Also, whenever runtime type error is found, it will be reported to
the original service requester so that it can be handled.


%\w %\paragraph{Interoperability with multiple standards}
\paragraph{Interoperable with multiple standards}
The Meadow system understands multiple standards, such as 
{\em AllJoyn\/}, {\em Open Interconnect
Consortium standards\/},  {\em Apple HomeKit\/}, {\em Google API/}, etc.
This capability will allow the devices which only conform to AllJoyn standard
to use the service of devices which only conform to Apple HomeKit. 

\paragraph{Persistent device services}
Unlike distributed languages (like Scala, Erlang, dRuby), properties,
   functions, etc. are persistent.\footnote{\textcolor{green2}{Should we make
       persistence as a system service rather than making it a default?}} 
   When a device shuts down for some reason, rebooting the device (and Meadow
   runtime) will restore the device to some previous checkpoint.

\paragraph{Both interpreted and compiled}
Users can add interpreted Meadowview script
but, for performance, users can opt to compile the code into native code and 
deploy the native code.

\subsection{Meadow devices}
In the Meadow system, {\em devices\/} are key elements of the
system. A \textcolor{blue2}{\bf{}device} is a uniquely identifiable entity,
which can provide and request services. 
In particular, a unique, persistent name, in the form of dot-separated
string\footnote{Let's say, for now, we follow Java package naming convention
  -- e.g. \texttt{com.sun.java.util}.},
is assigned to each device. The {\em behavior\/} of a device is characterized
by the set of services it provides, which can change dynamically as
services are added or removed. 

Conceptually, \textcolor{blue2}{\em any entity which can run a
  \textcolor{blue2}{\bf{}\em Meadow VM} is a {\bf\em Meadow device} \/}. 
For example, a Raspberry Pi board with a Meadow VM installed can be viewed as a
Meadow device. One can attach a LED light to the board and create a service 
which turns on or off the light.
Also, a Linux box can be viewed as a Meadow
device, if a Meadow VM is installed on the box.
When one can import MqSQL operations into this VM, this device can support
database operations.\footnote{\textcolor{green2}{Actually, a general Linux box
    as a Meadow device can be useful by itself in a non-IoT context. 
    Check if this can
    compete with Hadoop, Pig, etc. For this, design the Meadow language so that
    MapReduce job can be described.}}

\subsection{Device services}
A device can provide services to other devices. 
A \textcolor{blue2}{\bf{}service} provided by a device is either a
{\em property}, an {\em event\/}, or a {\em function\/}. 
A set of services provided by a Meadow device is called
\textcolor{blue2}{\bf{}device interface}. 
\textcolor{blue2}{\em The interface of a device can change
dynamically by adding or removing services\/}.

\paragraph{Properties}
A device can have properties.\footnote{\textcolor{green2}{That is, a device is
    one of possible entities which can ``own'' properties. Actually, the most
    common and important class of property owner.}} A
  \textcolor{blue2}{\bf{}property} 
of a device is a named value, which can be read or written
locally (by the device which contains the property) or remotely.
Properties support {\em pull-based dataflow model\/}.

For example, a device can have a property,
\textcolor{red2}{\texttt{remainingBattery}}, which is a real number
between 0 and 1. Consider a factory which has multiple battery-operated
devices. Then, one may want to check if all devices have enough battery life. 
Users can add a Meadowview script which monitors the battery life of devices
periodically (and do some action, such as sending an email, when the
battery life is low)  
or create a program which grabs and displays these properties
using the \textcolor{blue2}{\bf{}Meadow API}\footnote{\textcolor{green2}{The
    Meadow system provides a C library (also in C++, Java, Python, etc) which
    allows to access device services.}}.


\paragraph{Events}
A device can {\em publish\/} and/or {\em subscribe to\/} events.
An \textcolor{blue2}{\bf{}event} is a named value, 
which can be read locally or remotely. 
An event is delivered to its subscribers, whenever
an event producer publishes an occurrence of the event.
Events support {\em push-based dataflow model\/}.

As an example, a device with a GPS receiver can produce an event,
\textcolor{red2}{\texttt{locationChanged}}. 
This event can be ``used'' by its subscribers by defining ``event handlers''
for the event. An event handler can be defined in the form of a {\em
  reactor\/} in the Meadow language or a program which uses the Meadow API
provided by the Meadow system.


\paragraph{Functions}
A device can have functions, which can be called locally or remotely.
A \textcolor{blue2}{\bf{}function} is a piece of code parametrized by
symbols, which can be executed by its local or remote consumers.
A function can be invoked in either blocking or nonblocking manner.

Consider a ``MySQL Server'' device as an example. One can add
a Meadow function, \textcolor{red2}{\texttt{addUser(name, age, address,
    phone)}}, to the device by \textcolor{blue2}{\em importing\/} a
corresponding C function which performs a DB operation over the MySQL server.


\paragraph{Imported vs user-defined services}
A service of a device can be either {\em imported\/}  or {\em defined\/} by
the user. 

As an example of an \textcolor{blue2}{\bb{imported service}}, consider an LCD
panel device. The manufacture of the panel device can provide a static library 
compiled for Atmel AVR micro-controller.
This library may provide a function, say
``\textcolor{red2}{\texttt{void display\_message(char *)}}'', which allows to 
write a string to the panel. The Meadow system provides a method for importing
such functionality into a device, say as a {\em Meadow function\/} so that 
the functionality can be used locally or remotely.

\textcolor{blue2}{\em Device services through another software frameworks,
  such as {\bf\em AllJoyn}, can also be imported into Meadow devices.
We could also support {\bf\em automatic discovery and registration of
external services\/} such as AllJoyn services.}


In addition to imported services, \textcolor{blue2}{\bb{user-defined
    services}} can be described using the {Meadowview  language} and {\em
  installed\/} (or deployed) to the device. After being installed, the
function can be used locally or remotely.


\subsection{Coordinating device interaction}
In the Meadow system, we are most interested in coordinating interaction
between devices to perform tasks.
%% In particular, we want to define collective behavior of devices by 
%% specifying how each device uses services of other devices.
%% one device can query the value of a property of another
%% device. 
For pull-based services, such as properties or functions, we can 
directly use such services inside the Meadowview code. To use events, which 
is a push-based service, we need to add some code which represents the
behavior that we want when the given event occurs.
For this purpose, the Meadow system allows to define and install {\em
  reactors\/} to devices.

\paragraph{Reactors}
A \textcolor{blue2}{\bf{}reactor} is a piece of code,
which is executed whenever the associated event occurs. 
It can also be called as a \textcolor{blue2}{\em callback\/} or an
\textcolor{blue2}{\em event handler\/}.  

As an example, let two devices be given: \textcolor{red2}{\texttt{Switch}} and 
\textcolor{red2}{\texttt{LedLight}}, where the former can generate
\textcolor{red2}{\texttt{changed}} event 
and the latter provides functions, 
\textcolor{red2}{\texttt{turnOn}} and 
\textcolor{red2}{\texttt{turnOff}}.
We want the toggle switch device to control the LED light device.
Consider the following Meadowview code as an example.
\begin{alltt}
  \textcolor{red2}{reactor LedLightControl(Swtich:changed ev) \{
    if (ev.value == 1)
      LedLight:turnOn();
    else
      LedLight:turnOff();
  \}}
\end{alltt}
This reactor code can be deployed either to the
\textcolor{red2}{\texttt{Switch}} device 
or the \textcolor{red2}{\texttt{LedLight}} device. Deploying the code to a
third device is also possible.



\subsection{Meadow processes}\label{welcomeproc}
The Meadow system allows to describe collective behavior of the
devices in a {\em process description\/}. 
A \textcolor{blue2}{\bf process} is a collection of properties, events,
functions, and reactors. 
A process can be {\textcolor{blue2}{\em instantiated with actual devices\/},
  which amounts to {\em installation of the process into devices\/}. 
After installation, a process can be executed either by explicitly calling
some ``start'' function or by generation of an event to which some reactors
of the process is sensitized to.

As an example, let four devices be given.
One device, called \textcolor{red2}{\texttt{detector}} which detects any Meadow
device which comes nearby. Whenever it detects a device, it generates an event
called \textcolor{red2}{\texttt{detected}}, which contains the ``detected
device''\footnote{\textcolor{green2}{Technically, the unique id of the
    device.}}, as its payload. 
A device named \textcolor{red2}{\texttt{door}} is an automatic door, which
opens the door. 
Also, an LCD display, called \textcolor{red2}{\texttt{lcd}}, which can display
any string on the panel.
Finally, the \textcolor{red2}{\texttt{logger}} device provides the function
\textcolor{red2}{\texttt{logEntry(name, time)}}. 
\begin{alltt}
  \textcolor{red2}{process WelcomeProc(detector, door, lcd, logger) \{
    reactor (detector:detected ev) \{
      // open the door
      door:open();

      // display the welcome message on the LCD panel
      name = ev.device:getOwnerName();
      lcd:display("welcome" + name);

      // log the entry; time is a predefined attribute of any event,
      // which contains the time of event occurrence
      logger:logEntry(name, ev.time);
    \}
  \}}
\end{alltt}

%. Then, , installed into devices
%and then executed.
%% The process consists of a set of events, to which it is
%% reactive, and a piece of code. The code is executed when the given events
%% occur.  The code can perform computation using the services from the involved
%% devices. 
%%A process (definition) can be instantiated with actual devices and executed by
%% the devices. 


%% \subsection{Who ``fires'' actions?}
%% The world of the Meadow system (which is a software system) consists of
%% ``Meadow devices'', each of which provides a set of services.
%% These devices cannot do any work unless 1) ``someone'' request services
%% or 2) ``some events'' occur. So some sort of triggering from the outside
%% world is required to making the system running. 

%% In the Meadow system, the following are ways how things begin to work.
%% \bit
%% \w \bb{reactors sensitized to timers}: this is the only case where the event
%%     is generated from inside the Meadow system. 
%% \w \bb{events generated from external stimulus}: e.g. button pressed by human
%% \w \bb{``start'' function of a process}:
%% \w \bb{external programs which uses Meadow API}
%% \eit



%% \textcolor{red2}{\em The notion of {\bf\em devices} is very similar to that of {\bf\em classes\/} in a
%% (class-based) object-oriented programming language, such as C++. \/}
%% Just like a C++ class represents a set of objects which has the same set of
%% fields and member functions, a device represents a set of device instances
%% which has the same set of properties, events, and native functions.
%% Differences include that devices support``events'' which facilitate
%% {\em asynchronous programming\/} through event-based, reactive programming
%% model of the Meadowview language.
%% \footnote{The event-based programming model is
%%   arguably the most scalable approach to build large, complicated, distributed
%%   systems. The Verilog hardware description language~\cite{Verilog2001} is based on event-based
%%   programming model and has been used to build highly-complex chips with
%%   billions of gates. One characteristics of distributed systems is that we are
%%   don't have a knowledge about when ``something will happen'', since we don't
%%   have a full control/knowledge over all computational activities. In many
%%   cases, only way to add some behavior is through adding callbacks to events
%%   and let these events--reaction chains do the job.}

%% \paragraph{\small\bf{}Lifetimes of devices and device instances}
%% Another difference is the lifetimes of classes and objects are different than
%% those of devices and device instances. For example, let a C++ program that
%% defines a class \verb+MyClass+ be given. Then, this class begins to exist when
%% object file of the program which uses this class is executed. An object of
%% the \verb+MyClass+ class is created when the runtime generates an internal
%% structure for the object, at the time when ``\verb+new MyClass+'' is to be
%% executed. Also, it is destroyed as this object is destroyed by the C++
%% semantics. 

%% Device is more like a database scheme, which is created and registered into 
%% a global registry or a name directory. Unless deregistered, the device
%% persists. 
%% While objects are abstract entities which can be
%% created and destroyed by the runtime, a device instance is an ``handle'' to a
%% concrete, physical entity. So, its creation and deletion is more like a
%% registration and deregistration to a registry or a name directory.


%% While objects are created by the runtime, the existence of device instances is 
%% possible inside the runtime by the device instances themselves by {\em
%%   sign-on\/} and {\em sign-off} procedures. 

%% \paragraph{Workflows}
%% A \textcolor{blue2}{\bf{}workflow} is a set of processes, prarameterized by
%% {\em devices\/}. A workflow can be instantiated with actual device instances.  

%% For further details about the concepts discussed in this section, see the
%% companion document, ``\textcolor{blue2}{Meadowview: A Language for Device
%%   Orchestration}''.


\section{The Meadowview Language}
The Meadowview is a language for describing behavior of individual devices
or collective behavior of multiple devices.
The language consists of two parts: \textcolor{blue2}{\bb{the core language}}
for describing individual device behavior
and  \textcolor{blue2}{\bb{the process language}} for describing
collective behavior of devices.

%% \subsection{Elements of the language}
%% The Meadowview language allows to manipulate the following concepts:
%% \bb{properties},
%% \bb{events},
%% \bb{functions},
%% \bb{reactors}, and
%% \bb{processes}.
%% Actually, each of these are first-class values in Meadowview. 

%% The behavior of a ``Meadow device'' is initially defined by its native
%% services -- native properties, native events, and native functions. Using
%% the Meadowview language, uses can define user-defined capabilities
%% on top of these native capabilities. 

%% \paragraph{Properties}
%% A \textcolor{blue2}{\bf{}property} is a variable which is contained 
%% in a Meadow device.
%% The value of a property can be {\em pulled from outside\/}.
%% Also, properties can be used locally by a local function or a reactor.

%% \paragraph{Events}
%% An \textcolor{blue2}{\bf{}event} is a variable which is contained in a Meadow
%% device.  The value of a property can be {\em pushed to the outside upon any
%%   change in its value\/}. 
%% Also, events can be used locally by a local function or a reactor.

%% \paragraph{Functions}
%% A \textcolor{blue2}{\bf{}function} is executable code which is contained in a
%% Meadow device. A function can be invoked from other devices.
%% Also, functions can be used locally by a local function or a reactor.

%% \paragraph{Reactors}
%% A \textcolor{blue2}{\bf{}reactor} is executable code which is contained 
%% in Meadow device. Each reactor is associated with an event and, whenever
%% the event changes its value, reactor is executed. 
%% By adding reactors to devices, we can add new behavior to a set of 
%% Meadow devices.

%% \paragraph{Processes}
%% A \textcolor{blue2}{\bf{}process} is a  description 
%% of collective behavior of devices to accomplish a task. 
%% A process is a collection of events, properties,
%% functions, and reactors, {\em parameterized with devices\/}.

\subsection{Core language}
The core language allows to define 
\textcolor{blue2}{\bf properties}, 
\textcolor{blue2}{\bf events}, 
\textcolor{blue2}{\bf functions} and 
\textcolor{blue2}{\bf reactors}. 
The core language is a \textcolor{green2}{\em native language of Meadow VM\/},
meaning that Meadowview code in core language can be directly interpreted by a
Meadow VM. 


\subsection{Process language}
The process (definition) language allows to define 
\textcolor{blue2}{\bf processes}.
The code written in process definition language {\em cannot\/} be 
directly interpreted by
Meadow VMs. Therefore, a process definition must be synthesized into code in
the core language before they are deployed to devices. 

\paragraph{\textcolor{red2}{\texttt{WelcomeProc}} revisited}
The \texttt{WelcomProc} process definition,
\begin{alltt}
   \textcolor{red2}{process WelcomeProc(detector, door, lcd, logger)}
\end{alltt}
discussed in Section~\ref{welcomeproc} can be  compiled into {\em object
  code\/}, which is a set of $(\arc{device}, \arc{service})$ pairs. There are
many ways to compile above definition but one possible object code is given
below. 
\begin{alltt}
  \textcolor{red2}{// code to be deployed to door
  event doorOpened;
  // pragma deploy(door) -- forces deployment of reactor to door
  reactor(detector:detected ev) \{ 
    :open(); 
    event_publish(EVENT(doorOpened, {device => ev.device}) /* ctor */);
  \}

  // code to be deployed to lcd
  event lcdDisplayed;
  reactor(door:doorOpened ev) \{
    ownername = ev.device:getOwnerName();
    :display("welcome" + name);
    event_publish(EVENT(lcdDisplayed, {name => ownername}));
  \}
 
  // code to be deployed to logger
  reactor(lcd:lcdDisplayed ev) \{
    logger.logEntry(ev.name, ev.time);
  \}}
\end{alltt}
Upon process instantiation using actual devices, the service code will be
instantiated with actuals and deployed to them. Let actual devices
be 
\textcolor{red2}{\texttt{com.mv.b1.l1.detector1}},
\textcolor{red2}{\texttt{com.mv.b1.l1.door}},
\textcolor{red2}{\texttt{com.mv.b1.l1.lcd}}, and
\textcolor{red2}{\texttt{com.mv.logger}}. 
Then, following instantiated code will be installed to above devices.
\begin{alltt}
  \textcolor{red2}{// instantiated code installed on com.mv.b1.l1.door
  event doorOpened;
  reactor(com.mv.b1.l1.detector:detected ev) \{
    :open();
    event_publish(EVENT(doorOpened, {device => ev.device}) /* request ctor */);
  \}
  // code to be installed on com.mv.bl.l1.lcd
  event lcdDisplayed;
  reactor(com.mv.bl.l1.door::doorOpened ev) \{
    name = ev.device:getOwnerName();
    :display("welcome" + name);
    event_publish(EVENT(lcdDisplayed, {name => ownername}));
  \})
  // code to be installed on com.mv.logger
  reactor(com.mv.bl.l1.lcd::lcdDisplayed ev) \{
    logger.logEntry(ev.name, ev.time);
  \})}
\end{alltt}
\footnote{\textcolor{green2}{Q: What is the \bf{FAILURE SEMANTICS}? 
What happens when a function or reactor fails in the middle of execution?
  %% There are two
  %% \texttt{getOwner} function call above. When only one of them succeeds, what
  %% will happen? When a process consists of multiple different code fragements,
  %% when one succeeds and other fails, what should we do. Rollback partial
  %% results back to original?  ACTUALLY, ABOVE CODE IS NOT A GOOD CODE. IF SOME
  %% EVENT (e.g. logging) TO TAKE PLACE ONLY AFTER ALL THINGS HAPPENED, WE COULD
  %% CREATE A ``JOIN'' and DO THAT AFTER JOIN.
}}





\section{The Meadow VM}
The Meadow system consists of Meadow devices, where each device runs a
Meadow VM (or Meadow runtime), which is characterized by the properties,
events,  functions, and reactors, it contains.
The \textcolor{blue2}{\bf{}Meadow VM} is a middleware  which is responsible for
the following functionalities: 
\bit
\w \textcolor{blue2}{\bb{Management of device services}}: Allows to
   dynamically add, remove, update {\em properties\/}, 
   {\em events\/}, {\em functions\/} (imported or user-defined), and
   {\em reactors\/}.

\w \textcolor{blue2}{\bb{Device service provider}}: allows local or
   remote service consumers to access properties, 
   events, and functions.

\w \textcolor{blue2}{\bb{Scheduling and evaluation of reactors}}: 
   Upon receipt of an event, fetch all reactors that subscribes to the
   event and evaluates them. As a result of evaluation, new reactors can
   be scheduled.

\w \textcolor{blue2}{\bb{Bridge to other software frameworks}}:
   Allows to import functions in AllJoyn framework, subscribe to event in
   Apple Home Kit, etc.

\w \textcolor{blue2}{\bb{Message bus}}: 
   All communication between devices are done in the form of messages.
   For example, a message can encode many different types of information:
   event occurrence, function call request, function call response, 
   some system information for bus maintenance.

   For now, use existing tools such as \textcolor{blue2}{ZeroMQ}.
\eit

\subsection{Architecture of Meadow VM}
\[\xymatrix@-0.5pc{
 & *+[F]\txt<1.4cm>{\textcolor{blue2}{\em{}Service Tables}} &&
  *+[F]\txt<1.4cm>{\textcolor{blue2}{\em{}Device Table}}\\
 & & *+[F]\txt<1.4cm>{\textcolor{blue2}{\bf{}Evaluator}}\ar[ld]\ar[lu]\ar[ru]\ar@/^7ex/[rdd]\\
  & *+[F]\txt<1.4cm>{\textcolor{red2}{Stratified Event
  Queue}} & *+[F]\txt<1.4cm>{\textcolor{red2}{\bf{}Scheduler}} \ar@{<->}[l]\ar[u]
\ar[d]& \\
&*+[F]\txt<1.4cm>{Message Decoder} \ar[u]&*+[F]\txt<1.4cm>{\textcolor{red2}{\em Event Reactor Map}}
&*+[F]\txt<1.4cm>{Message Encoder} \ar[d]&\\
&  *+[F]\txt<1.4cm>{Message Input Queue}\ar[u] && *+[F]\txt<1.4cm>{Message Output
    Queue} \\
& &  *+[F]\txt<1.4cm>{Message Bus} \ar[ul]\ar@{<-}[ru]
}\]

\paragraph{Event reactor map} A map from {\em events\/} to {\em event
  reactors\/}.  Whenever an event is fetched from an event queue, 
event reactor map is looked up for associated reactors. 


\subsection{Example flow: Device property query}
Assume that there are two devices, \textcolor{red2}{\texttt{A}} and 
   \textcolor{red2}{\texttt{B}}. 

\ben
\w \bb{Device \textcolor{red2}{\texttt{A}} sends request}

\ben
\w \textcolor{red2}{\texttt{A}} wants to access property named
\textcolor{red2}{\texttt{batteryLife}} of \textcolor{red2}{\texttt{B}},
e.g. when it executes the code \textcolor{red2}{\texttt{life =
    B:batteryLife}}. 
\w This assignment causes \textcolor{red2}{\texttt{evalExpr}}
  function over the \textcolor{red2}{\texttt{B:batteryLife}}. This, in turn, 
  causes evaluation of \textcolor{red2}{\texttt{propGet("B", "batteryLife")}}.

\w The body of \textcolor{red2}{\texttt{propGet}} will create a
\textcolor{blue2}{message} which encodes the ``get property'' request destined
to \textcolor{red2}{\texttt{B}}, and put the message into outgoing message
queue. Technically, the message request amounts to generation of
\textcolor{blue2}{``builtin event''}, say
\textcolor{red2}{\texttt{propGetRequest}}, 
which every device (VM) automatically supports
(i.e. subscribes).\footnote{Note that an {\em event provided by a device can
    be generated either by itself or from other device\/}.}

\w Message bus will route the message to \textcolor{red2}{\texttt{B}}. We can
use existing message bus libraries, such as ZeroMQ.

\w Also, two things are created (or instantiate pre-created ones):
   \bit 
   \w Event called \textcolor{red2}{\tt{}propGetResponse}.\footnote{\bf How to differentiate two
     different \textcolor{red2}{\texttt{propGetReponse}}, one called from
     during evaluation of function \textcolor{red2}{\texttt{foo}}, the other
     from \textcolor{red2}{\texttt{bar}}? When two responses come back, how
     can we pick proper continuation out of two?}
   \w Reactor that listens to this event, i.e. {\em continuation\/} from the
   point when  the value is ready.
   \eit
\een

\w \bb{Device \textcolor{red2}{\texttt{B}} handles request}
\ben
\w From the incoming message queue, the message requesting the property of
   B is retrieved, parsed.
\w Already scheduled reactor for \textcolor{red2}{\texttt{propGetRequest}} event
   is executed. Whose body will create a another message with the value of
   property as its payload. This message is technically an another event
   called \textcolor{red2}{\texttt{propGetResponse}}. 
\w Message is sent to \textcolor{red2}{\texttt{A}}.
\een

\w \bb{Device \textcolor{red2}{\texttt{A}} handles response}
\ben
\w From the incoming message queue, the message containing the response 
   is retrieved, parsed.
\w Already scheduled reactor for \textcolor{red2}{\texttt{propGetResponse}}
   is executed, whose body is essentially a \textcolor{blue2}{\bf
     continuation} from the point for property query.
\een
\een

\subsection{Events vs event reactors}
One technical issue to handle is that when there are two reactors to the same
event. Let's say there are two functions in a device,
\textcolor{red2}{\texttt{foo}} and \textcolor{red2}{\texttt{bar}}.
While executing these functions, \textcolor{red2}{\texttt{foo}} executed
property queries, \textcolor{red2}{\texttt{dev0.val0}} and
\textcolor{red2}{\texttt{dev1.val1}}, respectively. 
Then both functions are actually subscribing
\textcolor{red2}{\texttt{propGetResponse}} event, with its own reactors,
say \textcolor{red2}{\texttt{r0}} and \textcolor{red2}{\texttt{r1}}
respectively.\footnote{Or, we could consider two property queries which is
  executed from side {\em two invocations of \textcolor{red2}{\texttt{foo}}}}

When a \textcolor{red2}{\texttt{propGetResponse}} event arrives to this
device, which reactor should we evaluate? \textcolor{blue2}{\em This actually amounts to how to emulate ``session'' in
  stateless message passing system.\/}
There may exist several ways to handle this. For now, we can use a method
based on a \textcolor{blue2}{\bf{}token}, which allows to add
\textcolor{blue2}{\em causality\/} between the request and response. 

In particular, request carries a \bb{token}, a unique integer, 
which will be copied to response message when the servicing device responds, 
so that the original requester can figure out which request the response
corresponds to.
That is, when reactor (or reactor instance) is added to reactor queue,
we check the token to see if that's what the reactor is waiting for.

This suggests that we might need to revise \textcolor{blue2}{\bf event reactor
  map} such that it will be either a map from ({\em event, tag\/}) pairs to
reactors. The {\em tag\/} can be either a wildcard or a unqiue number.
Typical user-generated reactors are wildcard-reactors while some
internal reactors which require causality are added as a \textcolor{blue2}{\bf
  tagged reactor\/}.


\subsection{Handling exceptions}
In distributed systems, \textcolor{blue2}{\bf exceptions and failures are very
  common.} So, either linguistic and system-wide measures should be provided
to handle such situations. 


\section{The Meadow System Services}
In addition to Meadow devices and their services, there are 
system-wide services. Some are essential services but some other can be
provided as paid services.

Services can be implemented in different forms depending on the service.
For example, device registry service  is most likely implemented as functions
in a libray. However, logging service can be implemented a Meadow device which
uses a predefined Meadowview library.


\bit
\w \textcolor{blue2}{\bb{Meadow device registry service}}: 
   Maintains a global persistent namespace of devices.\footnote{\textcolor{green2}{Q: Should events, properties, functions, reactors
     managed in a global table? Or should each device which contains such
     services be the ``manager'' of such tables? Or hybrid?}}
   Also, maintains information on whether the device is connected to the
   ``system'' and how one can talk to the device (e.g. IP address and port).

\w \textcolor{blue2}{\bb{Meadow process service}}:
   Maintains a global persistent namespace of processes. 
   Also, supports instantiation of processes.

\w \textcolor{blue2}{\bb{Meadow device group service}}:
   Meadow devices can be grouped so that some per-group behavior (events,
   properties, functions, reactors) can be defined.\footnote{\textcolor{green2}{Q: implemented as P2P overlay network?}}

\w \textcolor{blue2}{\bb{Meadow transaction service}}:

\w \textcolor{blue2}{\bb{Meadow concurrency service}}:

\w \textcolor{blue2}{\bb{Meadow security service}}:
  Supports authentication, authorization, etc.

\w \textcolor{blue2}{\bb{Meadow logging service}}:
  Allows logging of activities. Needed for auditing purposes.

\eit

\subsection{Meadow device registry service}
The device registry service is implemented as functions in the C library.
External C programs or web programs can access these functions to add a new
device name. 

\paragraph{\small\textcolor{red2}{\tt{}int devicereg\_register(char *device\_name)}}
  Introduces a new device name into the Meadow system. The name must be
  globally unique and this function is supposed to check its unqiueness.

\paragraph{\small\textcolor{red2}{\tt{}int devicereg\_unregister(char *device\_name)}}
  Removes the device name from the Meadow system.

\paragraph{\small\textcolor{red2}{\tt{}int devicereg\_exists(char
    *device\_name)}} 
  Checks if the device name is already registered or not.


\subsection{Meadow process service}
The Meadow process service provides interface for management of processes.
In particular,
\bit
\w \textcolor{blue2}{\bf{}management of process definition table}
   \bit
   \w mapping from process name to ``compilation result of process
   definition''
   \eit
  
\w \textcolor{blue2}{\bf{}management of process lifecycle}
   \bit
   \w \bb{instantiation of process}, which amounts to
      \bit
      \w find compilation result of the given process 
      \w deploy code to bound devices
      \eit
   \w fire ``initial events'' which will cause chain reaction of device actions
   \w \bb{uninstallation} of processes if needed\footnote{Think more. Maybe
     quite challenging.}
   \eit
\w \textcolor{blue2}{\bf{}monitoring process activities}
\eit

\paragraph{\small\textcolor{red2}{\tt{}int proc\_register(char *proc\_name, char *proc\_def)}}
\paragraph{\small\textcolor{red2}{\tt{}int proc\_unregister(char *proc\_name)}}
\paragraph{\small\textcolor{red2}{\tt{}PROC\_HANDLE\_t proc\_lookup(char *proc\_name)}}

\paragraph{Meadowview process compiler}
A Meadowview process compiler takes programs written in \textcolor{red2}{\em
  Meadowview process definition language\/} as its input and produces  
programs in {\em Meadowview core language}. 

Let a process definition involve $n$ devices. 
Then, the compilation result is $n$ code fragment, each of which will be
deployed to a Meadow device. Technically, it will be a set of 
   \[ (\textrm{\arc{device\_class}}, \textrm{\arc{core\_item}})\]
pairs, where \arc{device\_class} is a ``symbol'' which denotes a class of
devices and \arc{core\_item} is one of the following:
\bit
\w property definition
\w event definition
\w function definition
\w reactor definition
\eit

\subsection{Meadow device group service}

\subsection{Meadow security service}
\paragraph{Authorization model}
Basically, authorization is where a given \bb{subject} is allowed to do
\bb[operation} on an \bb{object}.
In the Meadow system, authorization is performed at the following level:
\bit
\w \bb{subjects}: device
\w \bb{objects}: task, function, property
\w \bb{operations}: for now, just add/delete/read/write
\eit


\subsection{Meadow logging service}

%% \def\g#1{\save
%%   [].[ddrr]!C="g#1" \POS*\frm{.} \restore}
%% \xymatrix@-0.5pc{
%% \g1 \txt{VM$_0$}  & *+[F]\txt<1.8cm>{Meadowview Evaluator} \ar@{<->}[d]\\
%%   *+[F]\txt<0.9cm>{Service Tables} \ar@{<->}[r]\ar@{<->}[ru]&*+[F]\txt<1.8cm>{VM Controller} \ar@{<->}[r]\ar[d]&  *+[F]\txt<1cm>{Message
%%     Bus}\\
%%   &*+[F]\txt<1.8cm>{Global Tables}  &
%% }
%\section{System Overview}
%% \[
%% \xy
%% \xymatrix"*"@-0.5pc{
%% \txt{VM$_0$} & *+[F]\txt<1.5cm>{\textcolor{red2}{\bf{}MV Evaluator}} \ar@{<-}[d] \ar@{->}[rd]& \\
%%   *+[F]\txt<0.9cm>{Event Queue}
%%   \ar@{->}[r]&*+[F]\txt<1.5cm>{\textcolor{blue2}{\bf{}VM Controller}} \ar@{->}[r]\ar@{->}[rd]\ar@{<->}[d]&
%%   *+[F]\txt<1.1cm>{Service Tables} \\
%%   &*+[F]\txt<1.5cm>{\em{}Message Bus}\ar[lu] & *+[F]\txt<1.1cm>{Global Tables}
%% }
%% \POS*\frm{.}
%% \POS+(70,0)
%%  \xymatrix{
%%   & *+[F]\txt<1.5cm>{\textcolor{red2}{\bf{}MV Evaluator}} \ar@{->}[ld]\ar@{<-}[d]& \txt{VM$_1$}\\
%%  *+[F]\txt<1.1cm>{Service Tables}\ar@{<-}[r]&*+[F]\txt<1.5cm>{\textcolor{blue2}{\bf{}VM Controller}}
%%  \ar@{->}[r]\ar@{<->}[d]&  *+[F]\txt<0.9cm>{Event Queue}
%%  \ar@{<-}[l]\\
%% *+[F]\txt<1.1cm>{Global
%%     Tables} \ar@{<-}[ru]  &*+[F]\txt<1.5cm>{\em{}Message Bus} \ar[ru] \ar@/^8ex/@{<.>}["*"]&
%%  }
%% \POS*\frm{.}
%% \endxy
%% \]



\section{The Meadow API}
The Meadow system provides C++ library which can access devices. 
This allows external programs to access device services and to get information
about the devices and the Meadow
system.\footnote{\textcolor{green2}{\bf{}TODO: Currently, no
    security issues are taken into account. For example, each function should
    succeed or fail based on some security model. Add it later.}}

\subsection{Event service API}
\paragraph{\small\textcolor{red2}{\tt{}int event\_add(std::string device\_name, std::string event\_name)}}
\paragraph{\small\textcolor{red2}{\tt{}int event\_delete(std::string device\_name, std::string event\_name)}}
\paragraph{\small\textcolor{red2}{\tt{}int event\_exists(std::string device\_name, std::string event\_name)}}
\paragraph{\small\textcolor{red2}{\tt{}int event\_subscribe(std::string device\_name, std::string event\_name)}}
\paragraph{\small\textcolor{red2}{\tt{}int event\_unsubscribe(std::string device\_name, std::string event\_name)}}
\paragraph{\small\textcolor{red2}{\tt{}int event\_generate(std::string device\_name, std::string event\_name, VALUE\_t event)}}


\subsection{Property service API}
\paragraph{\small\textcolor{red2}{\tt{}int prop\_add(std::string device\_name, std::string prop\_name)}}
\paragraph{\small\textcolor{red2}{\tt{}int prop\_delete(std::string device\_name, std::string prop\_name)}}
\paragraph{\small\textcolor{red2}{\tt{}int prop\_exists(std::string device\_name, std::string prop\_name)}}
\paragraph{\small\textcolor{red2}{\tt{}int prop\_get(std::string device\_name, std::string prop\_name, VALUE\_t value)}}
\paragraph{\small\textcolor{red2}{\tt{}int prop\_set(std::string device\_name, std::string prop\_name, VALUE\_t value)}}


\subsection{Function service API}
\paragraph{\small\textcolor{red2}{\tt{}int function\_add(std::string device\_name, std::string func\_def)}}
\paragraph{\small\textcolor{red2}{\tt{}int function\_delete(std::string device\_name)}}
\paragraph{\small\textcolor{red2}{\tt{}int func\_call(std::string device\_name, std::string func\_name, VALUE\_t **args, VALUE\_t *rets, HANDLER\_t fail)}}

\subsection{Reactor API}
\paragraph{\small\textcolor{red2}{\tt{}int reactor\_add(std::string device\_name, std::string reactor\_def)}}
\paragraph{\small\textcolor{red2}{\tt{}int reactor\_delete(std::string device\_name, std::string reactor\_name)}}

\subsection{C value binding API}
\paragraph{\small\textcolor{red2}{\tt{}TYPE\_t value\_typeof(VALUE\_t value)}}
\paragraph{\small\textcolor{red2}{\tt{}VALUE\_t value\_encode\_int(int value)}}
\paragraph{\small\textcolor{red2}{\tt{}VALUE\_t value\_encode\_real(float value)}}

\section{Meadow Tools}
\subsection{Meadow shell}
Make a shell which each API function as a ``shell command''.

\subsection{Meadow interface checker}
Meadow devices continue to change its interface through dynamic addition and
removal of services. We may want to ensure if the device interfaces conform to
code at the given time.

Also, function signature checking can be done for ``arities'' not
``datatypes''. 


   %the Meadow system itself is a set of passive capabilities; if something
   %needs to be done, some ``firing'' is needed such as
   % \bit
   % \w calling functions from a non-device context (from a meta-level)
   % \w firing events from a non-device context (from a meta-level)
   % \eit
  
  %  Just like ``entering command at a Unix shell'', we can  a program in
  %  Linux. 


%% \section{Example Setup: Devices}
%% The rest of the document will present a walk-through of main concepts of the
%% Meadow system through an example. 
%% For this, we make a few assumptions.
%% \bit
%% \w We assume a common middleware for device communication, such as
%%    AllJoyn, is installed in each device.
%%    Also, native functionality of the device (e.g. switch channel of a TV) is
%%    programmatically registered as AllJoyn methods, signals, and properties.
  
%%    In addition to this, the common service interfaces which is defined in
%%    AllJoyn service frameworks (e.g. \verb+org.alljoyn.About+,
%%    \verb+org.alljoyn.Onboarding+, etc.) are already implemented. 

%%    \textcolor{red2}{\em Note that even if a industry consensus on a single common framework is not
%%    possible, we can always create a common wrapper framework, say using
%%    AllJoy,  which is located above proprietary frameworks such as
%%    Apple Home 
%%    Kit or Google Nest framework.\/}
   
%% %\w Devices are connected to the network. 
%% \w \textcolor{red2}{\em Manufacturers do not know about the Meadow
%%   system yet.\/}  
%% \eit

%% \subsection{Device \#1: An iPhone app}
%% Assume that the 24 hour fitness club developed an iPhone app using the AllJoyn
%% framework. While this app is being executed, this iPhone (i.e. app) is
%% discoverable from within the AllJoyn network and can provide services to other
%% AllJoyn network. We assume that it supports a very simple service -- a
%% \verb+userid+ property.
%% By this, we mean that other AllJoyn application can query the value of
%% \verb+userid+ in a programmatic way when this app joins an AllJoyn bus. 

%% \paragraph{Making the app known to the Meadow system}
%% Now, to make this app programmable in the Meadow system, we need to do
%% a few more things
%%   \bit
%%   \w register the app as the Meadow device
%%   \w register each device instance name
%%   \w install the Meadow VM (runtime) on top of the AllJoyn framework
%%   \eit
%% First, to register this app as a device, interface of the device must be
%% defined using the \textcolor{blue2}{\textbf{Meadowview Interface Definition
%%     Language (IDL)}}.\footnote{Typically, an IDL description is mandatory in
%%   binding statically compiled languages since it allows to generate stubs
%%   against which the callers can be compiled. However, in dynamically-typed
%%   languages, these are not mandatory. Meadowview is a dynamically-typed
%%   language but we still optionally allow MIDL descriptions to facilitate early
%%   error checking instead of runtime error.}
%% \begin{alltt}
%%   \textcolor{black}{device com.24.app.iphone \{
%%     // property name must match the ``AllJoyn'' property name
%%     string userid;
%%     string username;
%%   \};}
%% \end{alltt}
%% This textual definition needs to be compiled and be registered to the
%% \textcolor{blue2}{\bf Meadow name server} (for simplicity, we assume there is a central name
%% server. however, to avoid bottleneck, we may elect to implement a 
%% decentralized server.) by either the app developer or somebody who wants to
%% create a Meadow workflow. Let's assume that above definition is stored in
%% a file \verb+24app.dev+. 
%% Then, the following command will register a globally-unique name to the global
%% server and bind it to the device definition.
%% \begin{alltt}
%%   \textcolor{black}{# meadow-register 24app.dev}
%% \end{alltt}

%% Second, we need to assign a device instance name. Each installation of the
%% App is unique and we need to tell the Meadow system that
%% \bit
%% \w this app is an instance of \verb+com.24.app.iphone+, and
%% \w the name of this device instance is \verb+abc+, and
%% \w a password (or a public-key) is created for this device instance 
%%    and registered to the Meadow system. 
%% \eit
%% The fully-qualified name of the device instance will be
%% \verb+com.24.app.iphone[abc]+ in the Meadow system. 

%% Now, when this device signs in, the Meadow system will ask the device to
%% identify itself. The device will tell the system that ``I am
%% \verb+com.24.app.iphone[abc]+'', which is a globally-unique name, along with a
%% password\footnote{Or encryption of a system-provided text using the
%%   private-key so that the System can authentication of the device instance.}. 
%% When device instance name and password is correct, this device instance is
%% signed on and now this device instance is discoverable within the Meadow
%% network. 


%% \subsection{Device \#2: Treadmill}
%% The treadmills installed at the 24-hour fitness club in San Jose has the
%% following device definition.
%% \begin{alltt}
%%   \textcolor{black}{device com.24.sj.treadmill \{
%%     string instid;            // instance id
%%     int status;               // 0: stopped, 1: running

%%     void start();             // start the treadmill
%%     void stop();              // stop the treadmill
%%     void resetStepCount();    // resets the step count
%%     int getStepCount();       // number of steps since last reset

%%     event startButtonPressed;
%%     event stopButtonPressed;
%%   \};}
%% \end{alltt}
%% Also, we assume there are 50 treadmills in the club and each of them are given
%% unique names -- e.g. \verb+com.24.sj.treadmill[1F.01]+,\ \ 
%% \verb+com.24.sj.treadmill[1F.02]+, etc.

%% \subsection{Device \#3: TV monitor on Treadmill}
%% Also, for each treadmill, a third-party TV with larger screen was installed.
%% The device definition is given below.
%% \begin{alltt}
%%   \textcolor{black}{device com.24.sj.tv.samsung \{
%%     string id;                // instance id
%%     int status;               // 0: on, 1: off
%%     int channel;              // current channel

%%     void turnOn();
%%     void turnOff();
%%     void switchChannel(int ch);
%%     void printMessage(string msg);

%%     event turnedOn;
%%     event turnedOff;
%%     event channelSwitched \{
%%       int oldChannel;
%%       int newChannel;
%%     \};
%%   \};}
%% \end{alltt}
%% A total of 50 TVs were installed and their names were assigned to match those
%% of treadmills -- e.g. \verb+com.24.sj.tv.samsung[1F.01]+.

%% \subsection{Device \#4: Database server}
%% Also, a database server will be registered as a Meadow device. There will be 2
%% servers, each named as \verb+com.24.sj.dbserver[0]+ and
%% \verb+com.24.sj.dbserver[1]+, respectively. 

%% \begin{alltt}
%%   \textcolor{black}{device com.24.sj.dbserver \{
%%     void log(time t, string userid, string message); 
%%     void saveStepCount(time t, string userid, int count);
%%   \};}
%% \end{alltt}


%% %%\section{An Example}
%% %% The Meadow system allows to define workflows of devices. A workflow definition
%% %% can be instantiated and executed by the Meadow runtime.
%% %% Consider the following devices as an example (by a ``device'', we mean 
%% %% any computational entity which can execute the Meadow VM (or runtime)).
%% %% Each device will have a globally unique name, a device type, and has a
%% %% set of services provided to the outside world.
%% %% \bit
%% %% \w An \bb{ID device}, which can identify the owner of the device. Upon
%% %%    configuration, this device can authenticate the user id upon request for
%% %%    any context (in terms of 24 hour fitness membership, Starbucks membership,
%% %%    ...)\footnote{I think  this device is interesting by itself. \bb{Could this
%% %%        be a killer device by itself}? Basically, it's a customizable badge
%% %%      that can  be used in many different contexts. We might 
%% %%      be able to implement this with simple electronic components with 
%% %%      some ideas borrowed from public key cryptography. Might be related to
%% %%      Passbook app of iPhone but may have many usages -- both consumer and
%% %%      industrial.}. 
%% %%    %Has one small LED   screen and has two buttons: ``YES'' and ``NO''. 
%% %%     \bit
%% %%     \w globally registered device type: \verb+com.myid.model5+\footnote{DISCUSSION:
%% %%       what attributes should a device have? A ``device type'' can determine
%% %%       which type of exported methods are available, while ``device ID'' can
%% %%       uniquely identify a single device.}
%% %%     \w globally registered device name: \verb+com.myid.cj2005+
%% %%     %\w unique deivce ID:  \verb+1234+
%% %%     \w exported methods specific to \verb+com.myid.model5+: 
%% %%        \bit
%% %%        \w \verb+int getID(int context)+: returns a unique ID in the given
%% %%        context 
%% %%        \w \verb+string getOwnerName()+: returns the name of the owner
%% %%        \eit
%% %%     \eit
%% %% \w A \bb{treadmill} at a 24 hour fitness gym.
%% %%     \bit
%% %%     \w globally registered device type: \verb+com.precor.treadmill.model66.v3+
%% %%     \w globally registered device name: \verb+com.24fitness.sanjose.tm55+
%% %%     %\w unique device ID: \verb+9999+
%% %%     \w exported methods specific to \verb+com.precor.treadmill.model66.v3+: 
%% %%         \bit
%% %%         \w \verb+void printMsg(string msg)+: prints message to  screen 
%% %%         \w \verb+void resetStepCount()+: reset step count to 0
%% %%         \w \verb+int getStepCount()+: number of step count since last reset
%% %%         \w \verb+pairDevice(Device dev)+: pair device to dev (TODO: need some
%% %%         more details) 
%% %%         \w \verb+int isPaired()+: is paired with another device or not
%% %%         \eit
%% %%     \w exported signals: 
%% %%         \bit
%% %%         \w \verb+startButtonPressed()+: generated when start button pressed
%% %%         \w \verb+stopButtonPressed()+: generated when stop button pressed  
%% %%         \w \verb+deviceFound(Device dev)+: generated when the customer,
%% %%         who has Meadow-compatible device, stepped onto the treadmill; also as
%% %%         a payload, the found device handle is attached to the signal
%% %%               (please ignore how this can be done for now; actually, it's a
%% %%         topic for discussion since there could be multiple compatible devices
%% %%         around the treadmill.)
%% %%         \eit
%% %%     \eit
%% %% \w A \bb{database server} which stored any information about a person
%% %% (e.g. through Amazon service), owned by a healthcare company. 
%% %%     \bit 
%% %%     \w globally registered device type: \verb+com.unitedhealth.dbserver+
%% %%     \w globally registered device name: \verb+com.unitedhealth.dbserver51+
%% %%     %\w unique device ID: \verb+5555+
%% %%     \w device name: \verb+com.unitedhealth.dbserver+
%% %%     \w exported method: \verb+void addStepCount(int userid, int stepcount)+
%% %%     \eit
%% %% \eit

%% \section{Default interfaces}
%% There are some built-in functions, properties, and signals that are
%% supported by Meadow devices. 
%% \begin{alltt}
%%   \textcolor{black}{device Device \{
%%     list<Device> getPairedDevices();
%%     list<Device> getPairedDevicesByType(string type);
    
%%     event devicePairedEvent \{
%%       string devid;
%%     \}
%%   \}}
%% \end{alltt}

%% \section{Example \#1: Recording Step Counts}
%% Consider the following scenario.
%% \begin{quote}
%%   \textcolor{red2}{\em Whenever a person, who installed 24-hour fitness app on iPhone, steps
%%     onto a treadmill located in 24hour fitness San Jose site, store the step
%%     count in the DB. Also, print a welcome message on the treadmill screen.\/} 
%% \end{quote}
%% Maybe 24hour fitness wants to sell a new service, which will record all
%% customer exercise history, which can be browsed through Web. For this, they
%% can sell a small 10-dollar ``SecureID'' device, which will contain the ``ID''
%% of the owner. They want to program their treadmills that when a user with this
%% ID steps on, it will pair with the device and send the treadmill usage to the
%% DB server. 

%% \subsection{Static workflow}
%% One way to program the flow in the Meadowview language is given below.
%% \begin{alltt}
%%   \textcolor{black}{workflow StepCountFlow(com.24.app.iphone app,
%%                          com.24.sj.treadmill tms[],
%%                          com.24.sj.tv.samsung tvs[]);
%%   \{
%%     // get handle of any DB server instance
%%     Device dbserver = getDeviceType("com.24.sj.dbserver");

%%     // "foreach" may be similar to Verilog generate statement;
%%     // not sure if foreach construct is essential but could be
%%     // useful for configuring multiple devices 
%%     foreach tm in tms \{
   
%%       // whenever the treadmill is paired with a device,
%%       // the following process will wake up; "dev" will be 
%%       // bound to a Device object which was paired with the
%%       // the treadmill
%%       process @(tm.devicePaired(dev)) \{  
%%         if (dev.getDeviceType() == "com.24.app.iphone") \{
%%           string appowner = dev.username;
%%           tvdev = tm.getPairedDeviceByType("com.24.tv.samsung");
%%           if (tvdev)
%%             tvdev.printMessage("Welcome: " + appowner);
%%         \}
%%       \}

%%       // the following process wakes up when the start button 
%%       // of the treadmill is pressed; 
%%       process @(tm.startButtonPressed()) \{
%%         if (tm.isPaired()) \{
%%           list<Device> pdevs = tm.getPairedDevices();
%%           foreach pdev in pdevs \{
%%             int pdevid = pdev.getID();
%%             tm.resetStepCount();
%%             @(tm.stopButtonPressed());
%%             int steps = tm.stopStepCount();
%%             dbserver.addStepCount(dev.getID(24HOUR), steps);
%%           \}
%%         \}
%%       \}
%%     \}
%%   \}}
%%  \end{alltt}

%% \subsection{Dynamic discovery}
%% \begin{verbatim}
%%   workflow StepCountFlow(com.24.app.iphone app,
%%                          com.24.sj.treadmill tms[],
%%                          com.24.sj.tv.samsung tvs[]);

%%     list<Device> tms = getDevicesByName("24fitness.sanjose.treadmill*");
%%     Device dbserver = getDeviceByType("com.unitedhealth.dbserver");

%%     // actually "foreach" may be similar to Verilog generate statement;
%%     // I'm not sure if foreach construct is essential but could be useful for 
%%     // configuring multiple devices 
%%     foreach tm in tms {
   
%%       // whenever someone with a device steps on a treadmill, the following
%%       // process will wake up; "dev" will be bound to a Device object which 
%%       // represents a device that the user owns
%%       process @(tm.deviceFound(dev)) {
%%         if (dev.getDeviceType() == "com.myid") {
%%           tm.pairDevice(dev);
%%           String devowner = dev.getOwnerName();
%%           tm.printMsg("Welcome: " + devowner);
%%         }
%%       }

%%       process @(tm.startButtonPressed()) {
%%         if (tm.isPaired()) {
%%           Device dev = tm.getPairedDevice();
%%           int devid = dev.getID();
%%           tm.resetStepCount();
%%           @(tm.stopButtonPressed());
%%           int steps = tm.stopStepCount();
%%           dbserver.addStepCount(dev.getID(24HOUR), steps);
%%         }
%%       }
%%     }
%%   endflow
%% \end{verbatim}
%% The Meadow system compiles the above description into ``native'' code which
%% can be executed by Meadow runtime (VM). And then, the code is deployed to
%% relevant devices. (\bb{NOTE}: This somewhat reminds of the use model of Java
%% VM, though Java does not have the notion of automatic synthesis of
%% choreography of components. Might need to prepare what differentiates Meadow
%% VM from Java VM.) 

%% %In current technology, to support this functionality, all related 24 hour
%% %fitness treadmills should be programmed with the intended behavior.

%% \section{Example \#2: Streaming Producer and Consumer}
%% Next example and an interaction between the iPhone and the TV attached to each
%% treadmill. 
%% \begin{quote}
%%   \textcolor{red2}{\em Whenever a person, who installed 24-hour fitness app on iPhone, steps
%%     onto a treadmill located in 24hour fitness San Jose site, store the step
%%     count in the DB. Also, print a welcome message on the treadmill screen.\/} 
%% \end{quote}




%% \section{Components of the System}

%% \subsection{Infrastructure}
%% The Meadow system takes advantages of available infrastructures for device
%% communication. One such infrastructure is AllJoyn framework developed by
%% Qualcomm.  This framework allows devices to talk to each other through ``method
%% invocation''. In particular, under this framework, each device exports its
%% services which can be used in the form of a \bb{method call} (pull service) or
%%   \bb{signal notification} (push service). 

%%   Another alternative
%% is ZeroMQ message system. In ZeroMQ, it does not have any notion of function
%% calls but allows to deliver a free-form messages to be sent from one device to
%% another. Application can impose meaning on these messages by defining its own
%% format for service request/reply (i.e. ZeroMQ allows to build customized
%% application-level communication protocol easily). 

%% Basically, infrastructural software allows:
%% \bit
%% \w to name/discover devices
%% \w to export service of a device to outside world
%% \w to get service from other device
%% \w to define events
%% \w to publish events
%% \w to subscribe events
%% \w to add callbacks to the subscribed events
%% \eit

%% These frameworks typically support multiple protocols: low-level physical
%% protocols such as Bluetooth, ZigBee, WiFi, ethernet, etc. and high-level
%% application protocol such as HTTP, CoAP, etc.
%% Also, they support multiple platforms: Linux, Windows, iOS, MacOS, Andriod,
%% small embedded operating systems, etc. 

%% These infrastructures typically exist in two forms: software library (.so or
%% .a) and daemon. To use these infrastructure, developers could either link with
%% the library or use IPC/socket to communicate with the daemon.
%% Both AllJoyn and ZeroMQ support multiple languages through language
%% bindings. For example, they provide Java class library allows Java programs to
%% the call the software library. These Java class library could use JNI (Java
%% Native Interface) to call the native library in .a or .so. 

%% \paragraph{ASSUMPTION} For discussions below, we will assume that AllJoyn is
%% used as our infrastructure. ({\em However, we can relatively easily change our
%%   tool to target different infrastructures -- i.e. runtime can be configured
%%   to target multiple infrastructures.})

%% One major assumption is that all services (e.g. turn on LED, print
%%    message on a OLED screen, etc) provided by a device  are already
%%    developed and registered as an AllJoyn method or AllJoyn signal.
%%    Device manufacturer sells a device which has a set of exported services.
%%      That is, if we want to define workflow over some devices using Meadow, we
%%      expect that all services provided by these devices are exported as
%%      AllJoyn methods (also, these devices must internally have AllJoyn daemon
%%      inside). In addition to this AllJoyn, they also run Meadow runtime on top
%%      of AllJoyn. \bb{THIS COULD BE THE MOST DIFFICULT ISSUE TO OVERCOME. How
%%        can we bell the cat -- i.e. have manufacturers install Meadow
%%        runtime?}  

%% %These services must be provided as, say, AllJoyn method calls (which
%% %     means that the provider must have AllJoyn daemon/library pre-installed). 

%% When two AllJoyn-compatible devices, say A and B, are given, we can use AllJoyn API to
%% develop a code which make A and B interact with each other by manually
%% deploying the code. The Meadowview system tries to avoid this hard-coding. We
%% want to control the interaction of devices from a higher-level than than going
%% down to AllJoyn-level. 




%% \subsection{The Meadow VM (or Runtime)}
%% Each device runs a Meadow runtime (e.g. in the form 
%%     of daemon).  The runtime runs on top of the infrastructure and 
%%     is responsible for the following:
%%    \bit
%%    \w Maintains the set of processes (Verilog always-process like):
%%       some are \bb{built-in processes}, some are 
%%       \bb{user-defined processes}. 
%%        \bit
%%        \w A process is basically ``if event X happens, do Y.'' 
%%         The event X is either a ``method call request either internal or
%%         coming from other device'' or ``signal which is generated internally
%%         or by other device''.
%%        \w The ``action Y''  might need to be limited to very simple
%%        actions. For example,  
%%           \bit
%%           \w \bb{method call}, which is available locally or remotely 
%%           \w \bb{signal generation}: generate signal with some payload
%%           \w \bb{simple branching}: IF (simple-cond) THEN Y' ELSE Y''
%%           \eit
%%         \eit
%%    \w Maintains the set of events (i.e. method call request, 
%%       property get/set request, signals in AllJoyn 
%%       terminology\footnote{Essentially, method call or property 
%%       get/set calls is essentially an "event with a payload" that is sent from
%% other device to this device. A signal an "event (with our without a payload)"
%% that is sent from this device to other device.  So, all three different
%% services can be represented using a single notion of "events"}) that 
%% this runtime subscribes or publishes: an event is either a \bb{built-in
%%   event}, a \bb{manufacturer's event}, or \bb{Meadow's event}.
%%        \bit
%%        \w \bb{built-in event}: some events are very basic events provided by
%%        AllJoyn 
%%        \w \bb{manufacturer's event}: The method call request, signals defined
%%        and registered by the manufacturer
%%        \w \bb{Meadow's event}: Meadow will create internal events for
%%        instrumentation purpose.
%%        \eit
%%    \w Waits for the lower-level framework (e.g. AllJoyn Bus or 
%%       ZeroMQ message queue) on Meadowview events.
%%       \bit
%%       \w Though the device reacts to all three different types of events ,
%%        at Meadow runtime level, only Meadowview events are relevant.
%%        Built-in events and manufacturer's events should have been already
%%        registered to infrastructure (infrastructure does not know anything
%%        about Meadow 
%%        runtime) and should be handled directly by the infrastructure.
%%    \w The behavior of the runtime is quite similar to the event loop 
%%        executed by Verilog simulator.
%%    \eit
%%    \eit

%% \subsection{Two languages}
%% \bit
%% \w \bb{Meadowview: A workflow definition language}: 
%%    Users define workflows over some devices using a language, to
%%    \bb{orchestrate} or \bb{compose device services}.
%%    \bit
%%    \w This is like seeing the problem from a conductor viewpoint. 
%%    \eit
%% \w \bb{Native language for VM}
%%    Given that workflow is  defined as an orchestration of device services, now
%%    we have to derive (or \bb{synthesize}) what each device needs to do to
%%    accomplish the job described by the workflow. 
%%    \bit
%%    \w This is like creating a choreography of devices -- define which needs to
%%    be done from a viewpoint of ``each'' device. 
%%    A device involved in the workflow doesn't need to know the entire picture
%%    of the workflow. All it needs to know is ``under this circumstance, do this
%%    or do that, etc.'' 
%%    \w So, from a Meadowview workflow definition, we need to derive VM code for
%%    each devices involved. This code will be have the form of ``a set of
%%    Verilog process-like code fragments''. However, the syntax of such a
%%    process must be limited to very simple constructs for efficient VM
%%    implementation.

%%    For example, We may not even need loops since we could restrict any real computation to
%%    be provided by the manufacturer-provided method implementation.
   
%%    This VM language is a native language for the Meadow system, which
%%    is directly executable by the Meadow runtime (VM). See the next section for
%%    generated VM process examples.
%%    \eit
%% \eit

%% \subsection{Synthesis Tool}
%% Given a workflow definition written in Meadowview, this needs to be
%% synthesized to VM processes for execution. There may be multiple ways to
%% synthesize a single workflow definition (just like there are multiple ways to
%% synthesize an Verilog design using AND/OR gates) and synthesizing efficient
%% code will be a big challenge.  

%% The following is one possible example of how the workflow definition given in
%% Section~1 can be synthesized in to VM code. (\bb{NOTE}: The VM language is not
%% yet defined. So, there are plenty of holes, missing pieces, etc.)

%% \begin{verbatim}

%%   // signal definition; internal signal generated to represent
%%   // that function call reply has received from the function call
%%   // to getDeviceType of Device dev
%%   //
%%   // synthesis tool must define/register these internal signals
%%   // to infrastructure (e.g. AllJoyn)
%%   signal signal000 {
%%     string devtype;    // payload of the signal000 (return value of 
%%                        // getDeviceType)
%%   };
%%   process @(deviceFound(dev)) {
%%     // funccall is a primitive which calls a function and,
%%     // upon termination of function call, will generate a signal
%%     // putting the return value of the function call as a 
%%     // payload of the signal
%%     funccall(dev,            // target device
%%              getDeviceType,  // function name
%%              NULL,           // args
%%              signal000);     // signal that will be generated when
%%                              // function returns
%%   }

%%   process @(signal000) {
%%     // process body can have exactly one statement;
%%     // either a function call or a IF-statement or IF-ELSE
%%     // statement
%%     if (signal000.devtype == "com.myid") {
%%       funccall(self, pairDevice, dev, signal001);
%%     }
%%   }

%%   signal signal002 {
%%     string ownername;
%%   };
%%   process @(signal001) {
%%     funccall(dev, getOwnerName, NULL, signal002);
%%   }

%%   signal signal002 { bool boolvalue; }
%%   process @(signal002) {
%%     // how do we support string concatenation? should we support this?
%%     funccall(self, printMsg, "Welcome: " + signal002.ownername, signal003);
%%   }

%%   process @(startButtonPressed) {
%%     funccall(self, isPaired, NULL, signal003);
%%   }

%%   process @(signal003) {
%%     if (signal003.boolvalue) {
%%       funccall(self, getPairedDevice, NULL, signal004);
%%     }
%%   }

%%   ...
%% \end{verbatim}
%% As seen above, synthesis tool might need to break down the original workflow definition into tiny small processes.
%% Each process can have one "event" to which the process sensitized 
%% and up to one condition checking at the beginning and one action.
%% An action can be either a function call or a signal generation.
%% A called function can be either local or remote. 


%% \subsection{Deployment of VM processes}
%% Let's say that a developer employed by 24hour fitness has created a workflow
%% and used a Meadowview compiler to get the synthesized VM code. 
%% Then, the problem is how to deploy this code to the treadmills in the San Jose
%% site. 
%% For this, several things might be needed. First, two built-in VM processes
%% inside each Meadow VM runtime may be needed:
%% \bit
%% \w \bb{int loadProcess(Process proc)}: a built-in Meadow VM process, which loads
%% the given process into the runtime 
%% \w \bb{unloadProcess(int procid)}: a built-in Meadow VM process, which
%% unloads the process with the given id
%% \eit

%% Also, a loader utility, which loads a VM process(es) to a device may be
%% needed. This loader utility can be triggered either by the person or
%% dynamically by a tool. 

%% \subsection{Meadow VM runtime}
%% A VM runtime maintains a list of processes. Whenever some event happens, 
%% it picks a process in the list and executes it.
%% Essentially, runtime executes an infinite event loop like that of Verilog S/W
%% simulator. 



%% %% \section{Background}

%% %% \subsection{Devices}
%% %% Devices are getting more intelligent and more capable.
%% %% As chips (from full-scale processor to limited
%% %% microcontroller) become cheaper, more devices are produced, equipped with
%% %% chips. Manufacturers would want to add chips to their products
%% %% for more value (e.g. more electronic devices are added to cars).

%% %% %Car manufacturer are trying to add
%% %% %more value to the cars and sell at a higher prices by adding more
%% %% %chips to the car. 

%% %% \paragraph{Device Example: Fitbit}
%% %% A fitbit (http://www.fitbit.com) produces pedometers and weight scales, which
%% %% are equipped with microcontrollers and wireless chips so that measured data
%% %% can be automatically transmitted to a cloud. Users can view 
%% %% many different statistics on the data.

%% %% \subsection{Lowered barrier to manufacturing}
%% %% It's getting easier to manufacture devices without a plant. 
%% %% If you have a design on what to be built, you can send an order to a Chinese
%% %% factory to build  small number of devices (like 10Ks to 100Ks) to be
%% %% manufactured. Then, you can sell it on Amazon.
%% %% You can search for ``portable battery chargers'' in Amazon. There are
%% %% many different vendors and I guess many of them were produced that way. 

%% %% \subsection{Maker culture\/}
%% %% Recent {\em Maker culture\/} will encourage people to build ``anything cool.''
%% %% This culture is advocated by people with all kinds of
%% %% backgrounds, not just engineers. Technologically, it might be nothing novel
%% %% about this (there are rather low-tech; like resurgence of 80's hobby
%% %% electronics)  but all kinds of creative and innovative ideas might come out of
%% %% this culture -- law of numbers1.

%% %% \subsection{Wearable computers}

%% %% \subsection{Augmented reality}

%% %% \subsection{Expansion of middle class}
%% %% It is widely expected that people will become richer, with 
%% %% the increasing size of ``middle class'' population. More people will be
%% %% able to afford reasonably expensive devices -- like 
%% %% \$100 pedometers.  This might open up a new industry which builds smart,
%% %% value-added devices.

%% %% \subsection{Devices used by corporates}
%% %% Companies have long been used special devices for many different purposes.
%% %% Factories have many different types of sensors, acturators, devices installed
%% %% and they need to operate/maintain them.
%% %% Delivery companies uses some special devices which can track items.

%% %% \subsection{Rise of robots}
%% %% In labs,
%% %% even humanoid-type of robots are being built.
%% %% AI might still have long way to go but ``robot programming'' might 
%% %% become a mainstream soon. (of course, people should have been doing
%% %% robot programming for long in related industries but I think they 
%% %% should have been doing this in an ad-hoc hard-coded way.)

%% %% Consider a very complicated robot, consisting of many parts. One way to
%% %% program a robot is to have a single brain (CPU) which controlls every part
%% %% (just like single-threaded program is built in a traditional way).
%% %% The other approach is to have several autonomously-working parts interacting
%% %% with each other. Now each part becomes a ``reactive process'' (just like a
%% %% Verilog process which wakes up on events to which it is sensitized) and
%% %% programmingis to add events and processes which cooperate to do some work. 

%% %% \subsection{Internet of Things}
%% %% Most major large corporates (including non-IT companies) 
%% %% are predicting the next Big Thing is
%% %% {\em Internet of Things\/}. You can search for ``Intenet of Things'' in Google
%% %% or in Youtube.
%% %% If some huge marketing/R\&D effort at this scale
%% %% continues, then the shape of industry is destined to change in one way or
%% %% another. 

%% %% \subsection{Event-based programming model}
%% %% Event-based programming model has been used only in limited contexts:
%% %% e.g. GUI (such as X-windows) or Javascript.
%% %% As things become more distributed, reactive programs are
%% %% getting a lot of attention as an alternative programming model.
%% %% %Reactive model is essentially, 
%% %% %Things are become nonblocking/asynchronous. We cannot
%% %% %send method call to another device and wait until it returns (it may take too
%% %% %long). 
%% %% The notion of Actors, which was proposed in the 70s but dormant over 40 years,
%% %% are getting some attention. Also, the notion of reactive
%% %% processes are getting attention (e.g. Akka).
%% %% Also, see Twisted, a network programming tool for an example.

%% %% \subsection{Verilog}
%% %% Verilog is a very unique language at least in two points: 1) it has the notion
%% %% of space (real estate) and 2) it's based in reactive (event-based) processes.
%% %% Traditional programming languages (excluding any thread-related discussion)
%% %% only have the notion of time but no space. 

%% %% To write a program, in old traditional approach, you can follow the thread
%% %% from the first statement of main() and until the point of exit(). In Verilog
%% %% it's not like that. there are many seemingly-independent processes waiting
%% %% some events. As stimulus comes in these independent processes work together.

%% %% Better yet, this model of computation has been proven to cope with 
%% %% billion-gate chips, one of the most complicated human inventions.

%% %% \subsection{Workflow systems}
%% %% %Software workflow market is huge.
%% %% These systems are chracterized by Web-based view-layer, Java-based logic layer, 
%% %% heavy middlewares, backend DB. In particular, I believe
%% %% that they are mostly centralized server-based and does not easily scale.





%\bibliographystyle{plain}
%\bibliography{bib/mac,bib/softsys,bib/compsys,bib/digital,bib/pl,bib/theory,bib/formal}
%\nocite{EN11,Agha86,Hewitt76,Verilog2005,JK13,SK09}
%\nocite{LABHJJLTZ05,VD10}
\end{document}

%%  LocalWords:  MeadowView EPL Hadoop proc FSM API insertEvent TCP IP ev MV
%%  LocalWords:  Runtime workflow lookup Namespace stateful Meadow Jeong VM
%%  LocalWords:  Cheoljoo runtime HDLs Verilog VHDL HDL APIs softsys compsys
x
%%  LocalWords:  workflows iPhone getID getOwnerName printMsg msg pairDevice
%%  LocalWords:  resetStepCount getStepCount dev TODO isPaired deviceFound ld
%%  LocalWords:  startButtonPressed stopButtonPressed healthcare addStepCount
%%  LocalWords:  userid stepcount SecureID AllJoyn Qualcomm ZeroMQ Bluetooth
%%  LocalWords:  ZigBee WiFi ethernet CoAP iOS MacOS Andriod IPC JNI OLED pre
%%  LocalWords:  MessageQueue Meadowview loadProcess unloadProcess procid abc
%%  LocalWords:  programmatically instid turnOn turnOff switchChannel strign
%%  LocalWords:  turnedOn turnedOff channelSwitched oldChannel newChannel ruu
%%  LocalWords:  saveStepCount middleware username printMessage HomeKit Scala
%%  LocalWords:  Google Erlang dRuby uninstalling MqSQL IoT MapReduce GPS lcd
%%  LocalWords:  dataflow remainingBattery locationChanged nonblocking MySQL
%%  LocalWords:  addUser LedLight LedLightControl Swtich logEntry WelcomeProc
%%  LocalWords:  VMs WelcomProc doorOpened ctor lcdDisplayed ownername func
%%  LocalWords:  actuals getOwner fragements functionalities namespace args
%%  LocalWords:  unregister lifecycle uninstallation unsubscribe async typeof
% LocalWords:  CORBA Atmel AVR pragma lu ru rdd ul batteryLife evalExpr rets
% LocalWords:  propGet builtin propGetRequest propGetResponse propGetReponse
