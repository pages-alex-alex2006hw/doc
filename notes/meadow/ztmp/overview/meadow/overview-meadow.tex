\documentclass{beamer}
\useinnertheme{rectangles}
\usepackage{graphicx}
\usepackage{mydef}
\usepackage{MinionPro}
\usefonttheme{serif}
\usepackage[all]{xy}

\title{The Meadow System: Overview}
\author{Cheoljoo Jeong}
\date{}
\begin{document}

\begin{frame}
\titlepage
\end{frame}


\begin{frame}
\frametitle{The Meadow System}

The Meadow system is an infrastructure for defining and executing
{\em workflows over devices and events\/}. 

\vspace*{0.4cm}

In particular,

\vspace*{0.2cm}

\bit
\w Provides a \bb{namespace for devices}.

\vspace*{0.2cm}

\w Provides a \bb{namespace for events and evet types}.

\vspace*{0.2cm}

\w Provides a \bb{language} for defining workflows.

\vspace*{0.2cm}

\w Provides a \bb{runtime} for execution of workflow definitions.
\eit

\end{frame}


\begin{frame}[fragile]
\frametitle{Device Namespace}

The device namespace of the Meadow system is an infinite space of
\bb{hierarchical names}: e.g.   \verb+person.cj.fitbit+.

\vspace*{0.5cm}

The system allows \bb{registration of devices}.
   \bit
   \w No authorative registry that manages names. 
   \w Only name conflict will be checked at the time of registration.
   \eit

\vspace*{0.4cm}

A name can denote either a device or a {\em group\/} of devices.

\end{frame}


\begin{frame}[fragile]
\frametitle{Event Namespace}

The event namespace is an infinite space of \bb{hierarchical names}: e.g.
\verb+person.cj.fitbit.RechargeNeeded+. 

\vspace*{0.5cm}

The system allows \bb{registration of events}.
   \bit
   \w No authorative registry that manages names. 
   \w Only name conflict will be checked at the time of registration.
   \w The namespaces of device and events must not overlap.
   \eit

\vspace*{0.4cm}

qA name can denote either a event or a {\em group\/} of events.

\end{frame}


\begin{frame}[fragile]
\frametitle{Workflow Definition}

The \bb{Meadowview} language allows to define {\em agents\/}, {\em events\/},
and {\em workflows}. 

\vspace*{0.4cm}

An \bb{agent} represents a ``device'' which can partitipate in a workflow.

\vspace*{0.4cm}

Through \bb{events}, agents can be correlated in a workflow.

\vspace*{0.4cm}

An {\bf workflow definition} is a program written in Meadowview, which defines
how agents work together to accomplish the given task.

\end{frame}


\begin{frame}[fragile]
\frametitle{Workflow Execution}

The \bb{workflow runtime} can {\em instantiate\/} a workflow definition into a 
\bb{workflow process} using actual devices.

\vspace*{0.5cm}

The runtime is responsible for two major activities:

\vspace*{0.3cm}

\bit

\w \bb{dynamically scheduling} of device processes.

\vspace*{0.3cm}

\w \bb{routing of data} between participating devices.

\eit

\end{frame}

\begin{frame}[fragile]
\frametitle{Example: Device Namespace}

An C-API for device/group registration is provided.

{\scriptsize
\begin{verbatim}
  int mv_add_device(char *device_name);
  int mv_remove_device(char *device_name);
  int mv_add_group(char *group_name);
  int mv_remove_group(char *group_name);
  int mv_group_add_device(char *group_name, char *device_name);
\end{verbatim}
}
\end{frame}

\begin{frame}[fragile]
\frametitle{Example: Workflow Definition}

An example code in Meadowview:

{\scriptsize
\begin{verbatim}
  namespace person.cj2005.fitbit {
    event DeviceIdRequest;
    event DeviceIdReply { 
      int device_id;
    };
    event DeviceCommand {
      int command, data;
    };

    process P(DeviceIdRequest e) {
      Device sender = e.sender();
      sender.send(sender, DeviceIdReply(self.getId()));
    }

    process Q(DeviceCommand e) {
      if (e.command == 1) 
        self.send(cmd1.run, e.data);
      ...
    };
  }
\end{verbatim}
}

\end{frame}

\begin{frame}[fragile]
\frametitle{Example: Workflow Execution}

{\scriptsize
\[\xymatrix@+0.5pc{
  & \txt{\bf\em Distributed tables} \ar@{.}[rrr]\ar@{.}[d]\ar@{.}[rrrd]
  &&&*+[F]\txt{Device\ \#1\\ runtime} \ar[d]^{\txt{\tiny{}\tt{}DeviceCommand}}&\\ 
 & *+[F]\txt{Device\ \#0\\ runtime}
  \ar@/^3ex/[rrr]^{\txt{\tiny{}\tt{}DeviceIDRequest}}
  \ar@/_3ex/@{<-}[rrr]_{\txt{\tiny{}\tt{}DeviceIDReply}}
   && &
  *+[F]\txt{{\scriptsize\tt{}person.cj2005.fitbit}\\ runtime} \ar@{<->}[d]
  \ar@{<->}[dl]&\\ 
&&  &P &Q \ar[dl]_{\txt{\tiny{}\tt{}cmd1.run}}\ar[d]^{\txt{\tiny{}\tt{}cmd2.run}}\\
&&  &\txt{cmd1 \\ program} &\txt{cmd2\\ program} \\
}\]
}

\end{frame}


%
% Implementation
%
%% \begin{frame}[fragile]
%% \frametitle{Q\#1: How to Specify Process Behavior}

%% \bb{\large CASE \#1: Provide a language}
%% \bit
%% \w Will the language be ever adopted by users at all?
%% \eit

%% \vspace*{0.3cm}

%% \bb{\large CASE \#2: Provide a library}
%% \bit
%% \w Library which contains functions which interact with
%%   the daemon.
%% \w Each ``\verb+@(Event e)+'' must be implemented as an event loop. So, the
%%    library may need to provide some thread management facility.
%% \eit

%% \vspace*{0.3cm}


%% \end{frame}

\end{document}

%%  LocalWords:  RFID DSMS SQL CEP IFP Dataflow cep compsys softsys
