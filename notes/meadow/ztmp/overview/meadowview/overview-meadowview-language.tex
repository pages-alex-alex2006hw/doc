\documentclass{beamer}
\useinnertheme{rectangles}
\usepackage{mydef}
\usepackage{MinionPro}
\usefonttheme{serif}
\usepackage[all]{xy}

\title{The Meadowview Language}
\author{Cheoljoo Jeong}
\date{}
\begin{document}

\begin{frame}
\titlepage
\end{frame}


\begin{frame}
\frametitle{Overview}

The {\bf Meadowview} is a programming language which can be used to 
define processes and interactions between processes.

\vspace*{0.3cm}

In particular, the Meadowview language
\bit
\w supports the notion of \bb{agents} as containers of processes,
\w supports the \bb{global namespace} of {\em events\/} and {\em agents\/}
\w supports the notion of \bb{(local) real time} {\em within\/} agents,
\eit

\end{frame}


\begin{frame}
\frametitle{Elements of Meadowview}

An \bb{agent}, which is unqiue in the entire runtime,  is a container of
processes. 

\vspace*{0.3cm}

An \bb{event}, which is unique in the entire runtime, represents any
``change of state'' in which some processes may be interested in.

\vspace*{0.3cm}

A \bb{channel}, which is unique in the entire runtime, is a special type of
agent which acts as a ``bulletin board'' for an event.

\vspace*{0.3cm}

A \bb{process}, which is unique within an agent, is a reactive code which can
perform simple testing over event instance, generation of new event instance, 
execution of foreign program (e.g. Python script, Java program). 

\end{frame}

\begin{frame}[fragile]
\frametitle{Agents}

An \bb{agent} has a name which is unique in the entire Meadowview
system. 

\vspace*{0.3cm}

An agent, which is a {\em container of processes\/}, can send and/or 
receive events.

\vspace*{0.3cm}

Agents are the only programming element which serve as the {\em source\/} or
the {\em destination of event delivery\/}.

\vspace*{0.3cm}
\end{frame}


\begin{frame}[fragile]
\frametitle{Events}

An \bb{event} is a ``change of state'' which might be interesting to some
agents (or processes). 

\vspace*{0.3cm}

An event is unique across the entire Meadow system and events cannot be
generated or removed from inside a Meadowview program. They are {\em given\/}
just like IP addresses in the Internet.

\vspace*{0.2cm}

Though one cannot define events using the Meadowview language, the Meadow
system provides API through which one can add/remove/update events. An event
definition consists of a \bb{name} and \bb{datatype} definition.
For example,

{\small
\begin{verbatim}
  Event com.meadow.VacationApprovalRequest {
    com.meadow.EmployeeID id;
    Date date;
  };
\end{verbatim}}

\end{frame}

\begin{frame}
\frametitle{Event Instances}

While an event represents a {\em class\/} of related state-changes, an 
an \bb{event instance} represents an actual instantiation of an event.
In the runtime (which is inevitably dynamic), event instances are
transferred between agents.


\vspace*{0.3cm}

Transfer of event instance from one device to another carries following
information:

\bit
\w notification that the given event has occurred

\vspace*{0.3cm}

\w payload associated with the event instance

\vspace*{0.3cm}

\w additional control information such as timestamp, sender, receiver

\eit

\end{frame}

\begin{frame}
\frametitle{Processes}

A \bb{process} is a specification of behavior which will be \bb{deployed} in
an agent. 
A process consists of two parts: \ee{event list} and \ee{process body}. 

\vspace*{0.4cm}

The \bb{event list} is an unordered list of events to which the process is
sensitive to.

\vspace*{0.4cm}

The \bb{process body} is the code which is executed when any event in the
event list occurs. Roughly, the code can perform following activities:
\bit
\w \ee{Testing over event instance}
\w \ee{Branching/looping based on the result of testing}
\w \ee{Fork/join constructs}
\w \ee{Terminal actions}: 
   \bit
   \w send a new event instance
   \w execute foreign code
   \w delay for real-time (e.g. sleep(2 msec))
   \eit
\eit

\end{frame}

\begin{frame}
\frametitle{Processes: Control Constructs}

Meadowview supports the following control constructs:

\bit
\w \bb{Sequencing}: ``;'' is a sequencing operator
\w \bb{Branching}: if-then-else
\w \bb{Fork/join}: fork-join
\w \bb{Loops}: for-loop, while-loop
\eit

\end{frame}

\begin{frame}
\frametitle{Processes: Terminal Actions}

There three types of \bb{terminal actions} that a process can perform in
reaction to events.  
\bit
\w Generate/send a new event instance to an agent.
\w Execute a foreign program (e.g. Python script or Java program). 
\w Delay consumption.
\eit

\vspace*{0.4cm}

Restriction on the type of terminal actions can result in a compact and
efficient runtime.
\end{frame}


\begin{frame}[fragile]
\frametitle{Processes: Example}

The following shows a simple process which models an approval of a vacation
request. 

{\scriptsize
\begin{verbatim}
process ManagerApproval @(com.meadow.ApprovalRequestFirst R) {
  Agent DB = Agent("com.meadow.DBEngine");
  reqhandle = SEND(DB, new com.meadow.DBQuery(R.employee, "manager");
  RECV(reqhandle, com.meadow.DBQueryResult mgr);
  reqhandle = SEND(DB, new com.meadow.Query(R.employee, "vac_used");
  RECV(reqhandle = com.meadow.DBQueryResult vacused);
  if (vacused.value() - R.days() < 100 /* MAX_VAC */) {
    SEND(R.request(), new ApprovalRequestGranted());
    SEND(DB, new DBUpdate(R.employee, "vac_used", vacused.value()-R.days()));
  }
  else
    SEND(R.requester(), new ApprovalRequestDenied(NO_BALANCE));
}
\end{verbatim}
}

\end{frame}


\begin{frame}[fragile]
\frametitle{Further improvements}

\bit
\w Must be able to designate an abstract event source: e.g. any
{\tt ApprovalRequestFirst} originated from an agent who is in ``team member''
relation with the agent where the {\tt ManagerApproval} process is deployed.
\eit

\end{frame}

\begin{frame}[fragile]
\frametitle{Semantics of Meadowview Elements}

Each (static) Meadowview element has its dynamic counterpart.

\vspace*{0.3cm}

\centerline{\begin{tabular}{|l|l|} \hline
Syntax & Semantics \\ \hline
Agent & Device handle \\
Event & Event instance \\
Channel & Event queue \\ 
Process & Labeled transition system \\ \hline
\end{tabular}}

\end{frame}



\begin{frame}
\frametitle{Semantics: Events}

Events exists to convey two pieces of information:
  \bit
  \w notification that ``something has happened''
  \w any associated data
  \eit

An \bb{event flow} is transfer of an event instance between two devices.

\vspace*{0.3cm}

Event flows occur between devices not between processes. That is, the 
runtime running on a device is responsible for routing of event instances. 
The device runtime manages multiple processes in the given device.
\end{frame}


\begin{frame}
\frametitle{Semantics: Processes}

A process is compiled into a \bb{labeled transition system} (or state machine
with propositional formulas associated with edges). 

\vspace*{0.4cm}

\[\xymatrix@+0.1cm{
  & *+[F]\txt{$S_0$} \ar[d]^{\txt{\tiny{?ApprovalRequestFirst}}} &\\ 
 & *+[F]\txt{$S_1$} \ar[dl]_{\txt{\tiny vac\_used $<$ max\_vac}}
  \ar[dr]^{\txt{\tiny vac\_used $>=$ max\_vac}} & \\
*+[F]\txt{$S_3$} \ar@/^12ex/[uur] & & *+[F]\txt{$S_4$}  \ar@/_12ex/[uul]
}\]


\vspace*{0.4cm}

An engine in an agent (device) is responsible for the execution of labeled
transition systems. 


\end{frame}



\end{document}

%%  LocalWords:  RFID DSMS SQL CEP IFP Dataflow cep compsys softsys
