\documentclass[twocolumn]{report}
\usepackage{mathptm,mydef,myenv}
\usepackage{MinionPro}
\usepackage[Lenny]{fncychap}
\begin{document}

\small

\title{\huge\bf Meadow EPL: Language Reference}
\author{}
\date{\Large\today}
\maketitle

%%
\tableofcontents

%%
\chapter{Introduction}

\section{Events}
By ``events'', we need to distinguish between two possible interpretations.
One refers to something that has happened (the {\em event occurrence}) and
the other refers to an actual and {\em event object\/}. 

\section{Semantics}
\section{Syntax}
\section{Notation and terminology}

%%
\chapter{Lexical Conventions}
  
%%
\chapter{Basic Concepts}
\section{Event Types}
An event type defines a set of values that an event can assume. An event type
is either a primitive type or a complex type.

\subsection{Primitive types}
A primitive event type is one of bool, char, short, integer, long, float,
double, and string. 

\subsection{Complex types}
A complex type is either an array or a structure of event types.

\subsection{Array type}

\begin{verbatim}
  type int_array = int[10];
\end{verbatim}

\subsection{Structure type}

\begin{verbatim}
  type bid_type = struct {
    string symbol;
    float price;
    int quantity;
  };
\end{verbatim}

\section{Events}
An event 

\section{Event Contexts}

\section{Event Streams}
A \bb{stream} is an open-ended set of event objects, ordered by time.

\section{Aggregates}

\section{Event Patterns}
An \bb{event patterns} is a predicate over events which evaluates to a Boolean
value. 

\section{Event Windows}
An \bb{event window} over sequences of events is used to create a finite
ordered sewquence of event instances to be consumed by processes.

\section{Processes}
A process is a piece of code which is invoked when some condition is
satisifed. 

\subsection{conditions}


\chapter{Aggregates}

\chapter{Event Patterns}
\chapter{Event Windows}

\chapter{Processes}


\bibliographystyle{plain}
\bibliography{bib/mac,bib/softsys}
\nocite{EN11}
\end{document}
