\chapter{Introduction}


\section{Why Accelerate Java?}
Java is one of the most popular programming languages along with C and
C++. Considering the fact that now the object-oriented language is becoming
the norm of programming languages thanks to Computer Science education, it is
likely that Java and C++ will be the two most popular programming languages
in the future. 

Java has many benefits compared to C++. In particular, garbage-collected heap
memory resources allow well-known problem of memory leak or corruption. Also,
its extensive standard library allows easier software development. 

However, virtual machine based software execution model makes Java an
inherently inefficient language. Though techniques such as JIT (Just In Time)
...

\section{How To Accelerate Java Programs}
There are multiple ways to accelerate execution of Java programs. The
following are some technologies which have been used or have potential to be
used for Java acceleration.

\subsection{Java processors}
Building a processor which implements the Java virtual machine in a real
silicon has been conceived of since the inception of the Java programming
language. For example, the PicoJava microprocessor specification~\cite{MO98}
was proposed for the native execution of Java bytecode without the
interpretation through virtual machines. This was originally intended to be
used in consumer electronic products that run Java application. 

Recently, there is a resurgent interest in hardware Java
machines~\cite{Schoeberl09}. 

\subsection{Reconfigurable computing}

\section{Challenges in Java Accleration}

