\chapter{Formalisms for Concurrency}

\section{Background}



\subsection{Synchronous vs. asynchronous communication}

A large number of calculi and languages are based on asynchronous
communication primitives based on {\em message passing\/}.
For example, CSP, Occam, CCS, Squeak, $pi$ calculus, Concurrent ML,
Pict, and JoCAML are based on messaging primitves.
In mainstream langauges, this messaging primitives are built into
higer-level primitives such as RPC and RMI.


\subsection{Equivalence of processes}
One of the most important differences between formalisms for describing
concurrent processes is on the way they identify equivalent processes.
van Glabbeek~\cite{Glabbeek90} gave a spectrum of equivalence relations
adopted by different formalisms.


\section{CSP}
Communicating Sequential Processes (CSP), developed by Hoare~\cite{Hoare85} is a widely-used formalism for both
hardware and software processes.

\section{CCS}

\section{$\pi$ Calculus}

\section{I/O Automata}

\section{Petri Net}
