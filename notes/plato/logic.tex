\chapter{Mathematical Logic}

\section{Background}

\subsection{Formal languages}
In order to describe a formal logic, we need to give a syntax of the formal
language that is used by the logic.
For a syntax of the formal langauge, three pieces of information
should be provided:
\ben
\w A set of {\bf symbols} should be specified,
  such as $(, ), \rightarrow, \neg, A_1, A_2, \cdots$.
\w We should specify the rules for forming the 
     {\em{}grammatically correct\/} finite sequences of symbols, 
     called {\bf{}well-formed formulas}
     or {\bf{}wffs}.
\w We may need to indicate the allowable translations between
   English and the formal language. {\em{}This information is
   dispensable for ``formal or symbolic logic''\/}\footnote{Translations 
   between the formal language and the
  mathematical structures are studied in {\em{}Model Theory\/}.}.
\een

\subsection{Logical vs. non-logical symbols}
Even though a formal logic itself can be the topic of a study, 
for a practical use of formal logic, some non-logical world (sometime, logic
itself in a metacircular way) is denoted using the formal logic.
For example, suppose we want to build a formal logic of natural numbers.
Then, our universe of discourse is the set of natural numbers.
In this case, we need two languages: one for the logic and the other
for describing the natural numbers business.

For a precise definition of the language, we need to classify symbols into
two different classes: {\em logical symbols\/} and {\em non-logical
  symbols\/}. Logical symbols are those used by the formal logic
while non-logical symbols are used by the 


\subsection{Comparison of formal logics}
In first-order logic, the quantifiers $\forall$ and $\exists$ always range
over elements of the domain $M$ of discourse.
By contrast, second-order logic allows one to quantify over subsets of $M$ and
functions $F$ mapping, say $M \times M$ into $M$.  Third-order logic goes on
to sets of functions, etc.

\subsection{Models of formal logic}
Let's consider a group $G = \arc{S, +, 0}$ where $S$ is a non-empty set, $0
\in S$, and $+$ is a function mapping $S \times S$ to $S$, which satisfies the
following first-order axioms, or sentences:
  \begin{eqnarray}
\label{eqn:g1}  &\forall{x}\forall{y}\forall{z}[x + (y + z) = (x + y) + z]&\\ 
\label{eqn:g2}    & \forall{x}[x + 0 = x]&\\ 
\label{eqn:g3}  & \forall{x}\exists{y}[x +y = 0]
  \end{eqnarray}
Then, in logical terms, we say that $G$ is a \bb{model} of above three
sentences, rather than saying that $G$ satisfies them.
We write that $G \models $ (\ref{eqn:g1}), (\ref{eqn:g2}), (\ref{eqn:g3}).



\subsection{Syntax of logical languages}
A logical language consists of a set of \bb{sorts} and a set of
\bb{operators}. 


\subsection{Decidability of theories}
A theory is \bb{decidable} if there exists an effective procedure that 
tells of an arbitrary formula whether or not it is a logical consequence of
the axioms. 


%%%%%%%%%%%%%%%%%%%%%%%%%%%%%%%%%%%%%%%%%%%%%%%%%%%%%%%%%%%%%%%%%%%%%


\subsection{Truth Assignments}
A {truth assignment} $\nu$ for a set $S$ of sentence
symbols is a function
  \[ \nu: {S} \rightarrow \{\mbox{T}, \mbox{F}\}. \]
Note that a truth assignment is an {\bf{}interpretation}.
Let $\overline{S}$ be the set of wffs generated from the five
formula building operations.
We want an extension $\bar{\nu}$ of $\nu$,
	\[ \bar{\nu}: \overline{S} \rightarrow \{\mbox{T, F}\},\]
which assigns the correct truth value to each wff in $\overline{S}$.
It should meet the following conditions:
\ben
\w For any $A \in {S}$, $\bar{\nu}(A) = \nu(A)$.
\w For any $\alpha \in \overline{S}$,
	\ben
	\w $\bar{\nu}(\neg\alpha) = \left\{\begin{array}{ll}
		\mbox{T} & \mbox{\ if $\bar{\nu}(\alpha)$ = F},\\
		\mbox{F} & \mbox{\ otherwise}.
		\end{array}\right.$
	\w $\bar{\nu}(\alpha\vee\beta) =
	    \left\{\begin{array}{ll}
		\mbox{T} & \mbox{\ if $\bar{\nu}(\alpha)$ = T or 
					$\bar{\nu}(\alpha)$ = T},\\
		\mbox{F} & \mbox{\ otherwise}.
		\end{array}\right.$
	\w $\bar{\nu}(\alpha\wedge\beta) =
	    \left\{\begin{array}{ll}
		\mbox{T} & \mbox{\ if $\bar{\nu}(\alpha)$ = T and
		  $\bar{\nu}(\alpha)$ = T},\\
		\mbox{F} & \mbox{\ otherwise}.
		\end{array}\right.$
	\w $\bar{\nu}(\alpha\rightarrow\beta) =
	 \left\{\begin{array}{ll}
	\mbox{F} & \mbox{\ if $\bar{\nu}(\alpha)$ = T and
		$\bar{\nu}(\alpha)$ = F},\\
		\mbox{T} & \mbox{\ otherwise}.
			\end{array}\right.$
	\w $\bar{\nu}(\alpha\leftrightarrow\beta) = 
               \left\{\begin{array}{ll}
               \mbox{T}&\mbox{\ if $\bar{\nu}(\alpha) = \bar{\nu}(\alpha)$},\\
		\mbox{F} & \mbox{\ otherwise}.
		\end{array}\right.$
	\een
\een

\subsection{Semantics of first-order logic}
The interpretation function $\overline\nu$ can be illustrated using
the following category theoretical diagram:
	\[ \xymatrix{
		{S} \ar[d]_{\{\EE_\cdot\}} 
		\ar[rr]^{\overline{\nu} \equiv \nu}  
			& &  \{T, F\}
			\ar[d]^{F,G} \\
		\overline{S} 
		\ar[rr]^{\overline{\nu}} & & \{T, F\}
	}\]
In the diagram, the meanings of  $F$ and $G$ are indicated in (b.1)--(b.5).

For any truth assignment $\nu$ for a set $S$ there is a unique
	function $\bar{\nu}: \overline{S} \rightarrow \{\mbox{T, F}\}$
	meeting the above conditions (a) and (b.1)--(b.5) by the
	{\em{}Recursion Theorem\/}.

\subsection{Models}
A truth assignment $\nu$ {\bf{}satisfies} a sentence 
$\varphi$ iff $\bar\nu(\varphi) = \mbox{T}$. 
In this case, we say that $\bar{\nu}$ is a {\bf{}model} of $\varphi$.
A set of wffs $\Sigma$ {\bf{}tautologically implies} $\tau$, written as
$\Sigma \models \tau$, iff every truth assignment
for the sentence symbols in $\Sigma$ and $\tau$ which
satisfies every member of $\Sigma$ also satisfies $\tau$.
$\Sigma$ can be thought of as hypotheses and $\tau$
can be thought of as a {\em{}possible\/} conclusion.
As a special case, $\emptyset \models \tau$ iff every 
truth assignment satisfies $\tau$; in this case we call $\tau$
a {\bf{}tautology} and write this as $\models \tau$.
\subsection{Compactness theorem}
Let $\Sigma$ be an infinite set of wffs such that for any 
finite subset $\Sigma_0$ of $\Sigma$, there is a truth
assignment which satisfies every member of $\Sigma_0$.
Then there is a truth assignment which satisfies every
member of $\Sigma$.
 $\Sigma; \alpha \models \beta$ iff $\Sigma \models (\alpha
	\rightarrow \beta)$.



\subsection{Theories}
A theory is a set of sentences {\em closed under logical consequence}.
For example, the theory of strict partial orders is the set of all
consequences of the axioms for strict partial orders;
in other words, it is the set of sentences true in all strict partial orders.

\subsection{Completeness of theories}
A theory is complete if, for every sentence $S$, either $S$ or $\neg S$ is in
the theory.

\subsection{Consistency of theories}
A set of sentences is consistent if it has a model.


\section{Propositional Logic}

\section{First-order Logic}


\section{Higher-Order Logic}

\section{Equational Logic}


\section{Categorical Logic}


% LocalWords:  wffs Equational
