\chapter{Automated Reasoning}

\section{Automated Deduction}
\subsection{Consequence relation $\vdash$} 
A \bb{consequence relation} $\vdash$ is a set-theoretic relation among the
syntactical well-formed formulae of the proposed logic which satisfies the
following conditions, with $P$, $Q$, and $R$ denoting sets of wffs, and
$A$ denoting a single formula:
  \ben
  \w [(a)] \bb{reflexivity}: $P \vdash Q$\ \ if\ \ $P \cap Q \ne \emptyset$
  \w [(b)] \bb{monotonicity}: $P \vdash Q$\ \ implies\ \ $P \cup Q \vdash Q$
  \w [(c)] \bb{transitivity (cut)}: $P \cup \{A\} \vdash Q$ and $P \vdash Q
  \cup \{A\}$\ \ imply $P \vdash Q$
  \een
In many cases, the consequence relation is given in the form $P \vdash A$.
In this case the above conditions reduce to:
  \ben
  \w [(a)] \bb{reflexivity}: $P \vdash A$\ \ if\ \ $A \in P$
  \w [(b)] \bb{monotonicity}: $P \vdash A$\ \ implies\ \ $P \cup Q \vdash A$
  \w [(c)] \bb{transitivity (cut)}: $P \vdash A$ and $P \vdash Q \cup
  \{A\}$\ \ imply $P \vdash Q$ 
  \een

\subsection{Example: Intuitionistic Implication}

\subsection{Formal deduction systems}
A formal deduction system consists of a set of sentences, called \bb{axioms},
together with a set of functions, called \bb{deduction rules}, that map finite
set of sentences, called \bb{premises}, to a single sentence, called its
\bb{conclusion}. 
For example, a deduction rule states that $Q$ can be deduced from the premises
$P$ and $P \Ra Q$. We represent this rule as $(P, P \Ra Q) \vdash Q$.


\subsection{Herbrand's theorem}
Herbrand's theorem allows a reduction of first-order logic to propositional
logic. Although Herbrand originally proved his theorem for arbitrary
formulas of first-order logic, the simpler version restricted to formulas in
prenex form containing only existential quantifiers became more popular.
\begin{theorem}
Let $(\exists x_1, \cdots, x_n) F(x_1, \cdots, x_n)$ be a formula of
first-order logic with $F(x_1, \cdots, x_n)$ quantifier-free.
Then, $(\exists x_1, \cdots, x_n) F(x_1, \cdots, x_n)$ is valid if and only if
there exists a finite sequence of terms $t_{ij}$ ($1 \le i \le k, 1 \le j \le
n$, such that $F(t_{11}, \cdots, t_{1n}) \vee \cdots \vee F(t_{k1}, \cdots,
t_{kn})$ is valid.
\end{theorem}
Informally, a formula $A$ in prenex form is provable (valid) in first-order
logic if and only if disjunction comprising of substitution instances of the
quantifier-free subformula of $A$ is a tautology (i.e. propositionally
derivable). If $F(t_{11}, \cdots, t_{1n}) \vee \cdots \vee F(t_{k1}, \cdots,
t_{kn})$ is valid, it is called a \bb{Herbrand disjunction} for 
$(\exists x_1, \cdots, x_n) F(x_1, \cdots, x_n)$.

