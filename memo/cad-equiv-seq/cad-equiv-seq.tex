\documentclass{memo}
\usepackage{mathptm,mydef,myenv}
\usepackage{MinionPro,MnSymbol}
\usepackage{newalg}
\begin{document}
\small
\noindent{\large\bf{}Sequential Equivalence}

\paragraph{General verfication problem} 
Given two sequential systems (finite state machines), $M_1$ and $M_2$,
determine if they have the same input/output behavior, i.e. $M_1$ and $M_2$
produce the same output sequence for the same input sequence.

\paragraph{Restricted verification problem}
Given a finite state machine, $M$, with a single output $\lambda(x, e)$ over
the output alphabet $\{0, 1\}$, determine if $M$ always produce the output
value 1 for each possible input sequence.

\paragraph{Product machine}
Let $M_1 = (Q_1, I, O, \delta_1, \lambda_1, q_1)$ and
$M_2 = (Q_2, I, O, \delta_2, \lambda_2, q_2)$ be two sequential systems. 
The \bb{product machine} $M = (Q, I, O, \delta, \lambda, q_0)$ is defined by
  \bit
  \w $Q = Q_1 \times Q_2$
  \w $\delta((s_1, s_2), e) = (\delta(s_1, e), \delta(s_2, e)$
  \w $\lambda((s_1, s_2), e) = (\lambda_1(s_1, e) \equiv \lambda_1(s_2, e), 
            \cdots \lambda_m(s_1, e) \equiv \lambda_m(s_2, e))$, where $m$ is
            the number of output bits (bitwise comparison)
  \w $q_0 = (q_1, q_2)$
  \eit


\paragraph{Symbolic verification}
\bit
\w Given a set of states, each state are encoded using $n$ state bits
($s_0s_1\cdots{}s_n$. Each such state can be represented by a boolean formula
over these state bits. 
\w A set can be represented using a boolean function (and a BDD) through a
\bb{characteristic function\/} over the element encoding.
\w A state transition relation is after all a set and we can use BDD to
represent it.
\eit

%\paragraph{Representation of state sets}
%A set of states $S \in {\cal S}$ can be represented using a \bb{characteristic
%function},


\paragraph{Operator \#1: Generalized cofactor}
{\em Shannon decomposition\/} is performed w.r.t. \bb{literals}, $x_i$ and
$\overline{x_i}$, as in:
   \[ f = x_if_i + \overline{x_i}f_i.\]
Shannn decomposition is performed relative to a special function (one
variable) but we can generalize this to a general function.

Let $f, g \in B^n$, and let
   \[ f = g\cdot{f_g} + \overline{g}\cdot{f_{\overline{g}}}\]
be a decomposition of $f$ w.r.t. the orthonormal set $\{g, \overline{g}\}$. Then
the cofficient $f_g$ is called \bb{positive generalized cofactor} of $f$
w.r.t. $g$ and the coefficient $f_{\overline{g}}$ is called \bb{negative
  generalized cofactor} of $f$ w.r.t. $g$. 
\paragraph{Operator \#2: Constrain operator}
Usually, generalized cofactors of a function $f$ w.r.t. a function $g$ is not
uniquely determined.  

Let the variables $x_1, \cdots, x_n$ be ordered in the order $\pi$ according
to $x_{j_1} < x_{j_2} < \cdots < x_{j_n}$. Let $r = (r_1, \cdots, r_n), s =
(s_1, \cdots, s_n) \in B^n$. the \bb{distance} $\mid\mid r - s \mid\mid$ of
$r$ and $s$ w.r.t. the order $\pi$ is defined by
   \[ \mid\mid r - s \mid\mid = \sum_{i = 1}^n|r_{j_i} - s_{j_i}| 2^{n-i}.\]
For $f, g \in B^n$, the \bb{constrain operator} $f \downarrow g$ is defined by
  \[  (f\downarrow g)(r) = \left\{ \begin{array}{lll}
   f(r) & \mbox{if} & g(r) = 1, \\
   f(s) & \mbox{if} & g(r) = 0, g(s) = 1 \mbox{\ and\ } \mid\mid{}r-s\mid\mid
   \mbox{\ minimal}, \\
   0 & \mbox{if} & g = 0
  \end{array}
  \right.\]
\paragraph{Operator \#3: Quantification}
For $f \in B^n$, the \bb{existential quantification w.r.t. the variable $x_i$}
  is defined by
  \[ \exists_{x_i}f = f_{x_i} + f_{\overline{x_i}}.\]
The \bb{universal quantification w.r.t. $x_i$} is defined by
  \[ \forall_{x_i}f = f_{x_i} \cdot f_{\overline{x_i}}. \]

\paragraph{Operator \#4: Restrict operator}

\paragraph{Reachability analysis} 
Reachability analysis denotes the efficient computation and compact
representation of all states which can be reached from the initial state.

Let $M = (Q, I, O, \delta, \lambda, q_0)$ be a finite state machine. A state
$s \in B^n$ is said to be \bb{reachable in exactly $k$ steps from the state
  $r$} if there is an input sequence $e_0, \cdots, e_{k-1}$ and a state
sequence $s_0, \cdots, s_k$ s.t. $s_0 = r, s_k = s$ and
  \[ \delta(s_i, e_i) = s_{i+1}, \ \ 0 \le i \le k \]

\paragraph{Images}
For a finite state machine $M$ with $p$ input bits, $n$ state bits, and
next-state function $\delta: B^{n+p} \rightarrow B^n$, let
  \[ \chi_k(x_1, \cdots, x_n): B^n \rightarrow B\]
denote the \bb{characteristic function} of all states that are reachable in at
most $k$ steps.

Let $f: B^n \rightarrow B^m$. The \bb{image} $Im(f)$ of the function $f$ is
defined by
  \[ Im(f) = \{ v \in B^m: \mbox{\ there exists some\ } x \in B^n
  \mbox{\ s.t.\ } f(x) = v\}.\]
For a subset $C$ of $B^n$, the \bb{image of $f$ w.r.t. $C$} is defined by
  \[ Im(f, C) = \{ v \in B^m: \mbox{\ there exists some\ } x \in C
  \mbox{\ s.t.\ } f(x) = v\}.\]

\paragraph{Reachability algorithm based on image computation}
The following algorithm, given a state machine $M$, computes the set $Reachable$
of reachable states. 

\vspace*{0.2cm}

\begin{algorithm}{Traverse}{\delta, q_0}
  \mbox{\tt{}/* R: aet of reachable states */}\\
  R \= S_0\\
  From \= S_0\\
  \begin{REPEAT}{}
    \mbox{\tt{}/* compute image of From */}\\
    To \= Im(\delta, From)\\
    \mbox{\tt{}/* newly-reached states */}\\
    New \= To - R\\
    \mbox{\tt{}/* update reachable sets */}\\
    R \= R \cup New\\
    From \= New
  \end{REPEAT} New = \emptyset\\
  \RETURN R
\end{algorithm}

%% \begin{verbatim}
%%   traverse(delta, q0) {
%%     reached = from = S0;
%%     do {
%%       to = Im(delta, from);
%%       /* compute newly reached states */
%%       new = to \ reached;
%%       from = new;
%%       reached = reached CUP new;
%%     } while (new != emptyset);
%%     return reached;
%%   }
%% \end{verbatim}
%\bit
%\w \bb{set-difference} $To - R$ can be computed from $BDD(To) \wedge \overline{BDD(R)}$
%\eit

\paragraph{Image computation}



\end{document}
