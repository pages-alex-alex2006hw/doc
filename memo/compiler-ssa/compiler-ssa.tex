\documentclass{memo}
\usepackage{mathptm,mydef,myenv}
\usepackage[all]{xy}
%\usepackage{MinionPro}
\begin{document}
\small
\noindent{\large\bf{}Static Single Assignment Form}


\paragraph{Overview} 
Many dataflow analyses need to find the \bb{use-sites} of each {\em defined\/}
variable or the \bb{def-sites} of of each variable {\em used\/} in an
expression. The \bb{def-use} chain is a data structure that makes this
efficient: for each statement in the flowgraph, the compiler can keep a list
of pointers to all the use-sites of the variables defined there,  and a list
of pointers to all def-sites of the variables used there.  But when a variable
has $N$ definitions and $M$ uses, we need $N\cdot{}M$ pointers to connect
them. 

SSA avoids this problem by ``getting the right number of names.'' Even if we
see only one variable, $x$, based on its definition, the variable has many
versions ($x$ defined in statement $S_1$, $x$ defined in statement $S_2$,
etc.), each deserve its own name. 

In SSA, \bb{each variable in the program has only one definition} -- it is
assigned only once. The assignment might be, in a loop, which is executed many
times; so single assignment is a {\em static property\/} of the program text,
not a dynamic property of program execution.

To achieve single-assignment, we make up a new variable name for each
assignment to the variable. SSA simplifies and improves the results of various
compiler optimizations, by simplifying the properties of variables. For
example, given:
\begin{verbatim}

   original:  y = 1;        SSA: y1 = 1
              y = 2;  ===>       y2 = 2
              x = y;             x1 = y2
\end{verbatim}

{\em Reaching definition analysis\/} which determines that the first
\verb+y = 1+ is useless can be easily performed in SSA form. The following
compiler optimizations can benefit from the use of SSA:
\bit
\w constant propagation
\w sparse conditional constant propagation
\w dead code elimination
\w global value numbering
\w partial redundancy elimination
\w strength reduction
\w register allocation
\eit


%% \paragraph{Data dependence graph}
%% A \bb{data dependence graph} or \bb{precedence graph} is a form which
%% represents the flow of data between {\em definition points\/} and {\em use
%%   points\/}. This is an auxiliary intermediate representation used along with
%% the major representation such as flowgraphs. 

\paragraph{Converting flowgraph to SSA graph} Converting flowgraph into an SSA
graph involves {\em replacing the target of each {\bfseries definition}
  with a new variable, and the {\bfseries use} of a variable with the
  ``version'' of the variable   {\bfseries reaching} that point\/}. Given:
\[\xymatrix@-0.5pc{
  & *+[F]\txt{$x \leftarrow 5$ \\ $x \leftarrow x - 3$ \\ $x <
    3?$}\ar[rd]\ar[ld]& \\
*+[F]\txt{$y \leftarrow x * 2$ \\ $w \leftarrow y$ } \ar[rd]&& *+[F]\txt{$y
  \leftarrow x-3$} \ar[ld]\\
& *+[F]\txt{$w \leftarrow x - y$ \\ $z \leftarrow x + y$} &
}\]
its SSA form is:
\[\xymatrix@-0.5pc{
  & *+[F]\txt{$x_1 \leftarrow 5$ \\ $x_2 \leftarrow x_1 - 3$ \\ $x_2 <
    3?$}\ar[rd]\ar[ld]& \\
*+[F]\txt{$y_1 \leftarrow x_2 * 2$ \\ $w_1 \leftarrow y_1$ } \ar[rd]&& *+[F]\txt{$y_2 \leftarrow x_2-3$} \ar[ld]\\
& *+[F]\txt{$y_3 \leftarrow \phi(y_1, y_2)$\\ $w_2 \leftarrow x_2 - y_3$ \\ $z_1 \leftarrow x_2 + y_3$} &
}\]

\paragraph{The $\phi$-function}
Creating SSA forms when there is no jumps is easy. However, when two
control-flow edges join together, {\bf carry different values of some variable
  $x$}, we must somehow merge the two values. This is done using  the
$\phi$-function. 

The $\phi$-function at the last block, in the figure above, is inserted to
reconcile the conflict between two definitions, $y_1$ and $y_2$, along
different paths.  A $\phi$-function tasks as arguments the SSA-names for the
value on {\em each   control-flow edge entering the block\/}. It defines a new
name for subsequent uses. 

\paragraph{Computing minimal SSA using dominance frontiers}
{\em Dominance frontiers capture the precise places at which we need
$\phi$-functions:\/} if the node $A$ defines a certain variable, then that
definition and that definition alone (or redefinitions) will reach every node
$A$ dominates. 


\paragraph{Global value numbering}
Global value numbering (GVN) eliminates redundancy by constructing a \bb{value 
  graph} of the program, and then determines which values are computed by
equivalent expressions. GVN is able to identify some redundancy that
\bb{common subexpression elimination} cannot, and vice versa.

\paragraph{Sparse conditional constant propagation}
This optimization is effectively equivalent to iteratively performing
\bb{constant propagation}, constant folding, and dead code elimination until there is no
change, but is much more efficient. 
This optimization symbolically executes the program,
simultaneously propagating constant values and eliminating portions of the
control flow graph that this makes unreachable. 

\paragraph{Appel: SSA is functional programming}
Consider the following program:
\begin{verbatim}
  i = 1;
  j = 1;
  k = 0;
  while (k < 100) {
    if (j < 20) {
      j = i;
      k = k + 1;
    }
    else {
      j = k;
      k = k + 2;
    }
  }
\end{verbatim}
Its flowgraph is given below:
\[ \xymatrix@-0.5cm{
   &   *+[F]\txt{$i \ \la 1$ \\ $j \ \la 1$ \\ $k \ \la 0$} \ar[d]& &\\
 & *+[F]\txt{if $k < 100$} \ar[d]\ar[ld]& &\\
  *+[F]\txt{if $j < 20$} \ar[d]\ar[rd]& *+[F]\txt{return $j$}  &\\
  *+[F]\txt{$j \ \la 1$ \\ $k \ \ \la k +1$} \ar[rd]& 
           *+[F]\txt{$j \ \la k$\\ $k \ \la k + 2$} \ar[d] &&\\
   &       *+[F]\txt{$\arc{\mbox{empty}}$} \ar@/_13ex/[uuu] & &
}\]


\end{document}

% LocalWords:  flowgraph ld Appel GVN subexpression versa dataflow uuu
