\documentclass{myproc}
\usepackage{mydef,myenv,amssymb}
\usepackage{MinionPro}
%\usepackage{mathptm}
\usepackage[all]{xy}
\def\EE{\mbox{\eufm{}E}}

\begin{document}
\small
\pagestyle{empty}

\noindent{\large\bf Notes on Set Theory}

\section{Introduction}
\begin{itemize}
\item {\bf{}Extensionality Property}: A set is determined by its members
	\[ A = B\ \Leftrightarrow\ (\forall{x})[x \in A 
		\Leftrightarrow x \in B] \]
\item A {\bf{}function} $f: X \rightarrow Y$ associates with each member $x$
	of the set $X$ some member $f(x)$ of $Y$. $(x \mapsto f(x))$ is
	a name-free notation for functions.
	\begin{itemize}
	\item $f$ is an {\bf{}injection} (one-to-one) if 
		\[ (\forall{x, y \in X})[f(x) = f(y) \Rightarrow x = y] \]
	\item $f$ is a {\bf{}surjection} (onto) if 
		\[ (\forall{y \in Y})(\exists{x \in X})[f(x) = y] \]
	\item $f$ is a {\bf{}bijection} (correspondence) if
		\[ (\forall{y \in Y})(\exists!{x \in X})[f(x) = y] \]
	\end{itemize}
\w Given a set $A$, the {\bf{}identity function} on $A$, $I_A$, is
	the total function $(x \mapsto x)$.
\w A function $f: A \rightarrow B$ is {\em{}surjective\/} iff there exists a 
	(total) function $g: B \rightarrow A$ s.t.
	$g\circ{f} = I_B$.
\w If there exists a (total) function $g: B \rightarrow A$ s.t. 
	$f\circ{g} = I_A$
	then $f: A \rightarrow B$ is {\em{}injective\/}.
\w If $f: A \rightarrow B$ is injective and $A \ne \emptyset$ then
	there exists a function $g: B \rightarrow A$ s.t. $f\circ{g} = I_A$.
\w A function $f: A \rightarrow B$ is {\em{}bijective\/} iff 
	there exists a unique function $f^{-1}$ s.t. 
	$f\circ{f^{-1}} = I_A$ and $f^{-1}\circ{f} = I_B$; $f^{-1}$ is
	called the {\bf{}inverse} of $f$.
\w For every $f: X \rightarrow Y$ and $A \subseteq X$, the set
	\[f[A] = \{f(x): x \in A\}\]
	is the {\bf{}image of $A$ under $f$}, and if $B \subseteq Y$, then
	\[f^{-1}[B] = \{x \in X: f(x) \in B\}\]
	is the {\bf{}pre-image of $B$ by $f$}.
	
\end{itemize}

\section{Equinumerosity}
\begin{itemize}
\item Two sets $A, B$ are {\bf{}equinumerous} if there exists a
	(one-to-one) 
	{\em{}correspondence} between their elements, i.e.,
	\[ A =_c B 
		\ \triangleq\ 
		(\exists{f})[f: A \rightarrow_{\mbox{\scriptsize{}bij}} B] \]
\item The set $A$ is {\bf{}less than or equal to $B$ in size} if
	it is equinumerous with some subset of $B$, i.e.,
	\[ A \le_c B 
		\ \triangleq\ 
		(\exists{C \subseteq B})[A =_c C] \]
		\begin{itemize}
		\item $A \le_c B \Leftrightarrow (\exists{f})[f: A 
			\rightarrow_{\mbox{\scriptsize{}inj}} B]$
		\end{itemize}
\item A set $A$ is {\bf{}finite} if there exists some natural number 
	$n \in \N1$
	such that
		\[ A =_c \{i : i < n\} = \{0, 1, \cdots, n-1\}, \]
	otherwise, $A$ is {\bf{}infinite}.
	\bit
	\w A set $A$ is {\bf{}Dedekind-infinite} if there exists an injection
		\[ f: A \inj\ B \subsetneq A \]
		from $A$ into a proper subset $B \subsetneq A$. 
		
	\eit
\item A set $A$ is {\bf{}countable} if it is {\em{}finite\/} or 
	{\em{}equinumerous with the set of natural numbers $\N$},
	otherwise {\bf{}uncountable}.
	\bit
	\item A set $A$ is {\em{}countable\/} iff $A = \emptyset$ or
		$A$ has an {\em{}enumeration\/}, where
		an {\bf{}enumeration} is a surjection $\pi: {\N} 
		\rightarrow_{\mbox{\scriptsize{}surj}} A$.
	\w If $A$ is countable and there exists an injection
		$f: B\ \inj\ A$, then $B$ is also countable.
	\end{itemize}

\item ({\bf{}Cantor's first diagonal method})
	For each sequence\footnote{this implies an `enumeration', which,
	in turn, implies the countability of the sequence} 
	$A_0, A_1, \cdots$ of {\em{}countable\/} sets,
	the {\em{}union\/}
		\[ A = \bigcup_{i = 0}^\infty{A_i} = A_0 \cup A_1 \cup \cdots \]
	is also a {\em{}countable\/} set.
	\begin{itemize}
	\w This technique is also called {\em{}dovetailing\/}, which
		is heavily used in computability proofs.
	\item The set ${\Z}$ of integers is countable.
	\item The set ${\Q}$ of rational numbers is countable. 
		\bit
		\w Since 
			\[ \Q^+ = \bigcup_{n=1}^\infty\left\{\frac{m}{n}: 
		m \in \N\right\},\]
		and each $\{m/n: m \in \N\}$ for fixed $n$ is countable,
		$\Q^+$ is countable. $\Q^-$ is countable similarly.
		\eit
	\end{itemize}
\item ({\bf{}Cantor's second diagonal method}) 
	The set of {\em{}infinite\/}, binary sequences
	\[ \Delta = \{(a_0, a_1, \cdots) : (\forall{i})[a_i = 0 \vee a_i = 1] \}\]
	is uncountable.
		\begin{itemize}
		\item The set $\R$ of real numbers is uncountable.
		\w The set of total functions on $\N$ is uncountable.
		\end{itemize}
\item If $A_1, \cdots, A_n$ are all countable, so is their Cartesian
	product $A_1 \times \cdots \times A_n$.
	\begin{itemize}
	\item For every countable set $A$, each $A^i$ ($i \ge 2$) and the
		union 
			$\bigcup_{i = 2}^\infty A^i$
		is countable.
	\end{itemize}
\item The set $\cal{}K$ of {\em{}algebraic real numbers\/} is countable
	(Cantor), and hence there exists a real number that is not
		 algebraic (Liouville).
\item (Cantor) For every set $A$,
	\[ A <_c {\cal{}P}(A), \] i.e.,
	$A \le_c {\cal{}P}(A)$ but not $A \ne_c {\cal{}P}(A)$,
	where ${\cal{}P}(A)$ is the 
	{\bf{}powerset} of $A$.
	\begin{itemize}
	\w Proof of $A \le_c {\cal{}P}(A)$:
	$(x \mapsto \{x\})$ is an injection from $A$ to ${\cal{}P}(A)$.
	\w Proof of $A \ne_c {\cal{}P}(A)$:
	To the contrary, 
	let $\pi$ be a correspondence between $A$ and ${\cal{}P}(A)$
	and let 
		\[ B = \{x \in A: x \not\in \pi(x)\}.\]
	Since $B \subseteq A$ (i.e., $B \in {\cal{}P}(A)$), 
	there should exist $b \in A$ such that
	$B = \pi(b)$, and either $b \in B$ or $b \not\in B$.
	Both cases lead to contradiction (Russell phenomenon!).
	\bit
	\w {\em{}This proof is a fairly straightforward generalization
	of Cantor's second diagonal method.\/} 
	\w (cont.) Why? {\em{}We constructed $B \in {\cal{}P}(A)$ so that 
		$B$ is not equal to any element of ${\cal{}P}(A)$
		using self-reference.\/}
	\eit
	\end{itemize}
\item ({\bf{}Schr\"{o}der-Bernstein Theorem})
	For any two sets $A, B$, 
	\[ (A \le_c B\ \wedge\ B \le_c A) \Rightarrow A =_c B \]
	\bit
	\item ${\cal{}P}({\N}) \le_c {\R}$ and
		${\R} \le_c {\cal{}P}({\N})$
	\w ${\cal{}P}(\N) = \R$.
	\eit
\w A set $A$ is {\bf{}transitive} if every set $B$ which is an
	element of $A$ has the property that all of {\em{}its\/}
	elements also belong to $A$.
	\bit
	\w For every set $B \in A$, $B \in {\cal{}P}(A)$.
	\w For every set $B \in A$, $B \subseteq A$.
	\w $\bigcup{A} \subseteq A$.
	\w For any set $A$, $TC(A)$ is the smallest transitive set including
		$A$; it is called the {\bf{}transitive closure} of $A$: e.g.,
		\[ TC(A) = \bigcup\left\{A, \bigcup{A}, \bigcup\bigcup{A}, 
		\cdots\right\}.\]
	\eit
\end{itemize}


\section{Paradoxes and Axioms}
\begin{itemize}
\item ({\bf{}Hypothesis of Cardinal Comparability})
	For any two sets $A, B$, either $A \le_c B$ or $B \le_c A$.
\item ({\bf{}Continuum Hypothesis; CH})
	There is no set of real numbers $X$ with cardinality
	intermediate between those of $\N$ and $\R$, i.e.,
	\[ (\forall{X} \subseteq {\R})[X \le_c {\N} \ \vee\ X =_c \R] \]
\item ({\bf{}Generalized Continuum Hypothesis; GCH})
		For every infinite set $A$,
		\[ (\forall{X} \subseteq {\cal{}P}(A))[X \le_c A\ \vee\ X =_c {\cal{}P}(A)]\]
\item ({\bf{}General Comprehension Principle}) For each $n$-ary
	{\em{}definite condition\/} $P$, 
		there is a set 
			\[A = \{\vec{x}: P(\vec{x})\} \]
		whose members are precisely all the $n$-tuples of objects
		which satisfy $P(\vec{x})$, such that for all $\vec{x}$,
			\[ \vec{x} \in A \Leftrightarrow P(\vec{x})\]

		\begin{itemize}
		\item By {\em{}extensionality principle}, only {\bf{}one} 
			set $A$	can satisfy the above equivalence, we call 
			this $A$ the {\bf{}extension} of 
			the {\em{}condition\/} $P$.
		\item A $n$-ary {\bf{}condition} $P$ is {\bf{}definite}  if 
			for each
			$n$-tuple of objects $\vec{x} = (x_1, \cdots, x_n)$, 
			it is
			determined `unambiguously' whether 
			$P(\vec{x})$ is
			true or false.
		\item {\em{}Note that we do not demand of a definite 
			condition that its 
			truth value be {\em{}effectively decidable\/}}
			(Turing-decidable or recursive).
		\item A $n$-ary {\bf{}operation} $F$ is {\bf{}definite}, if it
			assigns to each $n$-tuple of objects $\vec{x}$ a
				unique, unambiguously determined object
				$w = F(\vec{x})$.
		\item {\em{}Note again that we don't demand of a definite 
			operation
			be {\em{}effectively computable\/}}.
		\end{itemize}
\item ({\bf{}Russell's Paradox}) {\em{}The 
	General Comprehension Principle is not valid\/}.
		\begin{itemize}
		\item Russell's normal set
			\[ R = \{x : \mbox{$x$ is a set and\ } x \not\in x\}\]
		\item Is $R \in R$ or $R \not\in R$?
		\end{itemize}
\end{itemize}
\paragraph{Axiomatic set theory}
\begin{itemize}
\item Zermelo's {\bf{}axiomatic set theory} saved the Cantor's paradise.
\item {\bf{}Axiomatic setup}
	\begin{itemize}
	\item assumption of a {\bf{}domain} or {\bf{}universe $\cal{}W$ of
	objects}, {\em{}including} {\bf{}sets}
	\item {\bf{}definite conditions} and {\bf{}definite operations on 
		$\cal{}W$} {\em{}including} 
		{\em{}\bfseries{}identity},
		{\em{}\bfseries{}sethood}, and
		{\em{}\bfseries{}memebersip}
	\item We call the objects that are not sets {\bf{}atoms}; but we 
		do not require that any atoms exist.
	\end{itemize}
\item ({\bf{}I: Axiom of Extensionality}) For any two sets
	$A, B$,
		\[ A = B \ \Leftrightarrow\ (\forall{x})[x \in A \ \Leftrightarrow\ 
			x \in B] \]

\item ({\bf{}II: Emptyset and Pairset Axioms}) 
	\begin{enumerate}
	\item [(a)] There is a special object $\emptyset$, which is a set with
		no members.
	\item [(b)] For any two objects $x, y$, there is a set $A$ whose only
		members are $x$ and $y$, i.e.,
			\[ t \in A \ \Leftrightarrow\ t = x \mbox{\ or\ } t = y\]
	\end{enumerate}
\item ({\bf{}III: Separation Axiom; Axiom of Subsets})
	For each set $A$ and each unary, definite condition $P$, there
		exists a set $B$ which satisfies the equivalence
		\[ x \in B \ \Leftrightarrow\ x \in A \mbox{\ and\ } P(x)\]
		\begin{itemize}
		\item From the Extensionality Axiom, only one $B$ satisfies the
			above equivalence, which we will denote by
				$B = \{x \in A: P(x)\}$.
		\item {\em{}This axiom restricts 
			the General Comprehension
			Principle and frees us from the Russell's Paradox.\/}
		\end{itemize}
\item ({\bf{}IV: Powerset Axiom})
	For each object $A$, there exists a set $B$ whose members are
	the subsets of $A$, i.e.,
	\[ X \in B\ \Leftrightarrow\ Set(X) \mbox{\ and\ } X \subseteq A\]
		\begin{itemize}
		\item $X \subseteq A \triangleq (\forall{t})[t \in X \Rightarrow t \in A]$
		\item ${\cal{}P}(A) \triangleq \{X: Set(X) \mbox{\ and\ } X \subseteq A\}$
		\end{itemize}
\item ({\bf{}V: Unionset Axiom})
	For every object $\cal{}E$, there exists a set $B$ whose members
		are the members of the members of $\cal{}E$, i.e., it satisfies
		the equivalence,
		\[ t \in B\ \Leftrightarrow\ (\exists{X}\in {\cal{}E})[t \in X] \]
\item ({\bf{}VI: Axiom of Infinity})
	There exists a set $I$ which contains the empty set $\emptyset$
	and the singleton of each of its members, i.e.,
	\[ \emptyset \in I \mbox{\ and\ } (\forall{x})[x \in I \Rightarrow
	\{x\} \in I]\]
	\begin{itemize}
	\item In the 19th century, there was a belief that the existence of
		natural numbers could be proved. But now we know better:
		{\em{}Logic 
		can codify the valid forms of reasoning but it
		cannot prove the existence of anything, let alone infinite
		sets\/}.
	\end{itemize}
\item For every unary, definite condition $P$ there
	exists a {\bf{}class} 
		\[ A = \{x : P(x)\} \]
	such that for every object $x$,
		\[ x \in A \ \Leftrightarrow\ P(x)\]
		\begin{itemize}
		\item Every set will be a class; but because of the Russell Paradox,
			{\em{}there must be more classes than sets\/}.
		\item A unary definite condition $P$ is {\bf{}coextensive} with
			a set $A$ if the objects which satisfy it are
			precisely the members of $A$.
			\[ P =_e A \ \triangleq\ 
			(\forall{x})[P(x) \Leftrightarrow x \in A] \]
		\item By Russell Paradox we know that {\em{}not every $P$ is
			coextensive with a set\/}.
		\item A class is either a set or a unary definite condition
			which is not coextensive with a set.
		\end{itemize}
\item ({\bf{}VII: Axiom of Choice})
\item ({\bf{}VIII: Axiom of Replacement})
\item {\bf{}Principle of Purity}: ``Every object is a set; there are no
	atoms.''
	
\end{itemize}

\section{Are Sets All There Is?}
\begin{itemize}
\item In analytic geometry the geometric line $\Pi$ is 
	``identified'' with real numbers $\R$. The correspondence
	$P \mapsto x(P)$ gives a {\bf{}faithful representation} of $\Pi$
	in $\R$.
\item {\em{}``Now let's codify ordered pairs, relations, 
	functions, equinumerosity,
	$\cdots$ in the framework of set theory.''}
\item Representation of {\bf{}ordered pair} in {\bf{}sets}
	\begin{itemize}
	\item Characteristic properties of ordered pairs
		\begin{enumerate}
		\item [(a)] $(x, y) = (x', y') \ \Leftrightarrow\ 
			x = x' \mbox{\ and\ } y = y'$
		\item [(b)] $A\times B \triangleq 
			\{(x, y) : x \in A \mbox{\ and\ }
				y \in B\}$ is a set.
		\end{enumerate}
	\end{itemize}

	\item The {\bf{}Kuratowski pair} operation
		\[ (x, y)\ \triangleq\ \{\{x\}, \{x, y\}\} \]
		satisfies above two properties.
		\bit	
		\w {\em{}Proof of \/}(b): The condition
		\[ \mbox{\em{}OrdPair}_{A, B}(z) \triangleq
			(\exists{x}\in A)(\exists{y}\in B)[z = (x, y)] \]
		is evidently {\em{}definite\/}.
		Now if we can find for each $A, B$ some set $C$ such that
		\[ x \in A \mbox{\ and\ } y \in B \Rightarrow
			(x, y) \in C, 
		\]
		then, using $C$, we can construct a Cartesian set
		\[ A \times B \triangleq \{z \in C: 
			\mbox{\ \em{}OrdPair}_{A, B}(z)\}\]
		as a set by the {\em{}Separation Axiom\/}.
		We let $C = {\cal{}P}({\cal{}P}(A \cup B))$ since
		\begin{eqnarray*}
		x \in A, y \in B & \Rightarrow & \{x\}, \{x, y\} 
			\subseteq (A \cup B) \\
		& \Rightarrow & \{x\}, \{x, y\} \in {\cal{}P}(A \cup B)\\
		& \Rightarrow & \{\{x\}, \{x, y\}\} 
			\subseteq {\cal{}P}(A \cup B)\\
		& \Rightarrow & \{\{x\}, \{x, y\}\} 
			\in {\cal{}P}({\cal{}P}(A \cup B))
		\end{eqnarray*}
		\eit

\w Representation of {\bf{}disjoint union} in sets:
	\[ A \uplus B \ \triangleq\  (\mbox{\em{}blue\ } \times A) \cup
		(\mbox{\em{}red\ } \times B), \]
	where
	\[ \mbox{\em{}blue\ } \triangleq \emptyset \mbox{\ \ and\ \ }
		\mbox{\em{}red\ } \triangleq \{\emptyset\}.\]
\item A {\bf{}binary relation} on the sets $A, B$ is any subset of
	the Cartesian product $A \times B$. 
	\[ xRy \ \triangleq\ (x, y) \in R. \]
	\bit
	\w The obvious way to represent binary relations in sets is
	to {\em{}identify it with its {\em{}extension\/},
	te set of pairs which satisfy it.\/}
	\eit
\item {\em{}Every relation determines a definite condition but 
	the converse is not true.\/}
	\begin{itemize}
	\item Definite conditions $x = y$, $x \in y$, and $X \subseteq Y$ are
		not binary relations according to the above definition.
	\item Instead, {\em when restricted to some set $A$}, 
		{\em{}identity, membership\/}, and {\em{}subsethood\/} are indeed
		binary relations.
		\begin{eqnarray*}
		x =_A y & \triangleq & x \in A \mbox{\ and\ } y \in A \mbox{\ and\ }
			x = y\\
		x \in_A y & \triangleq & x \in A \mbox{\ and\ } y \in A \mbox{\ and\ }
			x \in y\\
		X \subseteq_A Y & \triangleq & X \subseteq Y \subseteq A
		\end{eqnarray*}
	\end{itemize}
\item A {\bf{}function} (or {\bf{}mapping} or {\bf{}transformation})
	$f: A \rightarrow B$ with {\em{}domain\/} the set $A$ and 
	{\em{}range\/}
	the set $B$ is any subset $f \subseteq (A \times B)$ which satisfies
	the condition
		\[ (\forall{x}\in A)(\exists{!y \in B})[(x, y) \in f]. \]
		\begin{itemize}
		\item $(A \rightarrow B) \triangleq \{f \subseteq A \times B: 
			f: A \rightarrow B\}$: a set of all functions 
			from $A$ to
			$B$.
		\item When $A$ and $B$ are sets, the set
	of {\em{}total\/} functions from $A$ to $B$ is sometimes denoted by
	\[ B^A = \{f: f \mbox{\ is a total function from $A$ to $B$}\}\]
	where $|B^A| = |B|^{|A|}$.
	An interesting isomorphic interpretation of $B^A$ is the set of 
	$|A|$-ary number with $|B|$ digits.
	For example, let $A = \{1, 2, \cdots, n\}$. Then
	\[ B^{\{1, 2, \cdots, n\}} = \{(b_1, \cdots, b_n): b_i \in B\}.\]
	That is, $B^{\{1, 2, \cdots, n\}}$ can be thought of as a set of 
	{\em{}strings over $A$ whose length is $n$\/}.
	Note that $(b_1, \cdots, b_n) \in B^{\{1, 2, \cdots, n\}}$ denotes `a'
	function $f$ such that
	\begin{eqnarray*}
	f(1) & = & b_1,\\
	f(2) & = & b_2,\\
	& \cdots & \\
	f(n) & = & b_n.
	\end{eqnarray*}
		\end{itemize}
\item An {\bf{}indexed family of sets} is a function
	\[ A = (i \mapsto A_i)_{i\in I}: I \rightarrow E\]
	for some $I \ne \emptyset$ and some $E$, where each $A_i$ is a
	set.
	\bit
	\w {\bf{}union} and {\bf{}intersection} of an indexed family of sets
		\begin{eqnarray*}
		\bigcup_{i \in I}A_i & \triangleq &
			\{x \in \bigcup E: (\exists{i \in I})[x \in A_i]\}\\
		\bigcap_{i \in I}A_i & \triangleq &
			\{x \in \bigcup E: (\forall{i \in I})[x \in A_i]\}\\
		\end{eqnarray*}
	\w {\bf{}product} of an indexed family
		\[ \prod_{i \in I}A_i \triangleq
		\{f: I \rightarrow \bigcup_{i\in{I}}A_i: 
		(\forall{i \in I})[f(i) \in A_i] \]
	\w brand-new definition of {\bf{}equinumerosity} and 
		{\bf{}size comparison}
		\begin{eqnarray*}
		A =_c B & \triangleq & (\exists{f})[f: A 
			\rightarrow_{\mbox{\scriptsize{}bij}}
			B]\\
		&  \Leftrightarrow & (A 
			\rightarrow_{\mbox{\scriptsize{}bij}} B) \ne 
			\emptyset\\
		A <_c B & \triangleq & (\exists{f})[f: A 
			\rightarrow_{\mbox{\scriptsize{}inj}}
			B]\\
		& \Leftrightarrow & (A 
			\rightarrow_{\mbox{\scriptsize{}inj}} B) \ne \emptyset
		\end{eqnarray*}
	\eit
\w For each $X \subseteq A$, the {\bf{}restriction} $f \upharpoonright X$ of
	a  function $f: A \rightarrow B$ is
	obtained by cutting $f$ down so it is
		defined only on $X$,
		\[ f\upharpoonright X \triangleq \{(x, y) \in f: x \in X\} \]
\w The basic condition of {\bf{}functionhood}
	\[ \mbox{\em{}Function}(f) \triangleq (\exists{A})(\exists{B})[f \in (A
		\rightarrow B)] \]
	is a definite condition.
\end{itemize}
\paragraph{Cantor's Notion of Cardinal Numbers}
\bit
\w ({\bf{}Problem of Cardinal Assignment})
	Define an operation $|{\cdot}|$ on the class of sets which satisfies
		\bit
		\w [(a)] $A =_c |A|$
		\w [(b)] $A =_c B \Leftrightarrow |A| = |B|$
		\w [(c)] $\mbox{for each ${\cal{}E}, 
			\{|X|: X \in {\cal{}E}\}$ is a set}$
		\eit
\w A {\bf{}$($weak$)$ cardinal assignment} is any definite operation
	$|{\cdot}|$ which satisfies the conditions (a) and (c) given above.
	The {\bf{}cardinal numbers} (relative to $|{\cdot}|$) are its values,
		\[ \mbox{\em{}Card}(\kappa)\ \Leftrightarrow\ 
			\kappa \in \mbox{\ \em{}Card}\
			\ \triangleq\ (\exists{A})[\kappa = |A|]. \]
	A cardinal assignment $|A|$ is {\bf{}strong}, if in addition, 
		for any two cardinal numbers $\kappa, \lambda$,
			\[ \kappa =_c \lambda \ \Leftrightarrow\ \kappa = \lambda\]
		which is equivalent to the condition (b) above.
\w Arithmetic operations on cardinal numbers
	\begin{eqnarray*}
	\kappa + \lambda & \triangleq & |\kappa \uplus \lambda| =_c 
		\kappa \uplus \lambda\\
	\kappa \cdot \lambda & \triangleq & |\kappa \times \lambda| =_c 
		\kappa \times \lambda\\
	\kappa^\lambda & \triangleq & |(\kappa \rightarrow \lambda)| =_c 
		(\kappa \rightarrow \lambda)
	\end{eqnarray*}
	\bit
	\w Infinitary operations may be defined similarly:
	\[\sum_{i \in I}\kappa_i  \triangleq 
		\left|\left\{(i, x) \in I \times\bigcup_{i \in I}\kappa_i: x 
			\in \kappa_i\right\}\right|\]
			and
			\[
		\prod_{i \in I}\kappa_i  \triangleq 
			\left|\prod_{i\in I}\kappa_i\right|\]
	\eit
%\w {\bf{}Cardinal arithmetic}
\w For every indexed family of sets $A = (i \mapsto A_i)_{i \in I}$,
	there exists a function $f: I \rightarrow f[I]$ s.t.
		\[ f(i) = |A_i| \quad (i \in I)\]
\eit

\paragraph{Structured Sets}
\bit
\w A {\bf{}topological space} is a set $X$ of points endowed with a
	{\bf{}topological structure}, which is determined by a collection
	$\cal{}T$ of subsets of $X$ satisfying the following three
	properties:
	\bit
	\w [(a)] $\emptyset \in \cal{}T$ and $X \in \cal{}T$
	\w [(b)] $A, B \in \cal{}T$ $\Rightarrow$ $A \cap B \in \cal{}T$
	\w [(c)] For every family $\cal{}E \in T$ of sets in
		$\cal{}T$, the unionset $\bigcup\cal{}E$ is also in $\cal{}T$.
	\eit
\w A family of sets $\cal{}T$ with the three properties is called a
	{\bf{}topology} on $X$, with {\bf{}open sets} its members and
	{\bf{}closed sets} the complements of open sets relative to $X$, 
	i.e., all $X \setminus G$ with $G$ open.
\w A {\bf{}structured set} is a pair $U = (A, \cal{}S)$
	where $A = \mbox{\ \em{}Field}(U)$ is a set, the {\bf{}field} or {\bf{}space} of $U$, 
	and $\cal{}S$ is an arbitrary object, the {\bf{}frame} of $U$.
		\bit
		\w A {\em{}topological space\/} is a structured set $(X, \cal{}T)$.
		\w A {\em{}group\/} is a structured set $U = (G, (e, \cdot)) 
			= (G, e, \cdot)$ where
			$e \in G$ and $\cdot:G\times{G} \rightarrow G$ is a binary function
			satisfying the group axioms. 
		\eit

\eit



\section{The Natural Numbers}
\bit
\w A {\bf{}system of natural numbers} is any structured set
	$({\N}, 0, S) = ({\N}, (0, S))$ which satisfies the
	following conditions (or {\bf{}axioms of Peano}):
	\bit
	\w [(a)] $0 \in \N$
	\w [(b)] $S: \N \rightarrow \N$
	\w [(c)] $S$ is an injection, i.e., $Sn = Sm \Rightarrow n = m$
	\w [(d)] $(\forall{}n \in \N)[Sn \ne 0]$
	\w [(e)] ({\bf{}Induction principle}) for each $X \subseteq \N$,
		\[ [0 \in X \mbox{\ and\ } 
			(\forall{n \in {\N}})[x \in X \Rightarrow Sn \in X]]
			\Rightarrow X = {\N} \]
	\eit
\w In a system of natural numbers $({\N}, 0, S)$, every
	element $n \ne 0$ is a successor, 
		\[ n \ne 0 \Rightarrow (\exists{m \in {\N}})[Sm = n] \]
	and for each $n$, $Sn \ne n$.
\w ({\bf{}Existence Theorem for the Natural Numbers})
	There exists at least one system of natural numbers
	$({\N}, 0, S)$.
	\bit
	\w Can be proved using the {\em{}Axiom of Infinity\/}.
	\eit
\w ({\bf{}Uniqueness Theorem for the Natural Numbers})
	For any two systems of natural numbers
	$({\N}_1, 0_1, S_1)$ and
	$({\N}_2, 0_2, S_2)$, there exists exactly one
	bijection (or {\bf{}isomorphism}) 
	$\pi: {\N}_1\ \bij \ \N_2$ s.t.
		\begin{eqnarray*}
		\pi(0_1) & = & 0_2 \\
		\pi(S_1n) & = & S_2\pi(n) \quad (n \in {\N}_1)
		\end{eqnarray*}
\w ({\bf{}Recursion Theorem}) 
	Assume that $({\N}, 0, S)$ is a system of natural numbers,
	$E$ is some set, $a \in E$, and $h: E \rightarrow E$ is some function.
	It follows that there exists exactly one function 
	$f: {\N} \rightarrow E$ which satisfies the identities
	\begin{eqnarray*}
	f(0) & = & a,\\
	f(Sn) & = & h(f(n)), \quad  (n \in {\N})
	\end{eqnarray*}
	\bit
	\w {\em{}The Recursion Theorem justifies the usual way by which
		we can define functions on the natural numbers by recursion
		$($or induction$)$\/}
	\w From a purely mathematical view, Recursion Theorem can be viewed
		as a theorem of 
		{\em{}existence and uniqueness of solutions of the system
		of the identities where $f$ is the unknown\/}.
	\w Recursion Theorem is a special case of {\bf{}Continuous
		Least Fixed Point Theorem}, which, in turn, is a special
		case of {\bf{}Fixed Point Theorem} of Zermelo.
	\eit
\w ({\bf{}Recursion with parameters}) For any two sets $Y, E$ and
	functions
		\[g: Y \rightarrow E, \quad h: E\times{Y} \rightarrow E,\]
	there exists exactly one function $f: {\N}\times{Y} \rightarrow E$
	which satisfies the identities
	\begin{eqnarray*}
	f(0, y) & = & g(y) \quad (y \in Y),\\
	f(n+1, y) & = & h(f(n, y), y) \quad (y \in Y, n \in {\N})
	\end{eqnarray*}
\eit
\paragraph{The Natural Numbers}
\bit
\w We denote the {\bb{}cardinal number of $\N$\/} by the first
	Hebrew letter,
		\[ \aleph_0 \triangleq |{\N}| \]
\w Functions $a: {\N} \rightarrow A$ with the domain $\N$ are
	called (infinite) {\bf{}sequences} and we often write their
	argument as a subscript,
		\[ a_n = a(n) \quad (n \in {\N}, a: {\N} \rightarrow A)\]
\w {\em{}Addition and multiplication\/}. The addition function on
	the natural numbers is defined by the recursion
		\begin{eqnarray*}
		n + 0 & = & n,\\
		n + Sm & = & S(n + m)
		\end{eqnarray*}
	and multiplication is defined next, using addition, by the recursion
		\begin{eqnarray*}
		n \cdot 0 & = & 0, \\
		n \cdot Sm & = & (n\cdot{}m) + m
		\end{eqnarray*}
\w A binary relation $\le$ on a set $P$ is a {\bf{}partial ordering}
	if it is reflexive, transitive, and antisymmetric.
\w The partial ordering $\le$ is {\bf{}total} (or {\bf{}linear})
	if any two elements of $P$ are {\bf{}comparable} in $\le$, i.e.,
		\[ (\forall{}x, y \in P)[x \le y \mbox{\ or\ } y \le x] \]
\w The binary relation $\le$ on $P$ is a {\bf{}wellordering} of $P$
	if it is a total ordering of $P$ and, in addition, {\em{}every
	non-empty subset of $P$ has a least element\/},
	\[ (\forall{X}\subseteq P)[X \ne \emptyset \Rightarrow
		(\exists{x \in X})(\forall{y \in X})[x \le y] \]
		\bit
		\w The order relation $\le$ on the natural numbers is 
			defined by the equivalence
			\[ n \le m \triangleq (\exists{s})[m = n + s] \]
		\w The ordering $\le$ on $\N$ is a wellordering.
		\eit
\w A set $A$ is {\bf{}finite} if there exists some natural
	number $n$ s.t. $A =_c [0, n)$, {\bf{}infinite} if it is not
	finite and {\bf{}countable} if it is finite or equinumerous
	with $\N$. The {\bf{}finite cardinals} are the cardinal
	numbers of finite sets.
\w ({\bf{}Pigeonhole Principle}) Every injection $f: A \inj A$
	on a finite set into itself is also a surjection, i.e.
	$f[A] = A$.

\w ({\bf{}Simultaneous Recursion Theorem}) For each two sets
	$E_1, E_2$, elements $a_1 \in E_1, a_2 \in E_2$, and functions
	$h_1: E_1\times{E_2} \rightarrow E_1, h_2: 
		E_1\times{E_2} \rightarrow E_2$,
	there exists unique functions
		\[f_1: {\N} \rightarrow E_1, \quad f_2: {\N} \rightarrow E_2 \]
	which satisfy the identities
		\begin{eqnarray*}
		f_1(0) & = & a_1,\\
		f_2(0) & = & a_2,\\
		f_1(n+1) & = & h_1(f_1(n), f_2(n)),\\
		f_2(n+1) & = & h_2(f_1(n), f_2(n)) 
		\end{eqnarray*}
\eit

\paragraph{The Cardinal Numbers}
\bit
\w For each set $A$, we define the set of {\bf{}finite sequences}
	(or {\bf{}words} or {\bf{}strings}) from $A$ by
	\begin{eqnarray*}
	A^{(n)} & \triangleq & \{u \in {\N}\times{A}: 
			\mbox{\ \em{}Function}(u) 
		\mbox{\ and\ }
       \mbox{\ \em{}Domain}(u) = [0, n)\},\\
	A^* & \triangleq & \bigcup_{i=0}^\infty A^{(i)}
	\end{eqnarray*}
	\bit
	\w The {\bf{}length} of the string $u$ is defined as
		\[ lh(u) \triangleq \mbox{\ max}\{i: i = 0 \mbox{\ or\ }
			i -1 \in \mbox{\ \em{}Domain}(u)\}, \]
			$u \in A^*$
	\w We let $u \sqsubseteq v$ if $u \subseteq v$ for $u, v \in A^*$
		and we call $u$ an {\bf{}initial segment} of $v$ if $u \sqsubseteq v$.
	\eit
\w For each cardinal number $\kappa$ and each $n \in {\N}$, we
	set 
		\[ \kappa^n \triangleq |\kappa^{(n)}| \]
\w For each countably infinite set $A$ and each $n > 0$,
	\[ A =_c A\times{A} =_c A^{(n)} =_c A^*. \]
	As equations of cardinal arithmetic, these read:
		\[ \aleph_0 =_c \aleph_0\cdot\aleph_0 =_c
			{\aleph_0}^n =_c |\aleph_0^*| \]
\w {\bb{}The Continuum\/}.
	The classical notation for the cardinal of ${\cal{}P}(\N)$ is 
		\[ \c \ \triangleq\ |{\cal{}P}(\N)| =_c 2^{\aleph_0} \]
	\bit
	\w $\c\cdot{\c} =_c 2^{\aleph_0}\cdot{}2^{\aleph_0} =_c
		2^{\aleph_0 + \aleph_0} =_c 2^{\aleph_0} =_c \c$
	\w $\c =_c {\aleph_0}^{\aleph_0} =_c \c^{\aleph_0}$
	\w {\em{}The Continuum Hypothesis} revisited
	\[ (\forall{\kappa \le_c \c})[\kappa \le_c \aleph_0 \mbox{\ or\ }
		\kappa =_c \c]\]
	\eit
\eit

\section{Fixed Points}
\bit
\w A {\bf{}partially ordered set} (or {\bf{}poset}) is a structured
	set \[P = (\mbox{\em{}Field}(P), \le_P),\] where
	$\mbox{\em{}Field}(P)$ is an arbitrary set and $\le_P$ is a partial ordering
	on $\mbox{\em{}Field}(P)$.
	\bit
	\w Note that $\le_P$ determines $P$ since it's reflexive, i.e.
		\[ x \in \mbox{\ \em{}Field}(P) \ \RA \ x \le_P x \]
	\w $\bot = \bot_P \triangleq \mbox{\ the least element of $P$ 
		(if it exists)}$
	\eit
\w Let $P$ be a poset, $S \subseteq P$ and $M \in P$ a member of $P$.
	\ben
	\w $M$ is an {\bf{}upper bound} of $S$ if it is greater than
		or equal to every element of $S$, i.e.
		$(\forall{x \in S})[x \le M]$.
	\w $M$ is a {\bf{}greatest element} in $S$ 
		if it is a member and an upper
		bound of $S$, i.e. $M \in S$ and $(\forall{x \in S})[x \le M]$.
	\w $M$ is a {\bf{}least upper bound} of $S$ if it is an upper
		bound and also less than or 
		equal to every other upper bound of 
		$S$, i.e.
		
		\[ (\forall{x} \in S)[x \le M]\] and
		\[(\forall{M'})[(\forall{x \in S})[x \le M']\ \Ra\ M \le M']\]
	\w $M$ is {\bf{}maximal} in $S$ if 
		\[ (\forall{x \in S})[M \le x\ \Ra\ M = x]. \]
		Note that maximal elements are not necessarily unique.
	\een
\w When it exists, the least upper bound of $S$ is denoted by
	\[ \sup S = \mbox{\ the least upper bound of $S$} \]
	\bit
	\w There exists {\em{}at most one\/} least upper bound of $S$.
	\w If $M$ is a greatest element 
		of $S$ then $M$ is the least upper bound of $S$.
	\eit
\w A {\bf{}partial function} on a set $A$ to a set $E$ is any function
	with domain of definition some subset of $A$ and values
	in $E$, in symbols
	\begin{eqnarray*}
	f: A \rightharpoonup E &\triangleq &
		\mbox{\ \em{}Function}(f) \mbox{\ and\ } \\
		&&\mbox{\ \em{}Domain}(f) \subseteq A
		\mbox{\ and\ } \mbox{\ \em{}Image}(F) \subseteq E
	\end{eqnarray*}

		\bit
		\w $(A \rightharpoonup E) \triangleq \{f \subseteq A \times E:
			f: A \rightharpoonup E\}$
		\eit
\w A {\bf{}chain} in a poset $P$ is any linearly ordered subset $S$
	of $P$, i.e. a subset satisfying
		\[ (\forall{x, y \in S})[x \le y \mbox{\ or\ } y \le x]. \]
	A poset $P$ is {\bf{}chain-complete} or {\bf{}inductive} if
	every chain in $P$ has a {\em{}least upper bound\/}.
	\bit
	\w For each set $A$, the powerset ${\cal{}P}(A)$ is inductive.
	\w For any two sets $A, B$, the poset $(A \rightharpoonup E)$
		of all partial functions from $A$ to $E$ is inductive.
	\w For every poset $P$, the set
		\[ \mbox{\em{}Chains}(P) 
		\triangleq \{S \subseteq P: S \mbox{\ is a chain}\} \]
		of all chains in $P$ (partially ordered under $\subseteq$) is
		inductive.
	\eit
\w ({\bf{}Zorn's Lemma}) Given a poset $(P, \le_P)$, if every nonempty
	chain has an upper bound, then $P$ has at least one
	greatest element.
\w A mapping $\pi: P \rightarrow Q$ on a poset $P$ to another is
	{\bf{}monotone} if for all $x, y \in P$,
		\[ x \le_P y\ \Ra\ \pi(x) \le_Q \pi(y) \]
\w A monotone mapping $\pi: P \rightarrow Q$ on an {\em{}inductive\/} poset
	to another is {\bf{}countably continuous} if for every
	non-empty, countable chain $S \subseteq P$,
		\[\pi(\sup S) = \sup \pi[S]\]
\w ({\bf{}Continuous Least Fixed Point Theorem})
	Every countably continuous, monotone mapping $\pi: P \rightarrow P$ 
	on an
	inductive poset into itself has exactly one
	{\bf{}strongly least fixed point} $x^*$, which is characterized by
	the two properties,
	\begin{eqnarray*}
		& \pi(x^*) = x^*, &\\
		&(\forall{y \in P})[\pi(y) \le y\ \Ra\ x^* \le y]&
	\end{eqnarray*}
\w A partial function $g: A \rightharpoonup E$ is {\bf{}finite}
	if it has finite domain, i.e. if it is a finite set of ordered pairs.
	A mapping $\pi: (A \rightharpoonup E) \rightarrow (B \rightharpoonup M)$ 
	from one partial function space to another
	is {\bf{}continuous}, if it is monotone and for each 
	$f: A \rightharpoonup E$, and each $y \in B$ and $v \in M$,
	\[ \pi(f)(y) = v \Rightarrow (\exists{g}\in f)[g \mbox{\ is finite\ and\ }
	\pi(g)(y) = v]\]
	\bit
	\w A function $f: X \rightarrow Y$ from one topological space to another
		is (topologically) {\bf{}continuous} if the inverse  image 
		$f^{-1}[G]$ of every open subset of $Y$ is an open subset of $X$.
	\eit
\w Every continuous mapping $\pi: (A \rightharpoonup E) \rightarrow
	(B \rightharpoonup M)$ is countably continuous, in fact, for every
	(not necessarily countable) non-empty chain $S \subseteq (A 
	\rightharpoonup E)$, 
		\[ \pi(\sup S) = \sup \pi[S] \]
\eit

\section{Well-Ordered Sets}
\bit
\w A {\bf{}well-ordered set} is a poset
	\[ U = (\mbox{\em{}Field}(U),\ \le_U), \]
	where $\le_U$ is a {\bf{}wellordering} on {\em{}Field\/}$(U)$, i.e.
	a linear (total) ordering on {\em{}Field\/}$(U)$ such that
	every non-empty $X \subseteq \mbox{\em{}\ Field\/}(U)$ has a 
	least member.
	\bit
	\w Associated with $U$ is also its {\bf{}strict ordering}
	$<_U$,
	\[ x < y\ \Leftrightarrow\ x <_U y \ \triangleq \ 
	x \le_U y \mbox{\ and\ } x \ne y.\]
	\eit
\w A set is {\bf{}well orderable} if it admits a wellordering, so
	it is the field of some well ordered set $(A, \le)$.
	\bit
	\w One important lesson in this section is that 
		{\em{}well orderable sets behave much better than
		arbitrary sets, for example, any two of them are comparable
		in cardinality\/}.
	\w If fact, {\em{}every set is well orderable\/}, which was
		proved by Zermelo.
	\eit
\eit



\bibliographystyle{plain}
\bibliography{bib/mac,bib/math}
\nocite{Moschovakis94,Barwise77,BM96}
%\pagebreak
%\tableofcontents
\end{document} 
% LocalWords:  Extensionality surjection Equinumerosity equinumerous Dedekind
% LocalWords:  iff countability Liouville powerset Schr der CH GCH ary Emptyset
% LocalWords:  extensionality Russell's Pairset equinumerosity subsethood Peano
% LocalWords:  functionhood Infinitary unionset wellordering poset surjective
% LocalWords:  Zorn's LocalWords
