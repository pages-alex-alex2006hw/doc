\documentclass{memo}
\usepackage{mathptm,mydef,myenv}
%\usepackage{MinionPro}
\begin{document}
\small
\noindent{\large\bf{}Unix Filesystems}

\paragraph{Overview} Unix filesystem is a collection of files and directories
that has the following properties:
  \bit
  \w It has a root directory (\verb+/+) that contains other files and
  directories. 
  \w Each file or directory is uniquely identified by its name, the directory
  in which it resides, and a unique identifier, typically called an
  \bb{inode}. 
  \w By convention, the root directory has an inode number of 2. Inode numbers
  0 and 1 are not used (file inode numbers can be seen using
  ``\verb+ls -i+'').
  \w It is self-contained. It does not depend on other filesystems.
  \eit

\paragraph{Disks, slices, partitions, and volumes}
Each hard disk is typically split into a number of separate, different sized
units called \bb{partitions} or \bb{slices}. Each disk contains some for of
partition table, called a \bb{VTOC (Volume Table of Contents)}, which describes
where the slices start and what their size is.  
Each slide may then be used to store bootstrap information, a filesystem, swap
space, or left as a {\em raw parition\/} for database access or other use

\paragraph{Mounting and unmounting filesystems}
The root filesystem is mounted by the kernel during system startup. Each
filesystem can be mounted on any directory in the root filesystem except
\verb+/+. A \bb{mount point} is simple a directory. \verb+mount+ program
displays all mounted filesystems.

\paragraph{Querying filesystem statistics}
We can use \verb+stat+ program to display file or filesystem status. For
example, ``\verb+stat -f /+'' shows the statistics of root filesystem. 

\paragraph{On-disk layout of Unix filesystem}
Typically a disk is viewed as an array of blocks by the Unix system. 
\bit
\w Block 0 (\bb{bootblock}): unused 
\w Block 1 (\bb{superblock}): holds info about the filesystem as a
whole such as the number of blocks in the filesystem, the number of inodes
(files), and the number of free inodes and data blocks (layout defined by
\verb+struct filesys+ 
\w Block $2 \sim k$ (\bb{inodes}): each inode is defined by \verb+struct inode+
\w Block $k+1 \sim $ (\bb{data blocks})
\eit

\paragraph{Inode}


\end{document}
