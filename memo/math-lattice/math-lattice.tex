\documentclass{myproc}
\usepackage{mydef,myenv,amssymb}
%\usepackage{MinionPro}
\usepackage{mathptm}
\usepackage[all]{xy}
\def\EE{\mbox{\eufm{}E}}
\def\sbf{\bfseries}

\begin{document}
\small
\pagestyle{empty}

\noindent{\large\bf Notes on Lattices and Order}

\section{Ordered Sets}
\subsection{Ordered sets}
\bit
\w Let $P$ be a set. An \bb{order} (or \bb{partial order}) on $P$ is a binary
relation $\le$ on $P$ such that, for all $x, y, z \in P$,
 \ben
 \w [(a)] $x \le x$ \hfill (reflexivity)
 \w [(b)] $x \le y$ and $y \le x$ imply $x = y$ \hfill (antisymmetry)
 \w [(c)] $x \le y$ and $y \le z$ imply $x \le z$ \hfill (transitivity)
 \een
\w A set $P$ equipped with an order relation $\le$ is said to be an 
 \bb{ordered set} (or \bb{partially ordered set} or \bb{poset}).
\w On any set, $=$ is an order, the \bb{discrete order}.
\w A relation $\le$ on a set $P$ which is reflexive and transitive but not
necessarily antisymmetric is called a \bb{quasi-order} (or \bb{pre-order}).
\w A subset $Q$ of an ordered set $P$ inherits an order relation from $P$,
which we say that $Q$ has the \bb{induced order}.
\eit
\subsection{Chains and antichains}
\bit
\w An ordered set $P$ is a \bb{chain} (or \bb{linearly ordered set} or
\bb{totally ordered set}) if, for all $x, y \in P$, either $x \le
y$ or $y \le x$.
\w The ordered set $P$ is an \bb{antichain} if $x \le y$ in $P$ only if
 $x = y$.
\eit
\subsection{Order isomorphisms}
\bit
\w Two ordered set $P$ and $Q$ are \bb{order-isomorphic} (or \bb{isomorphic}),
written $P \cong Q$, if there exists a map $\varphi$ from $P$ onto $Q$
such that 
 \[ x \le y \mbox{\ in\ } P \ \Leftrightarrow\ \varphi(x) \le \varphi(y)
 \mbox{\ in\ } Q.\]
\w $\varphi: P \rightarrow Q$ is necessarily bijective and there exists a
well-defined inverse $\varphi^{-1}: Q \rightarrow P$.
\eit

\subsection{The covering relation}
\bit
\w Let $P$ be an ordered set and let $x, y \in P$. We say $x$ is \bb{covered
  by} $y$ (or $y$ \bb{covers} $x$), and write $x \prec y$, if $x < y$ and $x
\le z < y$ implies $z = x$.
\w Using the covering relation, we can draw the \bb{Hasse diagram} for an
  ordered set, which is a {\em canonical\/} graphical representation of the set.
\w {\sbf Lemma}: Let $P$ and $Q$ be finite ordered sets and let $\varphi: P
\rightarrow Q$ be a bijective map. Then the following are equivalent:
 \ben
 \w [(a)] $\varphi$ is an order-isomorphism.
 \w [(b)] $x < y$ in $P$ if and only if $\varphi(x) < \varphi(y)$ in $Q$.
 \w [(c)] $x \prec y$ in $P$ if and only if 
      $\varphi(x) \prec \varphi(y)$ in $Q$.
 \een
\eit
\subsection{The dual of an ordered set}
\bit
\w Given an ordered set $P$ we can form a new ordered set $P^{\partial}$ (the
\bb{dual} of $P$) by defining $x \le y$ to hold in $P^\partial$ if and only if
$y \le x$ in $P$.
\w An ordered set $P$ has a bottom element if there exists $\bot \in P$
(called \bb{bottom}) with the property that $\bot \le x$ for all $x \in P$.
\w An ordered set $P$ has a top element if there exists $\top \in P$
(called \bb{top}) with the property that $x \le \top$ for all $x \in P$.
\w A finite chain always have top and bootom elements.
\w Top and bottom elements are unique when they exists.
\w {\sbf The duality principle}: Given a statement $\Phi$ about ordered sets
which is true in all ordered sets, the dual statement $\Phi^\partial$ 
is also true in
all ordered sets.
\eit
\subsection{Lifting}
\bit
\w When an ordered set $P$ (with or without a bottom element), 
  we can form $P_{\bot}$ with a bottom $\bot$ by \bb{lifting} $P$, where
  lifting is done as follows:
  take an element \bb{0} $\in P$ and define $\le$ on $P_\bot \defeq P \cup
  \{\mbox{\bb{0}}\}$ by
 \[ x \le y \mbox{\ in\ } P_\bot
  \mbox{\ \ if and only if\ \ } x = \mbox{\ \bb{0}  or\ }
  x \le y \mbox{\ in\ } P.\]
\w A \bb{flat} ordered set can be constructed from any {\em set\/} $S$ 
  as follows: get an antichain $\overline{S}$, and then form
  $\overline{S}_\bot$. 
\eit
\subsection{Maximal and minimal elements}
\bit
\w Let $P$ be an ordered set and let $Q \subseteq P$. Then $a \in Q$ is a
\bb{maximal element} of $Q$ if 
   \[ (\mbox{$a \le x$ and $x \in Q$})\ \Rightarrow\  a = x. \]
\w We denote the set of maximal elements of $Q$ by max$Q$.
\w If $Q$ has a top elment, $\top_Q$, then max$Q$ = $\{\top_Q\}$; in this case
$\top_Q$ is called the \bb{greatest element} (or \bb{maximum}) 
of $Q$ and we write $\top_Q =$ max$Q$. 
\w A \bb{minimal element} of $Q \subseteq P$ and min$Q$, and the \bb{least
  element} (or \bb{minimum}) of $Q$ (when these exists) are defined dually.
\eit

\subsection{Sum of ordered sets}
\bit
\w Let $P$ and $Q$ be disjoint ordered sets. The \bb{disjoint union} $P \cup
Q$ of $P$ and $Q$ is the ordered set formed by defining $x \le y$ in $P \cup
Q$ if and only if either $x, y \in P$ and $x \le y$ in $P$ or $x, y \in Q$ and
$x \in y$ in $Q$.
\w Let $P$ and $Q$ be disjoint ordered sets. The \bb{linear sum} $P \oplus Q$
is defined by taking the following order relation on $P \cup Q$: $x \le y$ if
and only if
  \ben
  \w [(a)] $x, y \in P$ and $x \in y$ in $P$, or
  \w [(b)] $x, y \in Q$ and $x \in y$ in $Q$, or
  \w [(c)] $x \in P$ and $y \in Q$.
  \een
\w The lifting construction is a special case of linear sum: $P_\bot$ is just
\mbox{1} $\oplus\ P$.
\eit

\subsection{Products}
\bit
\w Let $P_1, \cdots, P_n$ be ordered sets. The Cartesian product $P_1 \times
\cdots \times P_n$ can be made into an ordered set by imposing the
coordinatewise order defined by
\[ (x_1, \cdots, x_n) \le (y_1, \cdots, y_n) \ \Leftrightarrow\ 
(\forall{i})x_i \le y_i \mbox{\ in\ } P_i.\]
\w $P^n$ is a shorthand for an $n$-fold product $P \times \cdots \times P$.
\w Let $P$ and $Q$ be ordered sets. A \bb{lexicographic order} is defined by
 $(x_1, x_2) \le (y_1, y_2)$ if $x_1 < x_2$ or
  $(x_1 = y_1 \mbox{\ and\ } x_2 \le y_2)$.
\w {\sbf Proposition}: Let $X = \{1, 2, \cdots, n\}$ and define $\varphi:
   {\cal{}P}(X) \rightarrow \mbox{2}^n$ by 
$\varphi(A) = (\epsilon_1, \cdots, \epsilon_n)$
  where
  \[ \epsilon_1 = \left\{\begin{array}{ll}
  1 & \mbox{if\ } i \in A,\\
  0 & \mbox{if\ } i \not\in A,
			 \end{array}\right.
  \]
 Then $\varphi$ is an order-isomorphism.
\eit

\subsection{Down-sets and up-sets}
\bit
\w Let $P$ be an ordered set and $Q \subseteq P$.
  \ben
  \w $Q$ is a \bb{down-set} (or \bb{decreasing set} or \bb{order ideal})
  if, whenever $x \in Q, y \in P$ and $y \le x$, we have $y \in Q$.
  \w Dually, $Q$ is an \bb{up-set} (or \bb{increasing set} or \bb{order
    filter}) if, whenever $x \in Q, y \in P$ and $y \ge x$, we have $y \in Q$.
  \een
\w Loosely speaking, a down-set is one which is closed under going down and
  up-set is one which is closed under going up.
\w Given an arbitrary subset $Q$ of $P$ and $x \in P$, we define
  \[\downarrow Q \defeq \{y \in P: (\exists x \in Q) y \le x\}\]
  and
  \[ \downarrow x \defeq \{y \in P: y \le x\}.\]
  $\uparrow Q$ and $\uparrow x$ is defined similarly.
\w $Q$ is a down-set if and only if $Q = \downarrow Q$.
\w Down-sets (up-sets) of the form $\downarrow x$ ($\uparrow x$) are called
\bb{principal}. 
\w The  family of all down-sets of $P$
  is denoted by ${\cal O}(P)$. 
\w ${\cal O}(P)$ itself is an ordered set, under the inclusion order.
\w When $P$ is finite, every non-empty down-set $Q$ of $P$ is expressible in
the form $\bigcup_{i=1}^k \downarrow x_i$.
% (where $\{x_1, \cdots, x_k\} = 
%\mbox{\ max}Q$ is an antichain).
\w {\sbf Lemma}: Let $P$ be an ordered set and $x, y \in P$. Then the
following are equivalent.
 \ben
 \w [(a)] $x \le y$.
 \w [(b)] $\downarrow x\ \subseteq\ \downarrow y$.
 \w [(c)] $(\forall Q \in {\cal O}(P)) y \in Q \ \Rightarrow\ x \in Q$.
 \een
\w Besides being related by duality, down-sets and up-sets are related by
complementation: $Q$ is a down-set of $P$ if and only if $P - Q$ is an up-set
of $P$ (equivalently, a down-set of $P^\partial$).
\w {\sbf Proposition}: Let $P$ be an ordered set. Then
 \ben
 \w [(a)] ${\cal O}(P \oplus \mbox{{1}}) \cong
      {\cal O}(P) \oplus \mbox{{1}}$\ and\
      ${\cal O}(\mbox{{1}} \oplus P) \cong
      \mbox{{1}} \oplus {\cal O}(P)$.
 \w [(b)] ${\cal O}(P_1 \cup P_2) \cong {\cal O}(P_1) \times {\cal O}(P_2)$.
 \een
\eit

\subsection{Maps between ordered sets}
\bit
\w Let $P$ and $Q$ be ordered sets. A map $\varphi: P \rightarrow Q$ is said
to be 
 \ben
 \w [(a)] \bb{order-preserving} (or \bb{monotone}) if $x \le y$ in $P$ implies
 $\varphi(x) \le \varphi(y)$ in $Q$.
 \w [(b)] an \bb{order-embedding} (written as $\varphi: P \hookrightarrow Q$)
 if $x \le y$ in $P$ if and only if $\varphi(x) \le \varphi(y)$ in $Q$.
 \w [(c)] an \bb{order-isomorphism} if it is an order-embedding which maps $P$
 onto $Q$.
 \een
\eit


\section{Lattices and Complete Lattices}
\subsection{Lattices as ordered sets}
\bit
\w Let $P$ be an ordered set and let $S \subseteq P$. An element $x \in P$ is
an \bb{upper bound} of $S$ if $s \le x$ for all $s \in S$.
A \bb{lower bound} is defined dually.
\w The set of all upper bounds of $S$ is denoted by $S^u$ (read as `S upper')
and the set of all lower bounds by $S^l$ (read as `S lower').
\eit

\bibliographystyle{plain}
\bibliography{bib/mac,bib/math,bib/algo}
\nocite{DP02,Birkhoff67}
\end{document}


% LocalWords:  Cheoljoo Jeong vertices endvertices endvertex cutvertex bc algo
% LocalWords:  cutvertices
