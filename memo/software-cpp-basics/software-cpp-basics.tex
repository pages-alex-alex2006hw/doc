\documentclass{memo}
\usepackage{mathptm,mydef,myenv}
%\usepackage{MinionPro}
\usepackage[all]{xy}
\begin{document}
\noindent{\large\bf{}C++ Basics}
\small

\paragraph{Names}
\bit
\w a \bb{name} denotes an object, a (set of) function, an
enumerator, a type, a class memory, a template, a value, or a label
\w \bb{name} is introduced by a \bb{declaration}
\w \bb{name} is associated with a ``region of program text'' called its
\bb{scope}, where the name is usable
\w \bb{name} has a type which determines its use
\eit

\paragraph{Objects}
\bit
\w an \bb{object} is a {\em region of storage}
\w a \bb{named object} has a storage class (automatic, static, ...) that
determines its lifetime
\w \bb{meaning of values} found in an object is determined by the type of the
expression used to access it
\eit

\paragraph{Declarations and definitions}
\bit
\w  a \bb{declaration} introduces one or more names into a program.
\w a declaration is a \bb{definition} in following cases:
  \begin{verbatim}
  int a;
  extern const c = 1;
  int f(int x) { return x+a; }
  struct S { int a; int b; };
  enum { up, down };
  \end{verbatim}
\w definition must be unique while ``re''-declarations are allowed
\eit

\paragraph{Scopes}
\bit
\w there are four kinds of scopes: local, function, file, class
\w \bb{local}: a name declared in a block is local to the block
\w \bb{function}: labels can be used anywhere in the function in which they
are declared; only labels have function scope
\w \bb{file}: a named declared outside of blocks/classes has file scope; names
declared with file scope are said to be \bb{global}
\w \bb{class}: name of a class member is local to its class and can be used
only in a member function of that class
\eit

\paragraph{Program and linkage}
\bit
\w a \bb{program} consists of one or more \bb{files} linked together
\w a \bb{file} consists of sequence of (variable and function) declarations
(\bb{note that (function) definition is always a (function) declaration})
\w \bb{internal linkage}:
  \bit
  \w a name with ``file scope'' which is {\tt static} (or {\tt inline}) is
  local to its translation unit (file) and have \bb{internal linkage} 
  \w every other names are considered to have \bb{external linkage}
  \eit
\w \bb{external linkage}
\eit


\paragraph{Lvalues}
\bit
\w an \bb{object} is a region of \bb{storage}
\w an \bb{lvalue} is an ``expression'' that refers to an object (a region of
storage) or a function
\w \bb{obvious example}: ``name of an object'' is an lvalue
\w some \bb{operators} yield lvalues; e.g. given that \verb+E+ is an
expression of pointer type, \verb+*E+ is an \bb{lvalue expression} referring
to the object to which \verb+E+ points
\w \bb{why it's called lvalue?}: In
  \[ \txt{\tt E1 = E2;} \]
\verb+E1+ must be a lvalue expression.
\w lvalue is \bb{modifiable}| if it's not a functino name, array name or const
  
\eit

\paragraph{Example class: {\tt Expr}}
\bit
\w \bb{class definition}:
\begin{verbatim}
  class Expr {
    /* constructor */
    Expr() { ... }

    /* destructor */
    virtual ~Expr();

    /* copy constructor */
    Expr(const Expr& rhs) {
      field = rhs.field();
    }

    /* copy assignment operator */
    Expr& operator=(const Expr& rhs) {
      if (this == &rhs) /* check self-asgn */
        return *this;
      field = rhs.field();
      return *this;
    }

    /* constant member function */
    int foo() const { 
      /* no object state change */ 
    }
  };
\end{verbatim}
\w \bb{object initialization}:
\begin{verbatim}
   Expr e1;                   /* constructor */
   Expr e2(e1);          /* copy constructor */
   Expr e3 = e1; /* copy assignment opreator */
\end{verbatim}
\eit

\paragraph{Uses of constructors}
\bit
\w \bb{Initialization} (giving objects its first value) of objects generated
by {\em structs} and {\em classes} is performed by
\bb{constructors}. 
\w \bb{default constructor}: one that can be called with without any
arguments:
  \begin{verbatim}
  class A {
    A();
  }
  class B {
    explicit B(int x = 0);
  }
  \end{verbatim}
\w there are multiple types of constructors:
\w \bb{copy constructor}: used to \bb{initialize and construct} an object with
a different object of the same type
\w \bb{copy assignment operator}:  used to \bb{copy} the value from one object
to another of the same type
\eit

\paragraph{Operator overloading}
\begin{verbatim}
  class Expr {
    Expr* operator&();
    Expr operator++(int);
  };
\end{verbatim}

\paragraph{Functors}
Functor is a special object which acts as a function.
  \begin{verbatim}
  struct add_x {
    int x;
    add_x(int x) : x(x) {}
    int operator()(int y) { return x + y; }
  };

  add_x add42(42);
  int i = add42(8);  // returns 42 + 8
  \end{verbatim}


\paragraph{Overloaded functions}
\bit
\w Two functions can have \bb{same function name} with different signatures. 
\eit

\paragraph{Function pointers}
\bit
\w \bb{function that takes a function pointer as an argument}
  \begin{verbatim}
  int foo(int x, int (*funarg)(int, int));
  \end{verbatim}

\w \bb{using to funptr vars}
  \begin{verbatim}
  int add(int a, int b) { return a + b; }
  int (*sum)(int, int) = add;
  ... foo(10, add) ...
  \end{verbatim}
\eit

\paragraph{Structure definition}
  \begin{verbatim}
  struct tree_t { 
    int value;
    struct tree_t *left;
    struct tree_t *right;
  };
  typedef struct tree_t bintree_t, *bintree_p;
  \end{verbatim}

\paragraph{Virtual functions}
\bit
\w \bb{virtual  function}: runtime automatically invokes the proper
member function when it is overridden by a derived class
\w \bb{pure virtual function}: \verb+virtual void foo() = 0+; derived class
   {\em must\/} define the function.
\w \bb{virtual destructor}: always make classes with virtual functions contain
virtual destructor; this will ensure that correct destructor will be invoked
  \begin{verbatim}
  class Base { 
    ~Base throw(); /* non-virtual */
  };
  class Derived : public Base {
  };
  void wrongFunc(Base *b) {
    /* only fields related to base is removed */
    delete base;
  }
  ...
    Base *base = new Derived();
    wrongFunc(base);
  ...
  \end{verbatim}
\eit

\paragraph{Use consts}
\begin{verbatim}
  /* READ BACKWARDS! */
  /* p is a constant pointer to constant char */
  char greeting[] = "Hello";
  const char * const p = greeting;

  /* does not modify the object */
  char& Stream::getChar() const;
\end{verbatim}

\paragraph{References vs pointers}
\bit
\w reference must be initialized when it's created
\w once a reference is initialized to an object, it cannot be changed to
refer to another object
\w there is no ``NULL'' reference
\w (-) pointer arithmetic not possible
\w (+) no dereference needed
\eit

\paragraph{Argument passing: use pass-by-const-reference}
\bit
\w in C where only call-by-value was available, we needed to
pass-by-''pointer'' 
\w now, call-by-reference in C++ is just as efficient (no copy-in, copy-out as
in call-by-value, which involve constructor/destructor call)s and it's safer
\eit

\paragraph{Template specialization}
  \begin{verbatim}
  template<typename T>
  class A {
    T element;
    foo(T arg) { T.inc(); }
  }
  /* template specialization */
  template<>
  class A <int> {
    int element;
    foo(int arg) { arg++; }
  }
  \end{verbatim}

\paragraph{Type casting}
\bit
\w \bb{dynamic\_cast}: between pointers/references to objects; successfully
only casted to its base type (\bb{upcast}); {\em runtime-checking\/}
\w \bb{static\_cast}: between (related) pointer types  (can be used for \bb{downcast})
\w \bb{reinterpret\_cast}: between any (possibly unrelated) pointer types
\eit

\paragraph{Smart pointers: auto\_ptr}
\bit
\w deprecated; use \bb{unique\_ptr} instead
\w \verb+#include <memory>+
\begin{verbatim}
  template <class Y>
  struct auto_ptr_ref {};

  template<class X>
  class auto_ptr {
  public:
    typedef X element_type;
    explicit auto_ptr(X* p = 0);
             auto_ptr(auto_ptr&);
    template<class Y> auto_ptr(auto_ptr<Y>&);

    auto_ptr& operator=(auto_ptr&);
    template <class Y> auto_ptr& operator=(auto_ptr<Y>&);
  };
\end{verbatim}
\w \bb{usage}:
  \begin{verbatim}
  void f() {
    auto_ptr<int> pt(new int);
    /* get pointer */
    ... pt.get() ... 
  }
  \end{verbatim}
\eit


\end{document}

%%  LocalWords:  Expr structs Functors funptr runtime destructor consts const
%%  LocalWords:  dereference upcast ptr Functor
