\documentclass{memo}
\usepackage{mathptm,mydef,myenv}
\usepackage[all]{xy}
%\usepackage{MinionPro}
\begin{document}
\small
\noindent{\large\bf{}Control Flow Analysis}

\paragraph{Overview} Control flow analysis discovers the hierarchical flow of
control within each procedure. Typically {\em flowgraphs\/} are used for
the static representation of procedures. There are two approaches to control
flow analysis: one based on {\em dominators\/} and the other based on {\em
  interval analysis\/}. We focus on dominator-based analysis here.

\paragraph{Dominator-based analysis} Dominator-based control flow analysis is
used together with \bb{iterative dataflow analysis} later. Here are the steps:
\ben
\w [1)] identify {\em basic blocks}, 
\w [2)] build a {\em flowgraph} of the basic blocks, and 
\w [3)] build {\em dominance tree} to identify {\em loops}.
\een

%% \paragraph{Interval analysis-based control flow analysis} Interval-based
%% method is used later with a dataflow analysis method called \bb{elimination
%%   methods}. 
%%   \ben
%%   \w [(a)] identify basic blocks
%%   \w [(b)] build flowgraph
%%   \w [(c)] decompose the code into \bb{nested regions} called \bb{intervals}
%%        \bit
%%        \w nesting structure induces a \bb{control tree}
%%        \eit
%%   \een


\paragraph{Basic blocks}
A \bb{basic block} is a {\em maximal straight-line sequence of code\/}
   which can be entered only at the top and exited only at the bottom. There
   is no branchings (jumps) into or out of the `middle' of the basic block.

Basic blocks are identified using the notion of \bb{leaders}, i.e. the first
instructions in some basic block. The following are the leaders:
\bit
\w any first instruction in the code is a leader.
\w any instruction that is the target of a {\tt jump} is a leader.
\w any instruction that follows a {\tt jump} is a leader.
\eit
Then, for each leader, its basic block consists of {\em itself and all
  instructions up to but not including the next leader or the end of the
  program\/}. 

The {\tt call} instruction may or may not be considered as a basic block
boundary. Usually we don't need to consider it as a boundary (and this allows
larger basic blocks). However, there are cases when we need to. For example, 
Fortran alternate returns necessitates calls to be considered as
   boundaries and  C \verb+setjmp()+ and \verb+longjmp()+ also necessitates it.


%% An \bb{extended basic block} is maximal sequence of instructions beginning with
%% a leader that contains no join nodes other than its first node (which need not
%% be a join node itself if, e.g. it is the entry node)
%%    \bit
%%    \w has a single entry and possibly multiple exists
%%    \w can be though of as a tree with its entry basic block as the root
%%    \eit

\paragraph{Flowgraphs} A flowgraph is a directed graph $G = (V \cup
\{s, t\}, E)$, where $V$ is a set of basic blocks and $E$ is a set of all
directed edges between basic blocks. $\arc{u, v} \in E$ iff basic block $v$
{\em can\/} immediately follow $u$ in an execution. 
Two special nodes, $s$ and $t$ are entry
and exit blocks added for the convenience in later dataflow analysis.
A \bb{branching node} is one with outdegree $>1$ and a \bb{join node} is one
with indegree $> 1$. 

A {\em strongly connected subgraph\/} of a flowgraph is called a \bb{region}. 


\paragraph{Dominators and postdominators}
To determine the loops in a flowgraph, we first define a binary relation
called {\bf dominance\/} on flowgraph nodes.

A node $u$ \bb{dominates} node $v$, denoted by $u \mbox{\ dom\ } v$, if every
possible execution path from the entry node to $v$ includes $u$. 
Node $u$ is called a \bb{dominator} of node $v$. 
Node $u$ is the \bb{immediate dominator}, denoted by idom$(v)$, of node
$v$ if there is no other dominator of $v$ inside the $u$--$v$ path. 

\[\xymatrix@-0.2pc{
  \mbox{$u$ dom $v$}: &  s \ar@{~>}[r] & *+[F]{u} \ar@/^2ex/[r] \ar@/_2ex/[r] & v
  \ar@{~>}[r] & t
}
\]

Also, a node $u$ \bb{postdominates} node $v$, denoted by $u \mbox{\ pdom\ }
v$, if every possible path from $v$ to the exit node $t$ includes $u$. 

\[\xymatrix@-0.2pc{
  \mbox{$u$ pdom $v$}: & s \ar@{~>}[r] & v \ar@/^2ex/[r] \ar@/_2ex/[r] & *+[F]{u} \ar@{~>}[r] & t
}
\]

\paragraph{Strict dominance and dominance frontier}
A node $u$ \bb{strictly dominates} $v$ if $u$ dom $v$ and $u \ne v$.  Node $u$
is an \bb{ancestor} of $v$ if there is a $u$--$v$ path of dominator tree
edges. 

The \bb{dominance frontier} of a node $u$, DF($u$), is the set of all nodes
$\{v\}$ such that $u$ dominates a predecessor of $v$ but $u$ does not
strictly dominates $v$. 
That is, DF($u$) is the set of nodes where $u$'s dominance stops.
\[ \xymatrix@-0.5pc{
                 & *+[F]{u} \ar[r] & w_1 \ar[rd] &   \\
  s \ar@{~>}[ru] \ar@{~>}[r] & w_2 \ar[rr] & & \underline{v} \ar@{~>}[r] & t 
}
\]

\paragraph{Finding dominators}
The dominators of a node $v \in V$ are given by the maximal solution to the
following equations:
\begin{eqnarray*}
  \mbox{dom}(s) & = & \{s\}, \\
  \mbox{dom}(v) & = & \bigcup_{u \in \Gamma^{-1}(v)} dom(u) \cup \{v\}
\end{eqnarray*}


\paragraph{Dominator tree}
A {\em dominator tree\/} of a flow graph $G = (V, E)$ is a subgraph $T = (V,
E')$ of $G$, where 
\[ E = \{\arc{u, v}: \mbox{ $u$ idom $v$} \}. \]
Note that the resulting subgraph is a tree since each node has exactly one
immediate dominator except for the entry node.
Dominator tree can be computed using Lengauer-Tarjan algorithm.

\paragraph{Depth-first traversal of dominator trees}
A depth-first traversal of dominator trees paritions the set of edges into one
of four catetories: {\em tree edges\/}, {\em back edges\/}, {\em forward
  edges\/}, and {\em cross edges\/}.

A \bb{back edge} in a flowgraph is one where {\em the head dominates its
tail\/}. Back edges are critial in the identification of {\em natural
  loops\/}. 


%     \bit
%     \w \bb{loop} consists of all nodes dominated by a loop entry node with
%     all related edges
%     \eit

\paragraph{Loops in flowgraphs}
A \bb{loop} in a flowgraph is a set $L$ of nodes including a
\bb{header node} $h$ with the following properties:
\bit
\w From any node in $L$, there is a directed path leading to $h$.
\w There is a directed path from $h$ to any node in $L$.
\w There is no edge from any node outside $L$ to any node in $L$ other than
$h$. 
\eit

Given a back edge $\arc{v, h}$, the \bb{natural loop} of the back edge
$\arc{v, h}$ is the subgraph consisting of 1) the node $h$ (known as \bb{loop
header}) and 2) the set of nodes $\{u\}$ dominated by $h$ where there exists a
$u$--$v$ path not passing through $h$. The header of this loop will be $h$. 
Simply, a natural loop is a loop with a {\em single loop header\/}. 



\paragraph{Reducible flowgraph}
A flowgraph $G = (V, E)$ is \bb{reducible} or \bb{well-structured} iff 
$E$ can be partitioned into disjoint sets, $E_F$ of {\em forward edges\/} and
$E_B$ of {\em back edges\/}, such that $(V, E_F)$ forms a DAG in which each node
can be reached from the entry node, and the edges in $E_B$ are all back edges. 
In other words, a flowgraph is reducible iff all the loops in it are
{\em natural loops\/} which can be characterized by back edges. 
In reducible flowgraphs, any cycle of nodes have a unique header node. 

\bb{Irreducible flowgraphs} are graphs which has ``cyclic core'' 
as shown below, after node collapsing and edge deletions. Note that in the 
cycle, neither node 2 or node 3 dominates the other (the edges between nodes 2
and 3 are cross edges not back edges). 
Such graphs complicate the control flow analysis.
\[ \xymatrix@-0.5cm{
  & \ar[d] & \\
  &*++[o][F]{1} \ar[rd]\ar[ld] & \\
 *++[o][F]{2} \ar[d] \ar@/^2.5ex/[rr]& & *++[o][F]{3} \ar@/^2.5ex/[ll]  \\
& &
}
\]

When common control-flow constructs such as \bb{if-then}, \bb{if-then-else},
\bb{while-do}, \bb{repat-until}, \bb{for}, and \bb{break} can only generate
reducible flowgraphs. The use \bb{goto} statements can introduce irreducible
flowgraphs. 


\paragraph{Loop-nest trees}
If $L_1$ and $L_2$ are two loops with headers $h_1$ and $h_2$, respectively,
such that $h_1 \ne h_2$ and $h_1$ dominates $h_2$, then the nodes of $L_2$ are
a proper subset of the nodes of $L_1$. We say that loop $L_2$ is \bb{nested
  within} $L_1$, or that $L_2$ is a \bb{inner loop}. 

We can construct a \bb{loop-nest} tree of loops in a program. 
\bit
\w Compute dominators of the flowgraph $G = (V, E)$. 
\w Construct the dominator tree $T = (V, E')$.
\w Find all natural loops, and thus all the loop-header nodes.
\w For each loop header $h$, merge all the natural loops of $h$ into a single
loop, loop[$h$]. 
\w Construct the tree of loop headers (and implicitly loops), such that $h_1$
is above $h_2$ in the tree if $h_1$ dominates $h_2$.
\eit
The leaves of the loop-nest tree are the \bb{innermost loops}. 


\paragraph{Loop preheader}
Many loop optimizations will insert statements immediately before the loop
executes. For example, {\em loop-invariant hoisting\/}  moves a statement from
inside the loop to immediately before the loop. 
\bb{Loop preheaders} expediates the insertion of statements {\em before\/} the
loop entry.  For any loop entry nodes which has $n > 1$ indegree, we create
one more node right before the loop entry node. e.g.

\[\xymatrix@-0.5cm{
  *++[o][F]{1} \ar[rd] & &*++[o][F]{2} \ar[ld] & & &  *++[o][F]{1} \ar[rd] & &*++[o][F]{2} \ar[ld] \\ 
   & *++[o][F]{h} & & \ar@{=>}[r]& &&*++[o][F]{P} \ar[d] & \\
   &  && &&&*++[o][F]{h} &&
}\]



\end{document}

% LocalWords:  Inlining inline callees Scalarization structs dominators Fortran
% LocalWords:  flowgraph dominator dataflow subgraph setjmp longjmp WAW RAR iff
% LocalWords:  boolean stmt Unswitching flowgraphs outdegree indegree dom idom
% LocalWords:  postdominators postdominates pdom
