\documentclass{memo}
\usepackage{mathptm,mydef,myenv}
%\usepackage{MinionPro}
\begin{document}
\small
\noindent{\large\bf{}Procedure Call Transformations}

\paragraph{Overview} In most programming languages, procedure is an important
tool for control abstraction. However, this comes with a price of procedure
call overhead. {\em Procedure call transformations\/} try to reduce this
overhead in one of four ways:
\bit
\w eliminate the call entirely
\w eliminate the exeuction of the called function's body
\w eliminate some of the entry/exit overhead
\w avoiding some steps in making a procedure call when the behavior of the
called procedure is known or can be altered
\eit

\paragraph{Calling covnention: Register usage}
A calling convention to be used in this document is defined. 
\bit
\w \verb+R0+: always zero, writes are ignored
\w \verb+R1+: \bb{return value} when returning from a procedure call
\w \verb+R2+$\cdots$\verb+R7+: first 6 arguments to the procedure call
\w \verb+R8+$\cdots$\verb+R15+: 8 {\bf caller-save registers\/}, used as
temporary registers by callee
\w \verb+R16+$\cdots$\verb+R25+: 10 {\bf callee-save registers\/}; these
registers must be preserved across a call by callee 
\w \verb+R26+$\cdots$\verb+R29+: reserved for use by the operating system
\w \verb+SP+: \bb{stack pointer}
\w \verb+R31+: \bb{return address} during a procedure call
\w \verb+F0+$\cdots$\verb+F3+: first 4 FP argumments to procedure call
\w \verb+F4+$\cdots$\verb+F17+: 14 caller-save FP registers
\w \verb+F18+$\cdots$\verb+F31+: 14 callee-save FP registers
\eit

Each called procedure must ensure that the values in registers
\verb+R16+$\cdots$\verb+R25+ is preserved. So, if the callee needs to use any
of these registers, they need to restore back the original values after use. 

The \bb{stack} begins at the top of memory and grows downwards. There is no
explicit frame pointer but the stack pointer \verb+SP+ is decremented by the
size $s$ of the procedure's frame at entry and left unchanged during the call.
The value (\verb+SP+ $+ s$) serves as a {\em virtual frame pointer} that
points to the base of the stack frame, avoiding the use of a second dedicated
register. 

\paragraph{Calling convention: Procedure execution}
Execution of a procedure consists of six steps:
\ben
\w Space is allocated on the stack for the procedure invocation.
\w The values of registers that will be modified during procedure execution
(and which must be preserved across the call) are saved on the stack.
If the procedure makes any procedure calls itself, the saved registerrs should
include the \bb{return address}, \verb+R31+. 
\w The procedure body is executed.
\w The return value (if any) is stroed in \verb+R1+, and the registers that
were saved in step 2 are restored.
\w The frame is removed from the stack.
\w Control is transferred to the return address.
\een

\paragraph{Calling convention: Procedure call}
Calling a procedure consists of four steps:
\ben
\w The values of any of the registers \verb+R1+--\verb+R15+ that contain live
values are saved. If the values of any global variables that might be used by
the callee are in a register and have been modified, the copy of those
variables in memory is updated.
\w The arguments are stored in the designated registers and, if necessary, on
the stack.
\w A linked jump is made to the target procedure; the CPU leaves the address
of the next instruction in \verb+R31+.
\w Upon return, the {\em saved registers\/} are restored, and the registers
holding {\em global variables\/} are reloaded.
\een

\paragraph{Stack frame layout}
The following is a stack frame layout.

\vspace*{0.2cm}

\centerline{\begin{tabular}{c|c|} 
& \\ 
&caller's stack frame \\ \cline{2-2}
&arg n \\ 
&$\vdots$ \\
{\tt{}FP} & arg 1 \\ \cline{2-2}
& locals and temporaries \\ \cline{2-2}
& saved registers \\  \cline{2-2}
{\tt{}SP} & arguments for called procedures \\ \cline{2-2}
& \\ 
& \\ 
  \end{tabular}}


\paragraph{Leaf Procedure Optimziation}
A {\em leaf procedure} is one that does not call any other procedures. 
This is a leaf in the call graph. The simplest optimization for leaf
procedures is that {\em they do not need to save and restore the return
  address\/} (\verb+R31+). Also, if the procedure does not have any local
variables allocated to memory, the compiler does not need to create a stack
frame. 

\paragraph{Cross-call register allocation}
Separate compilation reduces the amount of information available to the
compiler about called functions. However, when both callee and caller are
available, the compiler can take advantage of the register usage of the callee
to optimize the call.

If the callee does not need (or can be restricted not to use) all the
temporary variables (R8$\cdots$R15), the caller can leave values in the unused
registers throughout execution of the callee. In addition, moving instructions
for parameters can be eliminated.


For this optimization, {\bf register allocation\/} must be performed in a
{\em depth-first postorder traversal\/} of the call graph, ensuring that each
caller will know its callees' register usage.

\paragraph{Parameter promotion}
When a parameter is {\em passed by reference\/}, the address calculation is
done by the caller, but the load of the parameter value is done by the
callee.  This wastes an instruction, since most address calculations can be
handled with the \verb+offset(+R$n$\verb+)+ format of load instruction. 

More importantly, if the operand is already in a register in the caller, it
must be spilled to memory and reloaded by the callee. If the callee modifies
the value, it must then be stored. Upon return to the caller, if the compiler
can not prove that the callee did not modify the operand, it must be loaded
again. Thus, as many as two unnecessary lods and two unnecessary stores can be
introduced.

An unmodified reference parameter can be passed by value, and a modified
reference parameter can be passed by value-result.  


\paragraph{Frame collapsing}

\paragraph{Procedure inlining}

\paragraph{Tail recursion elimination}

\paragraph{Function memoization}


\end{document}

% LocalWords:  Inlining inline callees Scalarization structs dominators Fortran
% LocalWords:  flowgraph dominator dataflow subgraph setjmp longjmp WAW RAR iff
% LocalWords:  boolean stmt Unswitching
