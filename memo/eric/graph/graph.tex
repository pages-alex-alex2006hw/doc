\documentclass{myart}
\usepackage{mydef,myenv}
\usepackage{mathptm}


%\def\sbf{\sf\bfseries}
\def\sbf{\bfseries}
\pagestyle{empty}

\begin{document}
%\small

\noindent{\large\bf Notes on Computer Science}


\paragraph{Sets}
\bit
\w A \bb{set} is a collection of distinguishable objects, called its
\bb{members} (or \bb{elements}). 

\w \bb{Set membership} is representation by the symbol $\in$. 
  \bit
  \w $x \in S$: object $x$ is a member of the set $S$
  \w $x \not\in S$: object $x$ is not a member of the set $S$
  \eit
\w Two ways to define a set:
  \bit
  \w \bb{set extension}: define a set by listing all members of the set
    \bit
    \w $S = \{1, 2, 3, 4, 5, 6, 7, 8, 9, 10\}$
    \w $E = \{0, 2, 4, \cdots\}$
    \eit
  \w \bb{set comprehension (definition by intension)}: 
        describe properties that its members must satisfy
    \bit
    \w $S = \{i : 1 \le  i \le 10\}$
    \w $E = \{2n : n \in \N\}$ is the set of even natural numbers, given that 
      $\N$ is the set of {\em natural numbers}.
    \eit
\w Set examples:
   \bit
   \w $\{1, 2, 2\}$ is NOT a set since objects are not
   distinguishable. However, it is called a \bb{multiset} -- multiset is a
   collectio of (possibly redundant) objects.
   \eit

\w \bb{Set inclusion}: 
   Given two sets $A$ and $B$, if $x \in A$ implies $x \in B$, 
   we way $A$ is a \bb{subset} of $B$.
  \bit
  \w This is also denoted by $A \subseteq B$.
  \eit
\w \bb{Set equivalence}: Given two sets $A$ and $B$, if $A \subseteq B$ and $B
\subseteq A$,  
   $A$ and $B$ are \bb{equivalent}. 

\w \bb{Set operations}
  \bit

  \eit
\eit

\paragraph{Graphs}

\end{document}


% LocalWords:  Cheoljoo Jeong vertices endvertices endvertex cutvertex bc algo
% LocalWords:  cutvertices iff uv cocycles coboundary
