\documentclass{note}
\usepackage{mathptm,mydef,myenv}
\usepackage[all]{xy}
\usepackage{graphicx}
%\usepackage{MinionPro}
\begin{document}
\small

\noindent{\Large\bf{}Cadence: IXCOM Internals}

\tableofcontents

\section{SystemVerilog Elaboration}


\section{SystemVerilog Optimization and Transformation}
\subsection{Assignment operator transformation}
\bit
\w  when there is a statement that contains an assignment operator expression,
we hoist the expression to statements; for example, given
  \begin{verbatim}
  a = c++ + b;
  \end{verbatim}
we create the following statements:
  \begin{verbatim}
  t0 = c;     // save the original value of c to t0
  c = c + 1;  // increment c;
  a = t0 + b; // use the saved value to compute a
  \end{verbatim}
\w \bb{complication \#1: dependency issue}: sometimes we need to hoist
non-assignment-operator expressions; given a task call
  \begin{verbatim}
  foo(j, i = j++, ...);
  \end{verbatim}
we should be careful when there are any dependencies between task arguments;
therefore, 
  \begin{verbatim}
  // wrong transformation       // correct transformation
  t0 = j;                       t1 = j;
  j = j + 1;                    t0 = j;
  foo(j/**/, i = t0, ...);      foo(t1, i = t0, ...);
  \end{verbatim}

  \w \bb{complication \#2: inout arguments}: we may also need restore
  temporary values; let another task call be given as below, where the first
  argument is an \bb{inout} argument
  \begin{verbatim}
  bar(b, ++b, c);
  \end{verbatim}
  \begin{verbatim}
  // wrong tranformation        // correct transformation
  t1 = b;                       t1 = b;
  t0 = b;                       t0 = b;
  b = b + 1;                    b = b + 1;
  bar(t1, t0, c);               bar(t1, t0, c);
                                b = t1;
  \end{verbatim}
\eit 

\subsection{Dead code elimination}
\bit
\w two types of DC performed:
   \bit
   \w per-statement-type DC 
   \w remove unused definitions (liveness analysis)
   \eit
\w \bb{DC optimization of DISABLE}: any statement that follws a disable
statement can be deleted as long as the statement is within the disabled scope
  \begin{verbatim}
  begin: blk
    ... 
    disable blk;
    ... // can be removed if disabled block is an ancestor scope
    ... // of the current scope
  end
  \end{verbatim}
  There are a few complications:
  \bit
  \w note that, if the disabled scope is either an external scope or a forward
  scope, this optimization is not applicable
  \w consider the following example:
  \begin{verbatim}
  begin: B1
    begin: B2
      if C
        disable B1;
      else
        disable B2;
    end // B2
  end // B1
  \end{verbatim}
   in this case, after the \bb{if-else} statement, the disabled scope is
   \bb{B1} $\cap$ \bb{B2} = \bb{B2}. To support this, we need a simple dataflow
   analysis. 
  \eit
\eit

\subsection{Loop unrolling}

\subsection{Constant propagation}
\bit
\w \bb{complication \#1: three special statements}: three types of statements complicates the implementation of constant
propagation: \bb{branches}, \bb{loops}, and \bb{scopes}

\w \bb{complication \#2: time-consuming statements}: wait/event/delay acts as
a barrier; since during the suspension of simulation, collected constants may
become obsolete since other parts of the design may update the values of the
corresponding variables
   \bit
   \w can be enhanced if the variable is not updated from outside
   \eit
\w \bb{complication \#3: function arguments}: for some functinos and tasks,
some arguments are not eligible for constant propagation
\w \bb{implementation sketch}: two main procedures
  \bit
  \w \bb{constant collector}: constants can be collected from 1) {\em explicit
    assignments\/} (e.g. \verb+a = 1+) or 2) {\em implicit assignments\/}
  (e.g. in \verb+if (a == 1) begin ... end+), the variable \verb+a+ has value
  1 in the then-branch
  \w \bb{constant propagator}
  \eit
\w \bb{context}: defined as a mapping from variable to constant values
  \bit
  \w implemented using a stack
  \w new context is created when enter a new ``scope''
   \eit
\eit

\subsection{Tristate buffer transformation}
\bit
\w let a process be given:
  \begin{verbatim}
  always @(posedge clk) 
    if (cond)
      q <= 1'bz;
    else
      q <= d;
  \end{verbatim}
  in this case, for hotswapping, we need to store the two values (enable, data)
\w \bb{transformation}:
  \begin{verbatim}
  always @(posedge clk)          always @(t_en, t_d)
    if (cond)                      if (t_en)
      t_en <= 1;                     q = 1'bz;
    else begin                     else
      t_en <= 0;                     q = t_d;
      t_d <= d;
    end
  \end{verbatim}
   
\eit

\subsection{DISABLE transformation}
\begin{verbatim}
  always @(posedge clk) begin
    S1;
    if (cond2 == 1'b1) 
      disable blk;
    S2;
  end  
\end{verbatim}

\begin{verbatim}
  always @(posedge clk) begin
    disable_blk = 0;
    S1;
    if (cond2 == 1'b1) begin
      disable_blk = 1;
    end
    if (!disable_blk) begin
      S2;
    end
  end  
\end{verbatim}


\subsection{Continue/Break statement}
\begin{verbatim}
  for (i = 0; i < 10; i++) begin
    S1;
    if (cond1)
      continue;
    S2;
    if (cond2)
      break;
  end
\end{verbatim}

\begin{verbatim}
  begin: blk1;
    for (i = 0; i < 10; i++) begin: blk2
      S1;
      if (cond1)
        disable blk2;
      S2;
      if (cond2)
        disable blk2;
    end
  end
\end{verbatim}

%% multi-event
\subsection{Multiple event transformation}
\paragraph{Same events}
\begin{verbatim}
  /* two indentical events */
  always @(posedge clk) begin
    S1;
    @(posedge clk);
    S2;
  end

  /* result */
  reg [0:0] state;
  initial state = 0;
  always @(posedge clk) begin
    case (state) begin
      0: begin
           S1;
           state = 1;
         end
      1: begin
           S2;
           state = 0;
         end
    endcase
  end
\end{verbatim}

\paragraph{Different events}
\begin{verbatim}
  /* two non-indentical events */
  always @(posedge clk) begin
    S1;
    @(negedge clk);
    S2;
  end

  /* result */
  reg [0:0] state;

  ixc_edge #(0) PEclk(peclkout, clk);  
  ixc_cap #(0, 0) Cap(capout0, pecklout, en0, reset0, set0, enxp0);

  ixc_edge #(0) PEclk(neclkout, clk);  
  ixc_cap #(0, 0) Cap(capout1, necklout, en1, reset1, set1, enxp1);
  
 
  initial begin
     en0 = 0;
     reset0 = 0;
     en1 = 1;
     reset1 = 0;
     state = 0;
  end
  always @(posedge capout0 or posedge capout1) begin
    case (state) begin
      0: begin
           reset0 = set0;
           S1;
           en1 = set1;
           state = 1;
         end
      1: begin
           reset1 = set1;
           S2;
           en0 = set0;
           state = 0;
         end
    endcase
  end
\end{verbatim}

\paragraph{Branching: same events}
\begin{verbatim}
  always @(posedge clk) begin
    if (cond) begin
      S1;
      @(posedge clk);
      S2;
    end
    S3;
  end

  /* result */
  always @(posedge clk) begin
    case (state) 
      0: begin
           en0 = 1;
           if (a) begin
             S1;
             state = 1;
             en0 = 0;
           end
           if (en0) begin
             S3;
             state = 0;
           end
         end
      1: begin
           S2;
           S3;
           state = 0;
         end
    endcase
  end
\end{verbatim}

\subsection{RTL blocking delay transformation}
Consider the following two DUT modules which contain \#-delays inside
processes. 
\begin{verbatim}
  module dut1;               module dut2;
    always begin               always begin
      #10;                       #20;
      S1;                        S2;
    end                        end
  endmodule                  endmodule
\end{verbatim}
To support the delay inside DUT modules, three components interact.
\bit
\w \bb{iscDelay modules}: transformed so that all \#-delays are
removed
\w \bb{ixc\_time module}: collects information from iscDelay modules and
computes the time until the earliest \#-event in iscDelay modules
\w \bb{runtime and xc\_top\_incl.v}: collects information from a)
\bb{ixc\_time} and \bb{IUS} to compute the time until the earliest next event
(either due to \#-event in iscDelay or dut to TB-side event); when such
computed time has passed, it triggers \bb{eClk}
\eit
Note that above example is a simpliest possible case where 1) each module has
only one \#-delay process and 2) each module has onlye one \#-delay. 

\paragraph{Transformation of iscDelay module}
Each process with RTL delay is transformed into one which wakes up at posedge
eClk. eClk makes posedge transition exactly when any \#-delay-related event
(i.e. simetime advances due to \#-delay) occurs. In the above example, eClk is
triggered at time 10, 20, 30, 40, ... The process in dut1 wakes up at time 10,
20, 30, while the process in dut2 wakes up at time 20, 40, 60, ...
\begin{verbatim}
  module dut1;                  module dut2;
    always @(posedge eClk)        always @(posedge eClk)
      if (TDL2 == DELTA) begin      if (TDL2 == DELTA) begin
        S1;                           S2;
        TDL2 = 10;                    TDL2 = 20;
      end                           end
      else                          else
        TDL2 = TDL2 - DELTA;          TDL2 = TDL2 - DELTA;
    assign TDM1 = TDL2;             assign TDM1 = TDL2;
  endmodule                     endmodule
\end{verbatim}
There are three important variables which are used for handling delays in DUT
modules:
\bit
\w \bb{TDL2}: this is a process-specific variable, which contains {\em the
  time the process has to wait until its \#-delay is consumed}
  \bit
  \w \bb{TDL2 = 10} means that ``my \#-delay will be fully consumed at 10ns
  later'' (or ``I need to wak up 10 ns later and executed S1 (or S2)'')
  \w in case there are multiple processes with \#-delays, we need multiple
  \bb{TDL2} variable
  \eit
\w \bb{DELTA}: this is shared by all iscDelay DUT modules, and contains the
``time until the earliest next TB event (in particular, the event due to full
consumption of \#-delay) over the entire modules''1
   \bit
   \w \bb{DELTA = 10} means that 10 ns has passed since the last \#-event
   \w \bb{DELTA} value is updated by \bb{ixc\_time} and is propagated to each
   iscDelay module (more will be discussed in \bb{ixc\_time} section)
   \eit
\w \bb{TDM2}: this is a module-specific variable, which contains ``the time
the module has to wait until any of its \#-delay processes wake up next''
   \bit
   \w when there is only one \#-delay process in the module, \bb{TDM2 = TDL2}
   \w when there are multiple \#-delay processes in the module, \bb{TDM2 =
     min(TDL2\_1, TDL2\_2, ..., TDL2\_n)}. 
   \eit
\eit
After the transformation, each process with \#-delay performsn the following
tasks.
\ben
\w each process with \#-delay checks if the \#-delay in the given process has
been consumed; this can be checked by comparing \bb{TDL2} value and \bb{DELTA}
value
  \bit
  \w if so, execute S1 (or S2)
  \w otherwise, update \bb{TDL2} with \bb{TDL2 - DELTA} (\bb{DELTA} time has
  passed since the last eClk; so we only need to wait \bb{TDL2 - DELTA} time
  until wakeup)
  \eit
\w update the \bb{TDM1} value which is the time until the next wakeup
  \bit
  \w \bb{TDM1} values from each iscDelay DUT modules will be collected later
  by \bb{ixc\_time} and will be used to compute \bb{DELTA} (more will be
  discussed in \bb{ixc\_time} sectino)
  \eit
\een

\paragraph{ixc\_time.v: Computing DUT-side delays}
The role of \bb{ixc\_time} module is:
  \ben
  \w to compute the time until the next \#-event in iscDelay modules, 
  \w to inform the runtime of the next iscDelay time (computed above), and
  \w to compare the next iscDelay time (computed above) and the next TB time
  and save the minmum of these to \bb{ixc\_time.delta}
    \bit
    \w note that \bb{ixc\_time.delta} will be read by all iscDelay modules
    (and copied into module var \bb{DELTA}
    \eit
  \een
After the transformation of iscDelay modules, each such module will updazte
its \bb{TDM2} variable (the time until any of its \#-delay process will wake
up). The \bb{ixc\_time} module collects all \bb{TDM2} values from all iscDelay
modules, and compute the minimum of these values, denoted by
\bb{nextClkTime}. 

The following code is a simplified versin of \bb{ixc\_time} module:
\begin{verbatim}
  module ixc_time;
    // compute the delay until the next "event"
    //   - minT: time until next iscDelay-event 
    //           (computed in previous eClk tick)
    //   - nextTbTime: time until next event
    //           (computed by xc_top_incl.v)
    assign delta = min(minT, nextTbTime);

    // compute minimum of TDM1 values over all iscDelay modules
    assign minT = min(TDM1_1, TDM1_2, ..., TDM1_n);
 
    // nextClkTime is the next time when #-event will happen in any 
    // of iscDelay modules
    assign nextClkTime = lastClkTime + minT;
  endmodule
\end{verbatim}

\paragraph{xc\_top\_incl.v and runtime}
\begin{verbatim}
  bit eClk; 
  event runSwEclk;
  import "DPI-C" pure function int xcMatchWriteTbTime();
  bit schSwEclk = 0;
  bit eotSwEclk;
  bit [63:0] nextTbTimeIUS = {64{1'b1}};
  assign ixc_time.nextTbTime = nextTbTimeIUS;

  // whenever, ixc_time computes the next iscDelay time, forward this
  // info to the runtime, so it can properly compute the time to next
  // earliest
  always @(ixc_time.nextClkTimePO)
    xcNextClkTime(ixc_time.nextClkTimePO);
  
  xc_sch_eval xc_sch_sw_eclk(EotSwEcl, schSwEclk);
  always @(runSwEclk) begin:swclk
    longint delay;
    while (1) begin 
      schSwEclk = ~schSwEclk;
      @eotSwEclk;

      delay = xcNextEclk();
      if (delay == 64'b0) break;
   
      nextTbTimeIUs = $time + delay;
      #(delay);
      if (cpi_capture_enable)
        void'(xcMatchWriteTbTime());
      eClk = 1;
      eClk <= 0;
    end
  end
\end{verbatim}

\paragraph{S/W run: Interaction between 3 components}
\begin{verbatim}
                 /* xc_top_incl.v */
                 always
                   while (1) begin
                     // schedule at program block
                     delay = xcNextEclk();
                     nextTbTime = $time + delay;
                     #(delay);
                     eClk = 1;
                     eClk <= 0;
                   end


  /* ixc_time.v */                              /* iscDelay dut1 */
  minT = min(dut1.TDM2, dut2.TDM2);             DELTA = ixc_time.delta;
  delta = min(minT, nextTbTime-lastClkTime);    always @(posedge clk)
  nextClkTime = lastClkTime + minT;               if (TDL2 == DELTA)
  nextDutTime = min(nextClkTime, nextTbTime);       S1; TDL2 = 10;
  always @(posedge eClk)                          else
    lastClkTime <= min(nextClkTime,                 TDL2 = TDL2 - DELTA:
                       nextTbTime);             assign TDM2 = TDL2;

 
                                                /* iscDelay dut2 */
                                                DELTA = ixc_time.delta;
                                                always @(posedge eClk)
                                                  if (TDL2 == DELTA)
                                                    S2; TDL2 = 20;
                                                  else
                                                    TDL2 = TDL2 - DELTA;
                                                assign TDM2 = TDL2;
\end{verbatim}
Let a simulation be given as shown in the diagram below:
\begin{verbatim}
               100                   120
   TB time:    *---------------------*-------------------------------->

   Time:       *------------------------------------------------------>
                                              130
   DUT time:   *------------------------------*----------------------->
   (iscDelay)
\end{verbatim}
We assume there was an event (either TB-event or iscDelay-event) at time
100. At time 100, eClk should have been triggered which had caused the
computation of main variables.
  \bit
  \w lastClkTime = 100;
  \w nextTbTime = 100;
  \w nextClkTime = 130;
  \eit
Now, another while loop iteration in the process at \bb{xc\_top\_incl.v} is
about to begin. (i.e. the first statement ``delay = xcNextEclk()'' in the
while loop of \bb{xc\_top\_incl.v} is about to be executed)

An example sequence of interactions between these components are given
below. Leftmost column shows the activity number, and the rest three columns
represent three components.
\begin{quote}
\centerline{\begin{tabular}{|r|p{5.0cm}|p{5.0cm}|p{4.0cm}|} \hline
 &\bb{xc\_top\_incl.v} & \bb{ixc\_time} & \bb{iscDelay module} \\ \hline
   \multicolumn{4}{|c|}{\bb{simtime: 100}} \\ \hline
@1 & delay = xcNextEclk() &  &  \\ \hline
@2 & nextTbTimeIUS = \$time + delay &  & \\ \hline
@3 & \#(delay); eClk generator process suspended & & \\ \hline
@4 & assign ixc\_time.nextTbTime = nextTbTimeIUS & & \\ \hline
@5 & & determine if next event is TB-event or iscDelay event (by comparing
\bb{nextTbTime} and \bb{nextClkTime} & \\ \hline
@6 &  & update \bb{delta} using the \bb{min(nextTbTime, nextClkTime)}; delta
means ``time to next event'' & \\ \hline
@7 & && DELTA = ixc\_time.delta \\ \hline
\multicolumn{4}{|c|}{\bb{simtime: 120}} \\  \hline
@8 & delay consumed; eClk generator process wakes up & & \\  \hline
@9 & eClk = ~eClk & & \\ \hline
@10 & & & @(eClk) if TDL2 delay consumed; executed S1; update \bb{TDL2} \\  \hline
@11 & & & TDM2 = min\{TDL2\} \\ \hline
@12 & & minT = min\{dut.TDM2\} & \\  \hline
@13 & & nextClkTime = lastClkTime + minT & \\ \hline
@14 & & @(eClk) lastClkTime $\leftarrow$ min(nextTbTime, nextClkTime) & \\ \hline
@15 & xcNextClkTime(ixc\_time.nextClkTime) & & \\ \hline
\end{tabular}}|
\end{quote}

\paragraph{S/W-side optimization}
\begin{verbatim}
  /* original */
  always begin
    #10;
    S1;
    #20;
    S2;
  end

  /* HW process */
  always @(posedge eClk) begin
    if (TDL2 == DELTA) begin
      case (state)
        0: begin
             S1;
             state = 1;
             TDL2 = 20;
           end
        1: begin
             S2;
             STATE = 0;
             TDL2 = 10;
           end
       endcase
    end
    else
      TDL2 = TDL2 - DELTA;

  /* SW process */
  always begin
    #(TD);
    if (!xc_top.hwOutInit) begin /* S/W run */
      case (state) begin
        0: begin
             S1;
             state = 1;
             TD = 20;
           end
        1: begin
             S2;
             state = 0;
             TD = 10;
           end
      endcase
    end
    else begin /* H/W run (wait until swapout */
      @(negedge xc_top.hwOutInit);
      TD = TDL2; // read swapout value of TDL2 and update TD 
    end
    localtime = $time;
  end

  always (posedge xc_top.cpi_capture_enable) begin  
    /* compute the value of TDL2 to download to HW */
    TDL2 = TD - ($time - localtime);
  end
 
  initial begin
    localtime = $time;
    TD = 10;
  end
\end{verbatim}
The following is brief explanation of the S/W process;
\bit
\item \bb{TDL2} and \bb{state} are the only variables which needs to be
  swapped in and out
  \bit
  \w this means that when we switch from S/W process to H/W process (i.e. when
  swapin), \bb{TDL2} and \bb{STATE} value should be correctly set
  \w \bb{TDL2} are computed on-the-fly when swapin
  \w \bb{state} always matches between H/W process and S/W process and we
  don't need recomputation before swapin
  \eit
\w \bb{TD} is used only in SW process and it means ``time until next
\#-event''
  \bit
  \w \bb{TD} is the ``SW couterpart'' of \bb{TDL2}  (i.e. \bb{TD} is for S/W,
  \bb{TDL2} is for H/W)
  \w when we switch from H/W process to S/W process (i.e. swapout) \bb{TD}
  should be correctly set using the \bb{TDL2} value which was just updated
  from H/W. 
  \eit
\eit

\subsection{Assignment Operator}


\section{Clocking}
\subsection{ICE-Mode Clocking}
\subsubsection{CAKE (Clock Averaging with Ko-incident Edges)}
\bit
\w \bb{oversampling ratio}: number of FCLK cycles per cycle of fastest
    design clock
    \bit
    \w CAKE 0.5X mode: 0.5 FCLK cycles to emulate one fastest clock cycle --
    i.e. 1 FCLK emulates two fastest cycles
      \bit
      \w 
      \eit
    \w CAKE 1X mode: 1 FCLK cycle to emulate one fastest clock cycle
      \bit
      \w large capacity cost -- since duplicated logic: one for original the
      other for shadow
      \eit
    \w CAKE 2X mode (DEFAULT): 2 FCLK cycles to emulate one fastest clock cycle --
        e.g. delays inserted for loop breaking requires multiple passes
      \bit
      \w 
      \eit
    \eit
\eit
\subsection{SCM (Simulation Control Module)}
\subsection{SA-Mode Clocking}


\section{ECM}


\section{Primitives}
\subsection{IXC\_CALL\_CLK}
\bit
\w It has a state variable, \verb+active+:
\eit

\subsection{IXC\_EDGE}
\begin{verbatim}
  module ixc_edge(ev, s);
    parameter DIR = 0; /* 0: posedge, 1: negedge, 2: duedge */
    parameter ASYNC = 0;
    output ev;
    input s;

    wire fclk;  // quickturn fast_clock fclk;
    wire xc_top_eventOn // quickturn name_map xc_top_eventOn xc_top.eventOn
    
    generate
      if (ASYNC) begin
        reg _zzsr = 0;
        always @(posedge fclk) begin
          if (xc_top_eventOn) 
            _zzsr <= s;
        end
        assign ev = xc_top_eventOn &
                    (((DIR == 0) & (s & ~_zzsr)) ||          
                     ((DIR == 1) & (~s & ~_zzsr)) ||          
                     ((DIR == 2) & (s ^ _zzsr)));
      end 
      else begin
        if (DIR == 0) begin
          Q_PEDECT detect(ev, s, 1'b1);
        end
        else if (DIR == 1) begin
          Q_NEDECT detect(ev, s, 1'b1);
        end
        else if (DIR == 2) begin
          Q_EVECT detect(ev, s, 1'b1);
        end
      end
    endgenerate
    
  endmodule
\end{verbatim}


\end{document}
