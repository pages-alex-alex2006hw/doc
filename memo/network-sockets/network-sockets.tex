\documentclass{memo}
\usepackage{mathptm,mydef,myenv}
%\usepackage{MinionPro}
\begin{document}
\small
\noindent{\large\bf{}Sockets}

\paragraph{Overview} Sockets are a method of {\em IPC\/} that allow data to be
exchanged between applications, either on the same host or on different hosts
connected by a network. 
In a typical {\em client-server\/} scenario, applications communicate using
sockets as follows:
\bit
\w both client and server applications create sockets
\w the server binds it socket to a predefined address so that client can
locate it
\eit

\paragraph{Socket address structures}
Most socket functions require a {\em pointer to a socket address
  structure\/}. Each supported protocol suite defines its own socket address
structure and the names of these structures begin with \verb+sockaddr_+ and
end with a unique suffix for each protocol suite.

\bb{IPv4 socket address strcuture} is defined as follows:
\begin{verbatim}
  #include <netinet/in.h>
  struct in_addr {
    in_addr_t s_addr; /* 32-bit IPv4 address */
                      /* network byte ordered */
  };
  struct sockaddr_in {
    uint8_t sin_len;        /* structlen (16) */
    sa_family_t sin_family; /* AF_INET */
    in_port_t sin_port;     /* 16-bit portno */
    struct in-addr sin_addr; 
    char sin_zero[8];       /* unused */
  };
\end{verbatim}

Four socket functions pass a socket address structure {\em from the process to
  the kernel\/}: \verb+bind+, \verb+connect+, \verb+sendto+, and
\verb+sendmsg+. Five socket functions that pass a socket address structure
{\em from the kernel to the process\/} are: \verb+accept+, \verb+recvfrom+,
\verb+recvmsg+, \verb+getpeername+, and \verb+getsockname+.  For these five
functions, the socket length argument is used as a \bb{value-result} argument.



\paragraph{Socket system call: { socket()}} Creates a new socket.
\begin{verbatim}
int socket(int domain, int type, int protocol);
\end{verbatim}
Communication domain, \bb{domain}, specifies which protocol family we will
use:
   \bit
   \w \verb+AF_UNIX+, \verb+AF_LOCAL+: local communication
   \w \verb+AF_INET+: IPv4 Internet protocol
   \w \verb+AF_INET6+: IPv6 Internet protocol
   \w \verb+AF_IPX+: Novell IPX  protocol
   \w \verb+AF_NETLINK+: kernel user interface device
   \w \verb+AF_AX25+: amateur radio AX.25 protocol, etc.
%   \w \verb+AF_PACKET+: low level packet interface
   \eit
The \bb{type} parameter indicates the {\em socket type\/} which determines
communication semantics:
  \bit
  \w \verb+SOCK_STREAM+: reliable, two-way connection-based
  \w \verb+SOCK_DGRAM+: connectionless, unreliable fixed-length datagrams 
  \w \verb+SOCK_SEQPACKET+: sequenced, reliable connection-based path for
  fixed maximum length datagrams; a consumer is required to read an entire
  packet with each input system call
  \w \verb+SOCK_RAW+: raw network protocol access
  \w \verb+SOCK_RDM+: reliable datagram layer w/o ordering
  \w \verb+SOCK_PACKET+: obsolete
  \eit
From Linux 2.6.27, we can BITWISE-OR the following values with above:
 \bit
 \w \verb+SOCK_NONBLOCK+: set \verb+O_NONBLOCK+ file status (save calling
 extra call to \verb+fcntl+
 \w \verb+SOCK_CLOEXEC+: see \verb+open(2)+ for its use
 \eit
The following is a brief
comparison between two most common types:
\begin{quote}
\begin{tabular}{r|c|c|}
& {\bf stream} & {\bf datagram} \\ \hline
{\em reliable delivery} & yes & no \\
{\em message boundary preserved} & no & yes \\
{\em connection-oriented} & yes & no  \\ \hline
\end{tabular}
\end{quote}
The {\bf protocol} specifies specifies a particular protocol for the given
protocol family. Normally a single protocol exists for a given protocol family
and we can set this value as 0. 

\paragraph{Socket system call: { bind()}} Binds a socket to an address. A
server employs this call to bind its socket to a well-known address so that
clients can locate the socket.
\begin{verbatim}
int bind(int sockfd, struct sockaddr *addr, 
         socklen_t addrlen)
\end{verbatim}
The \verb+struct sockaddr+ is just for casting the actual socket address
structure (e.g. \verb+sockaddr_in+ or \verb+sockaddr_in6+).
\begin{verbatim}
  struct sockaddr {
    sa_familiy_t sa_family;
    char sa_data[14];
  }; 
\end{verbatim}

\paragraph{Socket system call: { listen()}}
Allows a {\em passive stream\/} socket to accept incoming connections from other sockets.
The \verb+sockfd+ is a descriptor to a socket of type \verb+SOCK_STREAM+ or
\verb+SOCKSEQPACKET+. 
\begin{verbatim}
int listen(int sockfd, int backlog);
\end{verbatim}
The \verb+backlog+ is the maximum length of the queue of pending
connections. If the queue is full, the client may receive an
\verb+ECONNREFUSED+ error or the request will be ignored if underlying
protocol supports retransmission.

\paragraph{Socket system call: { accept()}}
Accepts a connection from a peer application on a {\em listening stream
  socket\/} (\verb+SOCK_STREAM+, \verb+SOCK_SEQPACKET+), and 
optionally returns the address of the peer socket. 
\begin{verbatim}
int accept(int sockfd, struct sockaddr *addr,
           socklen_t *addrlen);
\end{verbatim}
The \verb+sockfd+ must be a descriptor for a socket which was created with
\verb+socket()+, bound to a local address with \verb+bind+, and is listening
for connections after a \verb+listen()+.


\paragraph{Socket system call: { connect()}}
Establishes a connection with another socket. 
\begin{verbatim}
int connect(int sockfd, struct sockaddr *addr,
            socklen_t addrlen);
\end{verbatim}

\paragraph{Socket I/O} Socket I/O can be performed using \verb+read()+ and
\verb+write()+ system calls, or using a range of socket-specific system calls
(e.g. \verb+send()+, \verb+recv()+, \verb+sendto()+, and \verb+recvfrom()+). 

By default, these system calls perform {\bf blocking I/O}. {\bf Nonblocking
  I/O} is also possible by using the \verb+fcntrl() F_SETFL+ operation to
enable the \verb+O_NONBLOCK+ open file status flag. 

\paragraph{Stream sockets}
Operation of stream socke1ts are analogous to a \bb{telephone call}:
\ben
\w \bb{install telephone}: create a socket using \verb+socket()+
\w \bb{make a phone call}: caller connect its socket to another application's
socket for a communication; two sockets are connected as follows:
  \ben
  \w server calls \verb+bind()+ to a predefined address and then calls
  \verb+listen()+ to notify the kernel of its willingness to accept incoming
  connections.  
  \w client establishes the connection by calling \verb+connect()+ specifying
  the address of the socket to which the connection is to be made
  \w server that called \verb+listen()+ then accepts the connection using
  \verb+accept()+ 
  \een
\w \bb{talk}: once the connection has been established, data can be
  transmitted (e.g. \verb+read()+ and \verb+write()+) in both directions until
  one of the application calls \verb+close()+
\een

\paragraph{Active vs passive sockets}
By default a socket created using \verb+socket()+ is active. \bb{Active
  sockets} can be used in a \verb+connect()+ call to establish a connection to
a passive socket. This is called as performing an \verb+active open+.

A \bb{passive socket} (also called a \bb{listening socket}) is one which has
been marked to allow incoming connections by calling
\verb+listen()+. Accepting an incoming connection is referred to as performing
a {\bb{passive open}}. 

\paragraph{Datagram sockets}
Operation of datagram sockets can be explained by analygo with the \bb{postal
  system}: 
\ben
\w \bb{setup mailbox}: \verb+socket()+ system call is equivalent of setting up a mailbox
\w \bb{setup address}: server binds its socket to a well-known address so
client can initiate communication
\w \bb{send a mail}: client calls \verb+sendto()+ with the address of the
socket as its destination
\w \bb{receive a mail}: server calls \verb+recvfrom()+ which may block if no
datagram has not yet arrived
\w \verb+close()+ closes the communication
\een
\end{document}

% LocalWords:  IPC sockaddr IPv  sendto sendmsg recvfrom recvmsg getpeername
% LocalWords:  getsockname INET DGRAM datagram recv Nonblocking fcntrl SETFL
% LocalWords:  NONBLOCK
