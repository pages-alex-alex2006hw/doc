\documentclass{myproc}
\usepackage{mydef,myenv}
\usepackage{mathptm}
\usepackage[all]{xy}
%\usepackage{MinionPro}
\begin{document}
\title{\large\bf Notes on Web/Enterprise Technologies}
%\author{\normalsize\tt{}cjeong@live.com}
\date{}
\maketitle

\small

\tableofcontents
%\pagebreak

\section{Foundations}
\subsection{HTTP}
\bit
\w current standard: HTTP 1.1 (RFC 2616)
\w application protocol for distributed, collaborative, {\em hypermedia\/}
information systems 
\w \bb{HTTP {\em ``session''\/}}: sequence of network request-response transactions
  \bit
  \w TCP connection (typically) to port 80
  \w HTTP server listening on port 80 waits for client's request \underline{``message''}
  \w upon receipt of a request, server sends back a status line
  (e.g. \verb+"HTTP/1.1 200 OK"+); body of the message is typically the
  \bb{requested resource} or an error message (or \underline{else}: {\em what
    else could be? {\bf CODE\/} or {\bf CONTINUATION}?}  )
  \eit
\w defines 9 \bb{methods} (verbs) to indicate desired action to be performed
on ``{\em{}resources\/}''
  \ben
  \w \bb{HEAD}: asks for the response identical to the one that would
  correspond to a GET request but without the response body; this useful for
  retrieving {\em meta-information\/} written in response headers, without
  having to transport the entire content
  \w \bb{GET}: requests a representation of the specified resource
  \w \bb{POST}: submits data to be processed to the identified resource
  (submitted data is included in the body of request) 
  \w \bb{PUT}: uploads a representation of specified resource
  \w \bb{DELETE}: deletes the specified resource
  \w \bb{TRACE}: echoes back the received request, so that a client can see
  what (if any) changes or additions have been made by intermedite servers
  \w \bb{OPTIONS}: returns the HTTP methods that the server supports for
  specified URL; this can be used to check the functionality of a web server
  by requesting `*' instead of a specific resource
  \w \bb{CONNECT}: converts the request connection to a transparent \bb{TCP/IP
  tunnel}, usually to facilitate SSL-encrypted communication (HTTPS) through
  an unencrypted HTTP proxy
  \w \bb{PATCH}: used to apply partial modifications to a resource
  \een
\eit
\subsection{HTML}
\bit
\w HTML (hytertext markup language) is the main language for \bb{web pages},
where HTM elements are the basic building-blocks of webpages
\w \bb{HTML element} consists of \bb{tags} enclosed in {\em angle brakets}
(e.g. \verb+<html>+); HTML tags commonly come in pairs
\eit
\subsubsection{HTML DTD (Document Type Definition)}
\bit
\w a \bb{doctype} (document type declaration) at the beginning of HTML
specifies which SGML-based DTD (document type definition) to use for validation
of this document
\w e.g. \verb+<!DOCTYPE HTML "-//W3C//DTD HTML 4.01"..>+
\w usually, this line is used by browsers on which mode to render the document
(e.g. standard mode vs quirks (backward-compatible) mode)
 
\eit
\subsubsection{XHTML}
\bit
\w  To be \bb{deprecated} (to be replaced by \bb{HTML5}, which supports XML
and is endorsed by W3C)
\eit
\subsubsection{HTML5}
\bit
\w more support for XML; media-friendly; reduce the need for ``Flash''
\w \bb{HTML5 = Markup + JavaScript APIs + CSS}
\w HTML is mostly about ``interconnection'', now in HTML5, we are more
concerned with \bb{first-class handling} of \bb{audio}, \bb{image}, and
\bb{video}, etc. (not just inlining into webpages)
  \bit
  \w audio, video, etc. can be ``manipulated'' from JavaScript
  \eit
\w in summary, \bb{graphic toolkit} will be overrided by HTML5 and HTML5 will
be the presentation layer of applications
\w also, page elements will be programmable (reified)
  \bit
  \w HTML page is a ``iteratable'' container of page elements; can be
  modified/updated 
  \eit
\w \bb{How HTML works}
  \bit
  \w the \bb{browser} loads a document, which includes a markup in HTML and
  sytle in CSS
  \w when browser loads a page, it creates \bb{internal model of the document}
  which is a ``tree of markup elements''
  \w browser also loads \bb{JavaScript code} and begins execution after the
  page loads
    \bit
    \w  using JavaScript, we can interact with the page by manipulating the
    DOM, react to user or browser-generated events, or make use of all new
    HTML5 JavaScript APIs
    \w HTML5 JavaScript APIs include: \bb{sockets, web workers, forms, audio,
      video, local storage, offline caching, canvas, drag and drop,
      geolocation, forms}
    \eit
  \eit
\eit

%%%%%%%%%%%%%%%%%%%%%%%
% XML
%%%%%%%%%%%%%%%%%%%%%%%
\subsection{XML}
\bit
\w XML (eXtensible Markup Language) is used to {\em transport/store\/} data
while HTML is used to {\em display\/} data
\w XML documents for a \bb{tree structure}
%% \w to avoid ambiguity (w/ tokens), use:
%%    \verb+&lt;+ (\verb+<+), 
%%    \verb+&gt;+ (\verb+>+)
%%    \verb+&amp;+ (\verb+&+)
%%    \verb+&apos;+ (\verb+'+)
%%    \verb+&quot;+ (\verb+"+)
%% \w comments: \verb+<!-- comments -->+
\w \bb{XML element}: everything from start tag to end tag; can contain
   \ben
   \w another element (recursively),
   \w \bb{text}, or
   \w \bb{attribute}
   \een
\w using \bb{attributes} can make update of XML schemas difficult
\w \bb{well-formed XML documents}: satisfies syntactic constraints
(e.g. matches tags, exactly one root leement, proper nesting)
\w \bb{valid XML documents}: well-formed + conformity to \bb{DTD}
\eit

\subsubsection{XML DTD (Document Type Definition)}
\bit
\w used to define the structure of XML document, e.g.
{
\begin{verbatim}
  <!DOCTYPE note
  [
  <!ELEMENT note (to,from,heading,body)>
  <!ELEMENT to (#PCDATA)>
  <!ELEMENT from (#PCDATA)>
  <!ELEMENT heading (#PCDATA)>
  <!ELEMENT body (#PCDATA)>
  ]>
\end{verbatim}
}

\eit
\subsubsection{XML Schema}
\bit
\w XML-based alternative to DTD
{
\begin{verbatim}
  <xs:element name="note">
  <xs:complexType>
    <xs:sequence>
      <xs:element name="to" type="xs:string"/>
      <xs:element name="from" type="xs:string"/>
      <xs:element name="heading" type="xs:string"/>
      <xs:element name="body" type="xs:string"/>
    </xs:sequence>
  </xs:complexType>
  </xs:element>
\end{verbatim}}
\w Pattern \#1: \bb{Russian Doll}
  \begin{verbatim}
  <xsd:element name="book">
    <xsd:complexType>
      <xsd:sequence>
        <xsd:element name="title" 
                     type="xsd:string"/>
        <xsd:element name="price" 
                     type="xsd:decimal"/>
        <xsd:element name="category" 
                     type="xsd:NCName"/>
        <xsd:choice>
          <xsd:elemnet name="author" 
                       type="xsd:string"/>
          <xsd:element name="authors">
            <xsd:complexType>
              <xsd:sequence>
                <xsd:element name="author" 
                             type="xsd:string"/>
              </xsd:sequence>
            </xsd:complexType>
          </xsd:element>
        </xsd:choice>
      </xsd:sequence>
    </xsd:complexType>
  </xsd:element>
  \end{verbatim}

\w Pattern \#2:  \bb{Salami Slice}
  \begin{verbatim}
  <xsd:element name="book">
    <xsd:complexType>
      <xsd:sequence>
        <xsd:element ref="tns:title" />
        <xsd:element ref="tns:author" />
        <xsd:element ref="tns:category" />
        <xsd:element ref="tns:price" />
      </xsd:sequence>
    </xsd:complexType>
  </xsd:element>
  <xsd:element name="title"/>
  <xsd:element name="price"/>
  <xsd:element name="category"/>
  <xsd:element name="author"/>
  \end{verbatim}
\w Pattern \#3:  \bb{Venetian Blind}
  \begin{verbatim}
  <xsd:element name="book" type="tns:BookType"/>
  <xsd:complexType name="BookType">
    <xsd:sequence>
      <xsd:element name="title" 
                   type="tns:TitleType" />
      <xsd:element name="author" 
                   type="tns:AuthorType" />
      <xsd:element name="category" 
                   type="tns:CategoryType" />
      <xsd:element name="price" 
                   type="tns:PriceType" />
    </xsd:sequence>
    <xsd:simpleType name="TitleType">
      <xsd:restriction base="xsd:string">
        <xsd:minLength value="1"/>
      </xsd:restriction>
    </xsd:simpleType>
  </xsd:complexType>
  \end{verbatim}
\w Pattern \#4:  \bb{Garden of Eden}
\eit

\subsubsection{CSS (Cascading Style Sheets) \& CSS3}
\bit
\w not the most common method for formatting XML
\w W3C recommended using \bb{XSLT} instead
\w {\em However, what about CSS3 w/ HTML5?\/}
\eit

\subsubsection{XSLT (eXtensible Stylesheet Language Transformations)}
\bit
\w sort of language used to transform XML data into displayable HTML-format
data
\w e.g.

\eit

\subsubsection{XML DOM (Document Object Model)}
\bit
\w DOM defines a standard way of accessing and manipulating XML documents
 ({\bf get, change, add, delete XML {\em elements\/}})
\w DOM is an {\em internal(-to-browser) representation\/} for XML document, represented as
a tree-structure 
\w \bb{DOM nodes}:
  \bit
  \w document node
  \w element node
  \w test node
  \w attribute node
  \w comment node
  \eit
\w browsers have built-in XML parser: 
   \bit
   \w \bb{parser converts XML into a JavaScript accessible object (XML DOM)}
   \eit
\eit

\subsubsection{Creating XML on Servers}
\bit
\w XML can be statically available as a file (e.g. {\verb+books.xml+})
\w XML can be dynamically created
  \bit
  \w from ASP
  \w from PHP
  \w $\cdots$
  \eit
\eit

\subsection{Handling XML using Java}
\bit
\w XML parser: \bb{SAX} (Simple API for XML) is a de-factor standard for XML handling in Java
\w SAX is available as part of JavaSDK (\verb+org.xml.sax+)
\w for further reference on XML handling in Java (e.g. validating, etc.), see ``{\bb{}Java SOA Cookbook}'' and ``{\bb{}Java and XML}''
\eit

\section{Web Scripting: Client(Browser)-Side Scripting}
\subsection{JavaScript}
\bit
\w originated from \bb{ECMAscript}
\eit
\subsection{JQuery}
\subsection{JSON (JavaScript Object Notation)}
\bit
\w lightweight text-based open standard designed for {\em human-readable\/}
data exchange
\w used for representing data structures and associative arrays
\w ex. use in AJAX
  {
  \begin{verbatim}
  var my_json_obj = {}
  var http_req = new XMLHttpRequest();
  http_req.open("GET", url, true);
  http_req.onreadystatechange = function() {
    if (http_req.readyState == 4 && 
        http_req.status == 200) {
       my_json_obj = 
         JSON.parse(http_req.responseText);
    }
  }
  \end{verbatim}
  }
\w Inside Java program, there's JSON library which builds JSON object
(\bb{Builder} pattern)
  \begin{verbatim}
  mystr = new JSONStringer()
            .object()
               .key("JSON")
               .value("Hello, String!")
            .endObject();
            .toString()
  \end{verbatim}
  creates 
\begin{quote}
  \verb+{"JSON", "Hello, String!"}+
\end{quote}
\eit

\subsection{DHTML (Dynamic HTML)}
\bit
\w not a language or standard; it's mix of technologies to make websites more
dynamic 
\eit

\subsection{AJAX (Asynchronous JavaScript and XML)}
\bit
\w not a language or standard; it's mix of technologies to make websites more
dynamic and interactive
\w \bb{= fat client!}
\w \bb{involved technologies}: XHTML, DOM, JavaScript, CSS, XML
\eit

\subsection{Google Web Toolkit}
\bit
\w corss-compiles Java into JavaScript 
\w google chrome Web Apps
\eit
\subsection{Google Dart}

\section{Web Scripting: Server-Side Scripting}
\subsection{PHP}
\subsection{.NET}
\subsection{ASP}
\subsection{Node.js}
\subsection{Web Services}


%%%
%\subsection{WAP and WML}
%\subsection{WSDL}
%\subsection{SMIL}
%\subsection{RDF and OWL}

\section{Authorizations and Authentications}
\subsection{OpenID}
\bit
\w way of identifying yourself no matter which website you visit
\w open standard that describes {\em how users can be authenticated in a
  decentralized manner}, obviating the need for services to provided their own
ad hoc systems and allowing users to consolidate their digital identities
\eit
\subsection{OAuth}
\bit
\w allows users to share private resources (photos, contact list)
stored on one site with another site without having to hand out credentials
(id/password)
\w e.g. 3rd party tool/website is granted access to my google PPT file even if
it does not know my id/password
\w \bb{access delegation}
\eit

\section{Enterprise Computing}
\subsection{Web Services}
\subsubsection{Trends}
\bit
\w \bb{Containers}
  \bit
  \w old: EJB
  \w new: Spring
  \eit
\w \bb{Persistence Model}
  \bit
  \w old: EJB/JDBC
  \w new: iBATIS, JDO, Hibernate
  \eit
\w \bb{Web application framework}
  \bit
  \w old: Struts
  \w new: Tapestry, Ruby on Rails
  \eit
\w \bb{Java or PLs}
  \bit
  \w old: Java
  \w new: Python, Scala, Ruby
  \eit
\eit


\subsubsection{Java-WS (Java API for XML Web Services)}
\bit
\w offers three basic choises for connecting to web services: 
  \ben
  \w \bb{dynamic invocation}
  \w \bb{proxy}
  \w \bb{SAAJ}
  \een
\eit

\begin{figure*}[ht]
\small
\[\xymatrix@+3pc{
  *+[F]\txt{Java SE 6} \ar[d]_{\mbox{\scriptsize{}includes}} & *+[F]\txt{JAXB} \ar[dr]^{\mbox{\scriptsize{}transforms Java objects to/from}}& & \\ 
 *+[F]\txt{JAX-WS} \ar[r]^{\mbox{\scriptsize{}provides high-level}}_{\mbox{\scriptsize{}API for}} \ar[d]_{\mbox{\scriptsize{}is built on}} & *+[F]\txt{SOAP} \ar[r]^{\mbox{\scriptsize{}represented by}} & *+[F]\txt{XML} & *+[F]\txt{XML Schema} \ar[l]_{\mbox{\scriptsize{}defines}}\\ 
  *+[F]\txt{SAAJ} \ar[ur]^{\mbox{\scriptsize{}low-level API}} & *+[F]\txt{WSDL} \ar[u]_{\mbox{\scriptsize{}binds to}} & *+[F]\txt{JAX-P} \ar[u]_{\mbox{\scriptsize{}processes}}& \\
  *+[F]\txt{MTOM} \ar[uur]^{\mbox{\scriptsize{}performance}} & *+[F]\txt{UDDI}\ar[u]_{\mbox{\scriptsize{}directory for}} & *+[F]\txt{JAX-R} \ar[l]_{\mbox{\scriptsize{}provides client}}^{\mbox{\scriptsize{}access for}}&
}\]
\caption{The world of JAX-WS (many of thesea re quite dead)}
\end{figure*}

\subsubsection{WSDL (Web Services Description Language)}
\bit
\w XML-based language for describing web services and how to access them
(e.g. how to locate)
\w \bb{WSDL ports} (\verb+<portType>+)
  \bit
  \w defines a \bb{web service} (or \bb{operations}) that can be performed and
  the messages that are involved
  \w \bb{operation types}: \bb{one-way, request-response, solicit-response,
    notification} 
{
\begin{verbatim}
   <message name="getTermRequest">
     <part name="term" type="xs:string"/>
   </message>

   <message name="getTermResponse">
     <part name="value" type="xs:string"/>
   </message>

   <portType name="glossaryTerms">
     <operation name="getTerm">
       <input message="getTermRequest"/>
       <output message="getTermResponse"/>
     </operation>
   </portType>
\end{verbatim}
}
  \eit
\w \bb{WSDL messages}
\w \bb{WSDL types}
\w \bb{WSDL bindings}
\eit

\subsection{Business Processes}
\subsubsection{BPEL (Business Process Execution Language)}
\bit
\w origins of BPEL can be traced back to \bb{WSFL} (Web Services Flow
Language) and \bb{XLANG} 
\eit

\subsubsection{Workflows}

\subsection{SOA (Service-Oriented Architecture)}
\subsubsection{Service}
A \bb{service} is a {\em softwsare component\/} with the following properties:
\bit
\w defined by a \bb{interface} that may be {\bf platform-independent},
\w available across a \bb{network}
\w carries out {\em business functions\/} by operating on {\em business
  objects\/}
\w interface and implmenetation can be decroated with extensions that come
into effect at runtime
\eit

\subsubsection{SOA (Service-Oriented Architecture)}
An \bb{SOA} is a architecture that 
\bit
\w uses \bb{services} as building blocks,
\w facilitates enterprise integration and component reuse through 
\bb{loose coupling}. 
\eit

\subsubsection{Types of services}
\bit
\w \bb{entity service}: CRUD (create/read/update/delete) opreations over basic
entities (e.g. customer, invoice, employee, product) 
\w \bb{functional service}: performs a function such as sending an email, or
logging, or notification, security, etc
\w \bb{process service}: represents a series of related tasks; can be
represented as an orchestration by an ESB, with a coarse-grained contract that
causes the process service to appear as a unified whole to clients
\eit

\subsubsection{Service composition}
A \bb{composed service} is an aggregate of other existing services. 

\subsection{Enterprise Server-Side Langauges}
\subsubsection{Java}
\subsubsection{Python}
\subsubsection{Ruby}

\subsection{Enterprise Containers}
\subsubsection{EJB (Enterprise JavaBeans)}
\subsubsection{Spring}

\subsection{Enterprise Persistence}
\subsubsection{JDBC (Java Database Connectivity)}
\subsubsection{JDO}
\subsubsection{Hibernate}

\subsection{Web Applications Framework}
\bit
\w basically \bb{MVC framework}
  \bit
  \w \bb{Model}: data model + business logic
  \w \bb{View}: UI
  \w \bb{Controller}: glue (e.g. event system)
  \eit
\w \bb{functionalities of web application frameworks}
  \bit
  \w \bb{web template system}: static HTML + dynamically generated parts
     \bit
     \w e.g. \bb{real estate website}: 500 static pages vs. 1 dynamic page + 500
  records
     \eit
  \w \bb{caching}: instead of dynamic generation every time, cache generated
  files 
  \w \bb{database access and mapping}
  \w \bb{URL mapping}
  \w \bb{AJAX}: for interactive web applications (i.e. more intensive UI; more
  than just buttons)
  \w \bb{web services}
  \eit
\w \bb{push-based frameworks}: Struts, Django, Ruby on Rails, Spring MVC
\w \bb{pull-based (a.k.a. component-based) frameworks}: Struts2, Lift,
  Tapestry, JBoss Seam, Wicket, Stripes 
\eit
\subsubsection{Struts}
\subsubsection{Tapestry}
\subsubsection{Ruby on Rails}
\bit
\w 
\eit

\section{Software Bus and Message Queues}
\subsection{D-Bus}
\subsubsection{Overview}
D-Bus is a message bus system, a simple {\em way for applications to talk to
  each other\/}. In addition to the major role of \bb{interprocess
  communication} and \bb{remote procedure call} capabilities, D-Bus helps
coordinate process lifecycle (or \bb{workflow}); it makes it simple and
reliable to code a ``single instance'' application or daemon, and to launch
applications and daemons on demand when their services are needed.

D-Bus provides two kinds of daemons: a \bb{system daemon\/} for events such as
``new hardware device added'' or ``printer queue changed'' and a
\bb{per-user-login-session daemon} for general IPC needs among user
applications. 


\subsubsection{Language bindings}
APIs for D-Bus, or \bb{bindings}, are available in several languages --
typically one per language.  Each presents its own API as suits the language,
hiding the details of working with D-Bus from the programmer to different
extgents. The ideal is to fit the D-Bus API into the native language and
libraries as naturally as possible. 

\subsubsection{Buses}
There are two major components to D-Bus: a \bb{point-to-point communication
  dbus library}, which in theory could be used by any two processes in order
to exchange messages among themselves and \bb{a dbus daemon}. 
The daemon runs a actual {\em bus\/}, a kind of ``street'' that messages are
transported over, and to which any number of processes may be connected at any
given time. 

There are two kinds of buses: a single \bb{system bus} for system-wide
communication and a \bb{session bus} used by a single user's ongoing GNOME
session. A session bus normally carries traffic under only a single user
identity, but D-Bus is aware of user identities and does support flexible
authentication mechanisms and access controls. 

\subsubsection{Addresses}
Every bus has an \bb{address} describing how to connect to it. A bus address
will typically be the filename of a Unix-domain socket such as
``\verb+/tmp/.hiddensocket+,'' but it may also be a TCP port where a bus
daemon is listening on an IP-domain socket, or conceivably a descriptor for
some other low-level communications scheme.

The dbus library is responsible for hooking up clients to the bus daemon. This
process is said that a client process opens and uses a \bb{connection} to the
bus. 

\subsubsection{Connections}
Every connection to a bus can be addressed on that bus under one or more
names. These names are called the connection's \bb{bus names}. (note that bus
names are for ``connections'' not for ``buses.'') A bus name example is
\verb+com.acme.Foo+. 

When a conection is set up, the bus immediately assigns an {\em immutable\/}
bus name, called a \bb{unique connection name\/}. One example is
''\verb+:34-907+''. 


\begin{figure*}[ht]
\small
\centerline{\begin{tabular}{|l|l|p{7cm}|p{3.5cm}|}\hline
  & \bb{identified by} & \bb{looks like} & \bb{chosen by} \\ \hline
\bb{bus} & address & {\tt{}unix:path=/var/run/dbus/system\_bus\_socket} &
system configuration \\ \hline
\bb{connection} & bus name & {\tt :34-907} (unique) or {\tt com.mycompany.TextEditor}
(well-known) & D-Bus (unique) or the owning program (well-known) \\  \hline
\bb{object} & path & {\tt /com/mycompany/TextFileManager} & the owning program \\ \hline
\bb{interface} & interface name & {\tt org.freedesktop.Hal.Manager} &
the owning program \\ \hline
\bb{member} & member name & {\tt ListNames} & the owning program  \\ \hline
\end{tabular}}
\end{figure*}

\subsubsection{Object Model}
Message exchange on protocols like TCP or UDP is symmetric. 

\subsubsection{Objects}
One end of any exchange on a bus will always be a communications endpoint that
in D-Bus parlance is called an \bb{object}. An object is created by a client
process and exists within the context of that client's connection to the
bus. The object is a way for the client process to {\em offer its {\bfseries services\/} on the
bus\/} -- but one client may create any number of objects. 

The bus imposes an object-centric view of communications, where any message
carried by the bus is of one of three kinds: 
\ben
\w \bb{Requests} sent to objects by client processes.
\w \bb{Replies} to requests, going from an object back to a requesting process.
\w \bb{One-way messages} emanating from objects, {\em broadcast\/} to any
connected clients that have registered an interest in them.
\een
Thus at a higher level of abstraction, the bus supports two forms of
communication that we could call \bb{1:1 request-reply} going to an
object, and 
\bb{1:$n$ publish-subscribe} coming from an object. 

Every bus has at least one object, representing the bus itself. Clients can
obtain information about the status of the bus by sending requests to this
object. As you'll see later on, it represents the bus in other useful ways as
well. 


\subsubsection{Proxies}
Objects on the bus can be accessed through {\em references\/} that we call
\bb{proxies}. We 
call them that because a proxy is a local representation inside your own
program of an object that is really accessed through the bus, and typically
lives outside your program: you literally access the object ``by proxy.''

%% Objects have names, also called \bb{paths} because they look like Unix-style,
%% slash-separated filesystem paths. An object that represents a particular cell
%% in a particular spreadsheet might be called
%% ``\verb+/org/kde/kspread/sheets/3/cells/4/5+'', for instance. An object's name
%% needs to be unique only within the context of its connection to the bus. To
%% obtain a proxy to that spreadsheet cell, you would ask the bus to look up
%% object 
%% \verb+/org/kde/kspread/sheets/3/cells/4/5+ for you, to be found in the context
%% of the spreadsheet's connection.


Since any {\em object ``lives within'' the context of a connection\/}, it
takes a combination of that connection's bus name and the object's own name to
find  it. Once you have found the object you want, if you'll be using it again
soon, you'll usually want to keep a proxy to that object around as a variable
in your program. That will save you having to look up the object time and
again. 


%% Some bindings' proxies may support failover. If you have a proxy to an object
%% exported by some client connected to the bus under a well-known bus name, and
%% that client disconnects (removing the object), reconnects under the same
%% well-known name, and revivies the object, your own program may continue to use
%% the proxy without ever noticing that the object went away for a while. Not all
%% bindings support this, and of the two kinds of proxy in the GLib binding, only
%% one does. It is also not always desirable, e.g. when subsequent operations on
%% an object are meant to be a whole, and it's not acceptable for the object to
%% be disbanded and later reinstituted without your noticing. In those cases, you
%% may need to use a unique connection name in obtaining the proxy rather than a
%% well-known one.
 

\subsubsection{Methods}
{\em When a client sends a request to an object, it sees this request as
  invoking a method on the object: the object is asked to perform a specific,
  named action\/}. Normally, if a client tries to invoke a method on an object
that the 
object does not provide, this will raise an error. 

The method's definition may require certain information to be passed with the request as arguments (input parameters). For every request, a reply message carries the result back to the requester, along with either result data (output parameters) or, if the action could not be performed, exception information. Exceptions will contain at least an exception name and an error message.

Most D-Bus bindings make all this fit in with their environment's native
mechanisms, hiding the finicky details of encapsulating parameters in messages
and translating exception messages into exceptions (or whatever the native
error-handling mechanism is). For example, passing a string argument to a
method of some remote object will look to your program just like passing a
string argument to a function in your own program. There is no need for
tedious conversions and copying of the data into messages, and there is
usually no need to concern yourself with the sending of the underlying
message. The binding takes care of all that; the work of encapsulating your
data into the messages is called \bb{marshalling}. 

There is one interesting difference with conventional function calls: when
sending a request to an object, you don't necessarily have to sit around and
wait for a reply. In more complex programs you'll usually find other useful
things to do until the method completes. You may want to be ready to handle
user interaction, for example, or availability of data from a file or a
network connection; you may even have multiple method invocations "in flight"
and want to handle the results as they come in, rather than in some
pre-defined order. This style of invocation, where you go on to do other
things while waiting for an answer, is called \bb{asynchronous method
invocation}. 

\subsubsection{Signals}
The other form of communication is \bb{signals}. 
These one-way communications come from an object and go nowhere in
particular. Client processes can register an interest in signals of a
particular name coming from a particular object. Whenever an object emits a
signal, all interested clients will receive a copy of the signal. There may be
one client receiving it, or there may be many--or nobody may be
listening. There are no replies to signals: the object emitting the signal
would not know how many replies to expect, or where to expect them from. 


%% Of course,
%% since signals are a strictly one-way form of communication, signals do not
%% have input and output parameters like methods do. More recent versions of
%% D-Bus also allow clients to restrict their interest to cases where certain of
%% the signal's parameters match given values; they will only receive instances
%% of the signal that match those expectations. 

{\em Signals can carry parameters\/}, just like method invocations. 
Signals are used to {\em publish the occurrence of events that clients may be
interested in\/}, such as the closing of some other client's connection to the 
bus. That particular kind of signal is sent by the object representing the bus
itself. Because of this, the event can be announced properly regardless of
whether the departing client closed its connection in an orderly fashion, or
was killed, or crashed spectacularly. 


\subsubsection{Interfaces}
So every object supports particular \bb{methods} and may emit particular
\bb{signals}. These are known collectively as the \bb{object's members}. 

All of an object's members are specified in interfaces, which are {\em  sets
  of declarations\/}. "Implementing" an 
interface amounts to promising to provide all methods specified in the
interface, and announcing the availability of its signals for listeners. Each
of these members must accept and/or provide parameters exactly as specified in
the interface.  

%% Any object may implement a given interface, just as in Java any number of classes may implement the same interface. Conversely, a single object may implement any number of interfaces. (With D-Bus it probably wouldn't make much sense to have an object that implemented no interfaces at all, though, which is perfectly normal with Java classes.) The combination of all interfaces supported by the object is called the object's type.

%% When a client invokes a method or listens for a signal, it must indicate the object and the member it is referring to. In addition to object and member, the client may also name the interface in which that member was specified. This can be necessary in some cases. If an object implements two interfaces, for example, that both specify a method named foo, then the object may have separate implementations for foo in the two interfaces. When a client tries to invoke foo on the object without specifying which interface it had in mind, there is no guarantee as to which of the two foo methods will be invoked. The D-Bus implementation may even refuse to carry the request message in the first place. Similarly, you wouldn't want to receive signals that looked like what you're listening for but are really different ones that happened to have the same name. Older versions of D-Bus also had a bug where request or signal messages could be lost if they failed to specify an interface.

%% Whether there is "overloading" of members within interfaces, i.e. whether multiple members of the same interface may have the same name, depends on the binding. 
%\subsubsection{Addressing}


\subsubsection{Message ordering}
Requests from one connection to the same object are delivered in the same order in which they were sent. The same goes for multiple replies from one object to the same client.

This does not mean that all messages are always delivered in sending
order. For example, if two client processes send requests to the same object
around the same time, there is no documented guarantee that the object will
receive them in the same order. Even when one client sends two subsequent
requests to the same object, then waits for both replies, it is possible that
the reply to the second request comes in before the reply to the first
request. The "object" may really be a multithreaded server process with
multiple requests being handled in parallel, or it may prioritize requests
internally. 

\subsubsection{Activation}
So far we've assumed that objects are created by active clients. There is another way of offering services on the bus: the bus daemon can be instructed to start (or activate) clients automatically when needed. Activation of a client can be triggered in two ways, both keyed by a well-known bus name that the activated client must obtain:
\ben
\w Through an explicit request to the object representing the bus itself.
\w By invoking a method on an object in the context of the client's well-known bus name. 
\een
The latter can be inhibited through an option in the method invocation message. Some bindings may try to activate an appropriate client when you create a proxy on a well-known bus name that is not currently in use; others may defer this until you use the proxy to invoke the method. The difference can matter if you listen for a signal coming from an object: if the client that should provide the object is not actually running, you could wait in vain!

%% To create a client that can be activated, describe it in a service file. A service file looks like a human-readable ".ini" file, line-based and encoded in UTF-8. Its name must always end in ".service".

%% For example, you might want to register the fact that client program
%% /usr/local/bin/bankcounter can be run to provide well-known bus names
%% com.bigmoneybank.Deposits and com.bigmoneybank.Withdrawals. To do that, you'd
%% write a service file "bankcounter.service" (the name is arbitrary, so long as
%% it ends with .service) looking like: 
%% \begin{verbatim}
%% # Fixed section header (do not change):
%% [D-BUS Service]
%% Names=com.bigmoneybank.Deposits;
%%       com.bigmoneybank.Withdrawals
%% Exec=/usr/local/bin/bankcounter
%% \end{verbatim}
%% The Names line lists the well-known connection names that the client will provide, separated by semicolons (there may also be an extra semicolon at the end). The Exec line gives the name of the program to execute in order to activate the client.

%% The service files go into a directory indicated in a <servicedir> block in the bus' configuration file; the default location is /usr/share/dbus-1/services/. If you add service files while the bus is running, the bus daemon will notice and read them without any further prodding. 

\subsection{Message-Oriented Middleware (MOM)}

\subsection{ActiveMQ}

\subsection{RabbitMQ}


\section{Complex Event Processing}
\subsection{Event processing}
\bit
\w \bb{event processing}: computing that performs operations on ``events''
\w there are 3 components in event processing applications:
  \ben
  \w [(a)] \bb{event producers}
  \w [(b)] \bb{event processors}
  \w [(c)] \bb{event consumers}
  \een
\w \bb{event processing platform} consists of:
  \ben
  \w [(a)] \bb{language} for expressing event processing logic
  \w [(b)] \bb{tools} to design/test event processing logic
  \w [(c)] \bb{runtime} to execute event processing logic
  \w [(d)] \bb{event distribution mechanism}
  \w [(e)] operational management tools
  \een
\w \bb{traditional DB vs. event processing}:
  \bit
  \w traditional DB stores ``data'' and queries are incoming
  \w event processing stores ``queries'' and  data are incoming
  \eit
\eit

\subsection{Event Processing Languages}
\bit
\w \bb{languages}:  \vspace*{0.2cm}

\centerline{\begin{tabular}{|r|r|r|} \hline
style & name & vendor \\ \hline
\bb{stream-oriented:} & Aleri & Aleri \\
& CCL & Aleri/Coral8 \\ 
& Esper & open source \\
& CQL & Open ESB IEP \\
& Oracle CEP & Orcale \\
& RTM Analyzer & RTM \\
& SPADE & IBM \\
& StreamInsight & Microsoft \\
& StreamSQL & StreamBase \\ \hline
\bb{rule-oriented:}&&\\
\bb{ECA rules}&  Amit & IBM \\
& WebSphere &\\&Business Events & IBM \\
\bb{inference rules} & TIBCO & TIBCO\\
        &     DROOLS Fusion& JBoss \\ \hline
\bb{imperative:} & MonitorSCript & Progress \\ \hline
  \end{tabular}}
\eit

\subsubsection{Event-Based Programming}
\bit
\w \bb{request--response} 
   \bit
   \w decoupled (message-based) (cf. function call in x86)
   \w essentially: \bb{distributed function call} (as in RPC, REST-style web
   service and SOAP-style services)
   \w recall \bb{message sequence charts!}
   \eit
\w \bb{event processing network (EPN)}
\w \bb{event dispatch styles}
  \bit
  \w \bb{push-style}
  \w \bb{pull-style}
  \eit
\w \bb{stateless vs stateful event processing}:
  \bit
  \w \bb{stateless event processing}: 
  \w \bb{stateful event processing}:
  \eit
\eit

\subsection{Esper}
\subsubsection{Two methods for processing events}
\bit
\w \bb{event patterns}: event pattern language is provided for specifying
expression-based event pattern matching; underlying pattern matching engine is
based on \bb{state machines}; presense/absence of events/event-combinations;
also, time-based correlation of events
\w \bb{event stream queries}: windows, joining, analysis functinos for streams
of events; follows EPL syntax
\eit


\section{Data and Databases}
\subsection{Databases and Storage}
\subsection{Data Warehouses}

\section{Cloud Computing}
\subsection{Hadoop}
\subsubsection{Hadoop core}
\subsubsection{HDFS: Hadoop file system}
\subsubsection{MapReduce}
\subsubsection{ZooKeeper: Distributed coordination service}
\subsubsection{HBase: Distributed, column-oriented database}
\subsubsection{Hive: Distributed data warehouse}
\bit
\w provides a query lanauge based on SQL on top of HDFS (i.e. over data stored
in HDFS) 
\eit
\subsubsection{Chukwa: Distributed data collection and analysis}
\subsubsection{Pig: Dataflow language}


\subsection{OpenStack}
\bit
\w 
\eit


\section{Internet of Things}
\subsection{Wireless Sensor Networks}
\subsection{Arduino}
\subsection{Chips: MCUs vs FPGAs vs CPUs}
\subsubsection{AVR}
\subsubsection{ARM}

\subsection{FitBit}


\section{Virtualization}
\subsection{Virtualization}
\subsection{Hypervisors}

\section{Case Studies}
\subsection{Google}
\subsubsection{GFS}
\subsubsection{protobuf: Protocol Buffers}
\bit
\w Google's data interchange format
\w a way of encoding structured data in an efficient yet extensible format
\w Google uses Protocol Buffers for almost all of its internal RPC protocols
and file formats
\eit
\subsection{Facebook}
\bit
\w SDKs: 
  \bit
  \w JavaScript + DOM + CSS
  \w FLV
  \w PHP, XHP
  \eit
\w Infrastructure:
  \bit
  \w Apache Cassandra (Distributed Storage)
  \w Apache Hive (Data Warehouse)
  \w FlashCache 
  \w Scribe: data aggregation
  \eit
\eit

\section{Design Patterns}
\subsection{Service Provider Framework}
\bit
\w 3 (or 4) components
  \bit
  \w \bb{service interface}
  \w \bb{provider registration API}
  \w \bb{service access API}
  \w \bb{service provider API (optional)}
  \eit
\w e.g. in JDBC
  \bit
  \w service interface: \bb{Connection}
  \w provider registration API: \bb{DriverManager::registerDriver(n)}
  \w service access API: \bb{getConnection()}
  \w service provider APi: \bb{Driver}
  \eit
\eit
\subsection{MVC Framework}

\end{document}
