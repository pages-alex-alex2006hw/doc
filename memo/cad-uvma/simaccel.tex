\section{Remodeling for Simulation Acceleration}
Code in UVC/TB uses constructs which are generally considered non-RTL.
For UVMA migration, some of the code in TB must be moved into interfaces,
which in turn, are mapped to HW. 
So, those constructs need to be remodeled so that they can be synthesized by
UXE. 

\vspace*{0.2cm}

\noindent\textcolor{red2}{NOTE: Please do not remodel for now.}

\subsection{Level-sensitive sequence controls}
The following statement
\begin{verbatim}
FROM:
  event ev;
  wait(ev.triggered);

TO:
  @(ev);
\end{verbatim}

\subsection{Clock-gating expressions}
\bit
\w \bb{CASE \#1}:
\begin{verbatim}
FROM:
  always @(posedge clk iff cond) begin
    S1
  end

TO:
  always @(posedge clk) begin
    if (cond) begin
      S1
    end
  end
\end{verbatim}

\w \bb{CASE \#2}:
\begin{verbatim}
FROM:
  initial 
    forerver
      @(posedge clk iff cond);
      S1
    end

TO:
  always @(posedge clk) begin
    if (cond) begin
      S1
    end
  end
\end{verbatim}
\eit

\paragraph{Forever statement inside tasks}
One design pattern used in many parts is the following:
\begin{verbatim}
  initial 
    fork
      foo(0);
      foo(1);
    join
 
  task foo(int v);
    forever begin
      @(posedge clk iff cond);
      if (expected[i] != values[v][i])
        $display("error");
    end
  endtask

\end{verbatim}
Above code is effectively the same as the following, where ``\verb+forerver+''
can be changed to ``\verb+always+''. (The latter is macro-based syntactic transformation while the former is more programmatic.)
\begin{verbatim}
  generate for (i = 0; i < 1; i++) 
    always @(posedge clk iff cond);
      if (expected[i] != values[v][i])
        $display("error");
    end
  endgenerate
\end{verbatim}


