\documentclass{myart}
\usepackage{times,mydef,epsfig,myenv}
\def\sbf{\sf\bfseries}
\def\Q{{\sf{}Q}}
\def\A{{\sf{}A}}
\begin{document}
\title{\large\bf Computer Organization and Design, 3rd Ed: Solutions
  \vspace*{-0.5cm}} 
\author{\normalsize Cheoljoo Jeong $\arc{\mbox{cjeong@cs.columbia.edu}}$}
\date{}
\maketitle
\small

\section{Computer Abstractions and Technology}
\bit
\w \bb{Problem 1.50.}
  \bit
  \w [a.] For a single bit to propagate to the A--B link, it takes $m/s$
  seconds. Since $R$ bits are available per second, the time between the
  beginning and end of sending a file of size $L$ bits is 
   $L/R$ seconds. 
   Therefore, end-to-end delay is $L/R + m/s$ seconds.
  \w [b.] Since single burst of $R$ bits takes $m/s + t$ seconds to travel
  over the links and the router, it requires $L/R + (m/s + t)$ seconds in
  total. 
  \w [c.] $L/R + (m/s + t/2)$ seconds.
  \eit
  
\w \bb{Problem 1.51.}
   From a single wafer, we can get 750 dies since the die yield is 50\%.
   A single wafer costs \$6000 and the cost for a single die is
   6000/750 = 8\$.
   For a single die, we need \$10 for packaging and testing with a test yield
   of 90\%, resulting in 7500\$ with a 675 functional chips.

   Thus, to get 750*0.9 = 675 dies from a single wafer, there is a cost
   6000 + 7500 = 13500\$.
   The cost for a single functional chip is 13500/657=20\$.

   Also, we have a fixed cost of \$500,000, which can be divived into the
   number of chips produced. 
   Therefore, if we produce $x$ functional chips, the cost for a single chip
   is $20$.

   Since the retail price for the chips is 40\% more than the cost, we can 
   earn $20\times 0.4 = 8$\$ by selling a single chip.
   If we want to break even by selling $x$ chips, $8\times x$ should equal
   the fixed cost \$500,000. 
   Therefore, we have to sell
      \[ x = 500,000/8 = 62,500\]
   chips.
\eit


\section{Instructions: Language of the Computer}
\bit
\w 

\eit
\end{document}
