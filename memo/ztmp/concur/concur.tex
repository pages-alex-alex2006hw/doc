\documentclass{myproc}
\usepackage{minion,hfont,mydef,amssymb,epsfig}

\def\A{\mbox{\eurm{}A}}
\def\E{\mbox{\eurm{}E}}
\def\G{\mbox{\eurm{}G}}
\def\F{\mbox{\eurm{}F}}
\def\X{\mbox{\eurm{}X}}
\def\U{\mbox{\eurm{}U}}


\begin{document}
\small
\title{\large\bf Notes on Theories of Concurrency}
\author{\normalsize ��ö�� $\arc{\mbox{cjeong@cs.columbia.edu}}$}
\maketitle

\section{BACKGROUNDS}
\subsection{Non-determinism and non-determinacy}
\bit
\w \bb{Why non-deterministic?}
   \bit
   \w Real world is non-deterministic (unpredictable). We want to model this
   phenomenon. 
   \w Non-deterministic specification allows a degree of freedom in
   implementing  the specification.  
   \eit
\w \bb{non-determinism}: For a given input, if a system has multiple possible,
and unpredictable,
external behaviors (outputs), the system is said to be \bb{non-deterministic}.
\w \bb{Determinacy}: For a given input, if a system has a deterministic
external behavior(output), the system is said to be \bb{determinate}.
The system may have internal non-deterministic behavior.
\eit

\section{TEMPORAL LOGIC}
\subsection{Kripke structure}
\bit
\w Given a set $\Sigma$ of atomic proposition constants 
   and a set $\Gamma$ of atomic variables, let $A = \Sigma \cup \Gamma$.
   $A$ is a set of atomic propositions.
\w A {\em Kripke structure\/} is a triple $M = (S, R, L)$, where $S$ is a
finite set of {\em states\/}, $R \subseteq S \times S$ is a total relation, 
and $L: S \rightarrow 2^{A}$. 
\eit
\subsection{LTL}
\subsection{CTL}
\bit
\w Given that $A$ is a set of {\em atomic propositions\/}, 
  the syntax of CTL formulas are defined as follows:
\begin{eqnarray*}
P & ::= & a \in A\ \mid\ \neg{P} \ \mid\ P \vee P \ \mid\ 
   \A{P} \ \mid\ \E{P} \ \mid\ \\
  & &  \A[P\ \U\ P]\ \mid\ \E[P\ \U\ P]
\end{eqnarray*}
\eit
\subsection{CTL*}
\bit
\w Given that $A$ is a set of {\em atomic propositions\/}, 
  the syntax of CTL* formulas are defined as follows:
\begin{eqnarray*}
S & ::= & a \in A\ \mid\ \neg{S} \ \mid\ S \vee S \ \mid\ 
   \A{P} \ \mid\ \E{P} \\
P & ::= & s \in S\ \mid\ \neg{P} \ \mid\ P \vee P \ \mid\ 
   \X{P} \ \mid\ \F{P} \ \mid\\
 & &  \G{P}\ \mid\ P\ \U\ P
\end{eqnarray*}
  $s \in S$ is called a {\em state formula\/} and $p \in P$ is called a {\em
  path formula}
\eit


\subsection{Fixpoint characterization}
\bit
\w Let $M = (S, R, L)$ be a finite Kripke structure.
\w $Pred(S)$ denotes the lattice $\arc{P, <_P}$ of predicates over $S$
   \bit
   \w $p \in P$: a predicate over $S$
   \w $\denote{p}$: set of states in $S$ that satisfies $p$, i.e.
       the set of states in which $p$ evaluates to true
   \w $p_1 <_P p_2$ iff $\denote{p_1}_M \subseteq \denote{p_2}_M$
   \eit
\w A functional $\tau: Pred(S) \rightarrow
Pred(S)$, that maps $Pred(S)$ to $Pred(S)$ is called a {\em predicate
  transformer}. 
\w A predicate transformer $\tau: Pred(S) \rightarrow Pred(S)$ is 
  {\em monotonic\/} if
 $P \subseteq Q$ implies $\tau[P] \subseteq \tau[Q]$. 
\w  A predicate transformer $\tau$ is 
  {\em $\bigcup$-continuous} if 
  $P_1 \subseteq P_2 \subseteq \cdots \mbox{\ implies\ } 
  \tau[\bigcup_iP_i] = \bigcup_i\tau[P_i]$. 
\w A predicate transformer $\tau$ is 
  {\em $\bigcap$-continuous} if 
  $P_1 \supseteq P_2 \supseteq \cdots \mbox{\ implies\ } 
  \tau[\bigcap_iP_i] = \bigcap_i\tau[P_i]$. 
\w \bb{Tarski's theorem}: A monotonic predicate transformer $\tau$ on
$Pred(S)$ always has a least fixpoint and a greatest fixpoint.


\w A temporal logic formula $p$ with free variables $x_1, \cdots, x_n$ can be
interpreted as a mapping $p^{M}: (Pred(S))^n \rightarrow Pred(S)$. 
\eit

\section{HOARE'S CSP}
\subsection{Introduction}
\bit
\w There are three usages of CSP:
  \ben
  \w [(a)] as a notation for writing {\em programs\/} which are close to
  implementations,
  \w [(b)] a way of constructing {\em specifications\/} 
  which may be remote from
  implementation, and
  \w [(c)] as a {\em calculus for reasoning\/} about both of the above things.
  \een
\w Fundamental assumptions about {\em communications\/} in CSP are:
  (a) They are {\em instantaneous} and (b) They occur {\em if and only if\/} 
  both the process and the environment allow them.
\eit

\subsection{CSP operators}
\bit
\w \bb{prefixing}: given an event $a \in \Sigma$ and a process $P$, 
  $a \rightarrow P$; {\em indefinitely waits for $a$ to happen, and after 
  $a$ happens, behaves like $P$}
\w \bb{recursion}: given a process $P$, 
$\mu P.F(P)$ denotes the fixed point of $P = F(P)$, i.e. 
`the' $P$ that satisfies $P = F(P)$.
\w \bb{guarded alternative}: $(a_1 \rightarrow P_1\ |\ \cdots\ |\ 
a_n \rightarrow P_n)$ gives a \bb{choice} of actions to their environment;
{\em it can do any of the events $a_1, \cdots, a_n$ on its first step, and
subsequently behaves like $P_r$, if $a_r$ was chosen} 
\w \bb{prefix choice}: $?x:A \rightarrow P(x), A \subseteq \Sigma$ 
is a process which accepts any element $a \in A$ and then behaves like
$P(a)$
  \bit
  \w $?x:\{\} \rightarrow P(x) \equiv STOP$
  \w $?x:\{a\} \rightarrow P(x) \equiv a \rightarrow P(a)$
  \w $c?x: T \rightarrow P(x)$: wait for $x$ on \bb{channel} $c$
  \eit
\w \bb{external choice}: $P\ \square\ Q$ is a process which offers the
environment the choice of the first events of $P$ and $Q$ and then behaves
accordingly
   \bit
   \w $(a \rightarrow P\ |\ b \rightarrow Q) \equiv
       (a \rightarrow P\ \square\ b \rightarrow Q)$
   \eit
\w \bb{nondeterministic choice}: $P \sqcap Q$ {\em can\/} behave like either $P$ or $Q$
  and $\sqcap S$ can behave like any member of $S$
\eit
\subsection{More on CSP}
\bit
\w If $R$ is such that $R = R \sqcap P$ we say that $P$ is more deterministic
than $R$, or that $P$ \bb{refines} $R$. 
 When $R = R \cap P$, we write $R \sqsubseteq P$.
\eit

\subsection{Denotational semantics: The traces model}
\bit
\w 
\eit
\section{MILNER'S CCS}
\bit
\w 
\eit
\section{PETRI NETS}
\subsection{Definitions}
\bit
\w A {\em Petri net} is a $5$-tuple $PN = (P, T, F, W, M_0)$ where
  \bit
  \w $P = \{p_1, \cdots, p_m\}$ is a finite set of {\bf places},
  \w $T = \{t_1, \cdots, t_n\}$ is a finite set of {\bf transitions},
  \w $F \subseteq (P\times{T}) \cup (T\times{P})$ is a set of {\bf arcs} (flow
  relation),
  \w $W: F \rightarrow \Z^+$ is a {\bf weight function}, and
  \w $M_0: P \rightarrow \N$ is a the {\bf initial marking}, where
   $P \cap T = \emptyset$ and $P \cup T \not= \emptyset$.
  \eit
\w Each place represents a kind of a \bb{condition}.
\w Each transition represents a kind of a \bb{action}.
\w If all \bb{input places} of a transition have appropriate number (weight)
of tokens (if all \bb{preconditions} are satisfied), the transition is
\bb{fired} (action activated), and then \bb{output places} get the tokens
(\bb{postconditions} are satisfied).
\w A marking of a Petri net represents a \bb{state} of the system being
modeled. Or, {\em the state of a system is a set of conditions (places) which
  are true!} 
\w Two transitions are \bb{persistent w.r.t each other} if when both are
\bb{enabled}, the {\em firing of one does not disable the other\/}.
\eit

\subsection{Modeling idioms}
\bit
\w \bb{Sequencing}: trivial
\w \bb{Parallel composition}: just place them side by side
\w \bb{Fork/Join}: concurrency; JOIN=SYNCHRONIZATION
\w \bb{Choice/Merge}: non-determinism

 \vspace*{0.3cm}

  \centerline{\epsfig{figure=pics/petri-fork.eps,width=5cm}}
\eit
\subsection{Variants of Petri nets}
\bit
\w \bb{Timed Petri nets}: times (delays OR lower/upper bounds of delays) are
associated with places/transitions/arcs
\w \bb{Colored Petri nets (= typed Petri nets)}: places can have \bb{types}
and tokens are \bb{types}; type checking is done
\w Extension of Petri nets allows more expressive power but makes interesting
problems undecidable.
\w \bb{State machines}: each transition has exactly 1 input and 1 output
places; preserves the \# of tokens
  \bit
  \w supports choice/merge, does NOT support fork/join
  \w only for \bb{non-deterministic, sequential} systems (single thread of
  control) 
  \eit
\w \bb{Marked graphs}: each place has exactly 1 input and 1 output (so, we can
OMIT transitions)
  \bit
  \w supports fork/join, does NOT support choice/merge
  \w only for \bb{deterministic, concurrent} systems (multiple thread of
  control) 
  \eit
\w \bb{Free-choice Petri nets}:
\w \bb{STG (signal transition graphs)}: 
   \bit
   \w \bb{input-free choice}: selection among alternatives must only be
   controlled by mutually exclusive inputs (\bb{NO non-determinism})
   \w \bb{1-bounded}: at most one token in each place
   \w \bb{liveness}: no deadlock
   \eit
\eit
\subsection{Properties of Petri nets}
\bit
\w \bb{Safeness}: A net with initial marking $M_0$ is \bb{safe} IFF,
   in any marking reachable from $M_0$, every place contains no more than one
   token. 
\w \bb{Liveness}: A {\em finite safe\/} net is \bb{live} IFF its reachability
graph is strongly conntecd and each transition in $T$ is enabled in some
marking of the reachability grapyh. 
\eit

\section{Kahn Network}



\bibliographystyle{plain}
\bibliography{00bib/mac,00bib/theory,00bib/formal,00bib/pl,00bib/async}
\nocite{Reynolds98,Murata89,Roscoe98,CGP99,Kahn74}
\end{document}
