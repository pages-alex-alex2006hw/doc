\documentclass{article}
\usepackage{hfont,theorem,mydef,latexsym,amssymb}
%\usepackage{theorem,mydef,latexsym,amssymb}
\def\hm{{\sf{}Halting$_{\mbox{\scriptsize{}TM}}$}}
\def\am{{\sf{}Accepts$_{\mbox{\scriptsize{}TM}}$}}
\def\etm{{\sf{}AcceptNothing$_{\mbox{\scriptsize{}TM}}$}}
\begin{document}
\title{\Large\bf{}Notes on Computability Theory}
\author{�� ö��}
\maketitle

\section{Turing Machines}
\subsection{The Turing machine model}
\bit
\w A {\bf{}Turing machine} is a $7$-tuple 
	$M = (Q, \Sigma, \Gamma, \delta, q_0, q_A, q_R)$, where
	$Q, \Sigma, \Gamma$ are finite sets and
	\ben
	\w [(a)] $Q$ is the set of {\bf{}states},
	\w [(b)] $\Sigma$ is the {\bf{}input alphabet} not containing
		the {\bf{}blank symbol}, $\Box$,
	\w [(c)] $\Gamma$ is the {\bf{}tape alphabet}, where
		\[\{\Box\} \in \Gamma \mbox{\ and\ } 
			\Sigma \subseteq \Gamma,\]
	\w [(d)] $\delta: Q\times\Gamma \rightarrow 
		Q\times\Gamma\times\{L, R\}$
		is the {\bf{}transition function},
	\w [(e)] $q_0 \in Q$ is the {\bf{}start state},
	\w [(f)] $q_A \in Q$ is the {\bf{}accept state}, and
	\w [(g)] $q_R \in Q$ is the {\bf{}reject state}, where $q_A \ne q_R$.
	\een
\w A Turing machine $M$ receives its input $w = w_1w_2\cdots{w_n} \in 
	\Sigma^*$ on the leftmost $n$ squares of the tape, and the rest
	of the tape is blank.
	\bit
	\w The computation continues until it enters either the accept
		or reject states at which point it halts. 
	\w If neither occurs, $M$ goes on forever. A {\bf{}loop}!\footnote{not
		in the sence of `cycle'}
	\eit
\w A {\bf{}configuration}\footnote{or {\bf{}instantaneous description}} 
	of $M$ is defined to be $uqv$ where $q \in Q$ and $u, v \in \Gamma^*$.
	\bit
	\w $q$ is the {\bf{}current state}.
	\w $u, v$ indicates the {\bf{}current tape contents} and the 
		{\bf{}current head position}.
	\w $car(v)$ is the content of the table under the tape head.
	\eit
\w A configuration $C_1$ {\bf{}yields} configuration $C_2$, denoted
	$C_1 \vdash C_2$, if the Turing 
	machine can legally go from $C_1$ to $C_2$ in a single step. I.e.,
	for $u, v \in \Gamma^*$, $q_i, q_j \in Q$, and $a, b \in \Gamma$,
	\bit
	\w $uaq_ibv \vdash uq_jacv$ if $\delta(q_i, b) = (q_j, c, L)$.
	\w $uaq_ibv \vdash uacq_jv$ if $\delta(q_i, b) = (q_j, c, R)$.
	\eit
\w The {\bf{}start configuration} of $M$ on input $w$ is the configuration
	$q_0w$.
\w In an {\bf{}accepting configuration}, the state of configuration is
	$q_A$.
\w In a {\bf{}rejecting configuration}, the state of configuration is
	$q_R$.
\w Accepting and rejecting configurations are called
	{\bf{}halting configurations} and do not yield further computation.
\w A {\sl\bfseries{}Turing machine accepts input $w$\/} 
	if a sequence of configurations
	$C_1, C_2, \cdots, C_k$ exists s.t.
	\ben
	\w [(a)] $C_1$ is the start configuration of $M$ on input $w$,
	\w [(b)] $C_i \vdash C_{i+1}$, $1 \le i < k$, and
	\w [(c)] $C_k$ is an {\em{}accepting\/} configuration.
	\een
\w The {\bf{}language of $M$}, $L(M)$, is the
	set of string that $M$ {\em{}accepts\/}. 
\w A language $L$ is {\bf{}Turing-acceptable} or 
	{\bf{}recursively enumerable} ({\bf{}R.E.})
	if some Turing machine {\bf{}accepts}\footnote{Sipser prefers to
	say `recognizes' lest `accepts' is overloaded with two meanings:
	$M$ accepts $w$ and $M$ accepts $L$.} it.
	\bit
	\w In this case, $M$ may not halt for some inputs.
	\w Given a class $C$ of languages, 
		\[ \mbox{co-}C = \{L \subseteq \Sigma^*: 
		\overline{L} \in C\}.\]
	\w A language $L$ is {\bf{}co-R.E.} or {\bf{}co-Turing-acceptable}
		if some Turing machine
		accepts $\overline{L}$; that is, $\overline{L}$ is
		r.e. or Turing-acceptable.
	\eit
\w A language $L$ is {\bf{}Turing-decidable} or {\bf{}recursive}
	if some Turing machine {\bf{}decides} it.
	\bit
	\w For a decidable language $L$, there exists $M$ s.t.
		$M$ always halts (accepts or rejects) on inputs. 
	\w That is,
		$M$ always gives an answer to the question ``$w \in L$?''
	\eit
\w The following three `classes' of languages can be defined from the
	above definition:
	\begin{eqnarray*}
	\mbox{R}  & = & \{L: L\mbox{\ is Turing-decidable}\}\\
	\mbox{R.E.} & = & \{L: L\mbox{\ is Turing-acceptable}\} \\
	\mbox{co-R.E.} & = & \{L: \overline{L}\mbox{\ is Turing-acceptable}\} 
	\end{eqnarray*}
\eit

\subsection{Linear bounded automata}
\bit
\w A {\bf{}linear bounded automaton (LBA)} is a restricted type of Turing 
	machine wherein the tape head isn't permitted to move off the
	portion of the tape containing the {\em{}input\/}.
	\bit
	\w An LBA is a Turing machine with a limited amount of memory.
	\w LBAs are quite powerful as we shall see in that
	deciders for 
	{\sf{}Accepts$_{\mbox{\scriptsize{}DFA}}$},
	{\sf{}Accepts$_{\mbox{\scriptsize{}CFG}}$},
	{\sf{}AcceptsNothing$_{\mbox{\scriptsize{}DFA}}$}, and
	{\sf{}AcceptsNothing$_{\mbox{\scriptsize{}CFG}}$} are 
	LBAs.
	\eit
\w Let $M$ be an LBA with $q$ states and $g$ symbols in the tape alphabet.
	There are exactly $qng^n$ distinct configurations of $M$ for a tape
	of length $n$.
\eit

\subsection{Enumerators}
\bit
\w An {\bf{}enumerator} is a Turing machine variant which
	prints a possibly infinite list of string.
	\bit
	\w $E$ may generate the strings of the language in any order,
		{\em{}possibly with repetitions\/}.
	\eit
\w A {\bf{}language enumerated by $E$} is the collection of
	all strings that $E$ eventually prints out.
\w {\sl\bfseries{}A language $L$ is 
	Turing-acceptable (r.e.) iff some enumerator enumerates $L$\/}.
	\bit
	\w Proof of $(\Rightarrow)$ uses {\bf{}Cantor's first diagonal
		method\/} (a.k.a {\bf{}dovetailing}).
		\bit
		\w Let $M$ be a acceptor for $L$; then design $M'$ such that
			$M'$ simulates $M$ on all possible inputs 
			$w_1, w_2, \cdots$.
		\w This cannot be done just with `time-sharing' since 
			the possible inputs are infinite; hence the 
			dovetailing! 
		\eit
	\eit
\w {\sl\bfseries{}A language $L$ is 
	recursive iff some enumerator enumerates $L$ in
	canonical order.\/}
	\bit
	\w Proof of $(\Leftarrow)$ shows some difficulty: when $L$
		is finite, though we have an enumerator, 
		we have no way of building a Turing
		machine that decides $L$. 
	\w (cont.) Let $E$ be a enumerator and
		suppose that we want to build a TM $M$ that decides $L$
		using $E$ as a subroutine. 
	\w (cont.) That is given a string $w$, we want a TM $M$ that decides
		whether $w \in L$ by simulating $E$ to see where $E$ generates
		$w$.
	\w (cont.) If $L$ is infinite, the above strategy works, since 
		there are infinitely many strings in $L$ that comes 
		canonically after $w$:
		\[ \cdots, w, w', w'', \cdots. \]
		Since $w'$ should be generated by $E$, so should $w$.
	\w (cont.) Suppose that $L$ is {\em{}finite\/} 
		and $w_0$ is the last string
		in $L$. Given a input string $w$ 
		that may come canonically after $w_0$,
		we do not know whether $w$ is generated by $L$ or not.
	\w (cont.) Unfortunately, {\em{}we cannot determine whether $E$
		generates a finite set or, if finite, which finite set
		it is\/}. Hence we do not know what's the last string 
		$w_0$ generated by $E$.
			
	\eit
\eit


\subsection{Basic properties of recursive and r.e. sets}
\bit
\w {{}If $L$ is Turing-decidable then $L$ is Turing-acceptable\/}.
\w {{}The set of Turing-decidable languages is closed under
	{\em{}complementation\/}.}
	\bit
	\w I.e. if $L$ is recursive, then $\overline{L}$ is recursive.
	\w The set of Turing-decidable languages is closed under
		{\em{}union\/} and {\em{}intersection\/}.
	\eit
\w {{}$L$ is Turing-decidable iff both $L$ and $\overline{L}$ is 
	Turing-acceptable\/}.
	\bit
	\w In other words, 
		$L$ is decidable iff $L$ is both Turing-acceptable and
		co-Turing-acceptable.
	\eit
\w {{}The set of Turing-acceptable languages is {\sl\bfseries{}not\/}
	closed under complementation\/}.
		\bit
		\w So, Turing-acceptability of $L$ does not
			imply the Turing-decidability of $L$.
		\w {\em{}The set of Turing-acceptable languages is closed under
			union and intersection\/}.
		\eit
\w As a consequence of above theorems, 
	given a pair of complementary languages, $L$ and $\overline{L}$,
	only one of the three following condition holds:
	\ben
	\w [(a)] both $L$ and $\overline{L}$ is recursive
	\w [(b)] neither $L$ nor $\overline{L}$ is r.e.
	\w [(c)] one of $L$ and $\overline{L}$ is r.e. and the other is
		not r.e. (the r.e. one is not recursive, since the 
		complementary
		one would also be recursive otherwise)
	\w [(*)] {\em{}So, given $L$, if we could prove that $\overline{L}$ is
		not r.e., we know that $L$ is either non-r.e. or
			`r.e. but not recursive' by (b) and (c); which proves
			the {\sl\bfseries{}undecidability} of $L$}.
	\een
\w Let $A$ and $B$ be two disjoint languages. The language $C$
	{\bf{}separates} $A$ and $B$ if $A \subseteq C$ and
	$B \subseteq \overline{C}$. 
	\bit
	\w Any two disjoint co-Turing-acceptable languages are separable
	by some decidable language.
	\eit
\eit

\section{Decidable Languages}
\subsection{Decidable problems concerning regular languages}
\bit
\w ({\bf{}The Acceptance Problem for DFAs})
	\bit
	\w {\sf{}Accepts$_{\mbox{\scriptsize{}DFA}}$} $=
		\{\arc{B, w}:$ $B$ is a DFA that accepts input string $w\}$.
	\w {\bf{}Problem}: Does a DFA $B$ accept input string $w$?
	\w {\em{}The problem of testing whether a DFA $B$ accepts an input
		string $w$ is the same as the problem of testing
		whether $\arc{B, w}$ is a member of the language
		{\sf{}Accepts$_{\mbox{\scriptsize{}DFA}}$}}.
	\w {\sf{}Accepts$_{\mbox{\scriptsize{}DFA}}$} is decidable.
	\eit
\w ({\bf{}The Acceptance Problem for NFAs})
	\bit
	\w {\sf{}Accepts$_{\mbox{\scriptsize{}NFA}}$} $=
		\{\arc{B, w}:$ $B$ is a NFA that accepts input string $w\}$.
	\w {\bf{}Problem}: Does an NFA $B$ accept input string $w$?
	\w {\sf{}Accepts$_{\mbox{\scriptsize{}NFA}}$} is decidable.
	\w Note that an equivalent DFA can be constructed from an NFA
		by {\em{}subset construction\/}. That is, 
		{\sf{}Accepts$_{\mbox{\scriptsize{}NFA}}$} is reducible
		to {\sf{}Accepts$_{\mbox{\scriptsize{}DFA}}$}.
		
	\eit
\w ({\bf{}The Acceptance Problem for Regular Expressions})
	\bit
	\w {\sf{}Accepts\footnote{Perhaps the more plausible name for this
	problem would be 
	{\sf{}Generates$_{\mbox{\scriptsize{}REX}}$}.}$_{\mbox{\scriptsize{}REX}}$} $=
		\{\arc{R, w}:$ $R$ is a regular expressions that generates
		$w\}$.
	\w {\bf{}Problem}: Does a regular expression $R$ generate $w$?
	\w {\sf{}Accepts$_{\mbox{\scriptsize{}REX}}$} is decidable.
	\w Note that an equivalent DFA can be constructed from a regular
		expression by {\em{}Thompson's construction\/}.
	\eit
\w ({\bf{}The Emptiness Problem for DFAs})
	\bit
	\w {\sf{}AcceptsNothing$_{\mbox{\scriptsize{}DFA}}$} $=
		\{\arc{B}:$ $B$ is a DFA and $L(B) = \emptyset\}$.
	\w {\bf{}Problem}: Does a DFA accept nothing?
	\w {\sf{}Accepts$_{\mbox{\scriptsize{}DFA}}$} is decidable.
	\w A DFA accepts some string iff reaching an `accepting state'
		from the start state is possible, which can be found
		by well-known graph traversal algorithms.
	\eit
\w ({\bf{}The Equivalence Problem for DFAs})
	\bit
	\w {\sf{}Equiv$_{\mbox{\scriptsize{}DFA}}$} $=
		\{\arc{A, B}:$ $A, B$ are DFAs and $L(A) = L(B)\}$.
	\w {\bf{}Problem}: Do DFAs $A$ and $B$ accept the same language?
	\w {\sf{}Equiv$_{\mbox{\scriptsize{}DFA}}$} is decidable.
	\w Since regular sets are closed under union and complementation, 
		\[ L(C) = (L(A) \cap \overline{L(B)}) \cup
		(\overline{L(A)} \cap L(B))\]
		is {\em{}nonempty\/} iff $L(A) \ne L(B)$.
	\w That is, given that $C$ is a DFA that accepts $L(C)$,
		{\sf{}Equiv$_{\mbox{\scriptsize{}DFA}}$} is reducible
		to the problem:
		\[ \arc{C} \in \mbox{\ \sf{}AcceptsNothing$_{\mbox{\scriptsize{}DFA}}$}.\]
	\eit
\eit

\subsection{Decidable problems concerning context-free languages}
\bit
\w ({\bf{}The Acceptance Problem for CFGs})
	\bit
	\w {\sf{}Accepts$_{\mbox{\scriptsize{}CFG}}$} $=
		\{\arc{G, w}:$ $G$ is a CFG that generates string $w\}$.
	\w {\bf{}Problem}: Does a CFG $G$ generate $w$?
	\w {\sf{}Accepts$_{\mbox{\scriptsize{}CFG}}$} is decidable.
	\w If $G$ were in {\em{}Chomsky-normal form\/}, any derivation of
		$w$ has $2n-1$ steps if $|w| = n$. Note that any CFG $G$
		can be converted to Chomsky-normal forms.
	\eit
\w ({\bf{}The Emptiness Problem for CFGs})
	\bit
	\w {\sf{}AcceptsNothing$_{\mbox{\scriptsize{}DFA}}$} $=
		\{\arc{G}:$ $G$ is a CFG and $L(G) = \emptyset\}$.
	\w {\bf{}Problem}: Does a CFG generate nothing?
	\w {\sf{}AcceptsNothing$_{\mbox{\scriptsize{}CFG}}$} is decidable.
	\w This problem can be restated as that of 
		determining whether a start nonterminal can
		derive a non-$\epsilon$ string.
	\eit
\eit

\subsection{Decidable problems concerning linear bounded automata}
\bit
\w ({\bf{}The Acceptance Problem for LBAs})
	\bit
	\w {\sf{}Accepts$_{\mbox{\scriptsize{}LBA}}$} $= 
		\{\arc{M, w}:$ $M$ is an LBA, $w \in \Sigma^*$, and
			$M$ accepts $w\}$.
	\w {\bf{}Problem}: Does the LBA $M$ accept $w$?
	\w Simulate $M$ on $w$; if it halts and accepts/rejects, that's
		fine. The difficulty occurs if $M$ loops on $w$.
	\w (cont.) In case of loop, we need to detect looping so that
		we can halt and reject.
	\w (cont.) Since, $q$-state, $g$-symbol LBA with input string
		of length $n$ has exactly $qng^n$ distinct configurations,
		we know that if $M$ does not halt in $qng^n$ steps, 
		it is in a loop!
	\eit
\eit

\subsection{Relationship among classes of languages}
\bit
\w {\sl\bfseries{}Every context-free languages is decidable\/}.
\w The relationship on language class containment:
	\[\mbox{Regular $\subseteq$ Context-free $\subseteq$
	Decidable $\subseteq$ Turing-acceptable}\]
\eit


\section{Undecidability}
\subsection{Two main techniques for proving undecidability of languages}
\bit
\w \fbox{\sl\bfseries{}Diagonalization}
\w \fbox{\sl\bfseries{}Many-one reducibility}
\eit


\subsection{Proving the undecidability of the halting problem by 
	diagonalization}
\bit
\w {\sl\bfseries{}{\sf{}Halting$_{\mbox{\scriptsize{}TM}}$} is 
	not recursive (undecidable)}.
	\ben
	\w {\sf{}Halting$_{\mbox{\scriptsize{}TM}}$} $=
		\{\arc{M, w}:$ $M$ is a TM, $w \in \Sigma^*$, and
		$M$ halts on $w\}$.
	\w To prove this, we will prove that the related language $K$ is not
		recursive.
		\[ K =
		\{\arc{M}: \mbox{\ $M$ is a TM and $M$ halts on \arc{M}}\}.\]
		\bit
		\w $K$ is a special case of 
			{\sf{}Halting$_{\mbox{\scriptsize{}TM}}$}.
		\w So, if $K$ is undecidable then the more general version,
			{\sf{}Halting$_{\mbox{\scriptsize{}TM}}$} is
			also undecidable.
		\w That is, $K \le_m$ {\sf{}Halting$_{\mbox{\scriptsize{}TM}}$}.
		\w Now, what's left is to prove {\sl\bfseries{}$K$ is not
			recursive\/}.
		\eit
	\w {\sl\bfseries{}To prove that $K$ is not recursive, we prove
		that $\overline{K}$ is not r.e.\/}
		\bit	
		\w This is due to the fact that only one of the following
			condition holds:
			\ben
			\w [(a)] both $K$ and $\overline{K}$ is recursive;
			\w [(b)] neither $K$ nor $\overline{K}$ is r.e.
			\w [(c)] one of $K$ and $\overline{K}$ is r.e.
				and the other is non-r.e.; the non-r.e.
				one is not recursive.
			\een
		\w $\overline{K}$ is defined as:
		\[ \overline{K} =
			\{\arc{M}: \mbox{\ $M$ is a TM and $M$ does not halt 
			on \arc{M}}\}.\]
		\eit
	\w {\sl\bfseries{}Now, let's prove that $\overline{K}$ is not r.e.\/}
		\bit
		\w Suppose, to the contrary, that $\overline{K}$ is 
			Turing-acceptable.
		\w Then there exists a TM $M_0$ that halts exactly on
			the encodings of the elements of $\overline{K}$.
		\w That is, $M_0$ halts on $\arc{M}$ exactly when
			$M \in \overline{K}$.
		\w In other words, {\sl\bfseries{}$M_0$ halts exactly when 
			$M$ does not halt on the encoding of itself!\/}
		\eit
	\w Consider what $M_0$ does on ``its own'' encoding. That is,
		{\sl\bfseries{}let $M_0$ run with $\arc{M_0}$ as its input.\/}
		\bit
		\w {\sl\bfseries{}$M_0$ halts on $\arc{M_0}$ exactly
		when $M_0$ does not halt on $\arc{M_0}$.}
		\w {\em{}A contradiction!\/}
		\eit
	\een
\w ``Diagonalization'' used in the above proof (step 4 and 5)
	\bit
	\w Let $M_1, M_2, \cdots$ be a canonical ordering of {\em{}all\/}
		Turing machines.
	\w Let's build a 2-dimensional table $K^c$ indexed by Turing machines
		and its encodings where
		\[ K^c(i, j) = \left\{\begin{array}{ll}
				1, & \mbox{if\ } M_i \mbox{\ halts on\ } \arc{M_j},\\
				0, & \mbox{if\ } M_i \mbox{\ doesn't halt on\ } \arc{M_j}.
				\end{array}\right. \]
	\w We have defined $M_0$ so that $M_0(\arc{M_i})$ halts
		iff $M_i(\arc{M_i})$ does not halt.
	\w Thus the behavior of $M_0$ differs from the $i$th TM at
		the $(i, i)$th entry in the table.
	\eit
\eit

\subsection{Many-one reducibility}
\paragraph{Basic properties}
\bit
\w If $A \subseteq \Sigma^*$ and $B \subseteq \Sigma^*$ are two
	languages, we say that $A$ is {\bf{}many-one reducible}\footnote{or 
	{\bf{}mapping reducible}} to $B$,
	written $A \le_m B$	if there exists some (total) function
	$f: \Sigma^* \rightarrow \Sigma^*$ which is Turing-computable
	s.t. for all strings $x \in \Sigma^*$, $x \in A$ iff
	$f(x) \in B$.
\w {\sl\bfseries{}If $A \le_m B$ and $B$ is decidable then
	$A$ is decidable.}
	\bit
	\w {\sl\bfseries{}If $A \le_m B$ and $A$ is undecidable then
		$B$ is undecidable.}
	\eit
\w $A$ is {\bf{}many-one equivalent} to $B$, written $A =_m B$
	iff $A \le_m B$ and $B \le_m A$.
\w If $A \le_m B$ and $B$ is r.e. then $A$ is r.e.
	\bit
	\w If $A \le_m B$ and $A$ is not r.e. then $B$ is not r.e.
	\eit
\w $A \le_m B$ iff $\overline{A} \le_m \overline{B}$.
\w {\sl\bfseries{}If $A$ is r.e. but not recursive, then
	$A$ and $\overline{A}$ are `$\le_M$-incomparable';
	that is, neither $A \le_m \overline{A}$ nor
	$\overline{A} \le_m A$.}
\eit
\paragraph{Reductions via computation histories}
\bit
\w Given a Turing machine $M$ and an input string $w$,
	an {\bf{}accepting history for $M$ on $w$} is a sequence
	of configurations, $C_1, C_2, \cdots, C_l$, where $C_1$ is the
	start configuration of $M$ on $w$, $C_l$ is an accepting 
	configuration of $M$, and each $C_i$ legally follows from $C_{i-1}$.
\w A {\bf{}rejecting computation history for $M$ on $w$} is 
	defined similarly, except that
	$C_l$ is a rejecting configuration
\w Computation histories are finite sequences.
\w Reduction via computation histories was used 
	to prove the undecidability of Hilbert's Tenth Problem.
\eit

\subsection{Rice's theorem}
\bit
\w As we shall see, 
	{\sf{}Regular$_{\mbox{\scriptsize{}TM}}$} or 
	{\sf{}ContextFree$_{\mbox{\scriptsize{}TM}}$} is undecidable;
	in fact, Rice's Theorem states that {\sl\bfseries{}testing 
	`any' property
	of the languages accepted by Turing machines is undecidable\/}.
\w ({\bf{}Rice's Theorem})
	{\sl\bfseries{}Let 
	$P$ be any problem about Turing machines that satisfies
	the following two properties. As usual we express $P$ as a 
	language.
	\ben
	\w [\bf{}(a)] For any TMs $M_1$ and $M_2$, where $L(M_1) = L(M_2)$,
		we have $\arc{M_1} \in P$ iff $\arc{M_2} \in P$. 
	\w [\bf{}(b)] There exists TMs $M_1$ and $M_2$, where 
		$\arc{M_1} \in P$ and $\arc{M_2} \not\in P$. 
	\een
	Then, $P$ is undecidable.}
\w Condition (a) states that 
	the membership of a TM $M$ in $P$ depends only
	on the language of $M$.
\w Condition (b) states that
	$P$ is nontrivial -- it holds for some, but not all, TMs.
\w Both the conditions (a) and (b) are {\em{}necessary\/} for Rice's
	Theorem.
\eit






\section{Compendium of Undecidable Languages}
\subsection{Undecidable languages concerning context-free languages}
\bit
\w ({\bf{}The Totality Problem for CFGs})
	\bit
	\w {\sf{}AcceptsAll$_{\mbox{\scriptsize{}CFG}}$} $=
		\{\arc{G}:$ $G$ is a CFG and $L(G) = \Sigma^*\}$.
	\w {\bf{}Problem}: Does a CFG $G$ generate all strings?
	\w {\sf{}AcceptsAll$_{\mbox{\scriptsize{}CFG}}$} is undecidable.
	\eit
\w ({\bf{}The Equivalence Problem for CFGs})
	\bit
	\w {\sf{}Equiv$_{\mbox{\scriptsize{}CFG}}$} $=
		\{\arc{G, H}:$ $G, H$ are CFGs and $L(G) = L(H)\}$.
	\w {\bf{}Problem}: Do CFGs $G$ and $H$ generate the same language?
	\w {\sf{}Equiv$_{\mbox{\scriptsize{}CFG}}$} is undecidable.
	\w Since context-free languages are {\em{}not\/} closed under 
		complementation, we {\em{}cannot\/} use the following:
		\[ L(I) = (L(G) \cap \overline{L(H)}) \cup
		(\overline{L(G)} \cap L(H)).\]
	\eit
\eit
\subsection{Undecidable problems concerning linear bounded automata}
\bit
\w ({\bf{}The Emptiness Problem for LBAs})
	\bit
	\w {\sf{}AcceptsNothing$_{\mbox{\scriptsize{}LBA}}$} $=
		\{\arc{M}:$ $M$ is an LBA and $L(G) = \emptyset\}$.
	\w {\bf{}Problem}: Does an LBA $M$ accept nothing?
	\w {\sf{}AcceptsNothing$_{\mbox{\scriptsize{}LBA}}$} is undecidable.
	\w \am\ $\le_m$ {\sf{}AcceptsNothing$_{\mbox{\scriptsize{}LBA}}$}
	\eit
\eit
\subsection{Undecidable problems concerning Turing machines}
\bit
\w ({\bfseries{}The Acceptance Problem for Turing Machines})
	\bit
	\w {\sf{}Accepts$_{\mbox{\scriptsize{}TM}}$} $= 
		\{\arc{M, w}:$ $M$ is a TM, $w \in \Sigma^*$, and
			$M$ accepts $w\}$.
	\w {\bf{}Problem}: Does $M$ accept $w$?
%	\w {\sf{}Halting$_{\mbox{\scriptsize{}TM}}$} $\equiv$
%		{\sf{}Accepts$_{\mbox{\scriptsize{}TM}}$} 
	\w {\sf{}Accepts$_{\mbox{\scriptsize{}TM}}$} is r.e. but not
		decidable.
		\bit
		\w This implies that $\overline{\mbox{\am}}$
			is not r.e. since otherwise \am\
			would be decidable.
		\eit
	\w {\sl\bfseries{}Proof of the undecidability of
	the acceptance problem\/}:\\
	Assume that \am\ is Turing-decidable and let $H$ be a decider for 
	\am. Then
	\ben
	\w $H$ halts and accepts if $M$ accepts $w$.
	\w $H$ halts and rejects if $M$ doen not accept $w$.
	\een
	That is,
	\[H(\arc{M, w}) = \left\{\begin{array}{ll}
		accept & \mbox{if $M$ accepts $w$},\\
		reject & \mbox{if $M$ does not accept $w$}.
			    \end{array}\right.
	\]
	Now, let's construct a TM $D$ with input $\arc{M}$
	which uses $H(\arc{M, \arc{M}})$ as its subroutine\footnote{Note that
	this argument assumes that a TM can simulate the machine 	
	`described by' $\arc{M}$.}, where
	\[D(\arc{M}) = \left\{\begin{array}{ll}
		accept & \mbox{if $M$ does not accept $\arc{M}$},\\
		reject & \mbox{if $M$ accepts $\arc{M}$}.
			    \end{array}\right.
	\]
	Finally, run $D$ with its own description $\arc{D}$ as its input.
	Then, we get
	\[D(\arc{D}) = \left\{\begin{array}{ll}
		accept & \mbox{if $D$ does not accept $\arc{D}$},\\
		reject & \mbox{if $D$ accepts $\arc{D}$}.
			    \end{array}\right.
	\]
	This is a contradiction.
%	}.
	\eit
\w \fbox{\bfseries{}The Halting Problem for Turing Machines}
	\bit
	\w {\sf{}Halting$_{\mbox{\scriptsize{}TM}}$} $= 
		\{\arc{M, w}:$ $M$ is a TM, $w \in \Sigma^*$, and
			$M$ halts on $w\}$.
	\w {\bf{}Problem}: Does $M$ halt on $w$?
	\w \am\ $\le_m$ \hm
	\eit
\w ({\bf{}The Hanging Problem for Turing Machines})
	\bit
	\w {\sf{}Hanging$_{\mbox{\scriptsize{}TM}}$} $= 
		\{\arc{M, w}:$ $M$ is a TM, $w \in \Sigma^*$, and
			$M$ hangs on $w\}$.
	\w {\bf{}Problem}: Does $M$ hang on $w$?
	\w {\sf{}Hanging$_{\mbox{\scriptsize{}TM}}$} is r.e.
	\eit
\w ({\bf{}The Blank Tape Halting Problem for Turing Machines})
	\bit
	\w {\sf{}BlankTape$_{\mbox{\scriptsize{}TM}}$} $= 
		\{\arc{M}:$ $M$ is a TM, $M$ 
			halts on $\epsilon\}$.
	\w {\bf{}Problem}: Does $M$ halt on $\epsilon$?
	\w {\sf{}BlankTape$_{\mbox{\scriptsize{}TM}}$} is r.e.
	\w {\sf{}Halting$_{\mbox{\scriptsize{}TM}}$} 
		$\le_m$
		{\sf{}BlankTape$_{\mbox{\scriptsize{}TM}}$} 
		\bit
		\w {\sf{}BlankTape$_{\mbox{\scriptsize{}TM}}$} 
			$\le_m$
			{\sf{}Halting$_{\mbox{\scriptsize{}TM}}$} 
		\eit
	\eit
\w ({\bf{}The Emptiness Problem for Turing Machines})
	\bit
	\w {\sf{}AcceptNothing$_{\mbox{\scriptsize{}TM}}$} $= 
		\{\arc{M}:$ $M$ is a TM and $L(M) = \emptyset\}$.
	\w {\bf{}Problem}: Does $M$ accept nothing?
	\w \am\ $\le_m$ {\sf{}AcceptNothing$_{\mbox{\scriptsize{}TM}}$}
	\w Let $E$ be the decider for 
		{\sf{}AcceptNothing$_{\mbox{\scriptsize{}TM}}$} and
		let $\arc{M, w}$ be an input string for \am.
	\w (cont.) Now we'd like to build a decider $A$ that decides \am.
	\w (cont.) Modify $\arc{M}$ such that $M$ rejects all strings
		in $\Sigma^* - \{w\}$ and if the input is $w$ returns the
		result of running $M$ on $w$. Let's denote this modified
		description of $\arc{M}$ by $\arc{M'}$
	\w (cont.) Then $E(\arc{M'}) = accept$ iff $M$ does not accept $w$.
	\eit
\w ({\bf{}The Nonemptiness Problem for Turing Machines})
	\bit
	\w {\sf{}AcceptSomething$_{\mbox{\scriptsize{}TM}}$} $= 
		\{\arc{M}:$ $M$ is a TM and there exists $w \in \Sigma^*$
		such that $M$ accepts $w\}$.
	\w {\bf{}Problem}: Does $M$ accept some string? Stated otherwise,
		Is $L(M)$ non-empty?
	\w {\sf{}AcceptSomething$_{\mbox{\scriptsize{}TM}}$} 
		is r.e. since we can 
		design a TM $M'$ that perform an infinite {\em{}dovetailing\/} 
		procedure
		on all possile strings $w_1, w_2, \cdots$ of $\Sigma^*$
	\w {\sf{}Halting$_{\mbox{\scriptsize{}TM}}$} 
		$\le_m$
		{\sf{}AcceptsSomething$_{\mbox{\scriptsize{}TM}}$} 
		\bit
		\w {\sf{}AcceptSomething$_{\mbox{\scriptsize{}TM}}$} 
			$\le_m$
		{\sf{}Halting$_{\mbox{\scriptsize{}TM}}$}
		\eit
	\eit
\w ({\bf{}The Totality Problem for Turing Machines})
	\bit
	\w {\sf{}AcceptsAll$_{\mbox{\scriptsize{}TM}}$} $=
		\{\arc{M}:$ $M$ is a TM and $M$ halts on all 
		$w \in \Sigma^*\}$.
	\w {\bf{}The Problem}: Does $M$ halt on all inputs?
	\w {\sf{}Halting$_{\mbox{\scriptsize{}TM}}$} 
		$\le_m$
		{\sf{}AcceptsAll$_{\mbox{\scriptsize{}TM}}$} 
	\eit
\w ({\bf{}The Equivalence Problem for Turing Machines})
	\bit
	\w {\sf{}Equiv$_{\mbox{\scriptsize{}TM}}$} $=
		\{\arc{M_1, M_2}:$ $M_1, M_2$ are TMs and $L(M_1) = L(M_2)\}$.
	\w {\bf{}The Problem}: Does $M_1$ and $M_2$ accept the same language?
	\w {\sf{}AcceptsAll$_{\mbox{\scriptsize{}TM}}$} 
		$\le_m$
		{\sf{}Equiv$_{\mbox{\scriptsize{}TM}}$} 
	\eit
\w ({\bf{}The Finiteness Problem})
	\bit
	\w {\sf{}Finite$_{\mbox{\scriptsize{}TM}}$} $=
		\{\arc{M}:$ $M$ is a TM and $L(M)$ is finite$\}$.
	\w {\bf{}The Problem}: Does $M$ accept a finite language?
	\w Provable using {\em{}Rice's Theorem}.
	\eit
\w ({\bf{}The Regularness Problem})
	\bit
	\w {\sf{}Regular$_{\mbox{\scriptsize{}TM}}$} $=
		\{\arc{M}:$ $M$ is a TM and $L(M)$ is regular$\}$.
	\w {\bf{}The Problem}: Does $M$ accept a regular language?
	\w {\sf{}AcceptsAll$_{\mbox{\scriptsize{}TM}}$}
		$\le_m$
		{\sf{}Regular$_{\mbox{\scriptsize{}TM}}$} 
	\w Let $R$ be the decider for 
		{\sf{}Regular$_{\mbox{\scriptsize{}TM}}$}. Now, how can
		we use $R$ to build a decider for \am?
	\w (cont.) Simple! We need only to be modify $M$ into $M'$ such that
		{\em{}$R$ accepts $\arc{M'}$ 
		iff $M$ accepts $w$\/}; i.e. $L(M')$ is regular iff
		$M$ accepts $w$.
	\w (cont.) Design $M'$ to accept the nonregular language
		$\{0^n1^n: n \le 0\}$ if $M$ does not accept $w$ and
		the regular language $\Sigma^*$ if $M$ accepts $w$.
	\w (cont.) $M'$ is defined as: On input $w$,
		\ben
		\w If $w$ has the form $0^n1^n$, {\em{}accept\/};
		\w Otherwise, run $M$ on $w$ and {\em{}accept\/} if
			$M$ accepts $w$.
		\een
		
	\w Also proved using {\em{}Rice's Theorem}.
	\eit
\w ({\bf{}The Context-freeness Problem})
	\bit
	\w {\sf{}ContextFree$_{\mbox{\scriptsize{}TM}}$} $=
		\{\arc{M}:$ $M$ is a TM and $L(M)$ is context-free$\}$.
	\w {\bf{}The Problem}: Does $M$ accept a context-free language?
	\w Provable using {\em{}Rice's Theorem}.
	\eit
\w ({\bf{}The Post Correspondence Problem (PCP)})
	\bit
	\w An instance of the PCP is a collection $P$ of dominos:
	\[ P = \left\{\left[\frac{t_1}{b_1}\right], 
		\left[\frac{t_2}{b_2}\right], \cdots, 
		\left[\frac{t_k}{b_k}\right]\right\},\]
	and a {\em{}match\/} is a sequence $i_1, i_2,\cdots, i_l$ where
	$t_{i_1}t_{i_2}\cdots{}t_{i_l} = b_{i_1}b_{i_2}\cdots{}b_{i_l}$.
	Here, $1 \le i_m \le k$ ($1 \le m \le l$)
	and for every $i_m$ and $i_n$ ($m \ne n$),
	they need not be different positives.
	\w {\sf{}PCP} = $\{\arc{P}: \mbox{\ $P$ is a Post correspondence
		problem with a match}\}$.
	\w {\bf{}Problem}: Given a set of dominos, is there a match?
	\w {\sf{}PCP} is undecidable.
	\w A Proof of the undecidability uses {\em{}reduction from
		\am\ via accepting computation histories}.
	\eit
\eit



\section{The Recursion Theorem}
\subsection{Self-reference}
\def\self{{\em{}SELF\/}}
\bit
\w There's a `computable' function $q: \Sigma^* \rightarrow \Sigma^*$,
	where, for each string $w$, $q(w)$ is the description of
	the Turing machine $P_w$ that prints out $w$ and then halts. That is,
	\[ q(w) = \arc{P_w}.\]
\w Let's design a Turing machine {\em{}SELF} that ignores its input
	and prints out $\arc{\mbox{\em{}SELF\/}}$, 
	a copy of its own description.
	\bit
	\w {\em{}SELF\/} consists of two parts $A$ and $B$.
	\w We want \self\ to print out $\arc{\mbox{\self}} = \arc{AB}$.
	\w {\em{}$A$ prints out a description of $B$, and conversely $B$ prints
		out a description of $A$.\/}
	\w $A$ uses $q(\arc{B})$ to print out $\arc{B}$. That is,
		$A \equiv P_{\langle{}B\rangle}$.
	\w Our description of $A$ depends on having a description of $B$;
		so {\em{}we can't complete the description of $A$ until we
		construct $B$\/}.
	\w We cannot use the same strategy for $B$ since it incurs a
		{\em{}circularity\/}.
	\w {\em{}$B$ computes $A$ from the output that $A$ produces.\/}
	\w {\sl\bfseries{}Since $\arc{A} = q(\arc{B})$,
		if $B$ can obtain $\arc{B}$, it can apply $q$ to that
		and obtain $\arc{A}$.\/}
		\bit
		\w How does $B$ obtain $\arc{B}$? It was left on the tape
			when $A$ finished!
		\eit
	\eit
\eit
\subsection{The recursion theorem}
\bit
\w The recursion theorem extends the technique we used in constructing
	\self\ so that a program can obtain its own description
	and then go on to compute with it, instead of merely 
	printing it out.
\w ({\bf{}Recursion Theorem})\ 
	{\sl\bfseries{}Let $T$ be a Turing machine that computes a function
	$t : \Sigma^* \times\Sigma^* \rightarrow \Sigma^*$.
	There is a Turing machine $R$ that computes a function
	$r: \Sigma^* \rightarrow \Sigma^*$, where for every $w \in \Sigma^*$
	\[r(w) = t(\arc{R}, w).\]}
\eit

\subsection{Applications of the recursion theorem}
\bit
\w ({\bf{}Reification}) 
	When designing a Turing machine $M$, 
	we can include the phrase ``obtain own description $\arc{M}$.''
	\bit
	\w This result is the central theme in the field of
		{\em{}computational reflection\/}.
	\eit
\w If $M$ is a Turing machine, then we say that the {\bf{}length}
	of the description $\arc{M}$ of $M$ is the number of symbols
	in the string describing $M$. Say that $M$ is {\bf{}minimal}
	if there is no Turing machine equivalent to $M$ that has a shorter
	description. Let
	\[ \mbox{\sf{}Minimal$_{\mbox{\scriptsize{}TM}}$} =
		\{\arc{M}: \mbox{\ $M$ is a minimal TM}\}.\]
	\bit
	\w {\sf{}Minimal$_{\mbox{\scriptsize{}TM}}$} is not
		r.e.
	\eit
\eit

\subsection{Fixed Points}
\bit
\w A {\bf{}fixed point} of a `function' is a value that isn't changed
	by application of the function.
\w {\sl\bfseries{}Let $t: \Sigma^* \rightarrow \Sigma^*$ 
	be a computable function.
	Then there is a Turing machine $F$ wherein $t(\arc{F})$
	describes a Turing machine equivalent to $F$.}
\eit

\section{Decidability of Logical Theories}
\bit
\w If ${\cal{}M}$ is a model, we define the {\bf{}theory of ${\cal{}M}$},
	denoted by Th$(\cal{}M)$, be the collection of true sentences
	in the language of that model\footnote{For 
	details of mathematical logic, refer to `Notes on Mathematical
	Logic.'}.
\w Alonzo Church, building on the work of Kurt G\"{o}del, showed
	that {\em{}no algorithm can decide in general whether
	statements in number theory are true or false\/}.
\w Th$(\N, +)$ is decidable.
\w Th$(\N, +, \times)$ is undecidable.
	\bit
	\w Let $M$ be a Turing machine and $w$ a string. 	
		We can construct from $M$ and $w$ a formula
	$\phi_{M, w}$ in the language of Th$(\N, +, \times)$
	that contains a single free variable $x$,
	whereby the sentence $(\exists{x})\left[\phi_{M,w}\right]$ is
	true iff $M$ accepts $w$.
	\eit	
\w The collection of provable statements in Th$(\N, +, \times)$
	is Turing-acceptable (r.e.).
\w Some true statements in Th$(\N, +, \times)$ is not provale.
\w The sentence $\psi_{\mbox{\scriptsize{}unprovable}}$,
	as described in the proof of this theorem, is unprovable.
\eit

\section{Oracles and Turing Reducibility}
\subsection{Oracles}
\bit
\w An {\bf{}oracle} for a language $B$ is an external device that is capable
	of reporting whether any string $w$ is a member of $B$.
\w An {\bf{}oracle Turing machine} is a modified Turing machine that has 
	the additional
	capability of querying an oracle.
	We write $M^B$ to describe an oracle Turing machine
	that has an oracle for language $B$.
\eit
\subsection{Turing reducibility}
\bit
\w Consider an oracle for \am. An oracle Turing machine with an oracle
	for \am\ can decide more languages thatn an ordinary Turing
	machine can. Such a machine can obviously decide \am\ itself, by
	querying the oracle about the input.
	\bit
	\w Sup' that we'd like to decide \etm, with the following
		procedure called 
		$T^{{\mbox{\scriptsize\sf{}Accepts}_{\mbox{\tiny\sf{}TM}}}}$:
	\ben
	\w [$\diamond$.] On input $\arc{M}$, where $M$ is a TM:
	\w Construct the follwing TM $N$:	
		\ben
		\w [($\diamond$)] On any input:
		\w Run $M$ in parallel on all strings in $\Sigma^*$.
		\w If $M$ halts on any of these strings, {\em{}accept\/}.
		\een
	\w Query the oracle to determine whether $\arc{N, 0} \in $\ \am.
	\w If the oracle answers NO, {\em{}accept\/}; if YES, 
		{\em{}reject\/}.
	\een
	
	\w We say that a Turing machine $E$ is {\bf{}decidable relative to}
		$A$
	\eit
\w Language $A$ is {\bf{}Turing reducible} to language $B$, written
	$A \le_T B$, if $A$ is decidable relative to $B$.
\eit


\bibliographystyle{plain}
\bibliography{00bib/mac,00bib/theory,00bib/math}
\nocite{Sipser96,HU79,Rogers87,Kleene71,Jones97,FB94}
\pagebreak
\tableofcontents
\end{document}
