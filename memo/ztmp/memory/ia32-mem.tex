\chapter{Memory Organization}

\section{Physical Memory}
The real memory that the processor addresses on its bus is called \bb{physical
  memory}, which is a sequence of 8-bit bytes. Each byte is assigned a unique
address, called a \bb{physical address}. Physical address space ranges from 0
to $2^{36} - 1$ (i.e. 64GB) if the processor does not support Intel~64
architecture. In Intel 64 architecture, supports physical address space greater than 64GB. 

\section{IA-32 Memory Models}
Software programs do not directly address physical memory when accessing
processor's memory management facilities. 
Instead, it uses one of the following three memory models:
\bit
\w \bb{Flat memory model}
\w \bb{Segmented memory model}
\w \bb{Real-address mode memory model}
\eit

\subsection{Flat memory model}
Memory appears to a program as a single, continuous address space. This space
is called a \bb{linear address space}. Code, data, and stacks are all
contained in this address space. Linear address space is byte-addressable,
with addresses running contiguously from 0 to $2^{32} -1$ (if not in 64-bit
mode). An address of any byte in linear address space is called a \bb{linear
  address}. 

\subsection{Segmented memory model}
Memory appears to a program as a group of independent address spaces called
\bb{segments}. Code, data, and stacks are typically contained in separate
segments. To address a byte in a segment, a program issues a \bb{logical
address}. This consists of a \bb{segment selector} and an \bb{offset}
({\em logical addresses\/} are often referred to as \bb{far pointers}). 

The {\em segment selector\/} identifies the segment to be accessed and the
{\em offset\/} identifies a byte in the address space of the segment. Programs
running on an IA-32 processor can address up to $2^{14} - 1 = 16,383$ segments
of different sizes and types. Each segment can be as large as $2^{32}$ bytes
(4GB). 

Internally, all segments that are defined for a system are mapped into the
processor's {\em linear address space}. To access a memory location, the
processor thus {\em translates each logical address into a linear
  address\/}. This translation is transparent to the application program. 

The primary reason for using segmented memory is to increase the reliability
of programs and systems. For example, placing a program's stack in a separate
segment prevents the stack from growing into the code or data space and
overwriting instructions or data, respectively. 

\subsection{Real-address mode memory model}
This is the memory model for the Intel 8086 processor. It is supported to
provide compatibility with existing programs written to run on the Intel~8086
processor. The real-address mode uses a specific implementation of segmented
memory in which the linear address space for the program and the operating
system/executive consists of an array of segments of up to 64KB in size
each. The maximum size of linear address space in real-address mode is
$2^{20}$ bytes. 




\section{Paging and Virtual Memory}
With the {\em flat\/} or the {\em segmented\/} memory model, linear address
space is mapped into the processor's physical address space either
1) through \bb{direct mapping} or 2) through \bb{paging}. 

\subsection{Direct mapping}
When using \bb{direct mapping} (with paging disabled), each linear address has
a one-one correspondence with a physical address. Linear addresses are sent
out on the processor's address lines without translation. 

\subsection{Paging}
When using the IA-32 architecture's paging mechanism (with paging enabled),
linear address space is divided into \bb{pages} which are mapped to virtual
memory. The pages of virtual memory are then mapped as needed into physical
memory.  When an operating system uses paging, the paging mechanism is
transparent to an application program. All that the application sees is linear
address space. 

In addition, IA-32 architecture's paging mechanism includes extensions that
support:
\bit
\w \bb{Page Address Extensions (PAE)}: to address physical address space
greater than 4GB
\w \bb{Page Size Extensions (PSE)}: to map linear address to physical address
in 4MB 
\eit

\section{Modes of Operation vs. Memory Model}
When writing code for an IA-32 or Intel~64 processor, a programmer needs to
know the operating mode the processor is going to be in when executing the
code and the memory model being used. 
The relationship between \bb{operating modes} and \bb{memory models} is as
follows: 

\subsection{Protected mode}
When in protected mode, the processor can use any of the memory models
described above (the real-addressing mode memory model is ordinarily used only
when the processor is in the virtual-8086 mode). The memory model used depends
on the design of the operation system or executive. When multitasking is
implemented, individual tasks can use different memory models. 

\subsection{Real-address mode}
When in real-address mode, the processor only supports the real-address mode
memory model. 

\subsection{System management mode}
When in system management mode, the processor switches to a separate address
space, call the {\em system management RAM\/} (SMRAM). The memory model used
to address bytes in this address space is similar to the real-address mode
model. 

\subsection{Compatibility mode}
Software that needs to run in compatibility mode should observe the same
memory model as those targeted to run in 32-bit protected mode. The effect of
segmentation is the same as it is in 32-bit protected mode semantics. 

\subsection{64-bit mode}
Segmentation is generally (but not completely) disabled, creating a flat
64-bit linear-address space. Specifically, the processor treats the segment
base of CS, DS, ES, and SS as zero in 64-bit mode. {\em This makes a linear
  address equal an effective address. \/} Segmented and real address modes are
not available in 64-bit mode


\section{32-Bit and 16-Bit Address and Operand Sizes}
IA-32 processors in protected mode can be configured for 32-bit or 16-bit
address and operand sizes. With {\em 32-bit address and operand sizes\/}, 
the maximum linear address or segment offset is 0xFFFFFFF ($2^{32} - 1$)
(operand sizes are typically 8 bits or 32 bits). With {\em 16-bit address
  and operand sizes\/}, the maximum linear address or segment offset is
0xFFFF ($2^{16} - 1$) (operand sizes are typically 8 bits or 16 bits). 

When using 32-bit addressing, a logical address (or {\em far pointer\/})
consists of a 16-bit segment selector and a 32-bit offset. When 16-bit
addressing is used, a logical address consists of 16-bit selector and a 16-bit
offset. However, instruction prefixes allow temporary overrides of the default
address and/or operand sizes within a program.

\subsection{Segment descriptor}
When operating in \bb{protected mode}, the segment descriptor for the
currently executing \bb{code segment} defines the default address and
operand size.  A segment descriptor is a system data structure not normally
visible to application code. Assembler directives allow the default addressing
and operand size to be chosen for a program. The assembler and other tools
then set up the segment descriptor for the code segment appropriately. 

When operating in \bb{real-address mode}, the default addressing and operand
size is 16 bits. An address-size override can be used in real-address mode to
enable 32-bit addressing. However, the maximum allowable 32-bit linear address
is still 0x000FFFFF ($2^{20} - 1$). 

\section{Extended Physical Addressing in Protected Mode}
Beginning with P6 family processors, the IA-32 architecture supports
addressing of up to 64GB ($2^{36}$ bytes) of physical memory. A program or
task could not address locations in this address space directly. Instead, {\em
  it addresses individual linear address spaces of up to 4GB that mapped to 64GB
physical address space through a virtual memory management mechanism\/}. Using
this mechanism, an operating system can enable a program to switch 4GB linear
address spaces within 64GB physical address space. 

The use of extended physical addressing requires the processor to operate in
protected mode and the operating system to provide a virtual memory management
system.   

\section{Address Calculations in 64-Bit Mode}
In most cases, 64-bit mode uses flat address space for code, data, and
stacks. In 64-bit mode (if there is no address-size override), the size of
effective address calculations is 64 bits. An effective-address calculation
uses a 64-bit base and index registers and sign-extend displacements to 64
bits. 

In the flat address space of 64-bit mode, linear addresses are equal to
effective addresses because the base address is zero. 


% LocalWords:  PAE PSE SMRAM DS xFFFFFFF xFFFF FFFFF
