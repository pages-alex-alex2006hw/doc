\chapter{Caches}

\chapter{DRAM}
An individual DRAM device typically connects to a CPU {\em indirectly\/}
through a memory controller. In PC systems, the memory controller is part of
the {\bf northbridge} chipset that handles potentially multiple processors,
the graphics co-processor, communication to the {\bf southbridge} chipset
(which handles all of system's I/O functions), as well as the interface to the
DRAM system.  

\section{DRAM Basics}
Each DRAM die contains one or more {\bf memory arrays}, rectangular grids of
storage cells with each cell holding one bit of data. Memory arrays are
organized into \bb{rows} and \bb{columns}. By specifying the {\em row
  address\/} and {\em column address\/}, a {\bf memory controller} can access
an individual storage cell inside a DRAM chip to read or write the data. 

One way of characterizing DRAMs is by the {\em number of memory arrays inside
  them\/}. The memory arrays can be either work in unison (for the given row
and column addresses), they can act completely independently, or they can
  act in a manner somewhere in between. For example, \bb{x4 (``by four'')
    DRAM} has at least four memory arrays and the column width is 4. 

\subsection{Banks}
Each set of memory arrays that operates indenpendently of other sets is called
a \bb{bank}, not an array. Each bank is independent in that, with a few
restrictions, it can be activated, precharged, read out, etc. at the same time
that other banks (on the same or other DRAM devices) are being activated,
precharged, etc. 

\subsection{Interleaved banks}
Typically, multiple memory banks are {\bf interleaved}  to
achieve a high-bandwidth memory busses using low-bandwidth devices. 
In an interleaved memory system, the data bus uses a frequency that is faster
than any one DRAM bank can support. The control circuitry toggles back and
forth between multiple banks to achieve the bus data rate. This technique is
dated back to mid-1960s, where it was used in IBM/360 an Cray Control Data
6600.  

\subsection{Ranks}
In a memory system, a \bb{DIMM} can be thought of a set of independent
banks. Also, the DRAM devices inside each DIMM can implement internally
multiple independent banks. To distinguish these two types of banks, we call
DIMM-level indenpendent operations as \bb{ranks}. A rank represents a DIMM-level independent operation
while a bank represents a internal DRAM-device-level independent operation. 

\subsection{DIMMs, ranks, banks, and arrays}
In summary, a memory system consists of potentially many \bb{DIMMs}. 
Each DIMM may contain one or more independent \bb{ranks}. 
Each rank is a set of \bb{DRAM devices} that operate in unison. 
Internally, each DRAM device implements one or more independent \bb{banks}. 
Finally, each bank is composed of slaved \bb{memory arrays}.
