\chapter{Linux Memory Addressing}


\section{Memory Addresses}
There are three types of IA-32 memory addresses:
\bit
\w \bb{logical address}
\w \bb{linear address (or virtual address)}
\w \bb{physical address}
\eit

\begin{figure}[ht]
\[\xymatrix{
  \mbox{ } \ar[rrr]^{\mbox{\scriptsize{}logical address}} &&& *+[F]{\mbox{Segmentation Unit}}
  \ar[rrr]^{\mbox{\scriptsize{}linear address}}  &&& *+[F]{\mbox{Paging Unit}}
  \ar[rrr]^{\mbox{\scriptsize{}physical address}} &&&
}
\]
\caption{Logical address translation}
\end{figure}


\paragraph{Memory arbiters}
In multicore systems, all CPUs share the same memory and RAM chips can be
accessed concurrently by independent CPUs. To handle potential conflicts,
\bb{memory arbiter} is inserted between the bus and every RAM chip.



\section{Segmentation in Hardware}
\subsection{Linear addresses and segement selectors}
A \bb{linear address} consists of two parts: a \ee{segment identifier} and an
\ee{offset} that specifies the relative address within the segment. 
The segment identifier is a 16-bit field, called \bb{segment selector}, while
the offset is a 32-bit field.
The 16-bit segment selector consists of following fields:
\bit
\w bits 15--3 (13 bits): segment index
\w bit 2 (1 bit): table indicator (GDT or LDT)
\w bits 1--0 (2 bits): request privilege level (0: kernel mode, 3: user mode)
\eit


\subsection{Segment registers}
The x86 processors provide \bb{segement registers} whose only purpose is to
hold {\em segment selectors}: {\tt{}cs}, {\tt{}ss}, {\tt{}ds}, {\tt{}es},
{\tt{}fs}, and {\tt{}gs}:
\bit
\w code segement register ({\tt{}cs}): points to a segment that contains
program instructions
\w stack segment register ({\tt{}ss}): points to a segment that contains
current program stack
\w data segment register ({\tt{}ds}): points to a segment that contains
current global and static data
\eit

\subsection{Segment descriptors}
Each segment is represented by an 64-bit \bb{segement descriptor} that
describes the segement characteristics. Segement descriptors are stored either
in the \bb{global descriptor table (GDT)} or in the \bb{local descriptor table
  (LDT)}. 
There is only one GDT in Linux but each process can have its own LDT if it
needs to create additional segments besides those in GDT.


\subsection{Linux GDT and LDT}


\section{Paging}
The hardware paging unit translates {\em linear addresses\/} into {\em
  physical addresses\/}. Its major task is to check the requested access type
against the access rights of the linear address.
If the memory access is not valid, it generates a PFE (Page Fault Exception). 

In paging, linear addresses are grouped in {\em fixed-length intervals} called
\bb{pages}: contiguous linear addresses within a page are mapped into
contiguous physical addresses. 
A page can reside either inside RAM or inside disk. In the former case, a page
is mapped to a \bb{page frame} inside the RAM.


\subsection{Regular paging}
Each process owns pages for its code and data.  For this, one \bb{page
  directory} is assigned to it. The physical address of the {\em page
  directory\/} in use is stored in the control register {\tt cr3}.  

A 32-bit linear address is partitioned into three parts:
\bit
\w \bb{directory} (10 bits): index to the page directory
\w \bb{table} (10 bits): index to the page table
\w \bb{offset} (12 bits) : offset within a page 
\eit

Each entry of the page directory and page tables has the same structure:
\bit
\w \bb{present flag}: set to 1 if and only if the given page resides inside the
main memory; if it's 0, paging unit stores the linear address in {\tt cr2} and
generates exception 14: the \bb{Page Fault exceptino}.
\w \bb{accessed flag}: set each time paging unit address the corresponding
page frame; used by Linux to determine which page to swap out
\w \bb{dirty flag}: applies only to page table entries: 
\eit

\subsection{Extended paging}
Starting from Pentium, x86 processors allow page frames to be 4MB (instead of 4KB).


%%%%%%%%%%%%%%%%%%%%%%%%
\chapter{Physical Memory in Linux}
Physical memory is divided into \bb{banks}, an indenpently operating memory
units. In Linux, a bank is called a \bb{node}, and is represented in 
{\tt struct pglist\_data}. Each node is divided into a number of blocks called
\bb{zones}. With x86, a zone has one of the following three:
\bit
\w \bb{ZONE\_DMA} (firt 16MB):   memory in the lower physical memory ranges
that certain ISA devices require
\w \bb{ZONE\_NORMAL} (16MB -- 896MB): memory which is directly mapped by kernel
into the upper region of the linear address space; {\em most kernel operations
  only take place here} (so performance-critical)
\w \bb{ZONE\_HIGHMEM} (896MB -- end): remaining available memory and is not
directly mapped by the kernel
\eit

Each zone (actually, memory) is further broken dow to fixed-size memory chunks
called \bb{page frames}, which is represented by {\tt struct page} in Linux. 
