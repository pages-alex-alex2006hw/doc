\documentclass{myproc}
\usepackage{minion,mydef,amssymb}
\begin{document}
\small

\title{\large\bf Notes on Threshold Functions and Networks}
\author{\small Cheoljoo Jeong (cjeong@cs.columbia.edu)}
\maketitle

\section{Basics}
\bit
\w \bb{advantages}: ability of single threshold elements to realize a larger
class of functions than is realizable by any one conventional gate
\w \bb{threshold identification problem}: is the given Boolean function
realizable by a {\em single\/} threshold element? if so, what are the weights
and the threshold?
\w Given a $n$-variable function whose domain is representable with
$n$-dimensional space, a linear equation
   \[ w_1x_1 + w_2x_2 + \cdots w_nx_n = T\]
corresponds to an \bb{$(n-1)$-dimensional hyperplane} which cuts through the
$n$-cube. 
\w A Boolean function whose ON-vertices in $n$-cube can be separated by a
linear equation from its false ones is called a \bb{linear separable
  function}. 
\w \bb{(-1,1 configuration)}: (0,1)-valued threshold network can be changed
to (-1,1)-valued network by changing 
   \[ y_i = 2x_i - 1,   g = 2f - 1\]
\eit


\section{Theorems on Threshold Logic and Circuits}
\begin{theorem}
\cite[page 184]{Kohavi78} If a Boolean function is realizable
by a single threshold element, then it is realizable by an element with only
positive weights.
   \bit
   \w $f(x_1, \cdots)$ is realizable by $(w_1, \cdots; T)$
   \w $f(x_1', \cdots)$ is realizable by
      $(-w_1, \cdots; T - w_1)$
   \eit
\end{theorem}

\begin{theorem}
\cite[page 190]{Kohavi78} Every threshold function is unate. 
\end{theorem}

\begin{theorem}
\cite{Muroga71}
 Given a positive unate threshold function
$f(x_1, \cdots, x_l)$, with weight-threshold vector $(w_1, \cdots, w_l; T)$,
if $x_i$ is replaced by $x_i'$ to get $g(x_1, \cdots, x_i', \cdots, x_l)$,
then the weigh of $x_i$ in $g$ is simply $-w_i$ and the threshold of $g$ is
$T - w_i$.
\end{theorem}

\begin{theorem} \cite{Muroga71} Let $f(x)$ be an arbitrary $n$-variable
  threshold function. Then, $f(x)$ can be written as
  \[ f(x) = sign\left(w_0 + \sum_{i = 1}^nw_ix_i\right),\]
  where, for all $i = 0, \cdots, n$, 
 $w_i \in \Z$ and $|w_i| \le 2^{-n}(n+1)^{(n+1)/2}$.
\end{theorem}

\begin{theorem}
\cite{ZGZJ05} Given an expression for a unate Boolean
function, $f(x_1, \cdots, x_l)$, replace literal $x_i$ with literal $x_j'$
with some $j \ne i$, resulting in $g(x_1, \cdots, x_k)$ where $1 \le k \ne j
\le l$. If $g$ is not a threshold function, then $f$ is not a threshold
function. 
\end{theorem}

\begin{theorem}
\cite{ZGZJ05} If a Boolean function $f(x_1, \cdots, x_l)$ is
a threshold function, then $h(x_1, \cdots, x_{l+k}) = f(x_1, \cdots, x_l) \vee
x_{l+1} \vee \cdots \vee x_{l+k}$ is also a threshold function.
\end{theorem}


\section{Threshold Circuit Complexity}
\bit
\w \bb{Size} of a circuit is the number of non-input gates and \bb{depth} 
is the length of the longest directed path.
\w SIZE$_B(s)$: set of all sets $A \subseteq \{0, 1\}^*$ for which there is a
circuit family $C$ over basis $B$ of size $O(s)$ that accepts $A$
\w FSIZE$_B(s)$: set of functions $f: \{0, 1\}^* \rightarrow \{0, 1\}^*$ for
which there is $C$ over $B$ of size $O(s)$ that computes $f$.
\w A Boolean function $f$ is \bb{length-respecting} if whenever $|x| = |y|$
then also $|f(x)| = |f(y)|$.
\eit
\begin{theorem}
\mbox{\rm MAJORITY $\in$  DEPTH($\lg n$)}.
\end{theorem}
\begin{proof}
BCOUNT $\in$ FDEPTH($\lg n$). Count and then compare with $n/2$. \qedb
\end{proof}


\section{Algorithms on Threshold Networks}
\subsection{Aggregation of Inequalities}
\begin{problem}
Given a set of linear equations
\[\sum_{j=1}^n a_{ij}x_j = b_i \qquad (i = 1, 2, \cdots, m),\]
determine if there exists a single linear equation
\[\sum_{j=1}^n a_{j}x_j = b\]
such that these have precisely the same set of 0-1 solutions.
\end{problem}
\bit
\w Answer to this problem is always AFFIRMATIVE.
\eit

\begin{problem}
Given a set of linear inequalities
\[\sum_{j=1}^n a_{ij}x_j \le b_i \qquad (i = 1, 2, \cdots, m),\]
determine if there exists a single linear inequality
\[\sum_{j=1}^n a_{j}x_j \le b.\]
such that these have precisely the same set of 0-1 solutions.
\end{problem}
\bit
\w Can be solved in $O(mn^2)$ steps.
\eit


\section{Linear Programming Principles}
\bit
\w \bb{Decomposition principle}
\eit



\bibliographystyle{plain}
\bibliography{00bib/mac,00bib/theory,00bib/async,00bib/threshold,00bib/algo}

\end{document}
