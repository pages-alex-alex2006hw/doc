\documentclass{note}
\usepackage{minion,mydef}
\begin{document}

\title{\large\bf Notes on CMOS VLSI}
\author{cjeong@cs.columbia.edu}
\date{}
\maketitle


\section{Background on Electricity}
\subsection{Electric charge}
\bit
\w \bb{electric charge}: fundamental characteristic of particles
   \bit
   \w objects have both (+) charge and (-) charge
   \w {\em electrically neutral}: (+) charge $\equiv$ (-) charge
   \w {\em charged}: unbalanced (+)/(-) charges
   \eit
\w \bb{conductor}: material where (-) charge freely moves
   \bit
   \w cf. non-conductor or insulator
   \eit
\w \bb{grounding an object}: setup a path from the object to Earth's surface
   \bit
   \w discharging an object
   \eit
\w \bb{charge moves through a material}
   \bit
   \w we say \bb{electic current exists} in the material
   \eit
\w \bb{Electrostatic force} between charged particles
    \[ F = k \frac{|q_1||q_2|}{r^2} \]
    \bit
    \w Coulomb's law
    \w $k = \frac{1}{4\pi\epsilon_0}$: \bb{electrostaic constant}
    \w $\epsilon_0 = 8.99 \times 10^9$ N$\cdot{}m^2/C^2$: \bb{permittivity
             constant} 
    \w one C (coulomb): amout of charge through the wire \underline{1 second},
    when \underline{1 ampere current exists}

    
    \eit
\w $d q = i\ dt$
    \bit
    \w $dq$: charges in coulomb
    \w $i$: current in amperes
    \w $dt$: time interval
    \eit
\w \bb{elementary charge}, $e$
    \bit
    \w charge is quantized
    \w $e = 1.60 \times 10^{-19}$ C
    \eit
\eit

\subsection{Electric fields}
\bit
\w charges set up an {\bf electric field}
\w electric field is a vector field (has both {\em magnitude\/} and {\em
  direction\/} 
\w Electric field $\vec{E}$ at point $P$ due to a charge $q_0$
   \[ \vec{E} = \frac{\vec{F}}{q_0} \ \ \ (N/C)\]

\w \bb{charge density}: 
   \bit
   \w charge: $q$ (C)
   \w linear charge density: $\lambda$ (C/m)
   \w surface charge density: $\sigma$ (C/m$^2$) 
   \w volume charge density: $\rho$ (C/m$^3$) 
   \eit
\eit

\subsection{Electric potential}
\bit
\w \bb{electric potential energy} $U$ to a {\em system of charged
  particles\/}:
    an energy of a charged object in an external electric field
   \bit
   \w when a system changes from state $i$ to $f$, 
   electrostatic force does work $W$ on the particles
   \w resulting change in the potential energy of the system 
   \[\Delta U = U_f -  U_i = - W\]
   \w measured in \bb{joules}
   \eit
\w \bb{electric potential} $V$: characteristic of the electric field,
regardless of whether a charged object has been placed in that field
   \bit
   \w potential energy per unit charge (at some point)
       \[ V = \frac{U}{q} = - \frac{W_{\infty}}{q}\]
   \w 1 volt = 1 joule per coulomb
   \eit
\w \bb{work done by an applied force}
   \bit
   \w $W_{app} = q \Delta V$
      \bit
      \w $\Delta V$: difference between potentials of initial and final
         position 
      \w $q$: magnitude of the charge of the moved particle
      \eit
   \eit
\w \bb{how to compute potential from a field}
   \bit
   \w $dW = \vec{F}\cdot d\vec{s} = q_0\vec{E}\cdot d\vec{s}$
     \bit
     \w $d\vec{s}$: displacement
     \eit
   \w $W = q_0 \int^f_i \vec{E}\cdot{}d\vec{s}$
   \w $V = - \int^f_i \vec{E}\cdot d\vec{s}$
     \bit
     \w i.e. potential $V$ at any point $f$
       in the electric field {\em relative to the zero potential\/} at point
   $i$ 
     \eit
   \eit
\w \bb{how to compute field from the potential}
   \[E = -\frac{\Delta V}{\Delta s} \]
\eit

\subsection{Capacitance}
\bit
\w The charge $q$ and the potential difference $V$ for a capacitor are
proportional to each other; i.e.
   \[ q = CV\]
   \bit
   \w $C$: \bb{capacitance} opf the capacitor
   \w $C$ depends only on the geometry of the places and not on their charge
   or potential difference
   \w capacitance is a measure of {\em how much charge must be put on the
   plates to produce a certain potential difference between them\/}; the
   greater the capacitance, the more charge required
   \w measured in farads (F) = coulomb per volt (C/V)
   \eit

\w \bb{computing the capacitance}
   \bit
   \w 
   \eit
\w \bb{capacitors in parallel and in series}
  \bit
  \w \bb{in parallel}: $C = \sum_j{C_j}$
  \w \bb{in series}: $\frac{1}{C} = \sum_j{\frac{1}{C_j}}$
  \eit
\w \bb{energy stored in an electric field}
  \bit
  \w \bb{potential energy}: $U = \frac{1}{2}CV^2$
  \eit
\w \bb{energy density}:
\eit

\end{document}
