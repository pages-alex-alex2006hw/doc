% $Id: qs.tex,v 1.1.1.1 2002/08/10 03:28:03 cjeong Exp $
% Cheoljoo Jeong <cjeong@cs.columbia.edu>
\documentclass{myart}
\usepackage{mydef}
\usepackage[all]{xy}

\begin{document}
\title{\large\bf Notes on Queueing Systems}
\author{\normalsize Cheoljoo Jeong}
\date{\normalsize\today}
\maketitle


\section{Basics}
\subsection{The Kendall notation}
\bit
\w \bb{A/S/m/B/K/SD}
	\bit
	\w \bb{A}: arrival time distribution\footnote{if the jobs 
	arrive at times $t_1, t_2, \cdots$,
	$\tau_i = t_i - t_{i-1}$ are called the {\em interarrival times\/};
	it's usually assumed that $\tau_i$ form a sequence of i.i.d. 
	random variables}:
		M (exponential),
		E$_k$ (Erlang with parameter $k$), 
		H$_k$ (hyperexponential with parameter $k$),
		D (deterministic), G (general)
	\w \bb{S}: service time distribution
	\w \bb{m}: number of servers
	\w \bb{B}: buffer size (system capacity)
	\w \bb{K}: population size
	\w \bb{SD}: service discipline:
		FCFS, RR, SPT, $\cdots$
	\eit
\eit

\subsection{Key variables}
\bit
\w Assumption:
	\bit
	\w customers arrive at random times
	\w probabability distribution of interarrival times is given
	\w probabability distribution of service times is given
	\eit
\w $\tau$: interarrival time
\w $\lambda = \frac{1}{E(\tau)}$: mean arrival rate
\w $s$: service time per job
\w $\mu = \frac{1}{E(s)}$: mean service rate per server
\w $n_q$: number of jobs waiting to get service
\w $n_s$: number of jobs recieving service
\w $n = n_q + n_s$: number of jobs {\em in the system\/}; \bb{queue length} 
\w $r$: response time
\w $w$: waiting time
\w all the variables above except for $\lambda$ and $\mu$ are
	random variables
\eit

\subsection{Some basic properties}
\bit
\w $E(n) = E(n_q) + E(n_s)$: linearity of expectations
\w $N \equiv E(n)$: average number of jobs in the system
\w Number vs. time: the number of jobs in the system is proportional
	to $\lambda$, waiting times, and service times
	\bit
	\w $N \propto \lambda$
	\w $N \propto E(r)$
	\w $N \propto E(w)$
	\eit
\eit

\subsection{Little's theorem}
\bit
\w \bb{Little's theorem}:\\
	Mean number in the system = arrival rate $\times$
	mean response time
\w $p_n(t)$: probability of $n$ customers waiting in queue or 
	under service at time $t$
\w $\overline{N}(t)$: average number in the system at time $t$
\w \bb{Equilibrium}
	\bit
	\w $p_n = \lim_{t \ra \infty} p_n(t)$
	\w $N = \sum_{n=0}^\infty np_n = \lim_{t\ra\infty}\overline{N}(t)$
	\eit
\w $N(t)$: individual sample functions of the number of the customers
\w $N_t$: time average of such sample function in the interval $[0, t]$
	\[ N_t = \frac{1}{t}\int_0^tN(\tau)d\tau\]
\w Amost every system of interest is \bb{erdodic} in the sense that
	\[ \lim_{t\ra\infty} N_t = \lim_{t\ra\infty}\overline{N}(t) = N\]
	holds with probability one.
\w $\overline{T}_k$: average delay of $k$th customer
\w $T$: average delay per customer 
	\[T = \lim_{k\ra\infty}\overline{T}(k)\]
\w \bb{Little's theorem}:
	$N = \lambda T$
	\bit
	\w $N$: average number in the system
	\w $T$: average customer time in the system
	\w $\lambda$: average customer arrival rate
	\w $\lambda = \lim_{t\ra\infty}\frac{\mbox{Expected number of 
		arrivals in the interval $[0, t]$}}{t}$
	\eit
	\bit
\w $N_Q = \lambda W$
	\bit
	\w $N_Q$: average number of customers {\em waiting in queue\/}
	\w $W$: average customer {\em waiting time in queue\/}
	\w note that the ``queue'' itself is a system
	\eit
\w Utilization factor $\rho = \lambda\overline{X}$
	\bit
	\w $\rho$: average number of packets under transmission; i.e.
		the proportionof time that the line is busy transmitting
		a packet
	\w $\overline{X}$: average transmission time
	\eit
	\eit
	
\eit

\subsection{Poisson distribution}
\bit
\w $\lambda$:
\w Mean:
\eit

\subsection{Exponential distribution}
\bit
\w Mean:
\eit


\subsection{Birth-death process}
\bit
\w \bb{state}: corresponds to the number of jobs in the system
\w \bb{state probability} $P_n$: probability of having $n$ jobs in the system
	\bit
	\w $P_n = \lambda_{n-1} P_{n-1}, \qquad n > 1$
	\w $\lambda_{n-1} P_{n-1} = \mu_n P_n$
	\eit
\[\xymatrix{
&&&&\ar@{.}[dd]&\\
**{=<2.0pc>[o][F]}{0} \ar@/^1.7pc/[r]^{\lambda_0} &
**{=<2.0pc>[o][F]}{1} \ar@/^1.7pc/[l]^{\mu_1}\ar@/^1.7pc/[r]^{\lambda_1}&
**{=<2.0pc>[o][F]}{2} \ar@/^1.7pc/[l]^{\mu_2}\ar@/^1.7pc/[r]^{\lambda_2}&
**{=<2.0pc>}{\cdots} \ar@/^1.7pc/[rr]^{\lambda_{n-1}}\ar@/^1.7pc/[l]^{\mu_3}&&
**{=<2.0pc>[o][F]}{n} \ar@/^1.7pc/[ll]^{\mu_{n}}\ar@/^1.7pc/[r]^{\lambda_n}&
**{=<2.0pc>}{\cdots} \ar@/^1.7pc/[l]^{\mu_3}&\\
&&&&&
}\]
\w \bb{cut property}
	\[ P_n = \frac{\lambda_0 \lambda_1 \cdots 
		\lambda_{n-1}}{\mu_1 \mu_2 \cdots \mu_n} P_0 \qquad n > 1 \]
\[\xymatrix{
&&\ar@{.}[dd]&\\
**{=<2.0pc>[o][F]}{0} \ar@/^1.7pc/[rrrr]^{\lambda_0\lambda_1\cdots\lambda_{n-1}} &&&&
**{=<2.0pc>[o][F]}{n} \ar@/^1.7pc/[llll]^{\mu_1\mu_2\cdots\mu_n}&\\
&&&
}\]

\eit


\section{The M/M/$1$ queueing system}
\bit
\w M: {\em the nature of the arrival process} is \bb{memoryless};
	i.e. Possion process (or exponentially distributed interarrival
	times)
\w /M/: {\em the nature of service times} is \bb{exponential}
\w $1$: the number of servers is $1$
\w \bb{traffic intensity} $\rho = \lambda/\mu$
	\bit
	\w note that $\mu \le \lambda$
	\eit  
\w $P\{n = 0\} = 1 - \rho$
\eit


%bibliography
\bibliographystyle{plain}
\bibliography{00bib/mac,00bib/theory,00bib/math,00bib/algo,00bib/formal,00bib/se,00bib/pl,00bib/oop,00bib/sys,00bib/db,00bib/psl,00bib/misc,00bib/inet}
\nocite{Kleinrock75,Jain91}

%toc
%\pagebreak
%\tableofcontents


\end{document}
