% $Id: sched.tex,v 1.1.1.1 2002/08/10 03:28:03 cjeong Exp $
% Cheoljoo Jeong <cjeong@cs.columbia.edu>
\documentclass{myart}
\usepackage{charter,mydef}
\usepackage[all]{xy}
\includeonly{qos}

\begin{document}
\small
%preamble
\hrule
\vspace*{0.2cm}
\noindent
{\normalsize\bf Notes on Scheduling and Quality of Service}
\vspace*{0.1cm}\\
{\small{}Author: Cheoljoo Jeong $\arc{\mbox{\em{}cjeong@cs.columbia.edu\/}}$}\\
{\small{}Date: \today} \vspace*{0.1cm}
\hrule



\section{Background}
\bit
\w Assumption: there are two resources to share
	\bit
	\w \bb{buffer space}
	\w \bb{bandwidth}
	\eit
\w Ideal scheduling service is {\em easily implementable}.
	\bit
	\w sublinear time-complexity (wrt number of connections)
	\w memory is the bottleneck (avoid per-connection state)
	\eit
\w Ideal scheduling service is {\em fair\/}.
	\bit
	\w \bb{MAX--MIN fairness}: maximizes the minimum share of a source
		whose demand is not fully satisfied
	\w protection against hogs
	\eit
\eit

\section{GPS}
\bit
\w $\phi_i$: weight of connection $i$
\w $S_A(i, t, t')$: service by scheduling policy $A$
	to connection $i$ during $[t, t']$
\w MAX--MIN fair: maximize $S_A(i, t, t')/\phi_i$
	\bit
	\w connection $i$ gets a fraction
	$\phi_i/\sum_j \phi_j$ of the total bandwidth
	at time $t$
	\eit
\w GPS (bit-by-bit GPS) is great but {\em not implementable\/} since
	packets cannot be broken up
	\bit
	\w WFQ or PGPS (packet-by-packet GPS) is used
	\eit
\eit
\subsection{Approximations to GPS}
\bit
\w cannot compute the exact GPS finish-time ``on-line''
\w only able to deal with \bb{relative order of GPS finish-times}
\w finish time = start time + length of packet / normalized weight
	\bit
	\w normalized weight: $\phi_i/\sum_k \phi_k$, where
		$\sum_k \phi_k = 1$
	\eit
\eit
	


\section{Weighted fair queueing (WFQ)}
\bit
\w \bb{work-conserving round robin discipline}: looks for a packet of
	a given class but finds none will immediately check the next class
	in the round robin sequence
\w \bb{average rate}: long-term average rate (packets per time interval)
	\bit
	\w note that 100 packets/sec is more constrained than 6000
	packets/min;
		6000 packets/min allows 1000 packets/sec for some 'second'

	\eit
\w \bb{peak rate}: maximum number of packets allowed for a shorter
	period of time
	\bit
	\w the network may police average rate of 6000 packets/min, while
limiting the number of peak rate to 1500 packets/sec
	\eit
\w \bb{burst size}: maximum number of packets allowed for an ``extremely
	short'' period of time
\w 
\eit





%toc
%\pagebreak
%\tableofcontents

%bibliography
\bibliographystyle{plain}
\bibliography{00bib/mac,00bib/theory,00bib/math,00bib/algo,00bib/formal,00bib/se,00bib/pl,00bib/oop,00bib/sys,00bib/db,00bib/psl,00bib/misc00bib/inet}
\end{document}
