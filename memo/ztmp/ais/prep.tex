% $Id: prep.tex,v 1.1.1.1 2002/08/10 03:28:03 cjeong Exp $
% Cheoljoo Jeong <cjeong@cs.columbia.edu>
\documentclass{myart}
\usepackage{mydef,myenv}
\usepackage[all]{xy}
\begin{document}
\small
\title{\large {\sl Advanced Internet Services\/} Final Prep Material}
\author{\normalsize Cheoljoo Jeong}
\date{\normalsize\today}
\maketitle


\section*{Internet standards and Access technologies}
\ben
\w What are standard bodies? who assigns what kind of numbers? what
is the difference between RFC and internet-draft? Guess what does
G.726 deal with?
	\bit
	\w \bb{Internet-drafts}: working documents of the IETF, its areas, and
		working groups. not an archival document series. should not
		be cieted or quoted in any formal document.
	\w \bb{RFC}: a proposed standard
	\w \bb{G.726}: ITU standard. ADPCM. 
		merging of G.721 and G.723. allows for the encoding
		of speech at 40, 32, 24, and 16 kb/s
	\eit
\w Who is an ISP? What is an AS? What is an access link? What is a
peering relationship between AS? What are different access technologies?
	\bit
	\w read IP Networks manuscript by H. Schulzrinne
	\w \bb{ISP}:
	\w \bb{AS}: autonomous system.  each service provider has its
	own ideas about routing algorithms, what metrics to assign to
	links. this is why ``autonomous''
	\w \bb{access link}:
	\w \bb{peering relationship between AS}:
	\w \bb{different access technologies}:
	\eit
\w What is modulation? What is modulation efficiency? Bitrate vs
spectrum band.
	\bit
	\w \bb{modulation}:
	\w \bb{modulation efficiency}: b/s/Hz; given a spectrum bandwidth
		how much bit rate is achieved
	\w \bb{bitrate}:
	\w \bb{spectrum band}:
	\eit

\w Why are DSL asymmetric? Why are cable modems asymmetric? Given a
scenario recommend a particular access technology. Uplink vs
downlink. Sharing of spectrum in wireless. Time-division multiplexing
vs frequency division multiplexing. What does GSM use? Idea of CDMA.
Sharing of bandwidth in common lines (cable modem). What is modem
concentration ratio? Where does most of the cost lies in fiber
technology? How does the distance affect the technology to be used?
What is noise aggregation and how does it affect the effective
bandwidth?

\w Bandwidth (kHz) of a physical link vs Bandwidth (kb/s) of the
technology.  Shannon's limit on effective bitrate for given signal to
noise ratio.
	\bit
	\w see QoS slide
	\w \bb{bandwidth (kHz)}:
	\w \bb{bandwidth (kb/s)}: bit rate
	\w \bb{maximum data rate on a noiseless channel}: $2B\log_2 V$ b/s
		(each signal has $V$ discrete levels; 
		$B$: bandwidth)
	\w {\bf S/N (signal to noise ratio)}: signal {\em power\/} $S$, 
	noise power
	$N$. $S/N$. in this case, $10 \log_{10} S/N$ ``in \bb{decibel}''
	\bit
	\w e.g. S/N ratio: 10; 10 dB; S/N ratio 100; 20dB
	\eit
	\w $C = B \log_2 (1 + S/N)$
	\w \bb{baud}: number of changes in a sec; if each signal convey
		several bits (e.g. 3bits) 1 baud = 3 b/s
	\eit
\een

\section*{Internet technology review}

\ben
\w What are different network elements: host, router, repeaters,
bridges, gateways. Higher level elements: web servers, firewalls, DNS,
DHCP. Subnetting, mask address and IP addresses. IPv6 advantages and
disadvantages.
	\bit
	\w see internet technology slide
	\w \bb{host}: end-system 
	\w \bb{repeater}: a physical layer relay that forwards {\em bits\/},
		in contrast to routers and bridges, which forward packets
	\w \bb{bridges}: see Kurose-Ross
	\w \bb{router $=$ gateway $=$ intermediate system}
	\w \bb{firewall}: {\em router} placed bewtween org's intra-net
		and the external Internet
	\w \bb{IPv6}: see Kurose-Ross p.341
	\eit

\w Physical and Link layer technologies: fiber optics, OC-n, ATM,
DWDM.  Delay bandwidth product. How does the cell/unit size affects
the protocol. What is label switching? Virtual circuits. Permanent
circuits vs routing table.

\w Wireless: power conservation, band, bit-rate, distance.

\w VPN vs LAN. Peak traffic vs capacity vs average utilization.
	\bit
	\w \bb{VPN}: virtual private network. controlled connection.
		not real leashed line but virtual leashed private line.
		
	\eit
\w Routing vs forwarding
	\bit
	\w \bb{forwarding}: sending the packet
	\w \bb{routing}: building forwarding (routing) table
	\eit
\w Names, addresses, routes. Binding of names (DNS$\ra$IP).  MAC vs IP
vs Host name. Given a name how do you get the mac address? How is
routing done? How does a routing table look like? What is CIDR?
Firewall vs NAT? Firewall vs ALG? problems: security, dynamic ports,
unique identification. Multihoming. IP forwarding.
	\bit
	\w see internet tech slides (later part)
	\w \bb{CIDR}: classless interdomain routing a.b.c.d/x
	\w \bb{NAT}:
	\w \bb{ALG}:
	\w \bb{AS} (autonomous system) types:
	\bit	
	\w \bb{stub AS}: an AS with a single connection to other AS. carry
only  local traffic. small corporation
	\w \bb{multihomed AS}: that has connections to more than on other AS
but refuses to carry transit traffic. large corporation
	\w \bb{transit AS}: with more than one connections to other AS. carry
both transit and local traffic. backbone provider
	\eit
	\eit

\w What are the significance of different fields in IP/IPv6/UDP/TCP
headers.  why is TTL=0 allowed? Source routing. Type of service.
	\bit
	\w see internet tech slide
	\w \bb{source routing}: source determines the route path
	\eit
\w ICMP vs IP. Ping and traceroute. 
	\bit
	\w see internet tech slide
	\eit

\w What is ARP, RARP, Proxy ARP.

\w TCP vs UDP. Reliable vs Unreliable. Packet oriented vs stream
oriented.  What applications are suceptible to these? DNS: TCP or
UDP. Proxy or redirect.  Heirarchical. Service records. Dynamically
updatable records. bind.
\een

\section*{Multicast}
\ben
\w What is multicast? receiver oriented vs sender oriented. Multicast
vs broadcast vs directed broadcast vs anycast vs unicast. Router vs
Host vs Multicast router.
	\bit
	\w 
	\eit

\w Spanning tree. Shared tree vs per-source tree. Cost of edges in
the tree?  Steiner tree: min links for all sinks. Prim's vs Kruskal's
algorithm for MST.
	\bit
	\w \bb{Kruskal}: $O(m\log n)$
	\eit

\w What are connection oriented multicast? what are applications?
what are its problems.
	\bit
	\w see mcast slide
	\eit

\w Hard state vs soft state. host group model. Transient vs permanent
host groups.
	\bit
	\w see ``Fall 1996 final''
	\w \bb{soft state}: router has a {\em timer\/} for reservations.
		see Kurose-Ross p.548
	\eit

\w Multicast address. Address space. Scoping (TTL vs administrative).
	\bit
	\w see Schulzrinne's multicasting bookchapter
	\w \bb{mcast address}: 224.0.0.0 -- 239.255.255.255
	\w \bb{scope}: distance that a multicast packet allowed to
		travel on the network
		\bit
		\w \bb{TTL scoping}: ``time to live'' field of IP header.
		achieves {\em subnet}-limited scope. does not go
		beyond the router. DRAWBACKS: depends on the network 
		topology. A can send to B but B cannot send to A
		\w \bb{administrative scoping}:
		administriavely scoped multicast addresses 
		(239.0.0.0 -- 239.255.255.255)
		\eit
	\eit

\w Address allocation vs packet routing. IGMP vs Multicast
routing. Difference between IGMP v2 and V3. Source filtering.
	\bit
	\w see Schulzrinne's multicasting bookchapter
	\w \bb{mcast address allocation}: dymaically allocated.
		no central organization
	\w \bb{IGMPv3}: allows hosts to join groups from {\em specific
		sources}. source filtering
	\eit

\w RPF. Draw the multicast tree using RPF for given sources and
sinks.  Routing: link-state vs dense mode vs sparse mode. How does
DVMRP work? MOSPF? PIM-DM? PIM-SM? differences: PIM-DM vs
DVMRP. Router state vs flooding network. Given a shared tree for
PIM-SM how will new receiver join? CBT. Given a scenario which method
is best suited.

\w What is MBONE? what are problems in implementing multicast?
Inter-domain vs intra-domain routing for multicast. How does MSDP work.
BGMP vs MSDP vs BGP.
	\bit	
	\w \bb{MBONE}: multicast backbone. 
	a portion of the Internet which provides
	mcast service. there's some multicast-capable networks in the
	Internet. connect these islands together usign IP in IP tunnels.
	\w 
	\eit

\w Address allocation protocols. Different levels? MAAS, MASC, MADCAP, AAP.

\w What socket support is needed for multicast, connection oriented
multicast etc?

\een

\section*{Audio and video}
\ben
\w Digitization: sampling rate, quantization. What is packetization?
Why is low pass filter needed? How does sampling and quantization
affect the quality and bitrate. Quantization error. Signal to noise
ratio due to quantization. Nyquist bound.
	\bit
	\w see multimedia bookchapter by Henning
	\w \bb{sampling rate}: the number of samples per second
	\w \bb{bitrate}: number of bits required to represent the
		media stream
	\w \bb{quantization}: 
	\eit

\w Companding vs waveform vs model vs subband
compression. MOS. Quality of a codec: bitrate, MOS, delay
(algorithmic processing), robustness to loss, complexity, tandem, music.
Subjective vs objective speech quality measurement.
	\bit
	\w see audio slide
	\eit

\w A-law vs $\mu$-law. Silence detection and comfort noise. Hangover vs
pre-talkspurt.
	\bit
	\w fill up from Internet telephony
	\w see slide
	\w \bb{comfort noise}
	\w \bb{hangover \& pre-talkspurt}: ending and beginning of speach:
		soft
	\eit
 
\w How does ADPCM work? PCM vs DPCM vs ADPCM vs PCMU/PCMA. Difference
between linear prediction vs ADPCM vs analysis by
synthesis. G.711/G.723/G.729: applications of each. High quality:
G.722 at 16 kHz. Mp3 is MPEG-1 layer 3 audio. What is audio perceptual
encoding?
	\bit
	\w 
	\eit

\w Traffic model: talkspurt vs silence. Poisson distribution. Can
expect problems on this.

\w SNR, decibels. Energy vs amplitude.
	\bit
	\w \bb{dB}: a relative measure of voltage or power, named for A. G.
		Bell in a tenth of a bel. given two voltages $V_1$ and $V_2$
		their ratio is $20\log_{10}V_1/V_2$. given two power
	levels $P_1$ and $P_2$, their ratio is $10\log_{10}P_1/P_3$.
	a ratio of 3 dB corresponds to a doubling of power, while 
	1 dB is the smallest humanly detectable chagne in volume.
	also used as a measure of loudness, relative to the softest
	audible sound. a loudness of 74 dB is commonly used as a reference
	for the level produced by a speaker at a distance of 1 ft.
	\eit

\w Mixing: adding vs lookup.

\w Audio programming: delay buffer, clock drift, synchronization, 
timestamp, full duplex vs half duplex. What is simplex? programming API.

\w Video scanning. Why is interlace (interleaving) used?
Video frame rate vs audio frame rate. Color RGB vs YUV, Why?
Frequency domain? Vector quantization vs transform vs model-based vs lossless.
DCT. Subblocks of smaller size. Run-length vs huffman. MPEG vs JPEG. Image prediction frames: I, P, B. Why is I needed. Motion compensation.

\w H.261 vs H.263 vs MPEG-1 vs MPEG-2 vs MPEG-4. What happened to
MPEG-3?  Codec vs file formats. What does real-player use? What is
multiplexing? Audio codec vs Video codec: bitrate, packet size,
CBR/VBR, loss tolerance, loss recovery. RTP? What is FEC? what is
redundant encoding?

\w How does Audio and/or video traffic affect network? what
routing/scheduling methods affect them.
\een

\section*{RTP}
\ben
\w Application level framing (ALF)
	\bit
	\w \bb{ALF}: a new way of designing protocols for emerging multimedia
applications. these new applications are not likely to be well served by
existing protocols such as TCP and furthermore not well served by any
``one-size-fits-all'' protocol. the appliation knows its own needs
best. e.g. MPEG video application knows best on how to recover from lost
frames and segmentation, $\cdots$ RTP leaves so many of the protocol details
to the profile and format documents that are specific to an application
	\eit
\w What is a session, stream, media payload type. What is RTP? What
does RTCP do? Where all is RTP used? Is RTP secure? Does it work with
firewall? Can it work over TCP? Mixer vs translators.

\w How much bandwidth overhead. What are the alternatives?

\w Purpose of RTP header fields: timestamp, sequence number, payload
type, SSRC, marker, extension, etc. Purpose of RTCP header
fields. CSRC vs CNAMEs in RTCP.

\w How is sequence number assigned? how about timestamp? How is RTCP
sending interval calculated for both active senders and
receivers. Full use of RTP in multicast. How are RTP agents
implemented? Port numbers for RTP and RTCP. Separate audio video
sessions vs multiplexed sessions.

\w How to correlate RTP timestamp and Wall-clock time. Calculate
jitter for given sequence of packets. How is RTT calculated.

\w RTP for large groups: what are reconsiderations for calculating
interval? why do you need these? What are alternative.

\w Bundling multiple sessions. Collision detection and
recovery. Collision vs loop detection.

\w Adaptive application based on resources available. How does
audio/video traffic affect the TCP traffic.
\een

\section*{SIP}

\ben
\w What is SIP used for? Addressing. Proxy vs
redirect vs user agent. Registrar vs proxy. User agent client vs user
agent server.
	\bit
	\w \bb{SIP purpose}: find each other and signal to one another
	their desire to communicate (signaling)
	\eit

\w An example call/rounting in SIP. Message structure. Significance
of: request URI, To, From, Call-ID, Via, CSeq, Content-Type,
Content-Length, etc. Response codes. Why do we need status code and status
reason.
	\bit
	\w see IETF draft (around p.9x)
	\eit

\w Request forking. Parallel search. Difference between 4xx/5xx vs
6xx. Advantages and disadvantages. Why do we need a tag in To? why in
From?

\w Given an example of INVITE, format a final response for 401. For
200? Format the corresponding ACK. Write down BYE message format from
the other direction. Given a SIP message sequence find out the
problems (e.g., Call-ID is modified, CSeq is incorrect)

\w Reliability in SIP. Why is ACK needed? SIP over TCP vs UDP. Record
route.  Why is CANCEL needed? What is CSeq in CANCEL-same as original
msg, why? BYE vs CANCEL. 

\w SIP vs SAP vs SDP vs RTP. SIP vs H.323.

\een

\section*{QoS, RSVP and Diffserv}
\ben

\w Loss vs delay vs jitter. Perfect channel. Congestion. Arrival rate vs
service rate.  Why not in circuit switching?
	\bit
	\w see qos slide
	\w \bb{delay}: end-to-end delay. time elapsed bet'n sending and
	receiving a packet
	\w \bb{jitter}: variation in packet delay
	\w \bb{packet loss}: due to buffer overflow
	\eit
\w Delay: queuing delay, propogation delay, processing overhead.
	\bit
	\w see the slide ``figure''
	\w see the Henning's bookchapter: prop delay, transmission delay
	\eit

\w Queueing systems. Poisson model (M) vs general distribution (G)
Independent events. Markov model. Little's result. Number in queue vs
number in system. Service time vs waiting time vs system time.

\w M/M/1 = M/M/1/infinity example. Comparing queues: Calculate
service time, waiting time, efficiency (rho), number in system.

\w Solving markov chain for M/M/n/k. Take some example for n and k.

\w Policing vs scheduling vs reservation.
	\bit
	\w see resource control and reservation slide
	\eit

\w Leaky bucket vs token bucket. What does
initial token bucket being empty mean? What is GCRA? Burstiness. 
	\bit
	\w see slide
	\w \bb{GCRA}: generaic cell rate algorithm
	\eit
 
\w Scheduling. Work conserving vs non-work conserving.  FIFO+, HL,
Bit-by-bit round robin, WFQ (with weights and packets), DRR. Why does
DRR sets the counter to 0 when queue is empty. Evaluating scheduling
algorithms: fairness, priority, long-term average, burstiness,
complexity, implementability. Calculate the end-to-end delay for WFQ:
use hop count as rounter count plus one. Preemptive and non-preemptive
priority queueing.

\w Reservation: dynamic vs static, bulk service vs priority delivery.
Why is it needed? when is it not needed? Different levels of
reservation. Different intervals for reservation.
	\bit
	\w see Crowcroft chapter 2. network service models
	\w \bb{dynamic}:
	\w \bb{static}:
	\eit

\w RSVP: work out a given example for RSVP flows. Explicit vs wildcard
vs shared vs distinct reservation styles. Significance of PATH
message? RESV? When is reservation actually done? Explicit teardown vs
timeout? What are killer reservations. RSVP is just signaling!
	\bit
	\w see Kurose-Ross
	\w \bb{killer reservations}: see ``Spring 1998 final''
	\eit

\w RSVP service classes. RSVP vs ATM. Receiver oriented vs sender
oriented.  Scalability. Reliability. Intserv? Inter-domain vs
Intra-domain reservation. What is SRP?
	\bit
	\w \bb{receiver-oriented vs ...}: rather than having the senders
	keep track of the a potentially large number of receivers'
requirements, it makes more sense to let the receivers keep track of their own
needs
	\w RSVP receivers send refresh messages periodically (soft state)
	\w \bb{RSVP vs ATM} \vspace*{0.1cm}\\
\centerline{\begin{tabular}{l|l}\hline
	RSVP & ATM \\ \hline
	receiver generates reservation & sender generates \\
	soft state (refresh/timeout) & hard state (explicit delete) \\
	separate from route establishment & concurrent with route ... \\
	QoS can change dynamically & QoS static for life of connection\\
	receiver heterogeneity & Uniform QoS to all receivers \\ \hline
	\end{tabular}}
	\eit

\w Diffserv. TOS. Problems in ATM VCs and RSVP. Implementation and
deployment.  Better than best effort. Hop by hop. Aggregation. What is
an SLA? SLS? TCS?  Fits well in the Internet structure. Best effort vs
EF vs AF. AF is kind of relative. EF is absolute. What is RED? RIO? EF
must drop excess packets?
	\bit
	\w \bb{Diffserv}: rather than using a protocol like RSVP(intserv) 
	to tell all the routers that some flow is sending premium packets, it
would be much easier if the packets could just identify themselves to the
router when they are arrive. e.g. {\em TOS} bits
	
	\eit

\w Compare different interoperability models for Intserv and Diffserv.

\w Alternative best effort: lower delay vs high throughput.

\een

\section*{VoIP and RTSP}
\ben
\w What is internet telephony? Describe the two views of IP
telephony-internet based and telephony based? What is SS7? Peer to
peer vs centralized control. What roles do SIP, MGCP, H.323, RTP
play. Why does NAT not work with VoIP. How can we use RSVP for VoIP
calls (signaling path or data path). Exciting new services.

\w Media on demand. What is RTSP? why not just use HTTP? What role
does SDP and RTP play in media on demand. What is the use of single
aggregate control or per-stream control for different media in an RTSP
session. Can RTSP work on UDP? What are similarities and differences
between RTSP and HTTP. Can an RTSP server co-exist with an HTTP
server? what would be an application that requires such
co-existance. How does RTSP deal with firewalls. What is the use of
OPTIONS message? Can RTSP support live presentations? Format a message
sequence for a live-presentation. What would be the difference between
an RTSP client user interface and a regular VCR interface (the
physical buttons on your VCR box!)

\een

%toc
%\pagebreak
%\tableofcontents


\end{document}
