\documentclass{myart}
\usepackage{mydef}
\begin{document}
\small

\title{\Large\bf{}Audio \& Video}
\author{\normalsize{}Cheoljoo Jeong}
\date{\normalsize\today}
\maketitle

\section{Basics}
\bit
\w important factors that affect the quality of audio
	\bit
	\w \bb{sampling rate}: related to frequency bandwidth
		(e.g. CD-quality 44,100 samples/sec)
	\w \bb{number of bits to encode the signal}:
		affects the correctness of the digital representation
		of the signal
		(e.g. CD-quality 16 bits/sample)
	\eit
\w \bb{Nyquist frequency}: sampling frequency must be 
	equal or greater than twice the bandwidth to be transmitted
\w SNR (in dB) = 10 $\log$((peak signal level/root-mean-square level))
\eit

\section{Compression}
\bit
\w compression: $c(x) = x_c$
\w \bb{lossless compression}: $c^{-1}(x_c) = x$
\w \bb{lossy compression}: $c^{-1}(x_c) \not= x$
\w Shannon's {\em self-information\/}:
	\bit
	\w suppose we have an event (e.g. occurrence of a symbol) $A$
	\w $P(A)$: the probability that the event $A$ will occur
	\w $i(A)$: self-information associated with $A$
		\[ i(A) = \log_b\frac{1}{P(A)} = -\log_bP(A)\]
	\w if $P(A)$ is low, $i(A)$ is high; that is, {\em if the probability
	of the event is low, the amount of self-information associated with
	it is high\/}\footnote{if some rare event has occurred, it contains 
	much information! }
	\w if $P(A)$ is high, $i(A)$ is low
	\eit
\w Information `from' two independent events is the sum of the
	individual informations.
   \[ i(AB) = \log_b\frac{1}{P(AB)} = \log_b\frac{1}{P(A)P(B)}
	= i(A) + i(B).\]
\w If base $b = 2$, the unit is {\bf bits\/} (if base is $e$, unit is \bb{nats}
	and if base is $10$, the unit is \bb{hartleys})
	
\w Example: let $H$ and $T$ be two outcomes of coin flipping;
	$P(H) = P(T) = 1/2$ and $i(A) = i(B) = 1$ bit
\eit

\subsection{Lossless schemes}
\bit
\w \bb{Runlength-encoding}
\w \bb{Huffman coding}: based on statistics of frequency of occurrences
\w \bb{Lempel--Ziv family of algorithms}: based on substitutional or
	disctionary schemes
\eit
\subsection{Lossy schemes}
\bit
\w 
\eit


\end{document}
