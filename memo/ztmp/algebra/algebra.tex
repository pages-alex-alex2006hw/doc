\documentclass{myproc}
\usepackage{mydef}
\def\sbf{\sf\bfseries}
\begin{document}
\title{\large\bf Notes on Abstract Algebra \vspace*{-0.5cm}}
\author{\normalsize{}Cheoljoo Jeong $\arc{\mbox{cjeong@cs.columbia.edu}}$}
\maketitle
\small
%\vspace*{-2.5cm}

\section{Groups and Subgroups}
\subsection{Binary operations}
\bit
\w A \bb{binary operation} $*$ 
on a set $S$ is a \bb{function} mapping $S \times
S$ into $S$. For each $(a, b) \in S \times S$, we will denote the element
$*((a, b))$ of $S$ by $a * b$.
\w Let $*$ be a binary operation on $S$ and let $H$ be a subset of $S$. The
subset $H$ is \bb{closed under} $*$ if for all $a, b \in H$, $a * b \in H$. In
this Case, the binary operation on $H$ given by restricting $*$ to $H$ is the
\bb{induced operation} of $*$ on $H$.
\w A binary operation $*$ on a set $S$ is \bb{commutative} if and only if $a *
b = b * a$ for all $a, b \in S$.
\w A binary operation $*$ on a set $S$ is \bb{associative} if and only if $a *
(b * c) = (a * b) * c$ for all $a, b, c \in S$.
\w {\sbf Theorem}: Let $S$ be a set and let $f, g$ and $h$ be functions
mapping $S$ into $S$. Then, $f \circ (g \circ h) = (f \circ g) \circ h$.
\eit

\subsection{Isomorphic binary structures}
\bit
\w Let $\arc{S, *}$ and $\arc{S', *'}$ be binary algebraic structures. An
\bb{isomorphism of $S$} with $S'$ is a {\em one-to-one\/} $\phi$ mapping $S$
onto $S'$ such that
 \[ \phi(x * y) = \phi(x) *' \phi(y) \mbox{\ \ \ for all $x, y \in S$}.\]
If such a map $\phi$ exists, then $S$ and $S'$ are \bb{isomorphic binary
  structures} which we denote by $S \simeq S'$.
\w A \bb{structural property} of a binary structure  is one that must be
  shared by any isomorphic structure.
\w A structural property that is characterized in terms of the binary
  operation $*$, like associativity, is an \bb{algebraic property}.
\w Let $\arc{S, *}$ be a binary structure. An element $e$ of $S$ is an
  \bb{identity for} $*$ if $e * s = s * e = s$ for all $s \in S$.
\w {\sbf Theorem}: 
  A binary structure $\arc{S, *}$ has at most one identity. That is, if there
  is an identity, it is unique.
\w {\sbf Theorem}: Suppose $\arc{S, *}$ has identity $e$ for $*$. If $\phi: S
  \rightarrow S'$ is an isomorphism of $\arc{S, *}$ with $\arc{S', *'}$ then
  $\phi(e)$ is identity for the binary operation $*'$ on $S'$.
\eit
\subsection{Groups}
\bit
\w A \bb{group} $\arc{G, *}$ is a set $G$, closed under a binary operation
$*$, such that the following axioms are satisfied:
  \ben
  \w [(a)] (\bb{associativity}) 
  For all $a, b, c \in G$, $(a * b) * c = a * (b * c)$.
  \w [(b)] (\bb{identity})
  There is an element $e$ in $G$ such that for all $x \in G$,
  $e * x = x * e = x$.
  \w [(c)] (\bb{inverse})
  Corresponding to each $a \in G$, there is an element 
  $a'$ in $G$ such that for all $x \in G$,
  $a * a' = a' * a = e$.
  \een
\w A group $G$ is \bb{abelian} if its binary operation $*$ is commutative.
\w Vector space $V$ under vector addition is an abelian group.
\w {\sbf Theorem}: If $G$ is a group with binary operation $*$, then \bb{the
  left and right cancellation laws} hold in $G$, that is, $a * b = a * c$
implies $b = c$, and $b * a = c * a$ implies $b = c$ where $a, b, c \in G$.
\w {\sbf Theorem}: If $G$ is a group with binary operation $*$, and if $a$ and
$b$ are any elements of $G$, then the linear equations $a * x = b$ and $y * a
= b$ have unique solutions in $G$.
\w {\sbf Theorem}: In a group $G$,  the identity element $e$ is
unique and inverses for each element of $G$ are unique in the group.
\w {\sbf Corollary}: Let $G$ be a group. For all $a, b \in G$, we have $(a *
b)' = b' * a'$.
%\w A \bb{monoid} is a semigroup that has an identity element for the binary
%operation. 
\eit

\subsection{Subgroups}
\bit
\w If $G$ is a finite group, then the \bb{order} $|G|$ of $G$ is the number of
elements in $G$. 
\w If a subset $H$ of a group $G$ is closed under the binary operation of $G$
and if $H$ with the induced operation from $G$ is itself a group, then $H$ is
a \bb{subgroup of} $G$. We shall let $H \le G$ or $G \ge H$ denote that $H$ is
a subgroup of $G$, and $H < G$ or $G > H$ shall mean $H \le G$ but $H \ne G$.
\w If $G$ is a groiup, then the subgroup consisting of $G$ itself is the
\bb{improper subgroup} of $G$. All other subgroups are \bb{proper
  subgroups}. The subgroup $\{e\}$ is the \bb{trivial subgroup} of $G$. All
other groups are \bb{nontrivial}.
\w {\sbf Theorem}: A subset $H$ of a group $G$ is a subgroup of $G$ if and
only if 
  \ben
  \w [(a)] $H$ is closed under the binary operation of $G$,
  \w [(b)] the identity $e$ of $G$ is in $H$,
  \w [(c)] for all $a \in H$ it is true that $a^{-1} \in H$ also.
  \een
\w {\sbf Theorem}: Let $G$ be a group and let $a \in G$. Then
  \[ H = \{a^n: n \in \Z\}\]
 is a subgroup of $G$ and is the smallest subgroup of $G$ that contains $a$,
 that is, every subgroup containing $a$ contains $H$.
\w The group $H$ in the above theorem is the \bb{cyclic subgroup of $G$
 generated by $a$}, and will be denoted by $\arc{a}$.
\w An element $a$ of a group $G$ \bb{generates} $G$ and is a \bb{generator
 for} $G$ if $\arc{a} = G$. A group $G$ is \bb{cyclic} if there is some
 element $a$ that generates $G$.
\eit

\subsection{Cycle groups and generators}
\bit
\w {\sbf Theorem}: Every cyclic group is abelian.
\w {\sbf Theorem}: Every subgroup of a cyclic group is cyclic.
\w {\sbf Corollary}: The subgroups of $\Z$ udner addition are precisely the
group $n\Z$ under addition for $n \in \Z$.
\w Let $r$ and $s$ be two positive integers. The positive generator $d$ of the
cyclic group
 \[ H = \{nr + ms: n, m \in \Z\} \]
under addition is the \bb{greatest common divisor} (or \bb{gcd}) of $r$ and
$s$. 
\w Let $n$ be a fixed positive integer and let $h$ and $k$ be anyintegers. The
remainder $r$ when $h + k$ is divided by $n$ in accord with the division
algorithm is the \bb{sum of $h$ and $k$ modulo $n$}.
\w {\sbf Theorem}: The set $\Z_n = \{0, 1, \cdots, n-1\}$ under addition
modulo $n$ is a cyclic group.
\w {\sbf Theorem}: Let $G$ be a cyclic group with $n$ elements and generated
by $a$. Let $b \in G$ and let $b = a^s$. Then $b$ generates a cyclic subgroup
$H$ of $G$ containing $n/d$ elements, where $d$ is the greated common divisor
of $n$ and $s$.
\w {\sbf Corollary}: If $a$ is a generator of a finite cyclic group $G$ or
order $n$, then the other generators of $G$ are the elements of form $a^r$,
where $r$ is relatively prime to $n$.
\w Let $\{S_i: i \in I\}$ be a collection of sets. Here $I$ may be any set of
indices. The \bb{intersection $\cap_{i\in I} S_i$ of the sets $S_i$} is the
set of all elements that are in all the sets $S_i$; that is,
 \[ \bigcap_{i\in I}S_i = \{x: x \in S_i \mbox{\ for all\ } i \in E\}.\]
\w {\sbf Theorem}: The intersection of subgroups $H_i$ of a group $G$ for $i
\in I$ is a subgroup of $G$.
\w Let $G$ be a group and let $a_i \in G$ for $i \in I$. The smallest subgroup
of $G$ containing $\{a_i: i \in I\}$ is the \bb{subgroup generated by $\{a_i:
  i \in I\}$}. If this subgroiup is all of $G$, then $\{a_i: i \in I\}$
  \bb{generates} $G$ and the $a_i$ are \bb{generators of} $G$. If there is a
  finite set $\{a_i: i \in I\}$ that generates $G$, then $G$ is \bb{finitely
    generated}. 
\w {\sbf Theorem}: If $G$ is a group and $a_i \in G$ for $i \in I$, then the
subgroup $H$ of $G$ generated by $\{a_i: i \in I\}$ has as elements precisely
those elements of $G$ that are finite products of integral powers of the
$a_i$, where powers of a fixed $a_i$ may occur several times in the product.
\eit



\bibliographystyle{plain}
\bibliography{00bib/mac,00bib/math,00bib/algo}
\nocite{Fraleigh99}
\end{document}


% LocalWords:  Cheoljoo Jeong vertices endvertices endvertex cutvertex bc algo
% LocalWords:  cutvertices abelian
