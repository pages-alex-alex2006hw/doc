\section{NP-COMPLETENESS}
\subsection{Basic definitions}
\bit
\w A \bb{decision problem} $\Pi$ consists of a set $D_\Pi$ of \bb{instances}
and $Y_\Pi \subseteq D_\Pi$ of \bb{yes-instances}.
\w The problem $\Pi$ and the {\em encoding scheme} $e$ for $\Pi$ paritions
$\Sigma^*$ into three classes of strings:
   \ben
   \w [(a)] those that encode instances of $\Pi$ for which answer is ``YES'',
   \w [(b)] those that encode instances of $\Pi$ for which answer is ``NO'',
   and
   \w [(c)] those that are not valid encoding of instances of $\Pi$.
   \een
\w For (a), we can associate a \bb{language}
  $L[\Pi, e] = \{x \in \Sigma^*: \mbox{\ $x$ is the encoding under $e$ of an
    instance $I \in Y_\Pi$}\}.$
\w The class $P$ of problems are defined as follows:
  \[ \mbox{P}\ = \{L: \mbox{there is a polynomial time DTM program $M$ for
    which $L = L_M$}\}.\]
\eit

\subsection{Relationships between P and NP}
\bit
\w If $\Pi \in $NP, there exists a polynomial $p$ such that $\Pi$ can be
solved by a determinstic algorithm having time complexity $O(2^{p(n)})$.
\w If $L_1 \propto L_2$, $L_2 \in P$ implies $L_1 \in P$.
\w A language is \bb{NP-complete} if $L \in $\ NP and, for any other language
$L' \in $\ NP,\  $L' \propto L$.
\eit

\subsection{Cook's theorem: SATISFIABILITY is NP-complete}

\subsection{NP-hardness}
\bit
\w Any decision problem $\Pi$, {\em whether a member of NP or not}, to which
we can transform an NP-complete problem will have the property that it cannot
be solved in polynimial time unless P = NP. These problems are \bb{NP-hard}.
\w Formally, a string relation $R$ is \bb{NP-hard} if there is some
NP-complete language $L$ (itself state as a string relation) such that $L
\propto_T R$. 
\eit

\subsection{Polynomial hierarchy}
\bit
\w co-NP = $\{\Pi^c: \Pi \in \mbox{\ NP}\} =
   \{\Sigma^* - L: L \mbox{\ is a language over\ } \Sigma \mbox{\ and\ }
   L \in\ \mbox{NP}\}$.
\eit