\section{TEMPORAL LOGIC}
\subsection{Basics}
\bit
\w $Fq$ is true in the \ee{present} if $q$ is true {\em at some time in the
  future}. 
\w $Pq$ is true in the \ee{present} if $q$ is true {\em at some point in the
  past}. 
\w $Gq$ is true in the \ee{present} if $q$ is {\em always\/} true {\em in the
  future\/}.  
  That is, $Gq \equiv \neg P\neg q$. 
\w $Hq$ is true in the \ee{present} if $q$ is {\em always\/} true {\em in the
  past\/}.  
  That is, $Hq \equiv \neg P\neg q$. 
\eit
\subsection{Possible worlds semantics}
\bit
\w A \bb{frame} consists of
a class $S$ of \bb{states} through which the system evolves, and a relation
$<$ representing \bb{temporal order}. A \bb{model} of a temporal logic 
is a frame with a \bb{valuation} $L$, which assigns truth
values to every atomic proposition in every state. 
This model is essentially a \bb{Kripke structure} $(S, <, L)$.


\w A {\em temporal formula\/} is a term with one free parameter representing
the present state. That is, a formula defines a function $f: S \rightarrow
\{T, F\}$. 

Now, $f^{-1}(T)$ returns the set of states where the {\em given\/} 
formula evaluates to true.
\eit

\subsection{Basic deduction system for temporal logic}
\bit
\w The following choice of \bb{axioms} in the logic characterize the temporal
ordering relation $<$.
  \begin{eqnarray*}
   G(p \Rightarrow q) & \Rightarrow & (Gp \Rightarrow Gq)\\
   H(p \Rightarrow q) & \Rightarrow & (Hp \Rightarrow Hq)\\
   p & \Rightarrow & GPp\\
   p & \Rightarrow & HFp
  \end{eqnarray*}
\w Two \bb{inference rules} are used: \bb{modus ponens} and \bb{temporal
  generalization}. 
%  \begin{eqnarray*}
  \[\infer{q}{p, p \Rightarrow q} \quad \mbox{(MODUS PONENS)}, \qquad\qquad
  \infer{Gq, Hq}{q}\quad \mbox{(TEMPORAL GENERALIZATION)}\]
%  \end{eqnarray*}
  Temporal generalization means that the rules of sound inference does not
  change with time. That is, if $q$ is derivable at some point, it should be
  derived at any future or at any past.
  
\eit

\subsection{Linear time}
\bit
\w We think of time as a \bb{linearly ordered set}.
\w A frame is linearly ordered if the temporal order $<$ is \bb{total}.
\w Two more axioms are needed:
  \begin{eqnarray*}
   (FPq) & \Rightarrow & (Pq \vee q \vee Fq)\\
  (PFq) & \Rightarrow & (Pq \vee q \vee Fq)
  \end{eqnarray*}
\w Linear time temporal logic is usually extended with the \bb{until} operator
and the \bb{since} operator. 
\w $p U q$ states that $p$ will hold at some moment
in the future, until which time $q$ will hold at all moments.
\w $p S q$ states that $p$ held at some moment in the past, since which time
$q$ has held at all moments.
\eit

\subsection{Discrete time}
\bit
\w Discrete time frames can be characterized by adding the following two
axioms to those for \bb{linear time logic}:
  \begin{eqnarray*}
   p \wedge Hp   & \Rightarrow & FH p \\
   p \wedge Gp   & \Rightarrow & PG p 
  \end{eqnarray*}
\eit

\subsection{Branching (non-deterministic) time}