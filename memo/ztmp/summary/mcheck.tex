\section{MODEL CHECKING}
\subsection{Model checking problem}
\bit
\w While axiomatic approach is based on the notion of \bb{implication}, model
checking approach uses the notion of \bb{satisfaction}.
\w 
\eit

\subsection{Symbolic model checking}
\subsubsection{Symbolic representation of synchronous circuits} 
\paragraph{Representing a set $S$ of states}
\bit
\w To represent  $S$, we
   consider the set $V$ of \bb{variables}.
   That is, $V$ allows a kind of \bb{state encoding} of $S$.
   In case of {\em synchronous circuits\/}, $V$ is a set of \bb{input and
     output 
     variables}, and for {\em asynchronous circuits\/}, $V$ is a set of \bb{all
     nodes}.
   {\em Each state $s \in S$ is characterized by a valuation to $V$\/}.
\w Each valuation can be represented by a Boolean formula. 
   And, each Boolean formula represents a set of valuations, i.e. a set of
   states. (Example: $\{v_0 \leftarrow 1, v_1 \leftarrow 0\}$ can be
   represented by a formula $v_0v_1'$. And, a formula $v_0$ represents a set of
   valuations where 
   $v_0 = 1$)
\w Note that not all valuations are valid state encodings. For example, if
there are 5 states and 3 variables, only 5 out of 8 valuations are valid.
\w Any Boolean formula or Boolean function can be represented by a BDD. And,
the set of valuations which represents a valid state can be represented using
a BDD over the set of variables $V$. Let's denote this by $S(V)$.
\eit
\paragraph{Representing a transition relation over $S$}
\bit
\w A transition relation is an ordered set $(V, V)$ of states.
\w To represent a transition relation with a BDD, we create a \bb{copy}
$\tilde{V}$ 
of $V$. The BDD $N(V, \tilde{V})$ for a transition relation is over $V \cup
\tilde{V}$. 
\w For each variable $v \in V$, we have a Boolean function (next-state
function) over $V$ which determines the next value $v$. As an example, if
the next-state functions are given as
   \[ \tilde{v_0} = v_0', \qquad \tilde{v_1} = v_0 \oplus v_1, \]
$N(V, \tilde{V})$ is ANDing of the following
 two BDDs for each next-state function as follows:
  \begin{eqnarray*}
  N_0(V, \tilde{V}) & = & (\tilde{v_0} \Leftrightarrow v_0')\\
  N_1(V, \tilde{V}) & = & (\tilde{v_1} \Leftrightarrow v_0 \oplus v_1)
  \end{eqnarray*}
\eit

\subsubsection{Symbolic representation of asynchronous circuits} 
\bit
\w A state of an asynchronous circuit is determined by the set values on all
components in the circuits.
\w In speed-independent circuits, there can be an arbitrary delay between when
a transition is {\em enabled} and when it {\em actually occurs\/}. 
We model this by allowing each component to choose whether to transition its
output \bb{nondeterministically}, i.e.
   \[ N_i(V, \tilde{V}) = (\tilde{v_i} \Leftrightarrow f_i(V))\ \vee\
    (\tilde{v_i} \Leftrightarrow v_i).\]
\w In an \bb{interleaving semantic model}, we allow only one transition at a
time. 
\eit

\subsubsection{Computing reachable states from $S(V)$ and $N(V, \tilde{V})$}

\subsection{Abstract refinement}

