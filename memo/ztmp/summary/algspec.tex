\section{ALGEBRAIC SPECIFICATION}
\subsection{Basic definitions}
\bit
\w A \bb{signature} $SIG = (S, OP)$ consists of a set $S$ of \bb{sorts} and a
set $OP$ of \bb{constant and operation symbols}.
The set $OP$ is the pairwise-disjoint union of the set $K_s$ of constant
symbols and $OP_{w, s}$ of operation symbols, where $w \in S^+$.
\w An \bb{algebra} $A = (S_A, OP_A)$ of a signature $SIG = (S, OP)$, also
called \bb{$SIG$-algebra}, is given by two families $S_A = (A_s)_{s \in S}$
and 
$OP_A = (N_A)_{N \in OP}$ where
  \ben
  \w [(a)] $A_s$ are sets for all $s \in S$, called \bb{base sets} or
  \bb{domains} of $A$
  \w [(b)] $N_A$ are elements $N_A \in A_s$ for all contant symbols $N \in
  K_s$ i.e. $N: \rightarrow s$ and $s \in S$, called \bb{constant} of $A$.
  \w [(c)] $N_A: A_{s_1} \times \cdots \times A_{s_n} \rightarrow A_s$ are
  functions for all operation symbols $N \in OP_{s_1\cdots{s_n},s}$ (i.e. $N:
  s_1\cdots s_n \rightarrow s$) and $s_1\cdots s_n \in S^+$, $s \in S$ called
  \bb{operations} of $A$.
  \een

\w Let $T_{OP}$ be the set of terms of a signature $SIG = (S, OP)$ and $A$ a
$SIG$-algebra. The \bb{evaluation} $eval: T_{OP} \rightarrow A$ is recursively
defined by 
  \ben
  \w [(a)] $eval(N) = N_A$, \ \ for all constant symbols $N \in K$.
  \w [(b)] $eval(N(t_1, \cdots, t_n)) = N_A(eval(t_1), \cdots, eval(t_n))$ for
  all $N(t_1, \cdots, t_n) \in T_{OP}$. 
  \een
\w Given a set of variables $X$ for $SIG = (S, OP)$ and an \bb{assignment}
$\sigma: X \rightarrow A$ with $\sigma(x) \in A_s$ for $x \in X_s$ and $s \in
S$. The \bb{extended assignment}, or simply \bb{extension} 
$\overline{\sigma}: T_{OP}(X) \rightarrow A$ of the assignment $\sigma: X
\rightarrow A$ is recursively defined by 
  \ben
  \w [(a)] $\overline{\sigma}(x) = \sigma(x)$ for all variables $x \in X$;
     $\overline{\sigma}(N) = N_A$ for all constant symbols $N \in K$. 
  \w [(b)] $\overline{\sigma}(N(t_1, \cdots, t_n)) = 
      N(\overline{\sigma}(t_1), \cdots, \overline{\sigma}(t_2))$ for all $N(t_1,
      \cdots, t_n) \in T_{OP}(X)$. 
  \een
\eit

\subsection{Equational specifications and derivations}
\bit
\w Given a signature $SIG = (S, OP)$ and variables $X$ w.r.t. $SIG$,
a triple $e = (X, L, R)$ with $L, R \in T_{OP, s}(X)$ for some $s \in S$
is called an \bb{equation} of sort $s$ w.r.t. $SIG$.
\w The equation $e = (X, L, R)$ is called \bb{valid} in a $SIG$-algebra
 $A$ if for all assignments $\sigma: X \rightarrow A$ we have
$\overline{\sigma}(L) = \overline{\sigma}(R)$ where $\overline{\sigma}$
is the extended assignment of $\sigma$. 
\w If $e$ is \bb{valid} in $A$, we also say that $A$ \bb{satisfies} $e$.
\w \bb{Ground equations} are equations $e = (X, L, R)$ with $X =
\emptyset$. In this case $L$ and $R$ are \bb{ground terms}.
\eit

\subsection{Specification and $SPEC$-algebra}
\bit
\w A \bb{specification} $SPEC = (S, OP, E)$ consists of a \bb{signature} $SIG
= (S, OP)$ and a set $E$ of \bb{equations} $e$ w.r.t. $SIG$.
\w An \bb{algebra} of the specification $SPEC$, \bb{$SPEC$-algebra}, is an
algebra $A$ of the signature $SIG$ which satisfies all equations in $E$.
\eit


\subsection{Initial semantics of specifications}



