\section{GRAPH THEORY}
\subsection{Trees}
\bit
\w An edge $e$ of $G$ is a \bb{cut edge} iff $e$ is contained in {\em no
  cycle\/} of $G$. 
\w Let $T$ be a spanning tree of a connected graph $G$ and let $e$ be an edge
of $G$ not in $T$. Then $T + e$ contains a {\em unique\/} cycle.
\eit

\subsection{Connectivity}
\bit
\w The \bb{connectivity} $\kappa(G)$ of 
\eit


\subsection{Independent cycles and cocycles}
\bit
\w A \bb{0-chain} of $G$ is a linear combination $\sum \epsilon_i v_i$
of vertices and a \bb{1-chain} is a sum $\sum \epsilon_i e_i$ of edges,
where $\epsilon_i = 0$ or $1$. 
\w The \bb{boundary operator} $\partial$ sends 1-chains to 0-chains according
to the rules:
  \ben
  \w [(a)] $\partial$ is linear.
  \w [(b)] if $e = (u, v)$, then $\partial x = u + v$.
  \een
\w The \bb{coboundary operator} $\delta$ sends 0-chains to 1-chains by the
rules. 
  \ben
  \w [(a)] $\delta$ is linear.
  \w [(b)] if $\delta v = \sum \epsilon_i e_i$, where $\epsilon_i = 1$
  whenever $x_i$ is incident with $v$.
  \een
\eit


\subsection{Maximum matching}
\bit
\w A subset $M$ of $E$ is a \bb{matching} in $G$ if {\em no two edges in $M$
  are adjacent in $G$\/}.  A matching $M$ is said to \bb{saturate} a vertex
$v$ if some edge of $M$ is incident to $v$. 
\w $M$ is a \bb{maximum matching} if $G$ has no matching $M'$ with $|M'| >
|M|$. A \bb{maximal matching} is one which fails to be so when we add any new
edge. 
\w An \bb{$M$-alternating path} in $G$ is a path whose edges are alternately 
in $E - M$ and $M$. An \bb{$M$-augmenting path} is an $M$-alternating path
whose starting and ending vertices are {\em not $M$-saturated\/}.

\w \bb{Berge's Theorem}: A matching $M$ is a maximum matching iff
$G$ contains no $M$-augmenting path.

\eit

\subsection{Bipartite matching}
\bit
\w \bb{Hall's Theorem for bipartite matching}: 
   Let $G$ be a bipartite graph with bipartition $(X, Y)$. Then
$G$ contains a matching that saturates every vertex in $X$ iff
   $|\Gamma(S)| \ge |S|$ for all $S \subseteq X$, where $\Gamma(S)$ is the
   neighbor 
   set of $S$. 

\w \bb{Matching-covering lemma}: Let $M$ be a matching and $K$ be a covering
   s.t. $|M| = |K|$.  
   Then $M$ is a maximum matching and $K$ is a minimum covering.  
\w \bb{Theorem}: In a bipartite graph, the number of edges in a maximum
   matching is equal to the number of vertices in a minimum covering.
\eit

\subsection{Perfect matching}
\bit
\w \bb{Tutte's theorem}: $G$ has a perfect matching iff
  $o(G - S) \le |S|$ for all $S \subset V$, where $o(G)$ is the number of odd
  components of $G$. 
\eit

\subsection{Independence sets}
\bit
\w A subset $S$ of $V$ is an \bb{independence set} of $G$ if no two
vertices of $S$ are adjacent in $G$. 
An independent set $S$ is a \bb{maximum independent set} if
	$G$ has no independent set $S'$ with $|S'| \supseteq |S|$.
\w A subset $V'$ of $V$ is a \bb{covering} of $G$ if every edge
of $G$ has at least one end in $V'$.
\w {\bf Independent-set-covering Theorem}: 
A set $S \subseteq V$ is an independent set of $G$ iff $V -
S$ is a covering of $G$.

%% \w The size of a maximum independence set is called the
%% \bb{independence number} of $G$ and is denoted by $\alpha(G)$.
%% \w The size of a minimum covering of $G$ is the \bb{covering
%% number} of $G$ and is denoted by $\beta(G)$.
%% \w {\sbf Theorem}: {\sf
%% For any graph $G = (V, E)$, $\ \alpha(G) + \beta(G) = |V|$.
%% }
\w An \bb{edge covering} of $G$ is a subset $E'$ of $E$ such that
each vertex of $G$ is incident to some edge in $E'$.
\w Edge analogue of an independent set is a set of edges which are
pairwise non-adjacent, that is, a {\em matching\/}.
%% \w We denote the number of edges in a \bb{maximum matching} of $G$ by
%% $\alpha'(G)$ and call it the \bb{edge independence number}.
%% \w We denote the size of minimum edge cover of $G$ by $\beta'(G)$ and
%% call it the \bb{edge covering number}.
%% \w {\sbf Theorem}: {\sf
%%   For any graph $G = (V, E)$, if $\delta > 0$, 
%%   then $\alpha'(G) + \beta'(G) = |V|$.}
%% \w {\sbf Theorem}: {\sf
%%   In a bipartite graph $G$ with $\delta > 0$, the number of
%% vertices in a maximum independent set is equal to the number of edges
%% in a minimum edge cover.}
\eit

\subsection{Ramsey numbers}
\bit
\w Given any positive integers $k$ and $l$, there exists a smallest integer
$R(k, l)$ s.t. every graph on $R(k, l)$ vertices contains either a clique of
$k$ vertices or an independent set of $l$ vertices. $R(k, l)$ is called a
\bb{Ramsey number}. 
\w \bb{Erd\"{o}s-Szekeres theorem on Ramsey numbers}: 
   For any two integers $k \ge 2$ and $l \ge 2$,
   \[ R(k, l) \le r(k, l-1) + r(k-1, l).\]

\w \bb{Tur\'{a}n's theorem}: If $G$ is simple and contains no $K_{m+1}$, then
$\epsilon(G) \le \epsilon(T_{m,\nu})$. Moreover, $\epsilon(G) =
  \epsilon(T_{m, \nu})$ only if $G \cong T_{m, \nu}$. 
\eit


