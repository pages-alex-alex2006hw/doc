\section{AUTOMATIC THEOREM PROVING}
\subsection{Basic results}
\bit
\w\bb{G\"{o}del's completeness theorem}: there exist logical calculi
in which every {\em true\/} formula is provable
\w \bb{Church's undecidability theorem}:
there is no decision procedure for deciding whether a formula is true (valid)
\w Church's theorem states that it's impossible to find an algorithm
that halts and says NO when the theorem is not true.
\eit


\subsection{Horn clauses}
\bit
\w Boolean expression of the form $\bigwedge_i C_i$ is said to be in a
\bb{conjunctive normal form} and $C_i$, a disjunction of literals, is called a
\bb{clause}. 
\w Boolean expression $\bigvee_i D_i$ is said to be in a
\bb{disjunctive normal form} and $D_i$, a conjunction of literals, is called a
\bb{implicant}. 
\w A \bb{Horn clause} is a clause which has \bb{at most one positive
  literal}.
\w (EXAMPLE) $(\overline{x_1} \vee \overline{x_2} \vee x_3)$ is a Horn clause
   with one positive literal 
   and can be rewritten as $((x_1 \wedge x_2) \rightarrow x_3)$.
\w Satisfiability of CNF of Horn clauses can be determined in polynimial time.
   (HORNSAT $\in$ NP).
\eit
