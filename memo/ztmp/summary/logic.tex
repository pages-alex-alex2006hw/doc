\section{MATHEMATICAL LOGIC}
\subsection{Basic definitions}
\bit
\w A logical {\em language\/} consists of a set of \bb{sorts} and
\bb{operators}. 
\w An \bb{application} consists of an operator and a tuple of terms, each of
which has the same sort as the corresponding domain sort for the operator.


\w When a sentence is true in a structure, we say that the structure is a
\bb{model} of the sentence.

\w A \bb{sentence} is a predicate with {\em no free variables\/}.
\w A sentence $S$ is a \bb{logical consequence} of a set $T$ of sentences if
  every model of $T$ is also a model of $S$.

\w A set of sentences is \bb{closed} under logical consequence if it contains
all its logical consequences.
\w A \bb{theory} is a set of sentences closed under logical consequence.
\w A theory is \bb{complete} if for every sentence $S$, either $S$ or $\neg S$
is in the theory.

\w A set of sentences is \bb{consistent} if it has a model.
  \bit
  \w \bb{Theorem}: A sentence $S$ is a consequence of a set $T$ of sentences
  if and only if $T \cup \{\neg S\}$ is inconsistent. (foundation of \bb{resolution method})
  \eit
\w A theory is \bb{consistent} if and only if it does not contain a
  \bb{contradition}, that is, a sentence $\mbox{\tt{}TRUE} =
  \mbox{\tt{}FALSE}$.\eit

\subsection{Proofs and consequences}
\bit
\w The meaning ``$S$ is a logical consequence of $T$'' defined above
(e.g. every model of $T$ is a model of $S$) is semantic description not
{\em syntactic\/}. 
\w \bb{Syntactic characterization} of ``$S$ is a logical consequence of $T$'':
   is needed!
\w A formal \bb{deduction system} consists of sentences (\bb{logical axioms})
and a set of functions (\bb{deduction rules}). 
   \bit
   \w A deduction rule maps a finite
   set of sentences 
   (\bb{premises}) to a single sentence (\bb{conclusion}). As an EXAMPLE,
  \[\infer{Q}{P, P \Rightarrow Q} \qquad \ \mbox{(MODUS PONENS)}\]
   \eit
\w A \bb{proof} based on a set $T$ of sentences is a finite sequence of
sentences each of which is either a logical axiom, a member of $T$, or the
conclusion of a deduction rule applied to a set of sentences occurring earlier
in the proof.
\w A sentence $S$ is a \bb{theorem} of $T$ if it occurs in some proof based on
$T$. 
\w \ee{Three properties of a formal system}:
  \bit
  \w A system is \bb{sound} if, for any $T$, every theorem of $T$ is really a
  logical consequence of $T$. (NO SPURIOUS PROOF)
  \w A system is \bb{complete} if, for any $T$, every logical consequence of
  $T$ is also a theorem of $T$.
  \w A system is \bb{effective} if, for any computable set $T$ of sentences,
  the 
   set of proofs based on $T$ is also computable.
  \eit
\eit

\subsection{Notations}
\bit
\w $T \vdash S$: $T$ implies $S$
\w $T \models S$: $T$ satisfies $S$
\eit