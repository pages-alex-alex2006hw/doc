\documentclass{note}
\usepackage{mydef}
\usepackage[all]{xy}
\begin{document}
\title{\large\bf Notes on Computer-Aided Design of Digital Circuits}
\author{\normalsize Cheoljoo Jeong (cjeong@cs.columbia.edu)}
\date{}
\maketitle
\section{Backgrounds}
\bit
\w \bb{counter}: register with additional logic that cycle the register's
contents through a predefined sequence of states
\eit

\section{Sequential Synthesis}
\bit
\w \bb{storage elements}:
   \bit
   \w SR latch:
   \w gated SR latch:
   \w \bb{gated (level-sensitive) latches vs. flip-flops}:
Gated latches are memory elements that are enabled {\em during the entire time
  interval\/}, during which the control signal $C$ equals $1$.
These latches are often called \bb{level-sensitive latches\/} because they are
  enabled whenever the control signal is at $1$.
We say that at any point during the time when $C = 1$, the latches are
  \bb{transparent\/}, in the sense that any input changes will propagate to
  the output with some delay.
The latches behave as memory elements only after the falling edge of the
  control signal, when they retain the state set by the last input value that
  occurred before the falling edge of the control signal.
   \eit
\w \bb{steps of sequential synthesis}
   \bit
   \w [(1)] \bb{build a state-diagram}
   \w [(2)] \bb{derive next-state table and output table}
   \w [(3)] \bb{derive next-state equations and output equations}
   \w [(4)] \bb{state minimization}
   \w [(5)] \bb{state assignment}
   \eit
\eit
\subsection{FSMD (FSM with a data path)}
\bit
\w \bb{FSM} (no datapath): $(S, I, O, f, h)$
  \bit
  \w $S$: set of states
  \w $I$: set of inputs
  \w $O$: set of outputs
  \w $f: S \times I \rightarrow S$: next-state function
  \w $h: S \times I \rightarrow O$: output function (Mealy-type)
  \eit
\w \bb{FSMD}: $(S, V, I, O, f, h)$
  \bit
  \w $S$:
  \w $V$: set of \bb{variables}
  \w $I = I_C \times I_D$
  \w $O = O_C \times O_D$
  \eit
\eit

\section{Architectural Synthesis}

\[\xymatrix@-0.7pc{
 \txt<10pc>{Behavioral Model (Data-flow/Sequencing Graphs)} \ar[r] & 
  *+[F]{\txt<5pc>{Architectural Synthesis}} \ar[r] &
 \txt<8pc>{Structural View of the Circuit (Datapath + Control unit)}
}\]
\bit
\w \bb{datapath}: interconnection of 
   \bit
   \w \bb{resources}: implementation of arithmetic or logic functions
   \w \bb{steering logic}: sends data to the appropriate place at appropriate
 time (e.g. multiplexor, bus)
   \w \bb{registers/memory array}: store data
   \eit
\w objectives in implementations:
   small \bb{area}, small \bb{delay}, low \bb{power}, high \bb{testability}
   \bit
   \w multi-objective functions
   \eit
\w \bb{resource-bounded circuits}: a class of circuits where {\em area\/} and
   {\em performance} depend (mostly) \bb{only on resources}
\w \bb{design space}: collection of all feasible implementations (solutions)
 that satisfy the specification
\w \bb{Pareto points}: a point in the design space which are NOT dominated by
 other points in AT LEAST ONE objective (dimension)
\w \bb{architectural exploration}: traversing the design space and providing a
 spectrum of feasible non-inferior solutions, among which a designer can pick
 the desired implementation
\eit

\subsection{Circuit specification for architectural synthesis}
\bit
\w \bb{data-flow graph}: $G_d(V, E)$
  \bit
  \w $V = \{v_i: i = 1, 2, \cdots, n\}$: one-one correspondnce with the set of
  tasks (operations)
  \w an operations takes one more more operands and yield one or more results
  \w $E = \{(v_i, v_j): i, j = 1, 2, \cdots, n\}$: represents \bb{data
    transfer} from one task to another (\bb{data dependency})
  \w additionally, \bb{serialization dependency} can be added ``special'' edges
  \eit
\w \bb{sequencing graph}: a hierarchical CDFG (control/data flow graph), where
control-flow primitives such as branching and iteration are modeled through
the hierarchy, whereas data-flow and serialization dependency are modeled by
graphs; formally, $G_s(V, E)$
   \bit
   \w $v \in V$: \bb{extended data-flow graph}
   \eit
\w \bb{resources}:
   \bit
   \w \bb{functional resources}: implements logic/arithmetic functions
         (either {\em primitive\/} or {\em application-specific\/} resources)
   \w \bb{memory resources}: registers, read-only memory array, read-write
   memory array
   \w \bb{interface resources}: bus, I/O pads, interfacing circuits
   \eit
\w 
\eit
\subsection{Architectural synthesis and optimization problem}
\bit
\w \bb{Input to the problem}: 
   \bit
   \w a sequencing graph
   \w a set of functional resources (/w area, delay information)
   \w a set of constraints
   \eit
\w \bb{2 steps of architectural synthesis and optimization}:
   \bit
   \w [(1)] placing operations in time and in space
   \w [(2)] determining the detailed interconnections of data path and the
   logic-level specification of the control unit
   \eit
\w \bb{temporal domain: scheduling}
  \bit
  \w Let $D = \{d_i\}$ be the set of delays associated with the operations.
  \w We want to determine $T = \{t_i\}$, which are \bb{start times\/} of
  operations.
  \w A \bb{schedule} of a sequencing graph is a function $\varphi: V
  \rightarrow Z^+$, where $\varphi(v_i) = t_i$ such that
  $v_i \ge v_j + d_i$, for all $(v_j, v_i) \in E$.
  \w The \bb{latency} of a schedule is $t_n - t_0$, where $v_0, v_n$ are
   the source and the sink of the graph.
  \eit
\w \bb{spatial domain: resource binding}
  \bit
  \w A \bb{resource binding} is a mapping $\beta: V \rightarrow R \times Z^+$,
  where $\beta(v_i) = (t, r)$ denotes that the operation corresponding to $v_i
  \in V$, with type $T(v_i) = t$, is implemented by the $r$th instance of
  resource type $t \in R$ for each $i = 1, \cdots, n$. 
  \w A \bb{binding constraint} are upper bounds on the resource usage of each
  type, denoted by $\{a_k; k =1, \cdots, r\}$, where $r$ is the number of
  resources. 
  A resource binding represent the allocation of instances for each resource
  type. A resource binding satisfied resource bounds $\{a_k\}$ when
  $\beta(v_i) = (t, r)$ with $r \le a_t$ for each operation $v_i$. 
  \eit
\eit

\end{document}
