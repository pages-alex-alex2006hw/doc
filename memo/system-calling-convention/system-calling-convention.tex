\documentclass{memo}
\usepackage{mathptm,mydef,myenv}
%\usepackage{MinionPro}
\begin{document}
\small
\noindent{\large\bf{}x86 Calling Conventions on 32bit Linux}

\paragraph{Overview} 
Calling conventions describe the interface of the called code:
\bit
\w the order in which atomic (scalar) parameters, or individual parts of a
complex parameter, are allocated
\w how parameters are passed (pushed on the stack, placed in registers, or mix
of both)
\w which registers may be used by the callee without first being saved
(i.e. pushed)
\w how the task of setting up for and restoring the stack after a function
call is divided between the caller and the callee
\eit

\paragraph{Register usage}  The following is a summary of x86 registers.

\vspace*{0.3cm}

\centerline{\begin{tabular}{r|l|l} \hline
   type & name & usage \\ \hline
general & {\tt\%eax} & return value \\
 & {\tt\%edx} & dividened register \\
 & {\tt\%ecx} & counter register value \\
 & {\tt\%ebx}, {\tt\%esi}, {\tt\%edi} & local register variable \\
 & {\tt\%esp} & stack pointer \\
 & {\tt\%ebp} & frame pointer (optional) \\ \hline
FP & {\tt\%sp(0)} & FP stack top, return value \\ \hline
  & {\tt\%sp(0)} & FP next to stack top \\ 
  & $\cdots$ & \\  
  & {\tt\%sp(7)} & FP stack bottom \\  \hline
  \end{tabular}}

\vspace*{0.3cm}

\noindent{}The following are 32-bit Linux register usage. Note
that this is different from that of 32-bit Windows.

\vspace*{0.3cm}

\centerline{\begin{tabular}{r|p{4.0cm}}\hline
\bb{scratch registers} & {\tt\%eax}, {\tt{}\%ecx}, {\tt\%edx}, {\tt
  \%st(0)}--{\tt{}\%st(7)}, {\tt \%xmm(0)}--{\tt{}\%xmm(7)}, {\tt{}\%ymm(0)}--{\tt{}\%ymm(7)}\\ \hline 
\bb{callee-save registers} & {\tt\%ebx}, {\tt\%esi}, {\tt\%edi}, {\tt\%ebp}\\ \hline
\bb{argument registers} & none \\ \hline
\bb{registers for return} & {\tt\%eax}, {\tt\%st(0)}, {\tt\%xmm(0)}, {\tt\%ymm(0)}
\\ \hline 
\end{tabular}}


\paragraph{cdecl calling convention}
The \bb{cdecl} calling convention is used by many C systems (incl. GCC) for
the x86 architecture. 
\bit
\w Function \bb{arguments} are passed on the stack in a right-to-left order.
\w Function \bb{return values} are returned in the \verb+%eax+ register
(except for floating point values, which are returned in the x87 register
\verb+%st(0)+) 
\w The registers, \verb+%eax+, \verb+%ecx+, and \verb+%edx+ do not need to be
preserved, while others do.
\w \bb{caller} is responsible for \bb{stack cleanup}
\eit

\begin{verbatim}
  int foo(int, int, int);
  int a, b, c, x;
  x = foo(a, b, c);
\end{verbatim}

\begin{verbatim}
  push c;      ; arg3
  push b;      ; arg2
  push a;      ; arg1
  call foo;
  add esp, 12  ; pop funargs (a, b, c) from stack
  mov x, eax   ; fetch return value
\end{verbatim}

\centerline{\begin{tabular}{r|l|l} \hline
position & contents & frame \\ \hline
{\tt{}4n+8(\%ebp)} & argument $n$  & \\ 
& $\cdots$ &  previous \\ 
{\tt{}8(\%ebp)} & argument $0$ &  \\ \hline
{\tt{}4(\%ebp)} & return address &   \\ 
{\tt{}0(\%ebp)} & previous {\tt{}\%ebp} (optional) $0$ & current \\ \cline{1-2}
{\tt{}-4(\%ebp)} & unspecified &  \\ 
 & $\cdots$ &  \\ 
{\tt{}0(\%esp)} &  &  \\  \hline
\end{tabular}}

\vspace*{0.3cm}

In case, to force cdecl calling convention, we can add \verb+_cdecl+ modifier:
\begin{verbatim}
  void _cdecl foo(int, int);
\end{verbatim}

\paragraph{Functions returning scalars or no value}
A function that returns an integral or pointer value places its result in
register \verb+%eax+. A floating-point return value appears on the top of the
x87 register stack. {\em The caller is responsible for removing the value from
  the x87 stack, even if it does not use it\/}. 

\bit
\w \bb{function prologue}:
\begin{verbatim}
prologue:
  pushl %ebp       ; save frame pointer
  movl  %esp, %ebp ; set new frame pointer
  subl  $80, %esp  ; allocate stack space
  pushl %edi       ; save local register
  pushl %esi       ; save local register
  pushl %ebx       ; save local register
\end{verbatim}

\w \bb{call} instruction: {\em pushes the address of the next instruction
  ({\bfseries the return address}) onto the stack\/}. 

\w  \bb{function epilogue}: restores the state for the caller
\begin{verbatim}
  movl  %edi, %eax ; set up return value
epilogue:
  popl  %ebx       ; restore local register
  popl  %esi       ; restore local register
  popl  %edi       ; restore local register
  leave            ; restore frame pointer
  ret              ; pop return address
\end{verbatim}
\w \bb{ret} instruction: {\em pops the address off the stack and effectively
    continues execution at the next instruction after the {\bfseries call\/}
    instruction} 


\eit

\end{document}

% LocalWords:  Pthreads interprocess multithreaded sigaltstack errno realtime
% LocalWords:  datatypes pthreads pthread mutex pthrad mutexattr cond condattr
% LocalWords:  attr retval struct thr joinable mutexes reentrancy strtok callee
% LocalWords:  reentrant eax ecx edx xmm ymm ebx esi edi ebp cdecl GCC
