\documentclass{myproc}
\usepackage{mydef,myenv,amssymb}
\usepackage{MinionPro,MnSymbol}
%\usepackage{mathptm}
\usepackage[all]{xy}
\def\EE{\mbox{\eufm{}E}}

\begin{document}
\small
\pagestyle{empty}

\noindent{\large\bf Notes on Mathematical Logic}

\section{Sentential Logic}
In order to {\em{}describe a formal language\/} we have to give
  three pieces of information:
  \ben
  \w [(a)] We should specify the set of symbols (the {\em{}alphabet\/}).
%  For example,
%  $(, ), \rightarrow, \neg, A_1, A_2, \cdots$.
  \w [(b)] We should specify the rules for forming the 
     {\em{}grammatically correct\/} finite sequences of symbols, 
     called {\bf{}well-formed formulas}
     or {\bf{}wffs}.
     \w [(c)] We may need to indicate the allowable translations between
     English and the formal language. {\em{}This information is
       dispensable for ``formal or symbolic logic''\/}\footnote{Translations 
       between the formal language and the
     mathematical
     structures are studied in {\em{}Model Theory\/}.}.
   \een

\subsection{The Language of Sentential Logic}
\bit
\w Alphabet of sentential logic
	\bit
	\w {\bf{}Logical symbols}
		\bit
		\w {\em{}Sentential connectives}: 
			$\neg, \vee, \wedge, \rightarrow, \leftrightarrow$
		\w {\em{}Parentheses}:
			$(, )$
		\eit
	\w {\bf{}Sentence symbols} or non-logical symbols: $A_1, A_2, \cdots$
		\bit
		\w Only countably many sentence symbols exist.
		\eit
	\eit
\w We assume that no symbol is a finite sequence of other symbols.
	This assumption aims to assure that finite sequences of symbols be
	{\em{}uniquely decomposable\/}. That is, if  
		\[ \arc{a_1, a_2, \cdots, a_m} = \arc{b_1, b_2, \cdots, b_n}\]
	and each $a_i$ and $b_i$ is a symbol, then
	$m = n$ and $a_i = b_i$.
\w An {\bf{}expression} is a finite sequence of symbols.
\w A well-formed formula for sentential logic is defined as follows:
	\ben
	\w [(a)] Every sentence symbol is a wff.
	\w [(b)] If $\alpha$ and $\beta$ are wffs, then so are
		$(\neg\alpha)$, $(\alpha\vee\beta)$, $(\alpha\wedge\beta)$,
		$(\alpha\rightarrow\beta)$, $(\alpha\leftrightarrow\beta)$.
	\w [(c)] No expression is a wff unless it is compelled to 
		be one by (a) and (b).
	\een
\w {There are two equivalent ways to make
	the property (c) precise.\/}
	\ben
	\w {\bf{}Bottom-up}: using the notion of {\em{}inductive sets\/}
		\bit
		\w An expression is a wff iff it is a member
			of inductive set.
		\eit
	\w {\bf{}Top-down}: using the operational 
			notion of {\em{}formula-building operations\/}
		\bit
		\w An expression is a wff iff it can be built up from the sentence
			symbols by applying some finite number of times the 
			formula-binding operations (on expressions) defined by
			the equations
				\begin{eqnarray*}
				\EE_\neg(\alpha) & = & (\neg\alpha),\\				
				\EE_\vee(\alpha, \beta) & = & (\alpha\vee\beta),\\				
				\EE_\wedge(\alpha, \beta) & = & (\alpha\wedge\beta),\\				
				\EE_\rightarrow(\alpha, \beta) & = & 
						(\alpha\rightarrow\beta),\\				
				\EE_\leftrightarrow(\alpha, \beta) & = & 
						(\alpha\leftrightarrow\beta)
				\end{eqnarray*}
		\eit
	\een
\eit
\subsection{Induction}
\bit
\w Let's consider an initial set $B \subseteq U$ and a class ${\cal{}F}$ of
	functions containing just two members $f$ and $g$, where
		\[ f: U \times U \rightarrow U, \quad g: U \rightarrow U.\]
\w A set $S$ is {\bf{}inductive} iff $B \subseteq S$ and $S$ is 
	{\em{}closed\/}
	under $f$ and $g$.
	\bit
	\w A set $S \subseteq U$ is {\bf{}closed} under $f$ and $g$
		iff 
          \[x, y \in S \ \Rightarrow\ f(x, y),\ g(x) \in S.\]
	\eit
\w Let $C^*$ be the intersection of all the inductive subsets of $U$;
	thus $x \in C^*$ iff $x$ belongs to every inductive subset of $U$.
	\bit
	\w $C^*$ is itself inductive.
	\w $C^*$ is the smallest set that is included in all the other
		inductive sets.
	\w $C^*$ is called the {\bf{}inductive closure} w.r.t. $B, f, g$.
	\eit
\w Let $C_*$ be the set of things which can be obtained from $B$
	by applying $f$ and $g$ a finite number of times.
	\bit
	\w A {\bf{}construction sequence} is a finite
		sequence $\arc{x_0, \cdots, x_n}$ of elements of $U$ s.t.
			for each $i \le n$ we have at least one of
				\begin{eqnarray*}
				x_i & \in & B,\\
				x_i & = & f(x_j, x_k) \quad \mbox{\ for\ } j < i \mbox{\ and\ } k < i,\\
				x_i & = & g(x_j) \quad \mbox{\ for\ } j < i.\\
				\end{eqnarray*}
	\w $C_*$ is defined as $\{x: $ some construction
sequence that with $x\}$.
		\bit
		\w If $C_i$ is the set of all points $x$ s.t. some 
			construction sequence of length $i$ ends with $x$, then 
			\[ C_* = \bigcup_{i} C_i. \]
		\eit
	\eit
\w $C^* = C_*$; this means the above two definitions are equivalent.
\w Since $C^* = C_*$, we call the set simply $C$ and refer to it as the
	{\bf{}set generated from $B$ by the functions in $\cal{}F$}.
\w ({\bf{}Induction Principle}) 
	Assume that $C$ is the set generated from $B$ 
	by applying the functions
	in $\cal{}F$. If $S$ satisfies $B \subseteq S \subseteq C$
	and $S$ is closed under the functions of $\cal{}F$, then
	$S = C$.
\eit

\subsection{Recursion}
\bit
\w A set $C$ is {\bf{}freely generated from $B$ by $f$ and $g$} iff
	in addition to the requirements for being generated we have
	\bit
	\w [(a)] $f \upharpoonright C$ and 
		$g \upharpoonright C$ are one-to-one (injective), and
	\w [(b)] The range of $f \upharpoonright C$, 
		the range of $g \upharpoonright C$, , and the set $B$ are
		pairwise disjoint.
	\eit
\w ({\bf{}Recursion Theorem})
	Assume that a subset $C$ of $U$ is freely generated from $B$
	by $f$ and $g$, where
		\[ f: U\times{U} \rightarrow U, \quad
			g: U \rightarrow U.\]
	Further assume that $V$ is a set and $F,\ G$, and $h$ functions
		such that
		\begin{eqnarray*}
		h &: & B \rightarrow V, \\
		F &: & V \times V \rightarrow V, \\
		G &: & V \rightarrow V.
		\end{eqnarray*}
	Then there exists a unique function
		\[ \bar{h}: C \rightarrow V \]
	such that
		\ben
		\w [(a)] For all $x \in B$, $\bar{h}(x) = h(x)$.
		\w [(b)] For all $x, y \in B$, 
			\begin{eqnarray*}
			\bar{h}(f(x, y)) & = & F(\bar{h}(x), \bar{h}(y)), \\
			\bar{h}(g(x)) & = & G(\bar{h}(x)).
			\end{eqnarray*}
		\een
	\bit
	\w Pseudo-commutative diagram \& commutative diagram:
			\[ \xymatrix{
				x \in B \ar[d]_{g: U \rightarrow U} 
				\ar[rr]^{\overline{h}: U \rightarrow V} 
					& &  \overline{h}(x) \in V
					\ar[d]^{G: V \rightarrow V} \\
				g(x) \in C \ar[rr]^{\overline{h}} & &
				\overline{h}(g(x)) = G(\overline{h}(x)) 
			}\]
			\[ \xymatrix{
				B \ar[d]_{g} 
				\ar[rr]^{\overline{h} \equiv h} 
					& &  V
					\ar[d]^{G} \\
				C \ar[rr]^{\overline{h}} & & V
			}\]
	\w Algebraically, {\em{}recursion 
		theorem means that any map $h$ of
		$B$ into $V$ can be extended to a homomorphism $\bar{h}$
		from $C$ (with operations $f$ and $g$) into $V$
		(with operations $F$ and $G$)\/}.
	\eit
\w The wffs are freely generated from the sentence symbols
	by the five formula-building operations, 
	$\EE_\neg$, $\EE_\vee$, $\EE_\wedge$, $\EE_\rightarrow$, $\EE_\leftrightarrow$.
\eit

\subsection{Truth Assignments}
\bit
\w A {\bf{}truth assignment} $\nu$ for a set $\cal{}S$ of sentence
	symbols is a function
		\[ \nu: {\cal{}S} \rightarrow \{\mbox{T}, \mbox{F}\}. \]
	\bit
	\w A truth assignment is an {\bf{}interpretation}.
	\eit
\w Let $\overline{\cal{}S}$ be the set of wffs generated from the five
	formula building operations.
	We want an extension $\bar{\nu}$ of $\nu$,
		\[ \bar{\nu}: \overline{\cal{}S} \rightarrow \{\mbox{T, F}\},\]
	which assigns the correct truth value to each wff in $\overline{\cal{}S}$.
	It should meet the following conditions:
	\ben
	\w [(a)] For any $A \in {\cal{}S}$, $\bar{\nu}(A) = \nu(A)$.
	\w [(b)] For any $\alpha \in \overline{\cal{}S}$,
		\ben
		\w [(1)] $\bar{\nu}((\neg\alpha)) =
		\left\{\begin{array}{ll}
			\mbox{T} & \mbox{\ if $\bar{\nu}(\alpha)$ = F},\\
			\mbox{F} & \mbox{\ otherwise}.
			\end{array}\right.$
		\w [(2)] $\bar{\nu}((\alpha\vee\beta)) =$\\
		$\left\{\begin{array}{ll}
			\mbox{T} & \mbox{\ if $\bar{\nu}(\alpha)$ = T or 
						$\bar{\nu}(\alpha)$ = T},\\
			\mbox{F} & \mbox{\ otherwise}.
			\end{array}\right.$
		\w [(3)] $\bar{\nu}((\alpha\wedge\beta)) =$\\
		$\left\{\begin{array}{ll}
			\mbox{T} & \mbox{\ if $\bar{\nu}(\alpha)$ = T and
			  $\bar{\nu}(\alpha)$ = T},\\
			\mbox{F} & \mbox{\ otherwise}.
			\end{array}\right.$
		\w [(4)] $\bar{\nu}((\alpha\rightarrow\beta)) =$\\
		$\left\{\begin{array}{ll}
			\mbox{F} & \mbox{\ if $\bar{\nu}(\alpha)$ = T and
							$\bar{\nu}(\alpha)$ = F},\\
			\mbox{T} & \mbox{\ otherwise}.
			\end{array}\right.$
		\w [(5)] $\bar{\nu}((\alpha\leftrightarrow\beta)) = \left\{\begin{array}{ll}
			\mbox{T} & \mbox{\ if $\bar{\nu}(\alpha) = \bar{\nu}(\alpha)$},\\
			\mbox{F} & \mbox{\ otherwise}.
			\end{array}\right.$
		\een
	\een
\w Situation explained:
			\[ \xymatrix{
				{\cal{}S} \ar[d]_{\{\EE_\cdot\}} 
				\ar[rr]^{\overline{\nu} \equiv \nu}  
					& &  \{T, F\}
					\ar[d]^{F,G} \\
				\overline{\cal{}S} 
				\ar[rr]^{\overline{\nu}} & & \{T, F\}
			}\]
	\bit
	\w In the diagram, the meanings of 
		$F$ and $G$ are indicated in (b.1)--(b.5).
	\eit
\w For any truth assignment $\nu$ for a set $\cal{}S$ there is a unique
	function $\bar{\nu}: \overline{\cal{}S} \rightarrow \{\mbox{T, F}\}$
	meeting the above conditions (a) and (b.1)--(b.5) by the
	{\em{}Recursion Theorem\/}.
\w A truth assignment $\nu$ {\bf{}satisfies} a sentence 
	$\varphi$ iff $\bar\nu(\varphi) = \mbox{T}$.
	\bit
	\w $\bar{\nu}$ is said to be a {\bf{}model} of $\varphi$.
	\eit
\w A set of wffs $\Sigma$ {\bf{}tautologically implies} $\tau$, written as
	$\Sigma \models \tau$, iff every truth assignment
	for the sentence symbols in $\Sigma$ and $\tau$ which
	satisfies every member of $\Sigma$ also satisfies $\tau$.
	\bit
	\w $\Sigma$ can be thought of as hypotheses and $\tau$
		can be thought of as a {\em{}possible\/} conclusion.
	\w $\emptyset \models \tau$ iff every truth assignment 
		satisfies $\tau$; in this case we call $\tau$
		a {\bf{}tautology} and write this as $\models \tau$.
	\eit
\w ({\bf{}Compactness Theorem})
	Let $\Sigma$ be an infinite set of wffs such that for any 
	finite subset $\Sigma_0$ of $\Sigma$, there is a truth
	assignment which satisfies every member of $\Sigma_0$.
	Then there is a truth assignment which satisfies every
	member of $\Sigma$.
\w $\Sigma; \alpha \models \beta$ iff $\Sigma \models (\alpha
	\rightarrow \beta)$.
\eit

\subsection{Unique Readability}
\bit
\w Every wff has the same number of left as right parentheses.
\w Any proper initial segment of a wff contains an excess of
	left parentheses.
\w ({\bf{}Unique Readability Theorem})
	The five formula-building operations, when restricted
	to the set of wffs, 
	\ben
	\w [(a)] have ranges which are disjoint from each other and
		from the set of sentence symbols, and
	\w [(b)] are one-to-one.
	\een
\w {\em{}Unique readability theorem ensures that the set of
	wffs are {\em\bfseries{}freely\/} generated from 
	the sentence symbols by the formula-building operations\/}.
\eit

\subsection{Sentential Connectives}
\bit
\w A {\bf{}$k$-place Boolean function} is a function from
	$\{\mbox{T, F}\}^k$ to $\{\mbox{T, F}\}$.
\w Suppose that $\alpha$ is a wff whose sentence symbols are at most
	$A_1, \cdots, A_n$. We define an $n$-place Boolean function
	$B^n_\alpha$ (or just $B_\alpha$), the {\bf{}Boolean function realized
	by $\alpha$}, by 
	\[ B_\alpha(X_1, \cdots, X_n) = \bar\nu(\alpha) 
		\mbox{\ when $\nu(A_i) = X_i$}. \]
\w Let F $\le$ T. And let 
	$\alpha$ and $\beta$ be wffs whose sentence symbols are among
	$A_1, \cdots, A_n$. Then
	\ben
	\w [(a)] $\alpha \models \beta$ iff all $\vec{X} \in 
			\{\mbox{T, F}\}^n$,
			$B_\alpha(\vec{X}) \le B_\beta(\vec{X})$.
	\w [(b)] $\alpha \models =\!\!\!|\ \beta$ iff $B_\alpha = B_\beta$.
	\w [(c)] $\models \alpha$ iff ran$(B_\alpha) = \{\mbox{T}\}$.
	\een
\w Let $G$ be an $n$-place Boolean function, $n \ge 1$. We can
	find a wff $\alpha$ such that $G = B^n_\alpha$, i.e.,
	such that $\alpha$ realizes the function $G$.
\w For any wff $\varphi$, we can find a tautologically equivalent
	wff $\alpha$ in disjunctive normal form.
\w If every function $G: \{\mbox{T, F}\}^n \rightarrow \{\mbox{T, F}\}$
	can be realized by a wff using only the connective symbols 
	$\{s_i\}$, we say that the set $\{s_i\}$ is {\bf{}complete}.
	\bit
	\w Both $\{\neg, \vee\}$ and $\{\neg, \wedge\}$ are complete.
	\w $\{\wedge, \rightarrow\}$ is not complete.
	\eit
\eit

\subsection{Compactness and Effectiveness}
\bit
\w A set $\Sigma$ of wffs is {\bf{}satisfiable} iff there is a 
	truth assignment which satisfies every member of $\Sigma$.
\w ({\bf{}Compactness Theorem}) A set of wffs is satisfiable 
	iff every finite subset is satisfiable.
\w If $\Sigma \models \tau$, then there is a finite $\Sigma_0$
	such that $\Sigma_0 \models \tau$.
\w A set $\Sigma$ of expressions is {\bf{}decidable} iff there
	exists an effective procedure which, given an expression
	$\alpha$, will decide whether or `not' $\alpha \in \Sigma$.
	\bit
	\w Some infinite sets are undecidable since there are 
		$2^{\aleph_0}$, i.e. uncountably infinite, 
		sets of expressions but only countably
		many effective procedures.
	\eit
\w There is an effective procedure which, given a finite
	set $\Sigma; \tau$ of wffs, will decide whether 
	$\Sigma \models \tau$.
	\bit
	\w The truth-table method enables us the effective decision.
	\w Note that this theorem is for ``sentential logic.''
	\eit
\w For a finite set $\Sigma$, the set of tautological consequences
	of $\Sigma$ is decidable. In particular, the set of 
	tautologies is decidable.
\w A set $A$ of expressions is {\bf{}effectively enumerable} iff
	there is an effective procedure which lists, in some order,
	the members of $A$.
	\bit
	\w When $A$ is infinite, the procedure may not halt but {\em{}for
		any specific member of $A$, it should eventually
		$($in a finite length of time$)$
		appear on the list.}
	\eit
\w A set $A$ of expressions is effectively enumerable iff there
	is an effective procedure which, given any expression $\epsilon$,
	produces the answer ``yes'' iff $\epsilon \in A$.
	\bit
	\w Note that this procedure may not produce an answer when 
		$\epsilon \not\in A$.
	\eit
\w {\em{}A set of expressions is decidable iff both it and 
	its complement
	(w.r.t the set of all expressions)
	are effectively enumerable\/}.
\w If $\Sigma$ is a decidable set of wffs, then the set of tautological
	consequences of $\Sigma$ is effectively enumerable.
\eit


\section{First-Order Logic}
\subsection{First-Order Languages}
\bit
\w Alphabet of first-order logic
	\bit
	\w {\bf{}Logical symbols}
		\bit
		\w {\em{}Parentheses}: $(, ), [, ]$
		\w {\em{}Sentential connective symbols}: $\rightarrow, \neg$
		\w {\em{}Variables} (one for each positive integer $n$):
			$v_1, v_2, \cdots$
		\w {\em{}Equality symbol} (optional): $\approx$
		\eit
	\w {\bf{}Parameters}
		\bit
		\w {\em{}Quantifier symbol}: $\forall$
		\w {\em{}Predicate symbols}: for each $n$, some set of symbols,
				called {\bf{}$n$-place predicate symbols}
		\w {\em{}Function symbols}: for each $n$, some set of symbols,
				called {\bf{}$n$-place function symbols}
		\w {\em{}Constant symbols}: some set of symbols; this can be 
			treated as $0$-place function symbols
		\eit
	\eit
\w Example: Language of elementary number theory
	\bit
	\w Equality: $=$
	\w Predicate parameter: two-place predicate symbol $<$
	\w Constant symbol: symbol $0$
	\w One-place function symbol: $S$ for successor
	\w Two-place function symbol: $+, \cdot$ and $E$ for exponentiation
	\eit
\w A {\bf{}term} is an expression that can be built up from the
	{\em{}constant symbols\/} and the {\em{}variables\/} by prefixing the
	{\em{}function symbols\/}. 
		\bit
		\w Formally, if {\eufm{}F}$_f$ is a 
		$n$-place term-building operation for function symbol $f$ such that
		\[ \mbox{\eufm{}F}_f(x_1, \cdots, x_n) = f(x_1, \cdots, x_n) \]
		then the set of terms is the set of expressions generated from the
		constant symbols and variable by the {\eufm{}F}$_f$ operations.
		\eit
\w An {\bf{}atomic formula} is an expression of the form
	\[ P(t_1, \cdots, t_n) \]
	where $P$ is an {\em{}$n$-place predicate symbol\/} and $t_1, \cdots,
	t_n$ are {\em{}terms\/}.
\w The set of {\bf{}well-formed formulas} is the set of expressions
	generated from the {\em{}atomic formulas\/} by the operations
	\EE$_\neg$, \EE$_\rightarrow$, and {\eufm{}A}$_i$ 
	($i = 1, 2, \cdots$), 
	where
		\begin{eqnarray*}
		\EE_\neg (\gamma) & = & (\neg\gamma) \\
		\EE_\rightarrow (\gamma, \delta) & = & (\gamma \rightarrow \delta) \\
		\mbox{\eufm{}A}_i (\gamma) & = & (\forall{v_i})[\gamma]
		\end{eqnarray*}
\w We define, for each wff $\alpha$, what it means a variable $x$ 
	{\bf{}occur free} in $\alpha$ recursively:
	\ben
	\w [(a)] For atomic $\alpha$, $x$ occurs free in $\alpha$ iff
		$x$ is a symbol of $\alpha$.
	\w [(b)] $x$ occurs free in $\neg\alpha$ iff $x$ occurs free in $\alpha$.
	\w [(c)] $x$ occurs free in $\alpha\rightarrow\beta$ 
		iff $x$ occurs free in $\alpha$ or in $\beta$.
	\w [(d)] $x$ occurs free in $(\forall{v_i})[\alpha]$ iff
		$x$ occurs free in $\alpha$ and $v_i \ne x$.
	\een
\w If no variable occurs free in the wff $\alpha$, then $\alpha$ is 
	a {\bf{}sentence}.
\eit

\subsection{Truths and Models}
\bit
\w In sentential logic, {\bf{}truth assignments} enables us to
	answer the true/false questions and, in first-order logic,
	{\bf{}structures} enables us to do.
\w A {\bf{}structure} $U$ for our first-order logic is a {\em{}function\/}
	whose domain is the set of parameters and such that
	\ben
	\w [(a)] $U$ assigns to the quantifier symbol $\forall$ a 
		nonempty set $|U|$, called the {\bf{}universe} of $U$,
	\w [(b)] $U$ assigns to each $n$-place predicate symbol $P$
		an $n$-ary relation $P^U \subseteq |U|^n$, i.e., 
		$P^U$ is a set of $n$-tuples of members of the universe,
	\w [(c)] $U$ assigns to each constant symbol $c$ a member
		$c^U$ of the universe $|U|$, and
	\w [(d)] $U$ assigns to each $n$-place function symbol $f$ an
		$n$-ary (total) operation $f^U$ on $|U|$, i.e.,
			$f^U: |U|^n \rightarrow |U|$.
	\een
\w Setup
	\bit
	\w Let $\varphi$ be a wff of our language.
	\w Let $U$ a structure for the language.
	\w Let $s: V \rightarrow |U|$ be a {\bf{}valuation} function
		from the set $V$ of all variables
		into the universe $|U|$ of $U$.
	\eit
\w Definition for what is meant by ``{\bf{}$U$ satisfies $\varphi$ with $s$}'',
	written as $\models_U \varphi[s]$
	\ben
	\w {\bf{}Terms} 
		\bit
		\w Extension of $s$, 
			\[ \bar{s}: T \rightarrow |U|, \]
			a function from the set $T$ 
			of all terms into the universe of $U$ is defined as
			\[ \bar{s}(x) = \left\{\begin{array}{ll}
				s(x) & \mbox{$x$ variable},\\
				c^U & \mbox{$x = $ constant $c$},\\
				f^U(\bar{s}(t_1), \cdots, \bar{s}(t_n)) &
					\mbox{$x = f(t_1, \cdots, t_n)$.}
                                            %for $f, t_i$}.
			\end{array}\right. \]
		\w Commutative diagram:
			\[ \xymatrix{
				T \ar[d]_{\mbox{\eufm{}F}_f} 
					\ar[r]^{\bar{s}} & |U| \ar[d]^{f^U} \\
				T \ar[r]^{\bar{s}} & |U|
			}\]
		\eit
	\w {\bf{}Atomic formulas}
		\ben
		\w [2.1] $\models_U (t_1\approx{t_2})[s]$ 
			iff $\bar{s}(t_1) = \bar{s}(t_2)$.
		\w [2.2] For an $n$-place predicate parameter $P$,
			\[ \models_U P(t_1, \cdots, t_n)[s] \mbox{\ iff\ }
				\arc{\bar{s}(t_1), \cdots, \bar{s}(t_n)} \in P^U.\]
		\een
	\w {\bf{}Other wffs}
		\ben
		\w [3.1] $\models_U \neg\varphi [s]$ iff $\not\models_U \varphi[s]$.
		\w [3.2] $\models_U \varphi \rightarrow \psi [s]$
			iff either $\not\models_U \varphi[s]$ or $\models_U \psi[s]$
			or both.
		\w [3.3] $\models_U (\forall{x})[\varphi][s]$ iff for every
			$d \in |U|$, we have $\models_U \varphi [s[d/x]]$
		\een
	\een
\w Let $\Gamma$ be a set of wffs, $\varphi$ a wff. Then
	{\bf{}$\Gamma$ logically implies $\varphi$}, $\Gamma \models \varphi$,
	iff for every structure $U$ for the language and every function
	$s: V \rightarrow |U|$ such that $U$ satisfies every member of 
	$\Gamma$ with $s$, $U$ also satisfies $\varphi$ with $s$.
	\bit
	\w We write ``$\gamma \models \varphi$'' instead of 
		``$\{\gamma\} \models \varphi$''.
	\w $\varphi$ and $\psi$ are {\bf{}logically equivalent}, 
		$\varphi \models =\!\!\!|\ \psi$, iff
		$\varphi \models \psi$ and $\psi \models \varphi$.
	\eit
\w A wff $\varphi$ is {\bf{}valid} iff $\emptyset \models \varphi$
	(written just ``$\models \varphi$'').
	\bit
	\w Thus $\varphi$ is valid iff for every $U$ and 
		every $s: V \rightarrow
		|U|$, $U$ satisfies $\varphi$ with $s$.
	\w {\em{}First-order analog of the tautologies are the valid 
		formulas\/}.
	\eit
\w Assume that $s_1, s_2$ are functions from $V$ into $|U|$ which
	agree at all variables (if any) which occur free in the wff $\varphi$.
	Then
		\[ \models_U \varphi [s_1] \mbox{\ iff\ } \models_U \varphi [s_2]. \]
\w For a {\em{}sentence\/}\footnote{Note that a sentence is
	a wff with no free variables.} $\sigma$, either
	\ben
	\w [(a)] $U$ satisfies $\sigma$ with {\em{}every\/}
		valuation $s$ from $V$ into $|U|$ (in this case
			we say that $\sigma$ is {\bf{}true} in $U$, $\models_U \sigma$,
			or that $U$ is a {\bf{}model} of $\sigma$), or
	\w [(b)] $U$ does not satisfy $\sigma$ with any such function.
	\een
\w For a set $\Sigma; \sigma$ of sentences, $\Sigma \models \tau$ iff
	every model of $\Sigma$ is a model of $\sigma$, i.e., at a ``meta''-level
		\[ (\forall{U})(\forall{s: V \rightarrow |U|})
		[\models_U \Sigma[s] 
		\Rightarrow \models_U \sigma[s]]. \]
\w {\em{}In contrast to the sentential logic, the set of valid formulas
	is {\bfseries{}undecidable}. But the notion of validity 
	turns out to be equivalent to the notion of {\bfseries{}deducibility\/}
	and using this equivalence we will be able to show that
	the set of valid wffs is {\bfseries{}effectively enumerable}.}
\w {\bf{}Definability of a class of structures}
	\bit
	\w {\bf{}Question}: {\em{}given a mathematical object, can we define it
		in first-order logic?}
	\w For a set $\Sigma$ of sentences, let Mod $\Sigma$ be the class 
		of all models
		of $\Sigma$, i.e., the class of all structures for the 
		language in which
		every member of $\Sigma$ is true.
	\w A class $\cal{}K$ of structures for our language is an 
		{\bf{}elementary
		class} (EC) iff ${\cal{}K} = \mbox{\ Mod\ } \tau$ for some 
		sentence $\tau$ (``elementary'' is synonymous with 
		``first-order'').
	\w $\cal{}K$ is an {\bf{}elementary class in the wider sense}
		(EC$_\Delta$) iff ${\cal{}K} = \mbox{\ Mod\ } \Sigma$
		for some set $\Sigma$ of sentences.
	\w Example: 
		\bit
		\w Let's consider a language with equality and the parameters
			$\forall$ and $P$, where $P$ is a 2-place predicate symbol.
			A structure $(A, R)$ for the language consists of a nonempty
			set $A$ and a binary relation $R$ on $A$.
		\w $(A, R)$ is an {\em{}ordered set\/} iff $R$ is transitive
			and satisfies the trichotomy condition.
		\w Since transitivity and trichotomy conditions can be 
			translated into a sentence of the formal language, 
			the {\em{}class of nonempty ordered sets is an 
			elementary class\/}.
		\w Essentially the class of nonempty ordered sets is 
			Mod $\tau$, where $\tau$ is the {\em{}conjunction\/} of the
			three sentences
			\begin{eqnarray*}
			& & (\forall{x,y,z})[xPy \rightarrow yPz \rightarrow xPz]\\
			& & (\forall{x, y})[xPy \vee x \approx y \vee yPx] \\
			& & (\forall{x, y})[xPy \rightarrow \neg{y}Px]
			\end{eqnarray*}
		\eit
	\eit
\w {\bf{}Definability within a structure}
	\bit
	\w {\bf{}Question}: 
		{\em{}Given a mathematical object and a relation on
		(or other mathematical object based on) 
		that object, can we define the relation (it) 
		in first-order logic?}
	\w Given a structure $U$ and a formula $\varphi$ such that 
		all variables
		occurring {\em{}free\/} in $\varphi$ are included among
		$v_1, \cdots, v_k$. Then for elements $a_1, \cdots, a_k$ 
		of $|U|$,
			\[ \models_U \varphi \denote{a_1, \cdots, a_k} \]
		means that $U$ satisfies $\varphi$ with some function
		$s: V \rightarrow |U|$ for which $s(v_i) = a_i$ for $1 \leq 
		i \le k$.
	\w For each such $\varphi$ and $U$, the $k$-ary relation
		\[ \{\arc{a_1, \cdots, a_k}:\ \models_U \varphi 
			\denote{a_1, \cdots, a_k}\}\]
		is said to be {\bf{}defined by $\varphi$ in $U$}.
	\w A $k$-ary relation on $|U|$ is {\bf{}definable in $U$} iff there
		a formula which defines it in $U$.
	\w Example:
		\bit
		\w Given a structure ${\cal{}N} = (\N, 0, S, +, \cdot)$, 
			some {\em{}relations on $\N$\/} are definable
			in $\cal{}N$ and some are not; Note that there are 
			uncountably
			many relations on $\N$ but only $\aleph_0$ possible
			defining formulas.
		\w Ordering relation $\{\arc{m, n}: m < n\}$ is defined in 
			$\cal{}N$ by the formula
				\[ (\exists v_3) v_1 + Sv_3 \approx v_2 \]
		\eit
	\eit
\w Let $U, B$ be structures for the language. A {\bf{}homomorphism
	$h$ of $U$ into $B$} is a function $h: |U| \rightarrow |B|$ such that
	\ben
	\w [(a)] For each $n$-place predicate symbol $P$ and each $n$-tuple
		$\arc{a_1, \cdots, a_n}$ of elements of $|U|$,
			\[ \arc{a_1, \cdots, a_n} \in P^U \mbox{\ iff\ }
				\arc{h(a_1), \cdots, h(a_n)} \in P^B,\]
	\w [(b)] For each $n$-place function symbols $f$ and each $n$-tuple
		\[ h(f^U(a_1, \cdots, a_n)) = f^B(h(a_1), \cdots, h(a_n)), \]
	\w [(c)] For each constant symbol $c$,
		\[ h(c^U) = c^B. \]
	\een
\w If $h$ a homomorphism of $U$ into $B$ and {\em\bfseries{}one-to-one\/}
	then $h$ is called an {\bf{}isomorphism} of $U$ into $B$.
	\bit
	\w In this case, $U$ is said to be {\bf{}isomorphic to} $B$. 
	\eit
\w Given two structures $U$ and $B$ ($U \subseteq B$), 
	when a homomorphism where
	\ben
	\w [(a)] $P^U$ is the restriction of $P^B$ to $|U|$, for each predicate
		symbol $P$ and
	\w [(c)] $f^U$ is the restriction of $f^U$ to $|U|$, for each function
		symbol $f$, and $c^U = c^B$ for each constant symbol $c$
	\een
	exists, $U$ is said to be a {\bf{}substructure} of $B$ and
	$B$ an {\bf{}extension} of $U$.

\w ({\bf{}Homomorphism Theorem}) Let $h$ be a homomorphism of $U$ into
	$B$, and let $s$ map the set of variables into $|U|$.
	\ben
	\w [(a)] For any term $t$, 
		\[ h(\bar{s}(t)) = \overline{h\circ{s}}(t) \]
		where $\bar{s}(t)$ is computed in $U$ and 
		$\overline{h \circ s}(t)$ is computed in $B$.
	\w [(b)] for any quantifier-free formula $\alpha$ not containing the
		equality symbol,
			\[ \models_U \alpha [s] \mbox{\ iff\ } 
				\models_B \alpha [h\circ s]. \]
	\w [(c)] If $h$ is an isomorphism, then we may delete the restriction
		``not containing the equality symbol'' in (b).
	\w [(d)] If $h$ is a homomorphism of $U$ {\em\bfseries{}onto\/} $B$,
		then in $b$ we may delete the restriction ``quantifier-free''.
	\een
\w Two structures $U$ and $B$ for the language are {\bf{}elementarily
	equivalent}, written as $U \equiv B$, iff for any sentence
	$\sigma$, 
		\[ \models_U \sigma \Leftrightarrow\ \models_B \sigma.\]
\w An {\bf{}automorphism} of the structure $U$ is an isomorphism
	of $U$ onto $U$.
	\bit
	\w {\em{}An automorphism preserves the definable relations\/.} Formally,
		let $h$ be an automorphism of the structure $U$ and $R$ be
		an $n$-ary relation on $|U|$. Then for any $a_1, \cdots, a_n$
		in $|U|$, 
			\[ \arc{a_1, \cdots, a_n} \in R \Leftrightarrow
				\arc{h(a_1), \cdots, h(a_n)} \in R.\]
	\w Above corollary is useful in showing that some relations are
		{\em{}not definable\/}.

	\eit
\eit

\subsection{Unique Readability}
\bit
\w In order to be able to apply Recursion Theorem, we need unique
	readability results as in sentential logic.
\w ({\bf{}Unique Readability Theorem for Terms}) 
	The set of terms is freely generated from the set of variables 
	and constant symbols by the {\eufm{}F}$_f$ operations.
\w ({\bf{}Unique Readability Theorem for Formulas}) 
	The set of wffs is freely generated from the set of atomic formulas
	by the operations \EE$_\neg$, \EE$_\rightarrow$, 
	{\eufm{}A}$_i$ ($i = 1, 2, \cdots$).
\eit

\subsection{Deductive Calculus}
\bit
\w Let's consider what constitutes a `{\bf{}proof}' of $\Sigma \models 
	\tau$. A proof should satisfy at least the followings.
	\ben
	\w [(a)] A proof should be finitely long. 
	\w [(b)] A person should be able to check the proof to ascertain
		that it is correct. This checking process should be 
		{\em{}effective\/}.
	\een
\w We will select an infinite set $\Lambda$ of formulas to be
	called {\bf{}logical axioms} and a {\bf{}rule of inference} which
	will enable us to obtain a new formula from certain others.
\w Given $\Lambda$ and inference rules, for a set $\Gamma$ of
	formulas, the {\bf{}theorems} of $\Gamma$ will be formulas
	which can be obtained from $\Lambda \cup \Gamma$ by use
	of the rule of inference (some finite number of times).
\w If $\varphi$ is a theorem of $\Gamma$, written as 
	$\Gamma \vdash \varphi$, then a sequence of formulas which
	records how $\varphi$ was obtained from $\Gamma \cup \Lambda$
	with the rule of inference will be called a {\bf{}deduction of
	$\varphi$ from $\Gamma$}.
	\bit
	\w The choice of $\Lambda$ and the rule(s) of inference is not
		unique. For example, we can let $\Lambda = \emptyset$ and
		have many rules of inference. Or we can have infinite set
		of logical axioms and just one rule of inference.
	\w We will take the latter approach.
	\eit
\w Our one rule of inference is traditionally known as
	{\bf{}modus ponens}.
\w ({\bf{}Modus Ponens}) {\em{}From the formulas $\alpha$ and 
	$\alpha \rightarrow \beta$ we may infer $\beta$}.
		\[ \frac{\alpha,\quad \alpha \rightarrow \beta}{\beta} \]
\w A set $\Delta$ of formulas is {\bf{}closed under modus ponens}
	iff whenever two formulas $\alpha$ and $\alpha \rightarrow \beta$
	is in $\Delta$, then also $\beta$ is in $\Delta$.
\w For a fixed set $\Gamma$, $\Delta$ is {\bf{}inductive} iff
	$\Gamma \cup \Lambda \subseteq \Delta$ and $\Delta$ is closed
	under modus ponens.
	\bit
	\w Then the set of theorems of $\Gamma$ is simply the
		{\em{}smallest inductive set\/}.
	\w This situation is similar to the one in 
		sentential logic but the only difference is we are
		closing the initial set under 
		{\em\bfseries{}partially-defined\/}
		function (its domain consists only of pairs of the form
		$\arc{\alpha, \alpha \rightarrow \beta}$).
	\eit
\w {\em{}We define $\varphi$ to be a {\bf{}theorem of $\Gamma$}, written
	as $\Gamma \vdash \varphi$ iff $\varphi$ belongs to the set
	generated from $\Gamma \cup \Lambda$ by modus ponens\/}.
	\bit
	\w The set of theorems of $\Gamma$ is not {\em\bfseries{}freely}
		generated, which implies that a theorem {\em{}never has 
		unique deduction\/}.
	\eit
\w A {\bf{}deduction of $\varphi$ from $\Gamma$} is a sequence
	$\arc{\alpha_0, \cdots, \alpha_n}$ of formulas s.t.
	$\alpha_n = \varphi$ and for each $i \le n$ either
	\ben
	\w [(a)] $\alpha_i \in \Gamma \cup \Lambda$, or
	\w [(b)] for some $j$ and $k$ less than $i$, $\alpha_i$ is
		obtained by modus ponens from $\alpha_j$ and
		$\alpha_k = \alpha_j \rightarrow \alpha_i$.
	\een
\w {\em{}There exists a deduction of $\alpha$ from $\Gamma$
	iff $\alpha$ is a theorem of $\Gamma$}.
	\bit
	\w Note that this is similar to `$C^* = C_*$' situation; 
		inductive closure versus formula-building sequences.
	\w We define $\varphi$ to be {\bf{}deducible from $\Gamma$}
		iff $\Gamma \vdash \varphi$.
	\eit
\w Say that a wff $\varphi$ is a {\bf{}generalization} of $\psi$
	iff for some $n \ge 0$ and some variables 
	$x_1, \cdots, x_n$,
		\[ \varphi = (\forall{x_1})\cdots(\forall{x_n})[\psi].\]
\w The {\bf{}logical axioms} are all generalizations of wffs
	of the following forms, where
	$x, y$ are variables and $\alpha, \beta$ are wffs:
	\ben
	\w [(a)] Tautologies;
	\w [(b)] $(\forall{x})[\alpha] \rightarrow \alpha_t^x$, where
			$t$ is substitutable for $x$ in $\alpha$;
		\bit
		\w $\alpha_t^x$ is the expression obtained from the formula
			$\alpha$ by replacing the variable $x$, wherever
			it occurs free in $\alpha$, by the term $t$.
		\eit
	\w [(c)] $(\forall{x})[\alpha \rightarrow \beta]$ $\rightarrow$
		$((\forall{x})[\alpha] \rightarrow (\forall{x})[\beta])$;
	\w [(d)] $\alpha \rightarrow (\forall{x})[\alpha]$, where
		$x$ does not occur free in $\alpha$;	
	\een
	And if the language includes equality, then we add
	\ben
	\w [(e)] $x \approx x$;
	\w [(f)] $x \approx y \rightarrow (\alpha \rightarrow \alpha')$,
		where $\alpha$ is atomic and $\alpha'$ is obtained from 
		$\alpha$ by replacing $x$ in zero or more places by $y$;
	\een
\w $\Gamma \vdash \varphi$ iff $\Gamma \cup \Lambda$ tautologically
	implies $\varphi$.
\w A set of formulas $\Gamma$ is {\bf{}inconsistent} iff 
	for some $\beta \in \Gamma$, both $\beta$ and $\neg\beta$
	are theorems of the set.
\w {\bf{}Deductions and metatheorems}
	\bit
	\w ({\bf{}Generalization Theorem}) If $\Gamma \vdash \varphi$ and
		$x$ does not occur free in any formula in $\Gamma$, then
		$\Gamma \vdash (\forall{x})[\varphi]$.
		\bit
		\w Generalization Theorem reflects our informal feeling
			that if we can prove 
			$\underline{\hspace{0.2cm}}x\underline{\hspace{0.2cm}}$
			without any special assumptions about $x$, then
			we are entitled to say that
			``since $x$ are arbitrary, we have 
			$(\forall{x})
			[\underline{\hspace{0.2cm}}x\underline{\hspace{0.2cm}}]$''.
		\eit
	\w ({\bf{}Rule T})
		If $\Gamma \vdash \alpha_1, \cdots, \Gamma \vdash \alpha_n$ and
		$\{\alpha_1, \cdots, \alpha_n\}$ tautologically implies
		$\beta$, then $\Gamma \vdash \beta$.
	\w ({\bf{}Deduction Theorem}) If $\Gamma; \gamma \vdash \varphi$,
		then $\Gamma \vdash (\gamma \rightarrow \varphi)$.
	\w ({\bf{}Contraposition}) 
		$\Gamma; \varphi \vdash \neg\psi$ iff
		$\Gamma; \psi \vdash \neg\varphi$.
	\w ({\bf{}Reductio Ad Absurdum}) If $\Gamma; \varphi$ is inconsistent
		then $\Gamma \vdash \neg\varphi$.
	\w ({\bf{}Generalization on Constants})
		Assume that $\Gamma \vdash \varphi$ and that $c$ is a constant
		symbol which does not occur in $\Gamma$. Then there is a variable
		$y$ (which does not occur in $\varphi$) s.t.
		$\Gamma \vdash (\forall{y})[\varphi^c_y]$. Furthermore,
		there is a deduction of $(\forall{y})[\varphi_y^c]$ from
		$\Gamma$ in which $c$ does not occur.
		\bit
		\w Assume that $\Gamma \vdash \varphi^x_c$, where the constant
			$c$ does not occur in $\Gamma$ or in $\varphi$.
			Then $\Gamma \vdash (\forall{x})[\varphi]$, and
			there is a deduction of $(\forall{x})[\varphi]$ from
			$\Gamma$ in which $c$ does not occur.
		\eit
	\w ({\bf{}Rule EI}) Assume that the constant symbol $c$
		does not occur in $\varphi, \psi$, or
		$\Gamma$, and that
			\[ \Gamma; \varphi^x_c \vdash \psi. \]
		Then
			\[ \Gamma; (\exists{x})[\varphi \vdash \psi] \]
		and there is a deduction of $\psi$ from $\Gamma;
		(\exists{x})[\varphi]$ in which $c$ does not occur.
	\eit
\eit


\subsection{Soundness and Completeness Theorems}
\bit
\w We need to prove that the information conveyed by
	our deductive calculus is no more (soundness) and no less
	(completeness).
\w A set of formulas $\Gamma$ is {\bf{}satisfiable} iff there is
	some $U$ and $s$ s.t. $U$ satisfies every member of $\Gamma$
	with $s$.
\w ({\bf{}Soundness Theorem}) If $\Gamma \vdash \varphi$,
	then $\Gamma \models \varphi$.
	\bit
	\w This theorem can be proved using the following lemmas:
		\ben
		\w {\em{}Every logical axiom is valid\/}.
		\w $\bar{s}(u^x_t) = \overline{s[\bar{s}(t)/x]}(u)$.
		\w ({\bf{}Substitution Lemma})
			If the term $t$ is substitutable for the 
			variable $x$ in
			the wff $\varphi$, then
				\[ \models_U \varphi^x_t[x] \mbox{\ iff\ }
				\models_U \varphi[s[\bar{s}(t)/x]].
				\]
		\w If $\vdash (\varphi \leftrightarrow \psi)$, then
			$\varphi$ and $\psi$ are logically equivalent.
		\w If $\varphi'$ is an alphabetic variant of $\varphi$,
			then $\varphi$ and $\varphi'$ are logically equivalent.
		\w {\em\bfseries{}If $\Gamma$ is satisfiable, 
			then $\Gamma$ is consistent\/.} (In essence, this is
			equivalent to the soundness theorem.)
			
		\een
	\eit
\w ({\bf{}Completeness Theorem}\footnote{By G\"{o}del (1930).})
	\ben
	\w [(a)] If $\Gamma \models \varphi$, then $\Gamma \vdash \varphi$.
	\w [(b)] Any consistent set of formulas is satisfiable.
	\een
\w ({\bf{}Compactness Theorem})
	\ben
	\w [(a)] If $\Gamma \models \varphi$, then for some finite
		$\Gamma_0 \subseteq \Gamma$ we have
		$\Gamma_0 \models \varphi$.
	\w [(b)] If every finite subset $\Gamma_0$ of $\Gamma$ is satisfiable,
		then $\Gamma$ is satisfiable.
	\een
\w ({\bf{}Enumerability Theorem})
	For a reasonable language, the set of valid wffs can be 
	effectively enumerable.
	\bit
	\w By a {\bf{}reasonable language}, we mean one whose set of 
		parameters can be effectively enumerated and such that
		the two relations
		\[ \{\arc{P, n}: \mbox{$P$ is an $n$-place predicate symbol}\} \]
		and
		\[ \{\arc{f, n}: \mbox{$f$ is an $n$-place function symbol}\} \]
		are {\em{}decidable\/}.
	\eit
\w Let $\Gamma$ be a decidable set of formulas in a reasonable language.
	\ben
	\w [(a)] The set of theorems of $\Gamma$ is effectively enumerable.
	\w [(b)] The set $\{\varphi: \Gamma \models \varphi\}$ of 
		formulas logically implied by $\Gamma$ is effectively
		enumerable.
	\een
\w Assume that $\Gamma$ is a decidable set of formulas in a 
	reasonable language, and for any sentence $\sigma$ either
	$\Gamma \models \sigma$ or $\Gamma \models \neg\sigma$.
	Then the set of sentences implied by $\Gamma$ is
	decidable.

\eit

\subsection{Models of Theories}

\bibliographystyle{plain}
\bibliography{bib/mac,bib/math}
\nocite{Enderton72,Barwise77,BM96}
\end{document}
% LocalWords:  XYMATRIX wffs wff iff ome therwise ary ff deducibility EC od EI
% LocalWords:  Definability ponens metatheorems Contraposition Reductio del
% LocalWords:  Absurdum Enumerability LocalWords
