% $Id: plt.tex,v 1.1.1.1 2002/08/10 03:28:03 cjeong Exp $
% Cheoljoo Jeong (cjeong@cs.columbia.edu)
%
\documentclass{myart}
\usepackage{mydef,epsfig}
\usepackage[all]{xy}
\begin{document}
\small
\hrule
\vspace*{0.2cm}
\noindent
{\normalsize\bf Notes on Programming Languages and Translators}
\vspace*{0.1cm}\\
{\small{}Author: Cheoljoo Jeong (\tt{}cjeong@cs.columbia.edu})\\
{\small{}Date: \today}
\vspace*{0.2cm}
\hrule

\section{Values}
\subsection{Types and values}
\bit
\w a {\em type\/} is a set of values
\w a \bb{primitive type} is one whose values are atomic and
	therefore cannot be decomposed into simpler values
\w \bb{composite types}: 
	\bit
	\w Cartesian products (tuples and records)
	\w disjoint unions (variants and unions)
	\w mappings (arrays and functions)
	\w powersets (sets)
	\w recursive types (dynamic data structures)
	\eit
\eit
\subsection{Type systems}
\bit
\w In a \bb{statically typed} languages, every variable and parameter
	has a fixed type chosen by the programmer.
\w In a \bb{dynamiically typed} languages, only the {\em values\/}
	have fixed types. 
	\bit
	\w A variable and parameter has no designated
	types, but may take values of different types at different
	times.
	\w The operands must be type-checked {\em immediately before\/}
	performing an operation at {\em run-time\/}
	\w Example: Lisp \& Smalltalk
	\eit
\w \bb{structural equivalence} of types:
	$T \equiv T'$ iff $T$ and $T'$ have the same set of values
\w \bb{name equivalence}:
	$T \equiv T'$ iff $T$ and $T'$ were defined in the same place
\w \bb{first-class values}: assigned as 
\w \bb{type completeness principle}: no operation should be
	{\em arbitrarily\/} restricted in the types of the values involved
\eit



\section{Storage}
\section{Bindings}
\bit
\w A \bb{binding} is a pair of \bb{name} and \bb{value} it denotes.
\w An \bb{environment} is a set of bindings.
\w \bb{static and dynamic biding}
\eit

%%%%
\section{Type Systems}
\bit
\w \bb{monomorphic} type system:
	not good for reuse; e.g. sorting procedures are inherently
	{\em generic}
\w \bb{overloading}: a single name enotes several abstractions
	simulataneously
\w \bb{polymorphism}: abstractions that operate uniformly on
	values of different types
\w \bb{inheritence}: subtypes inherit operations from their supertypes
\eit

\subsection{Monomorphism}
\bit
\w 
\eit















% Q1
\section{Lexical Analysis}
\subsection{The role of the lexical analyzer}
\bit
\w What does a lexical analyzers do?
	\bit
	\w reads the input characters and produce as output a sequence
of tokens that the parser uses for syntax analysis
	\w correlates error messages from the compiler with the source
program
	\w strips out from the source program comments and white
spaces
	\w sometimes, macro preprocessor functions may be implemented
as part of a lexical analyzer
	\eit
\w Why separate the lexical analysis from parsing
	\bit
	\w \bb{for simpler design}: parser assuming no white space and
comments are far easier to design than one that does not
	\w \bb{compiler efficiency}:  a large amount of time is spent
doing I/O (reading source file); speciailized techniques for reading
and processing tokens can speed up the compilation process
	\w \bb{compiler portability}: 
	input alphabet peculiarities and other device-specific
anomalies can be restricted to the lexical analyzer (e.g, Pascal's
$\uparrow$ operator)
	\eit
\w \bb{Tokens, patterns, lexemes}
	\bit
	\w \bb{pattern}: a rule describing the set of lexemes that can
represent a particular token in source programs
		 a set of strings for which the same token
		is produced
	\w a pattern is said to {\em match\/} each string in the set
	\w \bb{lexeme}: a sequence of characters in the source program
that is matched by the pattern for a token
	\w Example: {\tt const pi = 3.1416;} 
		\bit
		\w {\tt pi} is a lexeme for the token {\em IDENTIFIER\/}.
		\eit
	\w {\em {tokens} are terminal symbols in the grammar}
	\eit
\w \bb{Attributes for tokens}
	\bit
	\w when more than one pattern matches a lexeme, lexer must
provide additional information
	e.g. pattern \bb{num} may match both the string {\tt 0} and {\tt
1},
	though not useful for lexer but it's essential for the code generator 
	\eit
\eit



\section{Run-Time Environments}
\bit
\w We need to relate the {\em static source text\/} to the {\em
	actions	that must occur at run time\/}
\w Relationship between \bb{names} and \bb{data objects}
	\bit
	\w same name deonte different data objects as execution proceeds
	\eit
\w Allocation and deallocation of data objects is managed by the {\bf
	run-time support\/} package
	\bit
	\w run-time support package consists of routines loaded ``with''
		the generated target code
	\eit
\w Each execution of a procedure is referred to as an \bb{activation}
	of the procedure
	\bit
	\w For recursive procedure, there may be several of its
		activations
	\eit
\w The {\bf representation\/} of a data object at run time is
	determined
	by its {\em type\/}
\eit
\subsection{Source language issues}
\bit
\w A \bb{procedure definition} is a declaration that, in its simplest
	form, associates an identifier (procedure name) with a 
	statement (procedure body).
\eit
\paragraph{Activation trees}
	\bit
	\w Assumptions: 
		\bit
		\w control flows sequentially
		\w each execution of a procedure starts at the
beginning of
		the procedure and eventually returns control to the
point immediately following the place where the procedure was called;
	this means the flow of control between procedures can be
depicted
using {\em trees\/}
		\eit
	\w \bb{activation of a procedure}: each execution of the
procedure body
	\w \bb{lifetime} of an activation of a procedure $p$:
	the sequence of steps between the first and last steps in the
execution of the procedure body, including the lifetime of the 
procedures called by $p$, and so on.
	\w If $a$ and $b$ are procedure activations, then their
	lifetimes are either \bb{non-overlapping} or are \bb{nested}:
		this is why \bb{trees}! (recall $2n \choose n+1$)
	\w a procedure is \bb{recursive} if a new activation can begin
	before an earlier activation of the same procedure has ended.
	\w \bb{activation tree} is used to depict the way control
	enters and leaves activations.
	\eit
\paragraph{Control stacks}
	\bit
	\w the flow of control corresponds to the DFS of the
activation tree
	\w \bb{control stakcs} are used to keep track of live
procedure activations.
	\w idea is {\em put the node for an activation on the stack 
	as the activation begins and to pop the node when the
activation ends\/}
	\w when node $n$ is at the top of the stack, the stack
contains the nodes along the path from $n$ to the root.
	\eit
\paragraph{The scope of a declaration}
	\bit
	\w a \bb{declaration} in a language is a syntactic construct
	that associates information with a {\em name}
	\w the \bb{scope rules} of a language determines which
declarations of a name applies when the name appears in the text of a program.
	\w the portion of the program to which a declaration applies
is called the \bb{scope} of that declaration
	\w an occurrence of a name in a procedure is said to be {\em local\/}
to the procedure if it is in the scope of a declaration within the
procedure, o.w. {\em nonlocal\/}
	\eit
\paragraph{Bindings of names}
	\bit
	\w \bb{environment} refers to a function that maps a {\em
name\/} to a 
	{\em storage location\/}
	\w \bb{state} refers to a function that maps a {\em storage
	location\/} to the {\em value\/} held there
\[\xymatrix{
	\mbox{name} \ar[rr]^{\mbox{\scriptsize environment}} & &
	\mbox{storage: l-value}  \ar[rr]^{\mbox{\scriptsize state}} &&
	\mbox{value: r-value}
}\]
	
	\w \bb{assignment changes the state not the environment}
	\w when environment associates storage location $s$ with the
	name {\tt x}
		\bit
		\w {\tt x} is bound to $s$
		\w this association itself is referred to as a
	\bb{binding} of {\tt x}
		\eit
	\w \bb{A binding is the dynamic counterpart of a delcaration}.
\vspace*{0.2cm} \\
\centerline{\begin{tabular}{l|l} \hline
static notion & dynamic counter part \\ \hline 
definition of a procedure & activations of the procedure \\
declaration of a name & bindings of the name \\
scope of the delcaration & lifetime of a binding \\ \hline
\end{tabular}}
	\eit


\subsection{Storage organization}
\paragraph{Subdivision of run-time memory}



\bibliographystyle{plain}
\bibliography{00bib/mac,00bib/sys,00bib/inet,00bib/math}
\end{document}
