\documentclass{memo}
\usepackage{mathptm,mydef,myenv}
%\usepackage{MinionPro}
\begin{document}
\small
\noindent{\large\bf{}TCP/IP}

\paragraph{The Internet}
Internet uses TCP/IP protocol suite (a.k.a. Internet protocol suite).
The protocol adopts the well-known 4-layer stack.
\bit
\w \bb{application layer}: e.g. HTTP
\w \bb{transport layer}: TCP (Transmission Control Protocol)
\w \bb{network layer}: IP (Internet Protocol)
\w \bb{datalink layer}: e.g. Ethernet
\eit

\paragraph{Transport layer: TCP}
TCP (\bb{Transmission Control Protocol}) is a \bb{reliable},
\bb{connection-based}, \bb{full-duplex} protocol based on acknowledgement and
retransmission.  TCP \bb{sequences} the data by associating a sequence number
with every {\em byte\/} it sends. 

Also, TCP provides \bb{flow control}. TCP always tells it peer exactly how
many bytes of data it is willing to accept from the peer at one time (this is
called \bb{advertised window}). At any time, the window (which dynamically
changes) is the amount of room currently available in the receive buffer,
guaranteeing that the sender cannot overflow the receive buffer. 

\paragraph{Transport layer: UDP}
UDP (\bb{User Datagram Protocol}) is a \bb{connectionless}, \bb{full-duplex}
protocol and UDP sockets are an example of datagram sockets.  This is a
\bb{unreliable} protocol whith no guarantee that
UDP datagrams will ever reach their destination. 
So, if we want any reliable communication, we need to add the such features at
the application level (e.g. acknowledge, timeouts, retransmissions, etc.).

UDP does not provide \bb{flow control}. 





\end{document}
