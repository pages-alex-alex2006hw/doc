\documentclass{myproc}
%\addtolength{\topmargin}{-2cm}
%\addtolength{\textheight}{2cm}
\usepackage{mathptm,mydef}
\usepackage{courier}
\usepackage{epsfig}
\usepackage{alltt}
%\renewcommand{\ttdefault}{txtt}
\usepackage[all]{xy}
%\usepackage{MinionPro}

\usepackage{hyperref}
\hypersetup{
    colorlinks, 
    citecolor=black, 
    filecolor=black, 
    linkcolor=blue, 
    urlcolor=black
}

\begin{document}
\small


\begin{center}
{\large\bf Apple Home Kit: Overview}
\end{center}

\vspace*{1cm}

\tableofcontents

%\pagebreak

\section{Introduction}
API (and library) which can be used by iPhone App developers to discover
devices, request services to devices (synchronous call), and add hook to
events (attach nonblocking, asynchronous handler).

\subsection{Questions}
\bit
\w Then, what software API/middleware the device manufacturers
(e.g. Honeywell thermostat developers) use?
\eit

\section{Basic Classes}
\subsection{HMAccessory}
\textcolor{blue2}{Accessories (\textbf{HMAccessory}) are installed into homes
  and assigned to rooms. These are the actual physical home automation devices,
  such as a garage door opener or a thermostat. If the user doesn't configure
  any rooms, HomeKit assigns accessories to a special default room for the
  home. }  

{\em Each physical accessory in
  the home is represented by one and only one accessory object\/}. A single 
accessory
provides one or more services, represented by instances of \bb{HMService}.
\textcolor{red2}{Corresponds to AllJoyn or Meadow device.}


\subsection{HMAction}
\bb{HMAction} is an abstract base class for actions in Home Kit.

\subsubsection{HMActionSet}
 An \bb{HMActionSet} object represents a set of actions (instances of
 \bb{HMAction}) to 
 be applied as a single set. 

Action sets can be executed as a result of evaluating a
\textcolor{red2}{trigger} (instances of 
\bb{HMTrigger}) or manually with \bb{startExecutingActionSet:}. Actions in an
action set 
are performed in an unspecified order. You create new action sets using the
\bb{addActionSetWithName:completionHandler:} method of \bb{HMHome}. 
\textcolor{red2}{Very primitive but similar to the notion of ``Workflow'' in
  Meadow. Note that AllJoyn has no explicit, reified notion of
  ``Action''. In AllJoy, each ``work'' is directly implemented using
  C++/Java/etc -- i.e. there is NO class named ``Work''.} 


\subsection{HMCharacteristic}
An \bb{HMCharacteristic} object represents a specific characteristic of a
service -- for example, if a light is on or off, or what temperature a
thermostat is set to.  
\textcolor{red2}{Loosely related to AllJoyn/Meadow property.}

\subsection{HMHome}
\textcolor{blue2}{Homes (\textbf{HMHome}) are the top level container, and
  represent a structure that a 
user would generally consider to be a single home. Users might have multiple
homes that are far apart, such as a primary home and a vacation home. Or they
might have two homes that are close together, but that they consider different
homes  -- for example, a main home and a guest cottage on the same property. }

An \bb{HMHome} object allows you to communicate with and configure the
different accessories in a home. Homes are the central organizing object for
Home Kit. 

Homes have three main purposes:
\bit
\w Organize accessories into a number of rooms, which are themselves
optionally grouped into zones. 

\w Serve as the main access point for communicating with and configuring
accessories. 

\w Allow the user to define sets of actions that can be performed with a
single operation, and triggers that can cause an action set to be performed at
a specific time. 
\eit
You don’t create homes directly. Instead, you create them with the 
\bb{addHomeWithName:completionHandler:} method of \bb{HMHomeManager}.

\textcolor{red2}{AllJoyn/Meadow does not have explicit notion of ``Container''
  of devices. Instead, it has a ``tree-structured namespace and based on the
  name of device, we can impose some notion of containment.}.

\begin{figure*}[hpt]
{\scriptsize
\[\xymatrix@-0.4pc{
*+[F]\txt<2cm>{HMHomeManager} \ar@{-}[rr]^{\txt{manages}} &&
*+[F]\txt<2cm>{HMHome} \ar@{o-}[rr]^{\txt{contains}}\ar@{o-}[d] &&
*+[F]\txt<2cm>{HMAccessory} \ar@{-}[dd]^{\txt{provides}}
\ar@{o-}[d]\\ 
&&*+[F]\txt<2cm>{HmZone} \ar@{o-}[d]&& \\
&&*+[F]\txt<2cm>{HMRoom} && *+[F]\txt<2cm>{HMService} \ar@{-}[d]^{\txt{invokes}} \\
*+[F]\txt<2cm>{HMTrigger}\ar@{-}[rr]^{\txt{executes}} &&*+[F]\txt<2cm>{HMActionSet} \ar@{o-}[rr]&& *+[F]\txt<2cm>{HMAction} \\
}\]}
\caption{Class diagram}
\end{figure*}

\subsubsection{HMHomeManager}
A home manager object manages a collection of one or more homes. Use the home
manager to add homes, get the list of homes, and track changes to homes with
the home manager’s delegate. 

\subsubsection{HMRoom}
\textcolor{blue2}{Rooms (\textbf{HMRoom}) are optional parts of homes, and
  represent individual rooms in 
the home. Rooms don’t have any physical characteristics—size, location,
etc. They’re simply names that are meaningful to the user, such as “living
room” or ``kitchen''. Meaningful room names enable commands like, ``Siri, turn
on 
the kitchen lights.''  }

An \bb{HMRoom} object is used to represent a room in a home. Accessories can
be assigned to rooms. 

You create new rooms using the \bb{addRoomWithName:completionHandler:} method of
\bb{HMHome}. You can also group rooms into zones using instances of
\bb{HMZone}.  

\subsubsection{HMZone}
\textcolor{blue2}{Zones (\textbf{HMZone}) are optional groupings of rooms in a
  home. ``Upstairs'' and ``downstairs'' would be represented by zones. Zones are
  completely optional -- rooms don’t need to be in a zone. By adding rooms to a
  zone, the user is able to give commands to Siri such as, ``Siri, turn on all
  of the lights downstairs.''  }

 An \bb{HMZone} object represents a collection of rooms that the user thinks
 of as  a single area or zone -- for example, ``Living Room'' and ``Kitchen''
 might be  grouped into a zone called ``Downstairs''. 

You create new zones using the \bb{addZoneWithName:completionHandler:} method of
\bb{HMHome}. A zone can not span homes -- that is, you can’t create a zone
that includes rooms from more than one home. 


\subsection{HMService}
\textcolor{blue2}{Services (\textbf{HMService}) are the actual services provided by an accessory. Accessories have both user-controllable services, like a light, and services that are for their own use, like a firmware update service. Home Kit is most concerned with user-controllable services.}

A single accessory may have more than one user-controllable service. For example, most garage door openers have a service for opening and closing the door, and another service for the light on the garage door opener. 

{\em Services have
characteristics that can be queried to discover their state or modified to
cause the accessory to modify its behavior. \/}

\textcolor{red2}{Roughly corresponds to Meadow/AllJoyn methods}.

\subsubsection{HMServiceGroup}
An \bb{HMServiceGroup} object represents a collection of accessory services,
making it easier to address the services as a single entity. For example, a user
might choose to group a set of lights together as ``desk lamps'', and have
another set of lights grouped as ``ceiling lights''.  

You create service groups using the
\bb{addServiceGroupWithName:completionHandler:} method of \bb{HMHome}. Service
  groups are visible to Siri and allow users to control a group of services
  through Siri. 

\subsection{HMTrigger}
An \bb{HMTrigger} object represents a trigger event, used to trigger one or
more action sets (instances of \bb{HMActionSet}) when the conditions of the
trigger are satisfied. 

This class defines the basic behavior of triggers, but does not itself specify
any criteria for firing a trigger. You should use instances of subclasses of
\bb{HMTrigger} to set up concrete triggers for actions. 


\textcolor{red2}{Roughly corresponds to Meadow events or AllJoyn signals}.

\subsubsection{HMTimerTrigger}
An \bb{HMTimerTrigger} object represents a trigger based on periodic timers.

When a timer trigger is enabled using \bb{enable:completionHandler:}, the
system checks to verify that the timer trigger’s fire date, time zone, and
recurrence rules yield a next fire date that is in the future. 







\end{document}
