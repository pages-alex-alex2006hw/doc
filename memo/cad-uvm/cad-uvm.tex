\documentclass{note}
\usepackage{mathptm,mydef,myenv}
%\usepackage{MinionPro}
\usepackage{alltt}
\usepackage[T1]{fontenc}
\usepackage[all,knot]{xy}
\usepackage{hyperref}
\hypersetup{
    colorlinks ,
    citecolor=black, 
    filecolor=black,
    linkcolor=blue, 
    urlcolor=black
}

\begin{document}
\noindent{\textcolor{blue2}{\large\bf{}Notes on UVM}}
\small

%\tableofcontents

\paragraph{Overview}
\bit
\w  UVM consits of two parts: 
  \bit
  \w domain-independent class library which implements ``component
  technology'' (how to build systems by ``composing'' components like lego
  blocks) 
  \w domain-specific components used for EDA verification domain
  \eit
\w \bb{software reuse}
  \bit
  \w how to reuse code with minimal modification
  \w how to extract commonalities into ``fixed framework'' (which does not
  need to be changed) and put hotspots (customization points) outside
  \w or we want to change just to part we want, without changing the rest
  \w key to reuse is ``late binding'' 
     \bit
     \w hard coding is worst -- we need to update the ``hard coded caller''
     \w OOPL allowed late binding by dynamic method lookup
     \w only compile the subclass again, rest is ok. only revise/compile the
         callee not the caller
         \bit
         \w for this, \bb{Hollywood principle}: ``don't call us, we'll call you''
         \eit
     \eit
  \eit

\w \bb{component technology}
  \bit
  \w mostly about connecting ``resusable components''
  \w ``components'' an abstraction layer over ``classes''
  \eit

\w \bb{software framework}
  \bit
  \w no notion of components; just plain classes
  \w 
  \eit
\eit

\section{UVM Configuration}

\begin{verbatim}
class mycomp extends uvm_component;

  mysubcomp subcomp0;

  // `uvm_component_utils(mycomp)
  `typedef uvm_component_registry #(mycomp, "mycomp") type_id;
  static function type_id get_type(); 
    return type_id::get();
  endfunction
  virtual function uvm_object_wrapper get_object_type(); 
    return type_id::get();
  endfunction

  virtual function build_phase();
    // instead of just doing
    //   subcomp0 = new
    // 
    // uvm_component_registry#(mycomop, "mycomp")::create("mycomp", this) {
    //   uvm_factory f = uvm_factory::get();
    //   f.create_component_by_type(get(), // type-specific registry
    //                              ctx, name, parent);
    //   $cast(create, obj))
    subcomp0 = mysubcomp::type_id::create("mycomp", this);
  endfunction
endclass
\end{verbatim}


\subsection{Component instance pool}
\bit
\w maintains a ``component (object) system''
\w hierachy of component instances is explicitly maintainted through the help
of ``factory method'' -- \verb+create()+
  \bit
  \w \verb+new+ is a system function so, if we let ``component system users''
      to use ``new'' directly, we as component system developers cannot do
      anything
  \w \verb+create()+ actually adds some ``room'' around the actual call to
  ``new'' 
  \w 
  \eit
\w that's how \verb+config_db::get/set+ works
\eit


\section{Overview}
\bit
\w UVM consists of two parts:
  \bit
  \w domain-independent class library which implements ``component
  framework'', which is about how to building systems by ``composing''
  components (like composing lego 
    \bit
    \w to use these components, need to subclass their ``component'' class
      and follow some guidelines (guidelines to overcome that this is
      user-space library not system library)
    \w component framework is ``one'' \bb{software reuse} technology
    \eit
  \w domain-specific components used for EDA verification domain
  \eit
\w \bb{software reuse}
  \bit
  \w good reuse technology allows to reuse code with minimal modification:
     ``reused library'' and ``reuser's code'' must independently developed
    \bit
    \w if change in one causes change in the other; it's not a good reuse
    technology 
    \eit
  \w extract ``commonalities'' into ``fixed framework''  and
   put ``hot spots'' (customization points) outside

  \w \bb{late binding}
  \eit
\eit



\section{UVM Configuration}
\begin{verbatim}
  class my_uvc extends uvc_component;
    my_subuvc c0;
    my_subuvc c1;

    // `uvm_component_utils(mycomp)
    `typedef uvm_component_registry #(my_uvc, "my_uvc") type_id;
    static functino type_id get_type();
      return type_id::get();
    endfunction
    virtual function uvm_object_wrapper get_object_type();
      return type_id::get();
    endfunction
    
    virtual function build_phase();
      // instead of just doing
      //  c0 = new
      // uvm_component_registry#(my_uvc, "my_uvc")::create("my_uvc", this) {
      //   uvm_factory f = uvm_factory::get();
      //   f.create_component_by_type(get(), 
      //                              ctx, name, parent);
      //   $cast(create, obj);
      c0 = my_uvc::type_id::create("my_uvc", this);
    endfunction
  endclass  
\end{verbatim}

\section{UVM Factory}
Using the factory involves three basic operations
\bit
\w registering objects and components types with the factory
\w designing components to use the factory to create objects or components
\w configuring the factory with type and  instance override, both within and
  outside components 
\eit

\subsection{Registering objects and components types with the factory}
\begin{verbatim}
  class comp extends uvm_component;
    `uvm_component_utils(comp)
  endclass

  class comp #(type T=int, int WIDTH=32) extends uvm_component;
    `uvm_component_param_utils(comp #(T,WIDTH))
  endclass
\end{verbatim}
\bit
\w register the type with the factory 
\w define the \verb+get_type+, \verb+get_type_name+
\w create virtual methods inherited from \verb+uvm_object+
\w define static \verb+type_name+ variable in the class, which will allow to
   determine the type without having to allocate an instance
\eit

\subsection{Designing components that defer creation to factory}
\bit
\w 
\eit




\section{UVC Overview}
\bit
\w UVM \textcolor{red2}{testbench} consists of reusable verification
environments called \textcolor{red2}{verification components}.
\eit

\subsection{Verification components}
\subsubsection{Transaction (Data item)}
\bit
\w a structure of related data
\eit
\subsubsection{Driver (BFM)}
\bit
\w \bb{active} component, which drives DUT
\w receives a data from seqeucence and sends the data to DUT at signal-level
\w acts as proxy between transaction-level TB and signal-level DUT
\eit

\subsubsection{Sequencer}
\bit
\w 
\eit

\subsubsection{Monitor}
\bit
\w \bb{passive} entity
\eit

\subsubsection{Agent}
\bit
\w \bb{container} which integrates sequencer, monitor, driver into one
\w hooks up, configures, assign roles to each of these entities
\eit

\subsubsection{Environment}
\bit
\w has configuration capability
\w contains multiple agents
\eit

\subsection{UVM facilities}
\subsubsection{Factory}
\bit
\w 
\eit

\section{UVM Configuration Mechanism}
\begin{verbatim}
  class uvm_config_db#(type T=int) extends uvm_resource_db#(T);
\end{verbatim}


\end{document}

%%  LocalWords:  Expr structs Functors funptr runtime destructor consts const
%%  LocalWords:  dereference upcast ptr Functor
