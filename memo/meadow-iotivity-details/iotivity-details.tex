\documentclass{myproc}
%\addtolength{\topmargin}{-2cm}
%\addtolength{\textheight}{2cm}
\usepackage{mathptm,mydef}
\usepackage{courier}
\usepackage{epsfig}
\usepackage{alltt}
%\renewcommand{\ttdefault}{txtt}
\usepackage[all]{xy}
%\usepackage{MinionPro}

\usepackage{hyperref}
\hypersetup{
    colorlinks, 
    citecolor=black, 
    filecolor=black, 
    linkcolor=blue, 
    urlcolor=black
}

\begin{document}
\small


\begin{center}
{\large\bf IoTivity: Overview}
\end{center}

\vspace*{1cm}

\tableofcontents

%\pagebreak

\section{Introduction}

\section{Control Manager (CM)}
CM runs on top of IoTivity base framework.
It allows to
\bit
\w discover controllee devices,
\w control controllee using RESTful resource operations
\w provides PUB/SUB service for monitoring device operations or state changes
\eit

A controller APP (e.g. in Android) should have a CM for controlling devices. 

\subsection{Components of CM}
\subsubsection{Smart home data model}
\bit
\w based on Samsung Smart Home Profile
\w defines resource model for all the available home devices and appliances
\w hierarchical resources and their attributes
\w resources are classified into:
  \bit
  \w \bb{common set of resources}: device capabilities, device configuration
  \w \bb{function-specific set of resources}: resources specific to device function (e.g. Thermostat, Light, Door)
  \eit
\eit


\end{document}
