\documentclass{note}
\addtolength{\topmargin}{-2cm}
\addtolength{\textheight}{2.2cm}
\usepackage{mathptm,mydef,myenv}
\usepackage[all]{xy}
%\usepackage{MinionPro}
\usepackage{courier}
\usepackage{alltt}
\usepackage[T1]{fontenc}
\usepackage{graphicx}
\DeclareGraphicsExtensions{.png,.jpg}


\usepackage{hyperref}
\hypersetup{
    colorlinks, 
    citecolor=black, 
    filecolor=black, 
    linkcolor=blue, 
    urlcolor=black
}

\begin{document}
\small

\begin{center}
{\large\bf \textcolor{blue2}{Notes on Distributed Systems}}
\end{center}

\vspace*{1cm}

\tableofcontents
\pagebreak

\section{Introduction}
\subsection{Overview}
\bit
\w \bb{WHAT is a distributed system?} multiple networked, cooperating,
autonomous computers
\w \bb{WHY distributed?}
  \bit
  \w to connect separate computers for useful \bb{cooperative work}:
  e.g. remotely access expensive computing resource
  \w to achieve \bb{fault-tolerance} via replication
  \w to increase \bb{performance} by distributing workload
  \w to achieve \bb{security} via physical isolation
  \eit
\eit

\subsection{Distributed Architecture}


\subsection{Performance}
\bit
\w (-) network latency/bandwidth slows down performance
\w (+) 
\eit

\subsection{Fault tolerance}

\subsection{Consistency}


\section{MapReduce}

\section{RPC and Threads}


\section{Appendix: Distributed Systems}
\subsection{Transparency}
  \bit
  \w {access transparency}
  \w {location transparency}
  \w {migration transparency}
  \w {relocation transparency}
  \w {replication transparency}
  \w {concurrency transparency}
  \w {failure transparency}
  \eit
\subsection{Openness}
  \bit
  \w \bb{\textcolor{blue2}{Open} distributed system}: services are offered
  according to \textcolor{red}{\textit{\underline{standard rules} that describe
      the syntax and semantics of those services}}
     \bit
     \w \bb{standard rules}: format, contents, meaning of messages
     sent/received, 
     i.e. \textcolor{blue2}{protocols}
     \eit
  \w e.g. \bb{IDL (interface definition language)}
  \w \textcolor{blue2}{separating policty from mechanism}
  \eit
\subsection{Scalability}
\bit
\w \bb{characteristics of decentralized algorithms}
    \ben
    \w \textcolor{blue2}{No machine has complete info about system
      state.} 
    \w \textcolor{blue2}{Machines make decisions based only on local info.}
    \w \textcolor{blue2}{Failure of one machine does not ruin the algorithm.}
    \w \textcolor{blue2}{No implicit assumption that a global clock.}
    \een
\w scaling a distributed system across multiple, independent
  administrative domains
\w how to resolve conflicting policies, management, security
\w \bb{\textcolor{red2}{SCALING TECHNIQUES}} 
  \bit
  \w \textcolor{blue2}{\bf{}\#1: hiding communication latencies}:
    \bit
    \w \bb{asynchronous communication}: e.g. future in Scala (nonblocking call +
    callback when done)
    \w \bb{moving computation to client side}: e.g. well-formness of inquiry
    form is done in client (using Javascript) rather than sending the form to
    DB to check the correctness
    \eit
  \w \textcolor{blue}{\bf{}\#2: distribution}: split a compnent into smaller
  parts and distribute it across the system
    \bit
    \w \bb{DNS}: DNS name space is hierchically organized into a tree of
    \bb{domains}, which are divided into nonoverlapping \bb{zones}.
    \w \bb{WWW itself}: conceptually, a massive hypertext system indexed by
    URL but distributed to each machine.
    \eit
  \w \textcolor{blue}{\bf{}\#3: replication}: replicate the same function
  across the system; increases availability and helps to balance the load
  between components
    \bit
    \w \bb{caching}: special form of replication (``copy'' of resource)
    \w (-) \bb{consistency problem}
    \eit
  \eit
\w vertical vs horizontal scaling
\eit
\subsection{Availability} 


%% \subsection{Types of distributed systems}
%% \subsubsection{Distributed computing systems}
%% \bit
%% \w \bb{cluster computing systems}: homogeneous
%%   \bit
%%   \w single computationally-intensive program is run in parallel on multiple
%%   machines -- {\em collection of compute nodes controlled by a single master
%%     node} 
%%   \w Beowulf, MOSIX
%%   \eit
%% \w \bb{grid computing systems}: heterogeneous
%%   \bit
%%   \w 
%%   \eit
%% \eit
%% \subsubsection{Distributed information system}
%% \bit
%% \w \bb{transaction processing system}
%%    \bit
%%    \w \bb{transaction}:
%%      \bit
%%      \w \bb{atomic}: appear indivisible
%%      \w \bb{consistent}: does not violate system invariant
%%      \w \bb{isolated}: concurrent transactions do not interfere with each other
%%      \w \bb{durable}: changes are permanent when committed
%%      \eit
%%    \eit
%% \w \bb{enterprise application integration (EAI)}
%%   \bit
%%   \w \bb{MoM}
%%   \w \bb{publish-subscribe system}
%%   \eit
%% \w \bb{distributed pervasive system}
%%    \bit
%%    \w nodes are not fixed -- come and go
%%    \w \bb{instability} is the main characteristics (e.g. mobile devices,
%%    sensors) 
%%    \w small, battery-powered, mobile, wireless -- mostly
%%    \w no human administrative control
%%    \w three requirements for pervasive applications 
%%      \ben
%%      \w \textcolor{blue}{embrace contextual changes}: device must be aware
%%      that its env may change all the time
%%      \w \textcolor{blue}{encourage ad hoc composition}
%%      \w \textcolor{blue}{recognize sharing as the default}
%%      \een
%%    \w \bb{home systems}
%%    \w \bb{electronic health care systems}
%%    \w \bb{sensor networks}
%%    \eit
%% \eit


\end{document}

