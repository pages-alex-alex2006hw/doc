\documentclass{memo}
%\addtolength{\topmargin}{-2cm}
%\addtolength{\textheight}{2.2cm}
\usepackage{mathptm,mydef,myenv}
\usepackage[all]{xy}
%\usepackage{MinionPro}
\usepackage{courier}
\usepackage{alltt}
\usepackage[T1]{fontenc}
\usepackage{graphicx}
\DeclareGraphicsExtensions{.png,.jpg}


\usepackage{hyperref}
\hypersetup{
    colorlinks, 
    citecolor=black, 
    filecolor=black, 
    linkcolor=blue, 
    urlcolor=black
}

\begin{document}
\small
\noindent{\large\bf{}Notes on RPC}

\paragraph{Background}
\bit
\w Network protocols are about ``two autonomous 
       (state) machines talking to each other''.
  \bit
  \w \bb{data}: which data, which format, which meaning, when
  \w \bb{protocol spec}: just like BNF specifies possible strings, 
     protocol spec specifies how ``well-defined sequence of talking'' can
     occur  
  \eit
\w \bb{Two interacting state machines}: $M_0$ and $M_1$
   \bit
   \w Product of two state machines defines how they interact.
   \w `` Each talk'' changes each machine state (i.e. the collective 
      states of $M_0$ and $M_1$. 
   \eit
\eit

\paragraph{RPC Overview}
\bit
\w Transfer of control and data across ``network''
  \bit
  \w \underline{No shared address space}
  \w Two different machines (two PCs, register sets, etc.)
  \eit
\eit

\paragraph{Issues}
\bit
\w \bb{RPC failure semantics}: in presence of machine/communication failures

\w \bb{Shared address space}: {address-containing arguments in the absence of shared address space}  
\w \bb{Binding}:  how a caller determines the location and identity of cal\w
\bb{protocols}
\w \bb{security}:
\eit


\end{document}

