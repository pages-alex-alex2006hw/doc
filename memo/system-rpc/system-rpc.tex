\documentclass{memo}
%\addtolength{\topmargin}{-2cm}
%\addtolength{\textheight}{2.2cm}
\usepackage{mathptm,mydef,myenv}
\usepackage[all]{xy}
%\usepackage{MinionPro}
\usepackage{courier}
\usepackage{alltt}
\usepackage[T1]{fontenc}
\usepackage{graphicx}
\DeclareGraphicsExtensions{.png,.jpg}


\usepackage{hyperref}
\hypersetup{
    colorlinks, 
    citecolor=black, 
    filecolor=black, 
    linkcolor=blue, 
    urlcolor=black
}

\begin{document}
\small
\noindent{\large\bf{}Notes on RPC}

\paragraph{Background}
\bit
\w Network protocols are about ``two autonomous 
       (state) machines talking to each other''.
  \bit
  \w \bb{data}: which data, which format, which meaning, when
  \w \bb{protocol spec}: just like BNF specifies possible strings, 
     protocol spec specifies how ``well-defined sequence of talking'' can
     occur  
  \eit
\w \bb{Two interacting state machines}: $M_0$ and $M_1$
   \bit
   \w Product of two state machines defines how they interact.
   \w `` Each talk'' changes each machine state (i.e. the collective 
      states of $M_0$ and $M_1$. 
   \eit
\eit

\paragraph{RPC Overview}
\bit
\w Transfer of control and data across ``network''
  \bit
  \w \underline{No shared address space}
  \w Two different machines (two PCs, register sets, etc.)
  \eit
\eit

\paragraph{Complications}
\bit
\w \bb{RPC failure semantics}: in presence of machine/communication failures

\w \bb{Shared address space}: {address-containing arguments in the absence of
  shared address space} 
  \bit
  \w How can a \bb{remote address} represented? Or, just would \bb{remote
    handle} (e.g. HOST addr + handle id) would suffice? Just like remote
     name-value table? After all, we need higher-level sharing than machine
     addresses. 
  \w Even with this higher-level sharing, getting values from remote
  name-value table can be expensive.
    \bit
    \w may need a cache.
    \eit
  \eit

\w \bb{Blocking vs nonblocking calls}:
 
\w \bb{Binding}:  how a caller determines the location and identity of call

\w \bb{Protocols}:

\w \bb{Security}:
\eit

\paragraph{At-most-once semantics}
\bit
\w Server need to detect duplicate request
   \bit
   \w HOW: client send XID (universally-unique ID) for each request; uses same
       XID for re-send 
   
   \eit
\eit

\paragraph{Binding}
\bit
\w \bb{Binding}: to bind an importer of an interface to an exporter of an
    interface 
   \bit
   \w \bb{interface}: {type} + {instance}
     \bit
     \w \bb{type}: specify which interface the caller expects the callee to
     implement (e.g. ``MAIL SERVER'') -- governs how client/server interacts
     \w \bb{instance}: specify which particular implementor of an abstract
     interface is desired (e.g. {\tt{}smtp.live.com})
     \eit
   \eit
\w \bb{Locating exporter}:
   \bit
   \w methods:
     \ben
     \w use \bb{distributed lookup table}, which is replicated and reliable
     \w use \bb{static lookup table} saved at each machine: may contain
     non-up-to-update data
     \een
   \eit
\eit

\paragraph{\textcolor{blue2}{Protocols}}
\bit
\w There are two types of protocols: (i) \bb{latency}-oriented protocol vs
(ii) bandwidth-oriented protocol
  \bit
  \w \bb{latency-oriented protocol}: 
     \bit
     \w need small setup/teardown connection time
     \w cannot maintain large state for connection; since, otherwise, it will
     add latency
     \w typically, many requests coming -- need to serve them fast
     \eit
  \w \bb{bandwidth-oriented protocol}: opposite
  \eit
\w For RPC: we need latency-oriented protocol
   
\eit

\end{document}

