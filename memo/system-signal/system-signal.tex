\documentclass{memo}
\usepackage{mathptm,mydef,myenv}
%\usepackage{MinionPro}
\begin{document}
\small
\noindent{\large\bf{}Unix Signals}

\paragraph{Overview} 
A Unix \bb{signal}, also known as {\em software interrupts\/}, is a
notification to a process that an event has occurred.  Signals can be sent
\bit
\w by one process to another process (or to itself),
\w by the kernel to the process
\eit

\paragraph{Signal action}
Every signal has a \bb{action}, which is also called the \bb{disposition}
associated with the signal. The action of the signal can be set by
\verb+sigaction()+ function. There are three choices for the action:
\ben
\w We can set up a functino to be called whenever a specific signal
occurs. This function is called a \bb{signal handler} and this action is
called \bb{catching} a signal. Two signals \verb+SIGKILL+ and \verb+SIGSTOP+
cannot be called. Function prototype of signal handler is
  \begin{verbatim}
  void handler(int signo);
  \end{verbatim}
\w We can \bb{ignore} a signal by setting its disposition to
\verb+SIG_IGN+. The two signals \verb+SIGKILL+ and \verb+SIGSTOP+ cannot be
ignored. 
\w We can set the \bb{default} disposition for a signal by setting its
disposition to \verb+SIG_DFL+. 
\een

\paragraph{The {\tt sigaction()} function}
\begin{verbatim}
  int sigaction(int signum, 
                const struct sigaction *act,
                struct sigaction *oldact);
\end{verbatim}
\verb+signum+ can be any signal except for \verb+SIGKILL+ and \verb+SIGSTOP+.
\verb+act+ has the following structure type:
\begin{verbatim}
  struct sigaction { 
    void (*sa_handler)(int);
    void (*sa_sigaction)(int,siginfo_t*,void*);
    sigset_t sa_mask
    int sa_flags;
    void (*sa_restorer)(void);  // obsolete
  };
\end{verbatim}
The \verb+sa_handler+ can be one of \verb+SIG_DFL+, \verb+SIG_IGN+ or a
pointer to \verb+void signal_handler(int)+ function.

The \verb+sa_mask+ specifies a mask of signals which should be blocked
(i.e. added to the signal mask of the thread in which the signal handler is
blocked) during the execution of the signal handler. . 



\end{document}

% LocalWords:  Pthreads interprocess multithreaded sigaltstack errno realtime
% LocalWords:  datatypes pthreads pthread mutex pthrad mutexattr cond condattr
% LocalWords:  attr retval struct thr joinable mutexes reentrancy strtok
% LocalWords:  reentrant
