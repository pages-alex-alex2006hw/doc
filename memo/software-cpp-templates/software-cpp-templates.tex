\documentclass{memo}
\usepackage{mathptm,mydef,myenv}
%\usepackage{MinionPro}
\begin{document}
\small
\noindent{\large\bf{}C++ Templates}

\paragraph{Overview} There are two types of C++ templates: \bb{function
  templates} and \bb{class templates}. A {\em function template\/} is a
parameterized function over type parameters while a {\em class template\/} is
a parameterized class over type parameters.

\paragraph{Compile-time vs runtime polymorphism}
Template is for {\em compile-time polymorphism\/} while dynamic object binding
is for {\em runtime polymorphism\/}. 

\paragraph{Typical use in STL}
\begin{verbatim}
  template<typename C>
  C& printLast(const C& container) {
    typename C::const_iterator iter(container.begin());
    while (iter != container.end())
      ++iter;
    return *iter;
  }
\end{verbatim}

\paragraph{Case \#1: Function Templates} {\em Function templates\/} are special
functions that can operate with {\bf generic types\/}. 

\begin{verbatim}
  template <class T>
  inline T const& maximum(T const& a, T const& b) {
    return a < b ? b : a;
  }

  template <typename T> function_decl;
\end{verbatim}
When the compiler sees a call to a function \verb+maximum+, it
\bb{instantiates} the function template by replacing type parameters with {\em
concrete types\/}. 
\begin{verbatim}
  double f1, f2;
  std::cout << maximum(f1, f2) << std::endl;
\end{verbatim}

\paragraph{Case \#2: Class Templates}

\begin{verbatim}
  template<typename T>
  class Stack {
    std::vector<T> elems;

  public:
    Stack() { }
    ~Stack() { }
 
    /* copy constructor */
    Stack(T const& s);

    /* copy assignment operator */
    T& operator=(T const& rhs);

    void push(T const& t);
    void pop();
    T const& top() const;
    bool empty() const;
  };

  template<typename T>
  void Stack<T>::push(T const& t)
  {
    elems.push_back(t);
  }

  template<typename T>
  void Stack<T>::pop()
  {
    if (!elems.empty())
      elems.pop_back();
  }

  template<typename T>
  T const& Stack<T>::top() const
  {
    return elems.back();
  }

  template<typename T>
  bool Stack<T>::empty() const
  {
    return elems.empty();
  }
\end{verbatim}

\paragraph{Template specialization}
\begin{verbatim}
  template <typename T1, typename T2>
  class Pair {
    ...
  };

  /* partial specialization: both have same type */
  template <typename T>
  class Pair<T, T> { ... };
  };

  /* second type is int */
  template <typename T>
  class Pair<T, int> { ... };

  /* both are pointers */
  template <typename T1, typename T2>
  class Pair<T1 *, T2 *> { ... };
\end{verbatim}


\paragraph{C++ traits class} A {\em traits\/} class is a class used in place
of template parameters. As a class, it aggregates useful types and constants;
as a template, it provides an avenue for that {\em extra level of indirection}
that solves all software problems. 

Traits are basically {\em compile-time else-if-then\/}s. The unspecizliaed
template is the else clause, the specializations are the if clauses. 
Essentially, traits allow you to make {\em compile-time decisions based on
  {\bfseries types\/}\/}, much as you would make {\em run-time decisions based
  on   {\bfseries values\/}\/}. 

Concretely, a \bb{C++ traits class} is a {\em class template\/} used to
associate information or behavior to a compile-time entity, typically a
datatype or a constant, without modifying the existing entity.

A \bb{generic template} is defined that implements the default behavior.
In this case, all but one type is void, so \verb+is_void::value+ should be
\verb+false+, so we start with:
\begin{verbatim}
  template <typename T>
  struct is_void {
    static const bool value = false;
  };
\end{verbatim}
Next, we add a \bb{specialization} for \verb+void+:
\begin{verbatim}
  template <>
  struct is_void<void> {
    static const bool value = true;
  };
\end{verbatim}
Now, we have a complete traits type that can be used to detect if any give
type, passed in as a template parameter, is void. 

\paragraph{C++ traits: Example \#1}
Let's consider a class which works on {\tt float} and {\tt double}
datatypes. Each datatype has a size in number of bits.
We want to create a traits class which contains type-specific speciazliation
for the datatype size. 
\begin{verbatim}
  // general template
  template <class T>
  struct FP_traits { };

  // specialiation for float
  struct FP_traits<float> {
    typedef float FP_type;
    static inline FP_type width() { return 32; }
  };

  // speciailization for double
  struct FP_traits<double> {
    typedef double FP_type;
    static inline FP_type width() { return 64; }
  };
\end{verbatim}
Next, when we want to create a class 
\begin{verbatim}
  template <class T>
  class matrix {
    typedef T number_type;
    typedef FP_traits<number_type> traits_type;

    inline number_type width() { 
      return traits_type::width();
    }
  }
\end{verbatim}

\paragraph{C++ traits: Example \#2}
In the \verb+iostream+ library, the interface to \verb+streambuf+ depends on a
value of \verb+EOF+ which is distinct from all character values. In
traditional libraries, therefore, the type of \verb+EOF+ was \verb+int+, and
the function that retrieves characters returned an \verb+int+:
\begin{verbatim}
  class streambuf {
    ...
    int sgetc();  // return next char or EOF
    int sgetn(char *, int N); // get N chars
  };
\end{verbatim}
What happens when we parameterize \verb+streambuf+ on the character type? We
need not only a type for the character, but for the type of the \verb+EOF+
value. 
Here's a start:
\begin{verbatim}
  template <class charT, class intT>
  class basic_streambuf {
    intT sgetc();
    int sgetn(charT *, int N);
  };
\end{verbatim}

The {\em default traits\/} class template is empty. 
\begin{verbatim}
  template <class charT>
  struct ios_char_traits { };
\end{verbatim}
For real character types, we can {\em speciailize\/} the template and provide
useful semantics.
\begin{verbatim}
  struct io_char_traits<char> {
    typedef char char_type;
    typedef int int_type;
    static inline int_type eof() { return EOF; }
  };
\end{verbatim}

\begin{verbatim}
  template <class charT, 
            class traits = ios_char_traits<charT> > 
  class basic_streambuf {
  public:
    typedef traits traits_type;
    typedef typename traits_type::int_type int_type;
    int_type eof() { return traits_type::eof(); }
    ...
    int_type sgetc();
    int sgetn(charT*, int N);
  };
\end{verbatim}


\end{document}
