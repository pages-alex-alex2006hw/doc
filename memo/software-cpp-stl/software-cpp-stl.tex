\documentclass{memo}
\usepackage{mathptm,mydef,myenv}
%\usepackage{MinionPro}
\begin{document}
\small
\noindent{\large\bf{}C++ Standard Template Library}

\paragraph{Overview} STL contains six major kinds of components:
{\em containers}, {\em generic algorithms}, {\em iterators}, {\em function
  objects}, {\em adaptors}, and {\em allocators}.

\paragraph{Containers}
There are two categories of STL container types: {\em sequence containers\/}
and {\em sorted associative containers}.

\bb{Sequence containers} organize a collection of objects, all of the same
type \verb+T+, into a strictly linear arrangement. Examples are:
\bit
\w \verb+vector<T>+
\w \verb+deque<T>+
\w \verb+list<T>+
\eit

\bb{Sorted Associative containers} provide an ability for fast retrieval of
objects from the collection based on keys. There are four sorted associative
container types: 
\bit
\w \verb+set<Key>+
\w \verb+multiset<Key>+
\w \verb+map<Key, T>+
\w \verb+multimap<Key, T>+
\eit


\paragraph{Iterators} 
STL generic algorithms are written in terms of iterator parameters, and STL
containers provide iterators that can be plugged into the
algorithms. \bb{Iterators} are pointer-like objects that STL algorithms use to
traverse the sequence of objects stored in a container. There are five
iterator categories: {\em random access\/}, {\em bidirectional\/}, {\em
  forward\/}, {\em input\/} and {\em output iterators\/}.  See
Figure~\ref{fig:iterator} to see which type of operations are supported by the iterators.

\begin{figure*}[htpb]
\scriptsize
\centerline{\begin{tabular}{c|ccccc} \hline
operation &  input & output & forward & bidir & random\\ \hline
 copy/copy-constructor ({\tt{}X b(a)}, {\tt b = a}) & $\circ$ & $\circ$ & $\circ$ & $\circ$ & $\circ$ \\ \hline
 increment ({\tt a++, *a++}) & $\circ$ & $\circ$ & $\circ$ & $\circ$ & $\circ$ \\ \hline
 equality ({\tt ==}, {\tt !=}) & $\circ$ &  & $\circ$ & $\circ$ & $\circ$ \\ \hline
 dereferenced as rvalue ({\tt *a}, {\tt a->m}) & $\circ$ &  & $\circ$ & $\circ$ & $\circ$ \\ \hline
 dereferenced as lvalue ({\tt *a = t}, {\tt *a++ = t}) &  & $\circ$ & $\circ$ & $\circ$ & $\circ$ \\ \hline
 default constructor ({\tt X a; X()}) &  & & $\circ$ & $\circ$ & $\circ$ \\ \hline
 decrement ({\tt a--; *a--}) &  & & & $\circ$ & $\circ$ \\ \hline
 arithmetic operators ({\tt +} , {\tt -}) &  & & & & $\circ$ \\ \hline
 comparisons ({\tt >} , {\tt <}, {\tt >=}, {\tt{}<=}) &  & & & & $\circ$ \\ \hline
 compound assignment ({\tt +=} , {\tt -=}) &  & & & & $\circ$ \\ \hline
 offset dereference ({\tt a[n]}) &  & & & & $\circ$ \\ \hline
\end{tabular}}
\caption{Operations supported by iterators}\label{fig:iterator}
\end{figure*}

\bb{Input iterators} are used to read values from a sequence. An example is
its use in \verb+find+ algorithm as shown below:

\begin{verbatim}
  template <typename InputIter, typename T>
  InputIter find(InputIter first, InputIter last,
                 const T& value) {
    while (first != last && *first != value)
      ++first;
    return first;
  }
  
  int main() {
    // find the first elt equal to 7 in array
    int a[5] = {12, 3, 25, -7, 8};
    int *ptr = find(&a[0], &a[5], 7);

    list<int> lst(&a[0], &a[10]);
    list<int>::iterator i = find(lst.begin(), 
                                 lst.end(), 7);

    // print the first char after 'x'
    istream_iterator<char> in(std::cin);
    istream_iterator<char> std::eos;
    find(std::in, eos, 'x');
    std::cout << *(++in) << std::endl;
  }
\end{verbatim}


\bb{Output iterators} allow us to write values into a sequence. An example is
its use in \bb{copy} algorithm, which copies from one sequence to another:

\begin{verbatim}
  template <typename InIter, typename OutIter>
  OutIter copy(InIter first, InIter last,
               OutIter result) {
    while (first != last) {
      *result = *first;
      ++first;
      ++result;
    }
  }
\end{verbatim}

A \bb{forward iterator} is one that is both an input iterator and an output
iterator, and it thus allows both reading and writing and traversal in one
direction. 

\begin{verbatim}
  template <typename ForwardIter, typename T>
  void replace(ForwardIter first, 
               ForwardIter last, 
               Const T& x, const T& y) {
    while (first != last) {
      if (*first = x) {
        *first = y;
      ++first;
    }
  }
\end{verbatim}

A \bb{bidirectional iterator} is similar to a forward iterator, except that it
allows traversal in either direction. STL \bb{reverse} algorithm uses
bidirectional iterators.

While above iterators only allow sequence accesses to elements in containers, 
a \bb{random access iterator} allows access to any position inside a sequence
in constant time. For example, STL generic \bb{binary\_search} algorithm uses
random access iterators. 

Following is a meaning of some common iterator functions:
\bit
\w \bb{front()}: first element
\w \bb{begin()}: pointer to the first element
\w \bb{back()}: last element
\w \bb{end()}: pointer to last-plus-one element position
\eit

\paragraph{Generic Algorithms}
Many generic algorithms have a version which accepts a {\em function
  parameter\/}. For example, sorting algorithms accept a binary predicate
parameter which compares two values.


{\bf Nonmutating sequence algorithms} are those which does not modify the
containers which operate. Examples include: \verb+find+, \verb+adjacent_find+,
\verb+count+, \verb+for_each+, \verb+mismatch+, \verb+equal+, and
\verb+search+. 

{\bf Mutating sequence algorithms} modify the contents of the containers on
which they operate. For example, the \bb{unique} algorithm eliminates all
consecutive duplicate elements from a sequence. Other algorithms are:
\verb+copy+, \verb+fill+, \verb+generate+, \verb+shuffle+, \verb+remove+,
\verb+replace+, \verb+reverse+, \verb+swap+, and \verb+transform+. 

\bb{Sorting-related algorithms} are related to sorting. For example,
algorithms for sorting and merging sequences and for searching and performing
set-like operations on sorted sequences. 

\paragraph{Function Objects}
A {\em function object} is any entity  that can be applied to zero or more
arguments to obtain a value and/or modify the state of the computation. 
In C++, any ordinary function satisfies this definition, but so does an {\em
  object of any class that overloads the function call operator\/},
\verb+operator()+. 

\begin{verbatim}
  class multiply {
  public:
    int operator()(int x, int y) const {
      return x*y; 
    }
  };
  multiply multfuncobj;
  int product1 = multfuncobj(3,7);
\end{verbatim}


Consider the STL generic function \verb+accumulate+, which adds up \verb+init+
plus all the values between \verb+first+ and \verb+last+:
\begin{verbatim}
  template <typename InputIter, typename T>
  T acculmulate(InputIter first, InputIter last, 
                T init) {
    while (first != last) { 
      init = init + *first;
      ++first;
    }
    return init;
  }
\end{verbatim}

If we want to replace the ``addition'' with a new two-argument function, we
can modify the above function template as follows:

\begin{verbatim}
  template <typename InputIter, typename T>
  T acculmulate(InputIter first, InputIter last, 
                T init, T (*binfunc)(T x, Ty)) {
    while (first != last) { 
      init = (*binfunc)(init, *first);
      ++first;
    }
    return init;
  }
\end{verbatim}

\paragraph{Adaptors}
Adaptors are STL components which can be used to change the interface of
another component. They are defined as template classes that take a component
type as a parameter. There are three categories of adapters: {\em container
  adaptors\/}, {\em iterator adaptors\/}, and {\em function adaptors\/}. 

{\bf Container adaptors} are used to provide a new data structure such as
\verb+stack+ or \verb+queue+, using existing STL containers such as
\verb+vector+ or \verb+list+. Examples are: {\em stack container adaptor},
     {\em queue container adaptor\/}, and {\em priority queue adaptor\/}. 
Consider a stack container adaptor which can be
applied to a \verb+vector+, \verb+list+, or \verb+deque+.
The {\em stack container adaptor}, \verb+stack<T>+, is a stack of \verb+T+
with a default implementation using a
\verb+deque+. \verb+stack<T, vector<T> >+ is a stack of \verb+T+ with a 
\verb+vector+ implementation. Also, \verb+stack<T, list<T> >+ can be used. 

{\bf Iterator adaptors} are STL components which can be used to change the
interface of an iterator component.  Only one kind of iterator adaptor is
defined in STL: {\em reverse iterator adaptor\/}, which transforms a given
bidirectional or random access iterator into one which the traversal direction
is reversed. 

{\bf Function adaptors} help us construct a wider variety of function
objects. Using function adaptors is often easier than directly constructing a
new function object type with a struct or class definition. There are three
categories of function adaptors: {\em binders\/}, {\em negators\/}, and {\em
  adaptors for pointers to functions\/}. A \bb{binder} is a function adaptor
which converts binary function objects into unary function objects by binding
an argument to some particular value.
\begin{verbatim}
  int *where = find_if(&a[0], &a[1000], 
               bind2nd(greater<int>(), 200));
\end{verbatim}
A \bb{negator} is used to reverse the sense of predicate function objects.
There are two negator adaptors: \verb+not1+ and \verb+not2+. 
\begin{verbatim}
  int *where = find_if(&a[0], &a[1000], 
               not1(bind2nd(greater<int>(), 200)));
\end{verbatim}
An {\bf adaptor for pointers to functions} is provided to allow pointers to
unary and binary functions to work with the other function adaptors provided
in the library. 

\paragraph{Allocators}
C++ STL uses special objects, called {\em allocators\/} to handle the
allocation and deallocation of memory. An allocator defines a {\em memory
  model\/} and it is useful to enforce special memory models, such as
such as shared memory, garbage collection, and object-oriented databases,
without changing interfaces.  

Basic allocator operations are as follows:
\bit
\w {\tt{}a.allocate(n)}: allocate memory for $n$ elements
\w {\tt{}a.construct(p)}: initialize the element to which $p$ refers
\w {\tt{}a.destroy(p)}: destroy the element to which $p$ refers
\w {\tt{}a.deallocate(p, n)}: deallocate memory for $n$ elements to which $p$ refers
\eit

%% See default allocator below:
%% \begin{verbatim}
%%   template <class T>
%%   class allocator {
%%   public:
%%     typedef size_t size_type;
%%     typedef ptrdiff_t difference_type;
%%     typedef T* pointer;
%%     typedef const T* const_pointer;
%%     typedef T& reference;
%%     typedef const T& const_reference;
%%     typedef T value_type;
%%     ...
%%   };
%% \end{verbatim}


\end{document}

% LocalWords:  STL deque multiset multimap ccccc bidir dereferenced rvalue init
% LocalWords:  lvalue dereference Nonmutating struct negators unary negator
% LocalWords:  adaptors adaptor deallocation deallocate
