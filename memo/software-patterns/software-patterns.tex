\documentclass{memo}
\usepackage{mathptm,mydef,myenv}
%\usepackage{MinionPro}
\usepackage[all]{xy}
\begin{document}
\noindent{\large\bf{}Notes on Design Patterns}
\small

\paragraph{Creational Patterns}
\bit
\w \bb{creational patterns}: abstract the \bb{instantiation process}
  \bit
  \w \bb{class creational pattern}: uses inheritance to vary the class that's instantiated
  \w \bb{object creational pattern}: delegates instantiation to another object
  \eit
\w creational patterns gets more important as systems evolve to depend more on
\bb{object composition} than \bb{class inheritance}
  \bit
  \w i.e. emphasis shifts away from hardcoding a fixed set of behaviors
  towards defining a {\em smaller set of fundamental behaviors \/} which can
  be composed into any number of more complex ones
  \eit
\w \bb{Late binding}: when ``instantiating an object'', don NOT \bb{hard code} the
instantiation so that a concrete class will be given as an argument to
\verb+new+. This allows flexibility in:
  \bit
  \w \bb{what gets created}: any object of its subclass of LHS is ok
  \w \bb{who creates it}: 
  \w \bb{how it gets created}: fixed code uses ``new'' but create() can use
  object pool or lookup table or any suitable method  for object creation
  \w \bb{when it's created}: ``new'' creates object at the time we hit
  ``new''; create() can control the exact time when object is created
  \eit
\eit

\paragraph{BASELINE: Hard-coded}
{\scriptsize
\begin{verbatim}
  Maze *MazeGame::createMaze() {
    // hard-coded constructors
    Maze *maze = new Maze;
    Room *r1 = new Room(1);
    Room *r2 = new Room(2);
    Door *d = new Door(r1, r2); 

    maze->addRoom(r1);
    maze->addRoom(r2);
  }
\end{verbatim}
}

\paragraph{ABSTRACT FACTORY (C)}
\bit
\w provides an interface for creating families of related or dependent objects
without specifying their concrete classes
\w pass an ``(factory) object'' to {\tt CreateMaze}, which can be used to
create walls, doors, rooms, etc.

{\scriptsize
\begin{verbatim}
  class MazeFactory {
    virtual Maze *makeMaze() const { 
      return new Maze;
    }
    viirtual Wall *makeWall() const {
      return new Wall;
    }
    ... 
  };
  class EnchantedMazeFactory : public MazeFactory {
    ...
  };

  Maze *MazeGame::createMaze(MazeFactory& factory) {
    Maze *maze = factory.makeMaze();
    Room *r1 = factory.makeRoom(1);
    Room *r2 = factory.makeRoom(2);
    Door *d = factory.makeDoor(r1, r2);
  }
\end{verbatim}
}
\eit

\paragraph{FACTORY METHOD (C)}
\bit
\w {\tt createMaze()} calls virtual functions (instead of constructor calls to
creaate rooms, walls, etc.)
\w create a subclass of {\tt MazeGame\/} which redefines virtual functions
{\scriptsize
\begin{verbatim}
  class MazeGame {
    Maze *createMaze();
   
    // factory methods
    virtual Maze *makeMaze() const { ... }
    virtual Wall *makeWall() const { ... }
    virtual Door *makeDoor() const { ... }
    virtual Room *makeRoom(int i) const { ... }
  };

  Maze *MazeGame::createMaze() {
    Maze *maze = makeMaze();
    Room *r1 = makeRoom(1);
    Room *r2 = makeRoom(2);
    Door *d = makeDoor(r1, r2);

  }

  class BombedMazeGame : public MazeGame {
    virtual Maze *makeMaze() const {
      return new BombedWall; 
    }
    ...
  };
\end{verbatim}
}
\eit

\paragraph{BUILDER (C)}
\bit
\w pass an object that can create a new maze \bb{in its entierty} using
operations for adding rooms, doors, and walls
\w then you can use inheritance to change parts of the maze or the way the
maze is built
{\scriptsize
\begin{verbatim}  
  class MazeBuilder {
  public: 
    virtual void buildMaze();
    virtual void buildRoom(int);
    virtual void buildDoor(int, int);
    virtual Maze *getMaze();
  }; 
  Maze *MazeGame::createMaze(MazeBuilder& builder) {
    builder.buildMaze();
    builder.buildRoom(1);
    builder.buildRoom(2);
    builder.buildDoor(1, 2);
    return builder.getMaze();
  }

  class StandardMazeBuilder : public MazeBuilder {
  public:
    virtual void BuildRoom(int n) {
      if (_currentMaze->roomNo(n)) {
        Room *room = new Room(n);
        _currentMaze->addRoom(room);
      }
    }
  private:
    Maze *_currentMaze;
  };
\end{verbatim}
}
\eit

\paragraph{PROTOTYPE (C)}
\bit
\w parameterize {\tt createMaze} by various prototypical room, door, and wall
objects, then copy and add them to the maze; then change the maze's
composition by replacing these prototypical objects with different ones
{\scriptsize
\begin{verbatim}
  class MazePrototypeFactory : public MazeFactory {
  public:
    MazePrototypeFactory(Maze *, Wall *, Room *, Door *);
    virtual Maze *makeMaze() const;
    virtual Room *makeRoom() const;
    virtual Wall *makeWall() const;
    virtual Door *makeDoor(Room *, Room *) const;
  private:
    Maze *_protytypeMaze;
    Room *_protytypeRoom;
    Wall *_protytypeWall;
    Door *_protytypeDoor;
  };

  MazePrototypeFactory::MazePrototypeFactory(Maze *m, 
      Wall *w, Room *r, Door *d) {
    _protytypeMaze = m;
    _protytypeWall = w;
    _protytypeRoom = r;
    _protytypeDoor = d;
  }
  Door *MazePrototypeFactory::makeDoor(Room *r1, 
      Room *r2) const {
    Door *door = _protytypeDoor->clone();
    door->initialize(r1, r2);
    return door;
  }
  Wall *MazePrototypeFactory::makeWall() const {
    Wall *wall = _protytypeWall->clone();
  }

  // create prototypyical maze
  MazeGame game;
  MazeProtytypeFactory simpleMazeFactory(new Maze,
    new Wall, new Room, new Door);
  Maze *maze = game.createMaze(simpleMazeFactory);

  // to change maze, initialize MazePrototypeFactory
  // with different set of prototypes
  MazePrototypeFactory bomedMazeFactory(new Maze,
    new BombedWall, new RoomWithBomb, new Door);
\end{verbatim}
}
\eit

\paragraph{Structural Patterns}
\bit
\w \bb{COMPOSITE (S)}: 
  \bit
  \w most common pattern: trees, ASTs, etc.
  \eit
\w \bb{BRIDGE (S)}:
  \bit
  \w decouple abstraction from its implementation so the two can vary
  independently
{\scriptsize
\begin{verbatim}
  class Window {
    void drawText(Text& t) {
      getWindowImpl->drawText();
    }
  protected:
    WindowImpl *getWindowImpl();
  };

  class WindowImpl {
  public:
    void drawText(Text& t) { ... }
  }
\end{verbatim}
}
  \eit
\w \bb{ADAPTOR (S)}:
\w \bb{FLYWEIGHT (S)}: 
  \bit
  \w in case the objects are immutable and there're limited
number of possible objects, \bb{share} them 
  \eit
\eit


\end{document}
