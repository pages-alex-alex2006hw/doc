\documentclass{myproc}
\usepackage{mathptm,mydef,myenv}
%\usepackage{MinionPro}
\begin{document}
\small
\noindent{\large\bf{}Notes on OpenMP}

\tableofcontents

\section{Basics}
\subsection{Overview}
\bit
\w supports \bb{shared-memory} parallel programming
\w \bb{goal}: easy to migrate from sequential programming
\eit

\subsection{Definitions}
\bit
\w \bb{thread}: runtime entity that is able to independently execute a stream
of instructions
\w \bb{OS}: creates a process to execute a program; allocates resources to the
process (e.g. memory pages, registers)
\w \bb{one-program, multiple-threads}: resources are shared (incl. address space); 
\eit

\subsection{Fork-join programming model supported by OpenMP}
\bit
\w Program starts as a signgle thread of execution. This thread is called
\bb{initial thread}.
\w Whenever an {\em OpenMP construct is encountered}: initial thread creates a
\bb{team of threads} (i.e. \bb{FORK}) and becomes of the \bb{master} of the
team.
\w Initial thread (master of the team) collaborates with team members to
execute the code dynamically enclosed by the construct.
\w At the end of construct, nly the original master thread continues. All
other team members terminate (this is \bb{JOIN}). 
\eit

\subsection{OpenMP API}
\bit
\w \bb{compiler directives}
\w \bb{runtime library routines}
\w \bb{environment variables}
\eit

\subsection{Work-sharing among threads}
\bit
\w given a \bb{loop}, a work unit is \bb{chunk of iterations}
\w \bb{loop constraint}:
   \bit
   \w the number of iterations must be known upon entry
   \w the number should not change while the loop is executing
   \w iterations must be independent (i.e. execution order of iterations must
   not matter)
   \eit
\eit

\subsection{Flush: Memory fence}
\bit
\w Makes sure that the thread calling it has the same values for shared data
objects as does main memory.
\eit

\subsection{Thread synchronization}
\bit
\w to ensure of proper ordering of accesses to shared objects
\w \bb{barrier synchronization}: OpenMP gets threads to wait at the end of a
work-sharing construct or parallel region until all threads in the team
executing it have finished their portion of the work
\eit


\section{OpenMP Constructs}
\subsection{Parallel construct}
\subsection{Work-sharing constructs}
\subsection{Data-sharing, no wait, and schedule clauses}
\subsection{Barrier constructs}
\subsection{Critical construct}
\subsection{Atomic construct}
\subsection{Locks}
\subsection{Master construct}

\section{Data dependence: Complications in Parallelization}
\subsection{Overview}
\bit
\w one iteration writes, some other iteration reads
\w we cannot guarantee the order
\eit

\subsection{Aliasing: Use restrict}
\bit
\w \bb{restrict keyword}: notifies the C compiler that pointers occupy
disjoint regions in memory (i.e. \bb{no aliasing issue})
  \begin{verbatim}
  void foo(int m, int n, double * restrict a, double * restrict b) ...
  \end{verbatim}
\eit

\end{document}


% LocalWords:  GPU vertices SPMD GPUs
