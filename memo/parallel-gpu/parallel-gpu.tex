\documentclass{memo}
\usepackage{mathptm,mydef,myenv}
%\usepackage{MinionPro}
\begin{document}
\small
\noindent{\large\bf{}GPU Computing}


\paragraph{Overview}
The GPU is designed for a particular class of applications with the following
characteristics.
\bit
\w \bb{Computational requirements are large} Real-time rendering requires
billions of pixels per second, and each pixel requires hundreds of
operations.  
\w \bb{Parallelism is substantial} Operations on vertices and fragments are
mostly independent without little interaction between parallel computations.
\w \bb{Throughput is more important than latency} Due to nature of human
visual system, latency is less important. As a results, GPU pipelines are very
deep, like hundreds to thousands of cycles, with thousands of primitives in
flight at any given time. 
\eit

\paragraph{GPU Programming Model}
The programmable units of the GPU follow a {\em single program multiple-data
  (SPMD)} programming model. For efficiency, the GPU processes many elements
in parallel using the same program. Each element is {\em independent\/} of the
other elements, and in the base programming model, elements cannot communicate
with each other. All GPU programs must be structured in this way: {\em many
  parallel elements, each processed in parallel by a single program\/}. 

Each element can operate on 32-bit integer or floating-point data with a
reasonably generate-purpose instruction set. Elements can read data from a
{\em shared global memory\/} (a \bb{gather} operation) and, with the newest
GPUs, also write back to arbitrary locations in shared global memory
(a \bb{scatter} operation).

\paragraph{Branching in GPU elements}
Allowing a different execution path for each element requires a substantial
amount of control hardware. Instead, today's GPUs support arbitrary control
flow per thread but impose a penalty for incoherent branching.  Elements are
grouped together into blocks, and blocks are processed in parallel. If
elements branch in different directions within a block, the hardware computes
both sides of the branch for all elements in the block. The size of block is
known as the \bb{branch granularity} and has been decreasing with recent GPU
generations -- today, is on the order of 16 elements. 

In GPU programs, the branches are permitted but not free. To make the best use
of the GPU hardware, blocks should have coherent branches. 



\end{document}


% LocalWords:  GPU vertices SPMD GPUs
