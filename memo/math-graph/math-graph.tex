\documentclass{myproc}
\usepackage{mydef,myenv}
\usepackage{mathptm}


%\def\sbf{\sf\bfseries}
\def\sbf{\bfseries}
\pagestyle{empty}

\begin{document}
\small

\noindent{\large\bf Notes on Graphs}

\paragraph{Graphs}
\bit
\w A \bb{graph} $G$ consists of a finite nonempty set $V$ of \bb{vertices} and
   a set $E$ of \bb{edges}, where an edge is a unordered pair of
   distinct vertices of $V$.
\w We denote a graph $G$ with the vertex set $V$ and the edge set $E$ by 
   $G = (V, E)$. And the vertex set of a graph $G$ is denoted by $V(G)$ and
   the edge set of $G$ is denoted by $E(G)$.
\w Given that $e = (u, v) \in E$, we say that $u$ is \bb{adjacent} to $v$ and
   $u$ (or $v$) is \bb{incident} to $e$.
\w A \bb{directed graph} (or \bb{digraph}) consists of a finite nonempty set
$V$ of \bb{vertices} and a set $E$ of \bb{directed edges} (or \bb{arcs}),
  where a directed edge is an ordered pair of distinct vertices of $V$.
\eit

\paragraph{Subgraphs}
\bit
\w A \bb{subgraph} of $G = (V, E)$ is a graph having a vertex set $V'
   \subseteq V$ and 
   an edge set $E' \subseteq E$ such that, for each $(u, v) \in E'$, 
   $u, v \in V'$.
\w A \bb{spanning subgraph} is a subgraph containing all the vertices of $V$.
\w For any set $S \subseteq V$, the \bb{induced subgraph} $G\arc{S}$ is the
maximal subgraph of $G$ with vertex set $V$.
\w Thus, $(u, v) \in E(G\arc{S})$ if and only if $(u, v) \in E(G)$
\eit

\paragraph{Graph isomorphism}
\bit
\w Two graphs $G$ and $H$ are \bb{isomorphic}
  if there exists a one-to-one correspondence $\theta$ 
  between $V(G)$ and $V(H)$
  which preserves adjacency, i.e.
   \[ (u, v) \in E(G) \ \ \Leftrightarrow\ \ 
   (\theta(u), \theta(v)) \in E(H).\]
\w An \bb{invariant} of a graph $G$ is a number associated with $G$ which has
the same value for any graph isomorphic to $G$.
\w A \bb{complete set of invariants} determines a graph up to isomorphism.
\eit

\paragraph{Walks and connectedness}
\bit
\w A \bb{walk} of a graph is an alternating sequence of vertices
of edges $\arc{v_0, e_0, v_1, \cdots, v_{n-1}, e_n, v_n}$, 
beginning and ending with
vertices, in which each edge is incident to the two vertices immediately
preceding and following it.
\w A walk is \bb{closed} if $v_0 = v_n$ and is \bb{open} otherwise.
\w A walk is a \bb{trail} if all the edges are distinct.
\w A walk is a \bb{path} if all the vertices (and thus necessarily all the
edges) are distinct.
\w If the walk is closed, then it is a \bb{cycle} if its $n$ vertices are
distinct and $n \ge 3$.
\w We denote by $C_n$ the graph consisting of a cycle with $n$ vertices.
\w We denote by $P_n$ a path with $n$ vertices.
\w {\sbf Theorem}: The edge set of a graph can be partitioned into cycles
  if and only if every vertex has even degree.
\w A graph is \bb{connected} if every pair of vertices are joined by a path.
\w A maximally connected subgraph of $G$ is called a \bb{connected component}
(or \bb{component}) of $G$. 
\w The \bb{length} of a walk $\arc{v_0, e_0, v_1, \cdots, v_n}$ is $n$, the
 number of edges in it.
\w The \bb{girth} of a graph $G$, denoted $g(G)$, is the length of a shortest
cycle in $G$.
\w The \bb{circumference} of a graph $G$, denoted $c(G)$ is the length of any
longest cycle.
\w The \bb{distance} $d(u, v)$ between two vertices $u$ and $v$ in $G$ is the
length of a shortest path joining them if any. Otherwise, $d(u, v) = \infty$.

\w {\sbf Theorem}: In a connected graph, distance is a {\em
  metric\/}; that is, for all vertices $u, v$ and $w$,
 \ben
 \w [(a)] $d(u, v) \ge 0$, with $d(u, v) = 0$ iff $u = v$.
 \w [(b)] $d(u, v) = d(v, u)$.
 \w [(c)] $d(u, v) + d(v, w) \ge d(u, w)$.
 \een

%\w A shortest $u$-$v$ path is called a \bb{geodesic}.
\w The \bb{diameter} $d(G)$ of a connected graph $G$ is the length of any
longest $u$-$v$ path.
\w The \bb{square} $G^2$ of a graph $G$ has $V(G^2) =V(G)$ with $u, v$
adjacent in $G^2$ whenever $d(u, v) \le 2$ in $G$.
\w Adjacency matrices for powers $G^2, G^3, \cdots$ can be obtained by
 multiplying the adjacency matrix $A$ for $G$. Actually, $A_{uv}^k$
 contains the number of distinct paths between $u$ and $v$ whose length is
 $\le k$.
 Adjacency matrix $G^n$ is essentially a transitive closure.
\eit
\paragraph{Degrees}
\bit
\w The \bb{degree} of a vertex $v_i$ in graph $G$, denoted by $d_i$ or
degree$(v_i)$, is the number of edges incident to $v_i$.
\w {\sbf Theorem}: The sum of the degrees of the vertices of a graph $G = (V,
E)$ is twice the number of edges,
  \[ \sum_{v \in V} \mbox{degree}(v) = 2|E|.\]
\w {\sbf Corollary}: In any graph, the number of vertices of
odd degree is even.
\w The minimum degree among the vertices of $G$ is denoted $\delta(G)$.
\w The maximum degree among the vertices of $G$ is denoted $\Delta(G)$.
\w If $\delta(G) = \Delta(G) = r$, then all vertices have the same degree
 and $G$ is called \bb{regular} of degree $r$.
\w 3-regular graphs are called \bb{cubic}.
\w {\sbf Corollary}: Every cubic graph has an even number of vertices. 
\eit

\paragraph{Trees: Characterization}
\bit
\w A graph is \bb{acyclic} if it has no cycles.
\w A \bb{tree} is a connected acyclic graph.
\w Any graph without cycles is a \bb{forest}, where the components of a forest
are trees.
\w {\sbf Theorem}:
The following statements are equivalent for a graph $G$:
  \ben
  \w [(a)] $G$ is a tree.
  \w [(b)] Every two vertices of $G$ are joined by a unique path.
  \w [(c)] $G$ is connected and $m = n - 1$.
  \w [(d)] $G$ is acyclic and $m = n - 1$.
  \w [(e)] $G$ is acyclic and if two nonadjacent vertices of $G$ are joined by
  an edge $e$, then $G + e$ has exactly one cycle.
  \w [(f)] $G$ is connected, is not $K_n$ for $n \ge 4$, and if any two
  nonadjacent vertices of $G$ are joined by an edge $e$, then $G + e$ has
  exactly one cycle.
  \w [(g)] $G$ is not $K_3 \cup K_1$ or $K_3 \cup K_2$, $m = n - 1$, and if
  any two nonadjacent vertices of $G$ are joined by an edge $e$, then $G + e$
  has exactly one cycle.
  \een

\w {\sbf Corollary} \cite{Harary69}:
  Every nontrivial tree has at least two endvertices.
\eit

\paragraph{Trees: Centers and centroids}
\bit
\w The \bb{eccentricity} $e(v)$ of a vertex $v$ in a {\em connected\/} graph
$G$ is max$\{d(u, v)\}$ for all $u \in V$ (i.e. distance to the farthest
vertex). 
\w The \bb{radius} $r(G)$ is the minimum eccentricity of the vertices.
\w The \bb{diameter} $d(G)$ is the maximum eccentricity of the vertices.
\w A vertex $v$ is a \bb{central vertex} of $G$ if $e(v) = R(G)$, and
 the \bb{center} of $G$ is the set of all central vertices of $G$.
\w {\sbf Theorem}: Every tree has a center consisting of either one vertex or
two {\em adjacent\/} vertices.
\w A \bb{branch} at a vertex $u$ of a tree $T$ is a maximal subtree containing
  $u$ as an {\em endvertex\/}. 
\w The number of branches at $u$ is $\mbox{degree}(u)$.
\w The \bb{weight} at a vertex $u$ of a tree $T$ is the maximum number of 
  edges in any branch at $u$.
\w A vertex $v$ is a \bb{centroid vertex} of a tree $T$ if $v$ has minimum 
   weight, and the \bb{centroid} of $T$ consists of all such vertices.
\w {\sbf Theorem}: Every tree has a centroid consisting of either one vertex
of two {\em adjacent\/} vertices. 
\eit

\paragraph{Trees: Block-cutvertex trees}
\bit
\w For a connected graph $G$ with blocks $\{B_i\}$ and cutvertices $\{c_j\}$,
the \bb{block-cutvertex graph}\footnote{For an application, see \cite{TV84}}
 of $G$, denoted by $bc(G)$, is defined as the
graph having vertex set $\{B_i\} \cup \{c_j\}$, with two vertices adjacent if
one corresponds to a block $B_i$ and the other to a cutvertex $c_j$ and $c_j$
is in $B_i$.
\w A block-cutvertex graph is a bipartite graph.
\w {\sbf Theorem}:  A graph $G$ is the block-cutvertex graph of
some graph $H$ if and only if it is a tree in which the distance between any
two endvertices is even.
\eit

\paragraph{Independent cycles and cocycles}
\bit
\w A \bb{0-chain} of $G$ is a linear combination $\sum \epsilon_i v_i$
of vertices and a \bb{1-chain} is a sum $\sum \epsilon_i e_i$ of edges.
\w The \bb{boundary operator} $\partial$ sends 1-chains to 0-chains according
to the rules:
  \ben
  \w [(a)] $\partial$ is linear.
  \w [(b)] if $e = (u, v)$, then $\partial x = u + v$.
  \een
\w The \bb{coboundary operator} $\delta$ sends 0-chains to 1-chains by the
rules. 
  \ben
  \w [(a)] $\delta$ is linear.
  \w [(b)] if $\delta v = \sum \epsilon_i e_i$, where $\epsilon_i = 1$
  whenever $x_i$ is incident with $v$.
  \een
\eit


%% section 7 of bondy/murty
\paragraph{Independence sets}
\bit
\w A subset $S$ of $V$ is an \bb{independence set} of $G$ if no two
vertices of $S$ are adjacent in $G$. 
\w An independent set $S$ is a \bb{maximum independent set} if
	$G$ has no independent set $S'$ with $|S'| \supseteq |S|$.
\w A subset $V'$ of $V$ is a \bb{covering} of $G$ if every edge
of $G$ has at least one end in $V'$.
\w \bb{maximum independent sets VS minimum vertex covering}
\w {\bfseries Theorem}: 
A set $S \subseteq V$ is an independent set of $G$ if and only if $V -
S$ is a covering of $G$.

\w The size of a maximum independence set is called the
\bb{independence number} of $G$ and is denoted by $\alpha(G)$.
\w The size of a minimum covering of $G$ is the \bb{covering
number} of $G$ and is denoted by $\beta(G)$.
\w {\sbf Theorem}: For any graph $G = (V, E)$, $\ \alpha(G) + \beta(G) =
|V|$. 
\w An \bb{edge covering} of $G$ is a subset $E'$ of $E$ such that
each vertex of $G$ is incident to some edge in $E'$.
\w Edge analogue of an independent set is a set of edges which are
pairwise non-adjacent, that is, a {\em matching\/}.
\w \bb{maximum matching VS minimum edge covering}
\w We denote the number of edges in a \bb{maximum matching} of $G$ by
$\alpha'(G)$ and call it the \bb{edge independence number}.
\w We denote the size of minimum edge cover of $G$ by $\beta'(G)$ and
call it the \bb{edge covering number}.
\w {\sbf Theorem}: For any graph $G = (V, E)$, if $\delta > 0$, 
  then $\alpha'(G) + \beta'(G) = |V|$.
\w {\sbf Theorem}: In a bipartite graph $G$ with $\delta > 0$, the number of
vertices in a maximum independent set is equal to the number of edges
in a minimum edge cover.
\eit



\paragraph{Depth-first search}
\bit
\w DFS has a nice feature that partitions the edge set into forward edges,
backward edges, tree edges, and cross edges, which can be used for
binconnectivity, planarity algorithms.
\w \bb{topological sorting}: a vertex is \bb{finished} only after all vertices
reachable from it are finished; so ordering vertices in decreasing order of
finish time is a topological order
\w An \bb{articulation point} (a.k.a. \bb{cut vertex}) is a vertex whose
deletion disconnects the remaining graph into multiple components.
\w A graph is \bb{biconnected} if there is no articulation point.
\w A \bb{biconnected component} of a graph is a maximal subset of edges
s.t. the corresponding induced subgraph is biconnected. Typically,
an articulation point join different biconnected components.
\w \bb{Hopcroft-Tarjan algorithm for biconnected components}: 

   \[ low(v) = \min_{w \in V}\{D[v], D[w]\} \]
   where $D[v]$ is the discover time of $v$, 
    $(u, w)$ is a back edge for some descendent $u$ of $v$.
   That is $low(v)$ of a vertex $v$ is the discovery number of the vertex
   closest to the root that can be reached from $v$  by following zero or more
   tree edges downward and at most one back edge upward.
\w \bb{Theorem}: Let $T$ be a DFS tree of a connected graph $G$, and let $v$
be a nonroot vertex of $T$. Vertex $v$ is a \bb{cut vertex} if and only if
there is a child $w$ of $v$ in $T$ with $low(w) \ge D[v]$.
\w \bb{strong connectivity}: can be found by two DFS over $G$: once for $G$
and second time with $G^T$ (edges of $G$ are reversed).
  \ben
  \w [(a)] call \bb{DFS}($G$) to compute \bb{finish times} $F(v)$
  \w [(b)] call \bb{DFS}($G^T$) but visit vertices in the order of decreasing $F(v)$.
  \een
\eit

\paragraph{Minimum spanning trees}
\bit
\w \bb{Kruskal's algorithm}:

\w \bb{Prim's algorithm}:
\eit

\paragraph{Single-source shortest paths}

\paragraph{All-pairs shortest paths}
\bit
\w $D^k(i, j)$: distance between vertices $i$ and $j$ which goes through
vertices $\le k$.
\w Iterate the following for $1 \le k \le n$
\[ D^k_{ij} = \left\{\begin{array}{ll}
   w_{ij}  & k = 0, \\ 
   \min\{D^{k-1}_{ij},  D^{k-1}_{ik}  + D^{k-1}_{kj}\} & k \le 1
\end{array} \right.\]
\eit


\paragraph{Network flows}
\bit
\w A \bb{flow network} $G = (V, E)$ is a directed graph where each edge
$\arc{u, v} \in E$ has a nonnegative \bb{capacity} $c(u, v) \ge 0$.
There are two special vertices: a \bb{source} $s \in V$ and a \bb{sink} $t \in
V$. 
\w A \bb{flow} in $G$ is a function $f: V \times V \rightarrow \R$, which
satisfies the following constraints (i.e. is a valid assignment of flow):
  \ben
  \w \bb{capacity constraints}: for all $u, v \in V$, $f(u, v) \le c(u, v)$.
  \w \bb{skew symmetry}: for all $u, v \in V$, $f(u, v) = - f(v, u)$.
  \w \bb{flow conservation}: for all $u \in V - \{s, t\}$, 
     \[ \sum_{u \in V} f(u, v) = 0.\]
  \een
\w The \bb{value of a flow $f$} is defined as
   \[ |f| = \sum_{v \in V} f(s, v).\]
\eit


\bibliographystyle{plain}
\bibliography{bib/mac,bib/math,bib/algo}
\nocite{Harary69}
\end{document}


% LocalWords:  Cheoljoo Jeong vertices endvertices endvertex cutvertex bc algo
% LocalWords:  cutvertices iff uv cocycles coboundary
