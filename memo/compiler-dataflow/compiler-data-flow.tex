\documentclass{memo}
\usepackage{mathptm,mydef,myenv}
\usepackage[all]{xy}
%\usepackage{MinionPro}
\begin{document}
\small
\noindent{\large\bf{}Data Flow Analysis}

\paragraph{Overview}
Given a program, while its static representation (e.g. control flow graph) may
be finite but it can have infinitely many possible execution paths
dynamically. 
For each point in the program, data flow analysis abstraction tries to combine
information of all the instances of the same program point. 


\paragraph{Dataflow abstraction}
The execution of a program can be viewed as a series of transformations of the
\bb{program state}, which consists of the values of all the variables in the
program, including those associated with stack frames below the top of the
runtime stack. 
Each execution of a {\em statement\/} transforms an input state into an output
state. 

When the program behavior is analyzed, we need to consider {\em all possible
sequences of program points\/} through a {\bf flowgraph\/} that the program 
execution can take.

\paragraph{Dataflow analysis schema}
In each application of dataflow analysis, we associate with every program
point a \bb{dataflow value} that represents an abstraction of the {\em set of
  all possible program states\/} that can be observed at that point. 
The set of all possible dataflow values is the \bb{domain} for this
application. 

We denote the dataflow values before and after each statement $s$ by IN$(s)$
and OUT$(s)$, respectively. The \bb{dataflow problem} is to find a solution to
a set of constraints: 1) those based on the semantics of the statements
(``transfer functions'') and 2) those based on the flow of control.

\[ \xymatrix@-0.5cm{
    \mbox{IN}(s) \ar[d] \ar@{.>}@/^10ex/[dd]^{f_s} &&&&
    \mbox{IN}(s)  \\
      *++[F]{s} \ar[d]  &&&& 
      *++[F]{s} \ar[u]  \\ 
    \mbox{OUT}(s) &&&& 
    \mbox{OUT}(s) \ar[u]\ar@{.>}@/_10ex/[uu]_{f_s} 
}\]

\paragraph{Constraint \#1: Transfer functions}
The relationship between dataflow values before and after the assignment
statement is a \bb{transfer function}. There are two types of transfer
functions: {\em forward transfer functions} and {\em backward transfer
  functions}. 

In a \bb{forward-flow problem}, the transfer function of a statement $s$,
denoted by $f_s$, takes the dataflow value {\em before\/} the statement and
produces a new dataflow value after the statement, That is,
   \[ \mbox{OUT}(s) = f_s(\mbox{IN}(s)). \]
Conversely, in a \bb{backward-flow problem}, the transfer function of a
statement $s$ converts a value after the statement to a value before the
statement. That is,
   \[ \mbox{IN}(s) = f_s(\mbox{OUT}(s)). \]

\paragraph{Constraint \#2: Control flow constraints}
The second set of constraints on dataflow values is derived from the flow of
control. Within a basic block, control flow is simple. If a block $B$ consists
of statements $s_1, \cdots, s_n$, then the control flow value out of $s_i$ is
the same as the control flow value into $s_{i+1}$. That is,
   \[ \mbox{IN}(s_{i+1}) =\mbox{OUT}(s_i), \ 1 \le i < n. \]


\paragraph{Dataflow analysis on basic blocks}
By default, dataflow analysis involves computing dataflow values at each
{\em statement\/} but, to save time and space, we can perform analysis at the
{\em basic-block level\/}. For this, we extend the \bb{transfer functions}, IN 
and OUT functions to basic blocks. 
Given that the basic block $B$ consists of statements $s_1, \cdots, s_n$,
\begin{eqnarray*}
  \mbox{IN}(B) & = & \mbox{IN}(s_1) \\
  \mbox{OUT}(B) & = & f_{s_n} \circ f_{s_{n-1}} \circ \cdots \circ f_{s_1}
  (\mbox{IN}(B)) \\
                & = & f_B(IN(B))
\end{eqnarray*}
Also control flow constraints are extended to blocks as follows.
For \bb{forward-flow problems}:
\begin{eqnarray*}
  IN(B) & = & \bigcup_{P \in \Gamma^{-1}(B)} OUT(P)
\end{eqnarray*}
For \bb{backward-flow problems}:
\begin{eqnarray*}
  IN(B) & = & f_B(OUT(B)) \\
  OUT(B) & = & \bigcup_{S \in \Gamma(B)} IN(S)
\end{eqnarray*}

\[\xymatrix@-0.5cm{
  P_0 \ar[rd]& \cdots & P_k \ar[ld] &&&& &B\\
        &  B & &&&&  S_0 \ar[ru]& \cdots & S_l \ar[lu]
}\]

\paragraph{Three dataflow analysis problems}
We will consider three dataflow analysis problems: {\em reaching definition},
{\em live variable analysis\/}, and {\em available expressions}. 

\paragraph{Example \#1: Reaching definitions}
A definition $d$ of a variable $x$ \bb{reaches} a point $p$ if there is a
patch ferom the point immediately following $d$ to $p$, such that $d$ is not
``killed'' along that path. A definition of a variable $x$ is \bb{killed} if
there is any other definition of $x$ anywhere on the path. A definition can
happen through assginment, procedure calls, procedure parameters, or
aliases. We must be conservative in the notion of ``definition''. 

As for transfer function at a block $B$, we are mostly interested with which
new definitions are {\em generated\/} and which existing definitions are {\em
    killed\/} at $B$.

\paragraph{Example \#2: Live variable analysis}
The {\em live variable analysis problem\/} is to determine, for eavh variable
$x$ and point $p$, whether the value of $x$ at $p$ could be used along some
path the the flow graph starting at $p$. 
If so, variable $x$ is said to be \bb{live} at point $p$. Otherwise, $x$ is
\bb{dead} at $p$. One important use of this analysis is \bb{register
  allocation} for basic blocks. 
\bit
\w def$(B)$: the set of variables {\em defined\/} (i.e. definitely assigned
values) in $B$ prior to any use of that variable in $B$
\w use$(B)$: the set of variables whose values may be used in $B$ prior to any
definition of the variable
\eit

\paragraph{Example \#3: Available expressions}
An expression $x + y$ is \bb{available} at point $p$ if every path from the
entry node to $p$ evaluates $x + y$, and after the last such evaluation prior
to reaching $p$, there are no subsequent assignments to $x$ or $y$. 

For the available expressions problem, we say that a block \bb{kills}
expression $x + y$ if it assigns (or may assign)  $x$ or $y$ and does not
subsequently recompute $x + y$. A block \bb{generates} expression $x + y$ if
it defintely evaluates $x + y$ and does not subsequently define $x$ or $y$. 

\begin{figure*}[hbt]
\small
\centerline{\begin{tabular}{l|l|l|l} \hline\hline
& Reaching definition & Live Variables & Available Expressions \\ \hline\hline
domain & sets of definitions & sets of variables & sets of expressions
\\ \hline
direction & forwards & backwards & forwards \\ \hline
transfer function & $\mbox{gen}_B \cup (x - \mbox{kill}_B)$ & $\mbox{use}_B
\cup (x - \mbox{def}_B)$ & $\mbox{egen}_B \cup (x - \mbox{ekill}_B)$ \\ 
boundary & OUT($s$) = $\emptyset$ & IN($t$) = $\emptyset$ & OUT($s$) =
$\emptyset$ \\ \hline
meet ($\wedge$)  & $\cup$ & $\cup$ & $\cap$ \\ \hline
equations & OUT$(B) = f_B(\mbox{IN}(B))$ & IN$(B) = f_B(\mbox{OUT}(B))$ &
              OUT$(B) = f_B(\mbox{IN}(B))$ \\ 
      & IN$(B) = \bigwedge_{P \in \Gamma^{-1}(B)} \mbox{OUT}(B)$ & 
         OUT$(B) = \bigwedge_{S \in \Gamma(B)} \mbox{IN}(B)$ &
              IN$(B) = \bigwedge_{P \in \Gamma^{-1}(B)} \mbox{OUT}(B)$
              \\ \hline
initialize & OUT($B$) = $\emptyset$ & IN($B$) = $\emptyset$ & OUT($B$) = $U$
\\ \hline\hline
\end{tabular}}
\end{figure*}




\paragraph{Dataflow analysis framework}
nA {\em dataflow analysis framework} $(D, V, \wedge, F)$ consists of
\ben
\w A direction of dataflow $D$: either {\em forwards\/} or {\em backwards\/}.
\w A {\em semilattice}, which includes a {\em domain\/} of values $V$ and a
   {\em meet operator} $\wedge$. 
\w A family $F: V \rightarrow V$ of {\em transfer functions\/}. This family
must include functions for the boundary conditions, which are constant
transfer functions for the special entry and exit node of flowgraphs.
\een

\paragraph{Semilattices}
A \bb{semilattice} is a set $V$ and a binary meet operator $\wedge$ s.t. for
all $x, y, z \in V$: 
\ben
\w [(a)] $x \wedge x = x$ (meet is \bb{idempotent}).
\w [(b)] $x \wedge y = y \wedge x$ (meet is \bb{commutative}).
\w [(c)] $x \wedge (y \wedge z) = (x \wedge y) \wedge z$ (meet is \bb{associative}).
\een
A semilattice has a \bb{top} element, denoted $\top$, such that
   \[ \forall{x \in V},\ \top \wedge x = x. \]
Optionally, a semilattice may have a \bb{bottom} element, denoted $\bot$, such
that 
   \[ \forall{x \in V},\ \bot \wedge x = \bot. \]

\paragraph{Partial order for a semilattice}
The meet operator, $\wedge$, of a semilattice defines a partial order on the
values of the domain. A relation $\le$ is a \bb{partial order} on a set $V$ if
it's {\em reflexive, antisymmetric, and transitive}. 
The pair $(V, \le)$ is called a \bb{poset}. 

We can define a partial order $\le$ for a semilattice $(V, \wedge)$ as
follows: for all $x, y  \in V$, 
\[ x \le y \mbox{\ iff\ } x \wedge y = x. \]
We can easily prove that the order $\le$ is a partial order. 

\paragraph{Greatest lower bounds}
Suppose $(V, \wedge)$ is a semilattice. A \bb{greatest lower bound} (or
\bb{glb}) of domain elements $x$ and $y$ is an element $g$ such that
\ben
\w [(a)] $g \le x$,
\w [(b)] $g \le y$, and
\w [(c)] if $z$ is any element such that $z \le x$ and $z \le y$, then $z \le
g$. 
\een
It turns out that $x \wedge y$ is their {\em only\/} greatest lower bound.

\[ \xymatrix@-0.5cm{
    & \{\} \equiv \top \ar[ld]\ar[d]\ar[rd]&  \\ 
 \{d_1\}\ar[d]\ar[rd] & \{d_2\} \ar[ld]\ar[rd]& \{d_3\}\ar[ld]\ar[d] \\
 \{d_1,d_2\} \ar[rd]& \{d_1,d_3\} \ar[d]& \{d_2, d_3\}\ar[ld] \\
  & \{d_1,d_2,d_3\} \equiv \bot & 
}
\]
\end{document}

% LocalWords:  Dataflow runtime flowgraph dataflow semilattice flowgraphs poset
% LocalWords:  Semilattices iff glb ld
