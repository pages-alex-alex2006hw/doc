% examples.tex

\documentclass[11pt]{article}
\usepackage{times,boxdims,pstcol,pst-grad}
\usepackage[dvips]{lscape}
\usepackage[dvips]{graphicx}

% instead of \usepackage[cm]{fullpage} (letter size):
  \setlength{\topmargin}{-29.59087pt}
  \setlength{\headheight}{0pt}
  \setlength{\headsep}{0pt}
  \setlength{\textheight}{679.61174pt}
  \setlength{\footskip}{30.0pt}
  \setlength{\textwidth}{528.93675pt}
  \setlength{\oddsidemargin}{-29.59087pt}
  \setlength{\evensidemargin}{\oddsidemargin}

% To test a new installation, all but one of the diagram inclusions
% below might be commented out, and re-introduced one by one.

% Because this file consists essentially only of diagrams, judicious
% \clearpage statements have to be introduced to prevent TeX capacity overflow.

\newcommand{\src}[1]{{\tt [#1]}}
\newcommand{\makepic}{\box\graph} % Required only for gpic -t
\newbox\graph

\begin{document}
  \hfill
  {\large\bf Examples:
    \input{Version.tex}}
  \hfill\break

  This is a collection of diagrams the author has had occasion to
  produce using m4 circuit macros and others, and gpic or dpic.  In
  some cases, there are other or better m4 or pic constructs for
  producing the same drawings, but for reference the names of the
  source files are shown.  Most of the examples can be processed using
  either dpic -p or gpic -t, which has meant that in a few cases the
  source is slightly more complicated than it would be if only one
  processor had been assumed.

% For some more examples in the context of a textbook, see
% J. Dwight Aplevich, {\sl The Essentials of Linear State-Space
% Systems,} New York, John Wiley \& Sons Inc., 2000.

% To trick the html into inputting quick.m4
% \input quick
%\caption{The figure produced by <tt>quick.m4</tt>, as described in the manual.
%    \src{ex01.m4}.}

% \centerline works for both dpic -p and gpic -t:
  \begin{figure}[h!t]
    \centerline{\input CctTable \makepic}
    \caption{Two-terminal element macros, with some variations
    \src{CctTable.m4}.}
  \end{figure}

  \begin{figure}[h!t]
    \centerline{\input Diodes \makepic}
    \caption{Diode variants
    \src{Diodes.m4}.}
  \end{figure}

  \begin{figure}[h!t]
    \centerline{\input AmpTable \makepic}
    \caption{Amplifier, delay, and integrator
    \src{AmpTable.m4}.}
  \end{figure}

  \begin{figure}[h!t]
    {\small\centerline{\input Misc \makepic} }
    \caption{Miscellaneous basic elements
    \src{Misc.m4}.}
  \end{figure}

  \begin{figure}[h!t]
    \centerline{\input Relay \makepic}
    \caption{Variants of the {\tt contact} and {\tt relay} macros
    \src{Relay.m4}.}
  \end{figure}

  \begin{figure}[h!t]
    \centerline{\input Bip \makepic}
    \caption{Bipolar transistor variants (drawing direction: up)
    \src{Bip.m4}.}
  \end{figure}

  \begin{figure}[h!t]
    \centerline{\input fet \makepic}
    \caption{JFET, insulated-gate enhancement, depletion,
     and simplified MOSFETS
    \src{fet.m4}.}
  \end{figure}

  \begin{figure}[h!t]
    \centerline{\input ujt \makepic}
    \caption{UJT variants
    \src{ujt.m4}.}
  \end{figure}

  \begin{figure}[h!t]
    \centerline{\input scr \makepic}
    \caption{SCR variants
    \src{scr.m4}.}
  \end{figure}

  \begin{figure}[h!t]
    \centerline{\input Liblog \makepic}
    \caption{Basic logic gates
    \src{Liblog.m4}.}
  \end{figure}

  \begin{figure}[h!t]
    \centerline{\input ex01 \makepic}
    \caption{A simple labeled circuit
    \src{ex01.m4}.}
  \end{figure}

  \begin{figure}[h!t]
    \centerline{\input ex18 \makepic}
    \caption{Precision half-wave rectifier
      (illustrating {\tt opamp, diode, resistor, ground,} and labels)
    \src{ex18.m4}.}
  \end{figure}

  \begin{figure}[h!t]
    \centerline{\input ex19 \makepic}
    \caption{ Tunnel diode circuit
    \src{ex19.m4}.}
  \end{figure}

  \begin{figure}[h!t]
    \centerline{\input gpar \makepic}
    \caption{ A skewed circuit for used to test the macro
      {\tt gpar\_(}{\sl element}, {\sl element}, {\sl separation}{\tt )}
    \src{gpar.m4}.}
  \end{figure}

% \vspace{1ex}
  \begin{figure}[h!t]
    \centerline{\input ex12 \makepic}
    \caption{ A CMOS NAND gate and a balanced test circuit
    \src{ex12.m4}.}
  \end{figure}

  \begin{figure}[h!t]
    \centerline{\input ex11 \makepic}
    \caption{Transistor radio audio chain (a modest circuit with a few
      custom elements)
    \src{ex11.m4}.}
  \end{figure}

  \begin{figure}[h!t]
    \centerline{\input Variable \makepic}
    \caption{Arrows to indicate variability
    \src{Variable.m4}.}
  \end{figure}

  \begin{figure}[h!t]
    \centerline{\input ex02 \makepic}
    \caption{Elements at obtuse angles
    \src{ex02.m4}.}
  \end{figure}
\clearpage

  \begin{figure}[h!t]
    \centerline{\input ex04 \makepic}
    \caption{Labels on non-manhattan elements
    \src{ex04.m4}.}
  \end{figure}

  \begin{figure}[h!t]
    \centerline{\input ex16 \makepic}
    \caption{A rate $1/2$ binary convolutional coder and its state diagram
    \src{ex16.m4}.}
  \end{figure}

  \begin{figure}[h!t]
    \centerline{\input pwrsupply \makepic}
    \caption{An elementary power supply circuit
    \src{pwrsupply.m4}.}
  \end{figure}
%\clearpage

  \begin{figure}[h!t]
    \centerline{\input Drive \makepic}
    \caption{Synchronous machine driven by variable-speed drive and rectifier
    \src{Drive.m4}.}
  \end{figure}
\clearpage

  \begin{figure}[h!t]
    \centerline{\input sfg \makepic}
    \caption{Signal-flow graphs
    \src{sfg.m4}.}
  \end{figure}

  \begin{figure}[h!t]
    \centerline{\input ex03 \makepic}
    \caption{Digital filter
    \src{ex03.m4}.}
  \end{figure}

  \begin{figure}[h!t]
    \centerline{\input ex08 \makepic}
    \caption{General-purpose latch: a small logic circuit
    \src{ex08.m4}.}
  \end{figure}

  \begin{figure}[h!t]
    \centerline{\input ex10 \makepic}
    \caption{Non-planar graph (illustrating the {\tt crossover} macro)
    \src{ex10.m4}.}
  \end{figure}

  \begin{figure}[h!t]
    \centerline{\input ex07 \makepic}
    \caption{A line diagram
    \src{ex07.m4}.}
  \end{figure}

  \begin{figure}[h!t]
    \centerline{\input ex21 \makepic}
    \caption{Some flip-flops
    \src{ex21.m4}.}
  \end{figure}

  \begin{figure}[h!t]
    \centerline{\input control \makepic}
    \caption{Control-system block diagrams that do not require m4
    \src{control.m4}.}
  \end{figure}
  \clearpage

  \begin{figure}[h!t]
    \centerline{\input exp \makepic}
    \caption{Test of {\tt project} and other {\tt lib3D}
      macros, showing the projection of a solid onto
      the $y_1,z_1$ plane by sighting along the $x_1$ axis.
  \src{exp.m4}.}
  \end{figure}

  \begin{figure}[h!t]
    \centerline{\input graysurf \makepic}
    \caption{Plotting a surface using a gray scale
  \src{graysurf.m4}.}
  \end{figure}

  \begin{figure}[h!t]
    \centerline{\input ex09 \makepic}
    \caption{Illustrating the macro
      {\tt dimension\_(}{\sl linespec}, {\sl offset}, {\sl label},
      {\sl blank width}, {\sl tic offset},{\tt <-|->)}.
      A negative second argument signifies an offset to the right of the
      direction defined by {\sl linespec.} The {\sl label} argument is
      treated as a string, except that if its first character is a double
      quote then the argument is copied literally
    \src{ex09.m4}.}
  \end{figure}

  \begin{figure}[h!t]
    \centerline{\input ex05 \makepic}
    \caption{Use of {\tt darrow}
    \src{ex05.m4}.}
  \end{figure}
\clearpage

  \begin{figure}[h!t]
    \centerline{\input ex06 \makepic}
    \caption{Crosshatching by {\tt for} loops
    \src{ex06.m4}.}
  \end{figure}

  \begin{figure}[h!t]
    \centerline{\input ex17 \makepic}
    \caption{Repetitive network created by Pic looping
    \src{ex17.m4}.}
  \end{figure}

  \begin{figure}[h!t]
    \centerline{\input ex00 \makepic}
    \caption{Magnetic field
    \src{ex00.m4}.}
  \end{figure}

  \begin{landscape}
  \begin{figure}[h!t]
    \centerline{\input lcct \makepic}
    \caption{A digital circuit of moderate size,
      redrawn from M.~P.~Maclenan and G.~M.~Burns,
      ``An Approach to Drawing Circuit Diagrams for Text Books,''
      Tugboat (12)1, March 1991, pp.\ 66-69
    \src{lcct.m4}.}
  \end{figure}
  \end{landscape}

  \begin{figure}[h!t]
    \centerline{\input csc \makepic}
    \caption{Conestoga Sailing Club (illustrating the filling of arbitrary
      shapes)
    \src{csc.m4}.}
  \end{figure}

  \begin{figure}[h!t]
    \centerline{\input rose \makepic}
    \caption{Redrawn from a detail of the set design for the musical
      {\it Dracula,} used for testing {\bf dpic}.  This diagram consumes much
      \LaTeX\ main memory
    \src{rose.m4}.}
  \end{figure}
  \clearpage

  \begin{figure}[h!t]
    \centerline{\input diamond \makepic}
    \caption{Variations on M.~Goossens, S.~Rahtz, and F.~Mittelbach,
      {\em The \LaTeX\ Graphics Companion,} Addison-Wesley 1997, pp.\ 57-58
    \src{diamond.m4}.}
  \end{figure}

  \begin{figure}[h!t]
    \centerline{\input worm \makepic}
    \caption{An exercise in calculating RGB colours
    \src{worm.m4}.}
  \end{figure}
  \clearpage

% \begin{figure}[h!t]
%   \centerline{\input yinyang \makepic}
%   \caption{A partially-filled figure
%   \src{yinyang.m4}.}
% \end{figure}

  \begin{figure}[h!t]
    \centerline{\input Sierpinski \makepic}
    \caption{The Sierpinski triangle: a test of pic macros
    \src{Sierpinski.m4}.}
  \end{figure}

  \begin{figure}[h!t]
    \centerline{\input recycle \makepic}
    \caption{Modest repetition
    \src{recycle.m4}.}
  \end{figure}

  \begin{figure}[h!t]
    \centerline{\input ex15 \makepic}
    \caption{Simple diagrams that are easily drawn by looping
    \src{ex15.m4}.}
   \end{figure}

  \begin{figure}[h!t]
    \centerline{\input Btree \makepic}
    \caption{A binary tree
    \src{Btree.m4}.}
  \end{figure}

  \begin{figure}[h!t]
    \centerline{\input Flow \makepic}
    \caption{A flowchart sampler
    \src{Flow.m4}.}
  \end{figure}

 \begin{figure}[h!t]
   \centerline{\input Incleps \makepic}
   \caption{Overlaying an eps figure with line graphics
   \src{Incleps.m4}.}
 \end{figure}

\end{document}
